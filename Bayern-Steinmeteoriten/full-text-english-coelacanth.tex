\documentclass[a4paper, 12pt, oneside]{article}
\usepackage[nf]{coelacanth}
\usepackage[T1]{fontenc}
\usepackage{booktabs}
\setlength{\emergencystretch}{15pt}
\usepackage{fancyhdr}
\usepackage{graphicx}
\usepackage[figurename=]{caption}
\graphicspath{ {./} }
\begin{document}
\begin{titlepage} % Suppresses headers and footers on the title page
	\centering % Centre everything on the title page
	\scshape % Use small caps for all text on the title page

	%------------------------------------------------
	%	Title
	%------------------------------------------------
	
	\rule{\textwidth}{1.6pt}\vspace*{-\baselineskip}\vspace*{2pt} % Thick horizontal rule
	\rule{\textwidth}{0.4pt} % Thin horizontal rule
	
	\vspace{1\baselineskip} % Whitespace above the title
	
	{\LARGE Proceedings of the\\[0.05in] Mathematical and Physical\\[0.05in] Science Class of the\\[0.05in] Royal Bavarian Academy\\[0.15in] of Sciences in Munich}
	
	\vspace{1\baselineskip} % Whitespace above the title

	\rule{\textwidth}{0.4pt}\vspace*{-\baselineskip}\vspace{3.2pt} % Thin horizontal rule
	\rule{\textwidth}{1.6pt} % Thick horizontal rule
	
	\vspace{1\baselineskip} % Whitespace after the title block
	
	%------------------------------------------------
	%	Subtitle
	%------------------------------------------------
	
	{Volume 8 --- Year 1878} % Subtitle or further description
	
	\vspace*{1\baselineskip} % Whitespace under the subtitle
	
    {\small In Commission with G. Franz} % Subtitle or further description
    
	%------------------------------------------------
	%	Editor(s)
	%------------------------------------------------
	
	\vspace{1\baselineskip}

    \vspace*{\fill}

	Munich 1878
	
	Academic Publishing Office of F. Straub

    \vspace{0.5\baselineskip} % Whitespace after the title block

    Internet Archive Online Edition  % Publication year
	
	{\small Attribution NonCommercial ShareAlike 4.0 International } % Publisher
\end{titlepage}
\setlength{\parskip}{1mm plus1mm minus1mm}
\clearpage
\tableofcontents
\clearpage
\section{Mr. Carl Wilhelm von Gümbel speaks ``About the stone meteorites found in Bavaria.''}
\subsection*{Introduction}
\paragraph{}
Among the stone meteorites that have fallen and been located in Bavarian areas, there are quite a few whose chemical composition is known to us only from antiquated analyses, while still no chemical investigation has been undertaken on any of them up till the present moment. Moreover, since many of them lack an exhaustive survey, such as has been recently performed on types of rock by means of a thin section and microscope, it thus seemed to me sufficiently interesting to conduct such work and compare the results with the earlier findings. Through the special kindness of professor Dr. [Wolfgang Franz] von Kobell, the gentleman curator of the mineralogical state collection, I obtained the material needed for this purpose and I gladly use this opportunity to express my best thanks for his friendly assistance in my investigation. Several broad remarks, which are included in the conclusions, are sourced from other meteoritic stones that I have from time to time pulled into the circle of my study for comparison.

It turns out that there are just five known stone meteorites that have fallen in Bavaria. Among them is actually included a find which, due to the present territorial circumstances, no longer belongs to Bavaria but to Austria, namely that of Mauerkirchen. Because the municipality belonged to Bavaria at the time of the fall, it should at least seem warranted to a certain extent to list this stone here among the Bavarian ones.

These five stone meteorites are:
\begin{enumerate}
    \item The stone from Mauerkirchen, now in the Austrian Innviertel, from the fall on November 20, 1768, at four hours past midday.
    \item The stone from Eichstädt, which fell five kilometers from the town in the so-called Wittmes [a nearby forest] on the 19\textsuperscript{th} of February 1785, at twelve o'clock midday.
    \item The stone from Massing near Altötting in southern Bavaria from the fall on December 13, 1803, between the hours of ten and eleven in the morning.
    \item The stone from Schönenberg next to Burgau and Swabia, which fell on December 25, 1846, at two o'clock in the afternoon and
    \item The stone from Krähenberg by Homburg in the Rhenish Pfalz from the fall on the 5\textsuperscript{th} of May 1869, at six-thirty in the evening.
\end{enumerate}
\paragraph{}
I first came upon information of a sixth meteoritic rock in [Ludwig Wilhelm] Gilbert's \emph{Annals of Physics}, 15\textsuperscript{th} Volume, page 317, where it is cited that Gaspar Schott's \emph{Physica curiosa}, 11\textsuperscript{th} Volume, 19\textsuperscript{th} chapter, reports: ``here in our city of Herbipolis [Würzburg] is preserved in the temple of St. Jacobi across the bastion, in the monastery of the Scots,\footnote{The Scotch Monasteries were established in 1140, 1803 \emph{saeculo} 1819 part of the church was restored for worship, the choir in fact, the rest served as a military depot. The complete description and history of Wieland is in the archive of the \emph{Historical Association of Lower Franconia and Aschaffenburg}, Vol. 16.} chained to the temple column... it is hard and with an iron nature.'' Hence, it works out that it was presumably an iron meteorite. I put forward my inquiries about vestiges of this rock to the gentleman Professor [Fridolin von] Sandberger in Würzburg, who was nice enough to perform the most thorough search. The rock is missing. Owing to Sandberger's gracious communication, further information is given by [Friedrich] Schnurrer in his \emph{History of Epidemics}, 2\textsuperscript{nd} Volume: ``In the year 1103 (or 1104) a meteoritic rock fell in Würzburg, so big that four men were hardly able to carry the fourth part of it.''
\clearpage
\subsection{The Meteoritic Stone from Mauerkirchen}
\begin{figure}[h]
\centering
\includegraphics[keepaspectratio, scale=2]{Fig-1.png}
\caption{Figure 1}
\end{figure}
\paragraph{}
A short booklet initially talked about this fall: ``News and Reports on some Rocks Dropped out of the Air on November 20, 1768 in Bavaria not far from Mauerkirchen'' (Straubingen, 1769). Referring to the same, [Ernst] Chladni shares in his chronological list of stone and iron masses which have fallen down with a fiery meteor (Gilbert's \emph{Annals of Physics}, 1803, Vol. 15, p. 316) that sundry ordinary folk near Mauerkirchen, who swore to it when questioned, stated that in the evening on the aforementioned day after four o'clock the skies noticeably darkened against the west, and they heard an extraordinary roar and powerful bang in the air like thunder and with shooting fragments. Beneath this aerial turmoil a rock had fallen out of the air and, according to an authoritative visual inspection, made a pit in the ground two and a half schuh\footnote{1 schuh = 29.75 centimeters, 1 zoll is around 2.62 centimeters, 12 linie = 1 zoll} deep. The stone did not even hold up to be a schuh in length, was six zoll wide, and weighed 38 bavarian pounds. It was made of matter so soft that one could crush it with the fingers, the color bluish mixed with some white flows or streamers, and also coated by a black crust.

Professor [Maximus von] Imhof supplemented this account (\emph{Bavarian Electoral Palatinate Weekly Paper}, 1804, Section 4) with the following particulars: ``The fallen rock was located the day after hearing the noise, on the so-called Schinperpoint in an oblique hole going inward two and a half schuh deep.'' Imhof identified the specific weight as 3.452 and described the grayish-black, one-quarter linie [line (unit)] thick crust as giving sparks on steel, furthermore as constituent components:
\begin{enumerate}
    \item reguline iron, which has fused with much of the exterior crust in little kernels and tines, is very pliable and viscous and makes a white, thick shiny filing streak,
    \item pyrites,
    \item small, flattened, angular grains, which are distinguished by their dark gray color, shell-like breaks, glistening appearance, and greater hardness,
    \item still other tiny kernels of a white and yellow color that are translucent and shimmering. According to his analysis, the meteoritic stone is comprised of:\footnote{Numbers in tables are in percentages, unless otherwise specified.}
\end{enumerate}
\begin{center}
    \begin{tabular}{l r} 
    Silicon dioxide & 25.40\\
    Iron oxide & 40.24\\
    Iron & 2.33\\
    Nickel & 1.20\\
    Magnesium oxide & 28.75\\
    Sulfur and losses & 2.08\\
    \end{tabular}
\end{center}
\paragraph{}
(Compare with Otto Büchner's \emph{Meteorites in Collections}, 1863, p. 9)

Closer examination of the stone further revealed to me that the matt-black, slightly glossy in spots, 0.7 --- 0.3 millimeter thick crust, like with other meteoritic stones, is merely fusion crust, which merges against the inner main-mass without a sharp boundary because this is strengthened by the tiny iron pieces that border it, where sure enough faint amber granules are located and appear more glossy in the latter spots. Frequently the same small mineral pieces are melted and embedded in the crust or protrude into it. The main mass of the stone is colored light gray, dotted black due to the interspersed meteoritic iron, and, at many of these black spots, stained a rust color due to the effects of iron oxidation. The stone may easily be crushed between the fingers and has the impression of a trachytic tuff.

Out of the utterly fine, crumbly, almost dust-like matrix there arise quite a number of interspersed, roundish, poppy seed to millet sized granules, which are usually somewhat dark black or yellowish in color, matt on the exterior, and shiny like glass without cleavage surfaces when shattered, that have the character of chondrules and therefore imprint upon this stone the seal of the chondrites. Beneath the microscope these granules display a distinct quality. Some are very finely striated-in-parallel, such that predominantly opaque, wide strips alternate with narrow, small transparent or translucent ones, as if transversely organized. In polarized light the latter show up with finely dappled matt colors. (\emph{y} in the illustration from the accompanying table, Figure 1). Other granules are whitish, as if composed of the finest flour, opaque, but a little translucent around the edges, occasionally with the finest, slightly glimmering, separate, irregularly interspersed little needles (\emph{x} in the illustration).  Additionally, other granules have a type of radial fiber, though not clearly shown here. The smallest, rounded bits are water-clear and show up in polarized light as brilliant, motley colors.

Aside from the chondrules embedded in the powdery main mass, there are more numerous, usually short, angular, elongated little slivers of a whitened mineral, which are noticeably reflective at the cleavage surface and in places vaguely striated-in-parallel, and more roundish, angular, unevenly cracked, rarely striped-in-parallel granules that are distinguished by a yellowish or brownish color tone and a glass-like luster. To these are added metallically glistening, relatively small, botryoidal, angular clumps of meteoritic iron, in addition to the uncommon brassy-yellow ferrous sulfide and the deep black, not metallically glistening, small chromite rods. On the worn off parts of the stone the harder granules stick out and allow the character of the chondrites to be clearly perceived, more so than with the transverse breaks, in which one notices the spherical deposits only with greater attention. The finest dust particles, which have to be considered as the agglutinating material resulting from the progressive granulation of the larger slivers, are partly water-clear, partly opaque, translucent, and turn out to be even in the smallest detail little birefringent crystalline shards, although in polarized light the multicolored shades are matt. There is not a trace of a glass-like intermediate mass to be found.

After treating the finely crushed (not pulverized) material with saltpeter hydrochloric acid and potash solution --- apart from the metallic constituent parts --- the yellowish little slivers (olivine) have disappeared and the remains are only of white and brownish scraps, which can be easily distinguished under the microscope. The brownish fragments are heavily fissured, seldomly furnished with traces of tiny, obscure, parallel striations, are transparent, and in polarized light colored vibrantly with motley colors. They are undoubtedly little pieces of a mineral from the augite group. The little white slivers, in contrast, are in many cases only translucent, partly corroded by the acids, and in polarized light speckled with matt color tones, which here and there remind one of a striped design. The chemical analysis of the portion leftover following the action of the acids is also evidence that these little slivers have to be interpreted as feldspar-like constituent parts. The tiniest black particles are to be regarded as chromite. Thus, the stone consists of olivine, a feldspar-like augitic mineral, and meteoritic sulfur and chromite.

So that the chemical analysis was correct as well, the gentleman assistant Adolf Schwager was separately supervising examinations conducted at the same time. The measurement of the meteoritic and ferrous sulfide was done through individual experiments.\footnote{Anything extractable was taken out of the crushed powder with the magnet and these component parts containing meteoritic iron were specially analyzed with the application of copper vitriol and copper chloride.} The analyses yielded:
\begin{center}
    \begin{tabular}{ |l|p{1.5cm}|p{3.2cm}|p{2.9cm}| }
    \hline
    Compounds & Bulk analysis & 65.45\% portion decomposable in hydrochloric acid & 34.55\% remainder elemental parts\\\hline
    Silicon dioxide & 38.14 & 23.23 & 61.39\\\hline
    Aluminum oxide & 2.51 & 1.20 & 5.00\\\hline
    Iron(II) oxide & 25.70 & 32.72 & 17.59\\\hline
    Iron \& Nickel & 6.30 & 9.65 & -. -\\\hline
    Sulfur & 2.09 & 3.20 & -. -\\\hline
    Phosphorus & 0.14 & 0.22 & -. -\\\hline
    Chromium(II) oxide & 0.39 & -. - & 0.84\\\hline
    Calcium oxide & 2.27 & 1.51 & 4.35\\\hline
    Magnesium oxide & 21.73 & 29.13 & 7.70\\\hline
    Potash & 0.48 & Traces & 1.40\\\hline
    Natron & 1.00 & Traces & 2.91\\\hline
    Sum & 100.75 & 100.86 & 101.18\\
    \hline
    \end{tabular}
\end{center}
\paragraph{}
It therefore logically follows that the stone meteorite of Mauerkirchen tops the list of silica impoverished chondrites, like those of Seres, Buschhof, Ensisheim, and Château-Renard. The contents can be calculated thereof, namely:
\begin{center}
    \begin{tabular}{l r}
        Meteoritic iron & 2.81\\
        Iron(II) sulfide & 5.72\\
        Chromite & 0.75\\
        Silicates & 90.72\\
    \end{tabular}
\end{center}
\paragraph{}
As far as the interpretation of the silicates is concerned, we have to first envisage the essential elements decomposable in hydrochloric acid. The relatively low content of silicon dioxide here is especially striking. Nonetheless, a similar ratio repeats itself several times, for instance in the cases of the meteoritic stones of Seres, Tjabé (Java --- September 19, 1869), Khetri (India), and others. Removing the meteoritic iron and iron(II) sulfide content, we obtain for component elements:
\begin{center}
    \begin{tabular}{l r}
        SiO\textsubscript{2} & 26.45\\
        Al\textsubscript{2}O\textsubscript{3} & 1.35\\
        FeO & 37.30\\
        CaO & 1.70\\
        MgO & 33.20\\
    \end{tabular}
\end{center}
\paragraph{}
Wherein, if the aluminum oxide and calcium oxide are counted towards a decomposed feldspar, as is likely, and a fraction of the iron(II) oxide subtracted as still originating from meteoritic iron, then the constituent elements decomposed by acids may not be interpreted in any way except as good and proper olivine. That a portion of the iron is oxidized, and thereby appears to slightly increase the content of alkalis, is already indicated by the rust patches, which are present in the mass and sometimes quite widespread.

As far as this or the silicates of the leftover components are concerned, the relatively high silica and aluminum oxide content, in addition to that of the alkalis, arguably gives room to the presumption that, besides an augite mineral, there is also still a feldspar one present. At the same time though, even with this conjecture, there still remains a large excess of silica, which one cannot assume develops in the form of a precipitated quartz mineral, because on examination of thin sections in reflected light there is no trace of an admixture with anything usually recognizable due to the intense sparkle that can be observed in quartz. This behavior is only provisionally unexplained.

The same meteoritic stone has recently been subjected to a chemical analysis from another aspect. [Carl] Rammelsberg uses (\emph{The Chemical Nature of the Meteorites, Papers of the Academy of Sciences in Berlin} for 1870, p. 148 and following) as the result of the investigation performed by [Frank] Crook.\footnote{\emph{On the Chemical Constitution of the [Ensisheim, Mauerkirchen, Shergotty, and Muddoor] Meteoric Stones}, Göttingen Dissertation, (Not available to me).} Composition:
\begin{center}
    \begin{tabular}{r r}
        3.52 & Meteoritic iron\\
        1.92 & Iron(II) sulfide\\
        0.72 & Chromite\\
        92.68 & Silicates\\
        100.00 & and in fact:\\
    \end{tabular}
\end{center}
\paragraph{}
the silicates are present as:
\begin{center}
    \begin{tabular}{ |l|p{1.5cm}|p{3.2cm}|p{2.9cm}| }
        \hline
        Substance & Bulk analysis as a whole & in which 61\% is decomposable by acids. Fraction. & in which 39\% is undecomposable in acids. Fraction.\\\hline
        Silicon dioxide & 44.81 & 32.68 & 3.94\\\hline
        Aluminum oxide & 1.24 & 9.36 & 4.17\\\hline
        Iron(II) oxide & 24.55 & 28.91 & 17.71\\\hline
        Magnesium oxide & 26.10 & 37.44 & 8.20\\\hline
        Calcium oxide & 2.28 & 0.61 & 4.91\\\hline
        Natron & 0.26 & - & 0.67\\\hline
        Potash & 0.16 & - & 0.40\\
        \hline
    \end{tabular}
\end{center}
\paragraph{}
These results deviate so considerably from those communicated earlier, that for this no other grounds can be found except for the wide inequality in the composition of the meteoritic stones, which all the more expresses the greater level of importance of the findings of this examination, with one being obliged to work with ever smaller quantities. The microscopic examination of the thin sections directly supported this supposition, by allowing the broadest inconsistency in the manner of distribution of the constituent pieces to be perceived. A larger grain of this or that constituent member mixed into the expended sample, in the case of low quantities, affects the numbers in a sizable way. For instance, jagged little nodules of meteoritic iron pieces can be dislocated from the mass, whose magnitude has no relation, in general and as a whole, to the low percentage content of meteoritic iron in the stone. The interspersed, hard nodules and granules behave similarly.

The description referring to the composition of those constituent components decomposable in hydrochloric acid is particularly dissimilar. Yet, even in Crook's analysis the relatively low amount of silica comes out very clearly. The results of the analysis of the parts left undecomposed in hydrochloric acid prove to be less divergent. Precisely this proves that it does not lie in the course of the analytical work, as it might seem if the silica content here was likewise comparatively high, such as was detected in the portion decomposable in acids. Because this remaining part, as the microscopic examination of it shows, is comprised of dissimilar mineral substances, namely a white and a brown component part, the oxygen ratio taken as a whole is not able to provide us any special information.

The thin sections, which are challenging to produce because of the effortless friability of the mass, and which can only be obtained in a suitable condition by repetitive soaking with very dilute Canada balsam, provide, as the thin section image in Figure 1 of the accompanying table demonstrates, some instructive insights concerning the composition of the stones and the distribution of the constituent elements. Above all, the chondrules stick out with their partly powdery, friable, and in part fibrous composition. Despite their poor transparency they invariably turn out to be colorful, viewed in polarized light, and indeed, not just their bright little stripes, but their entire mass. Compared with these intermixtures, the remaining distinct tiny fragments are always irregularly defined, yellowish, brownish, and whitish. They are all crossed by uncountable bites, which only here and there run in parallel. Minor little pieces and dust particles of the seemingly same minerals constitute the matrix in which the larger \emph{débris} lay interspersed. In polarized light, color phenomena materialize down to the finest particles, so that the absence of a vitreous binding agent can be definitely noticed in the thin sections as well. Worthy of remark are countless tiny, round, water-clear granules that are the admixed matrix. Meteoritic iron and ferrous sulfide nodules approximately share the dimensions of the small mineral fragments, though their outlines do not generate the impression of destruction like the latter and are located quite uniformly dispersed in the mass. We see therefore that the meteoritic stone from Mauerkirchen has a structure that is not substantially different from other chondritic meteoritic stones.
\clearpage
\subsection{The Meteoritic Stone from Eichstädt}
\begin{figure}[h]
\centering
\includegraphics[keepaspectratio, scale=2]{Fig-2.png}
\caption{Figure 2}
\end{figure}
\paragraph{}
Concerning the fall of these stones, it was told that in the so-called Wittmes, a wooded area about five kilometers to the west of Eichstädt [Eichstätt], on February 19, 1785 in the afternoon between twelve and one o'clock, a laborer at a brick mill saw, after a thunder-like roar, a great black rock fall onto ground covered with snow on which bricks were lying around. When he went to the spot, he found the stone, which had shattered a brick, one hand deep in the ground and so hot that he at first had to cool it down with snow so that he could take hold of it. The stone was approximately a foot in diameter and, parenthetically, weighed three kilograms. [Carl Emil von] Schafhäutl (academic notice in \emph{The Academy of Sciences in Munich}, 1847, p. 559) describes it as follows: ``Its structure is considerably coarse-grained, the grains being more roundish than is the case in all those remaining aerolites; indeed, even completely elliptical, polished-looking granules of a grayish color are found, with compact, kind of matt, flat breaks in them, devoid of perceivable crystalline texture. Alongside these are situated greenish, olivine-like grains with glassy, conchoidal breaks. Ferrous sulfide, iron-nickel, and magnetite are disseminated among these grains, so that of all the meteoritic stones in our collection (Munich's State Collection), it has the strongest effect on the magnetic needle.''

The specific weight\footnote{Compare with [Carl von] Moll's \emph{Annals of Orography and Metallurgy}, Vol. 3, p. 251.} is given:
\begin{center}
    \begin{tabular}{l r}
        from [Carl Franz Anton von] Schreibers as & 3.700\\
        from Rumler as & 3.599\\
    \end{tabular}
\end{center}
\paragraph{}
[Martin Heinrich] Klaproth has analyzed this stone and gives (Gilbert's \emph{Annals of Physics}, Vol. 13, p. 338) as its component parts:
\begin{center}
    \begin{tabular}{l r}
        Solid iron & 19.00\\
        Nickel metal & 1.50\\
        Brown iron oxide & 16.50\\
        Magnesium oxide & 21.50\\
        Silicas & 37.00\\
        Loss (with sulfur) & 4.50\\
    \end{tabular}
\end{center}
\paragraph{}
The piece stored in the Munich State Collection shows a black, matt-glossed, rugose crust, and a whitish-gray, coarse-grained chondritic, easily broken main mass, dotted yellowish here and there by numerous rust stains, and from which huge chondrules can often easily be disengaged. They are found up to about three millimeters wide in diameter, they are very hard, at the surface matt, knobbed like strawberries and grubbed in such a manner that the connected little mineral fragments of the main mass appear as if cemented to the surface. Moreover, one notices small reflective strips in many places, whereby it appears faceted, so to speak. Tightly intergrown little meteoritic iron bits also occur, which are sometimes sunk into the surface. A smoothing of the surface never presents itself, as deposits must, if the globules were caused by abrasion and tumbling. They rather resemble, according to their external texture, the pig iron stone-pellets that are found in slags. If one shatters them, then they reveal a flat conchoidal surface break, a matt-glassy luster, a blackish grey color and with further fragmentation, they prove themselves under the microscope to be not a homogeneous, but a composite mass. One can clearly discern a transparent glass pervaded with numerous small vesicles, in polarized light exceptional motley colored constituent parts alongside slightly translucent, cloudy ones, as though composed of the tiniest dust particles, but, in polarized light still clearly colored, the main body is at times finely striped and distinctly a translucent, intense yellowish-brown, distinguished in polarized light by unaffected, tinted small stripes. In thin sections one sees their structure even more clearly, although here they are situated in a dark-colored main mass and obtaining good transparency is challenging. Due to the occurrence of quite a lot of mixed pieces of meteoritic iron that are in large part already slightly corroded and surrounded by a small ring of yellowish-brown color, the clarity of those little mineral pieces, which otherwise stand out by their transparency, also suffer. The yellow color is due to the ferric oxyhydroxide, which was formed by the exposure of the meteoritic iron to the humid air of our atmosphere, primarily retroactive throughout the time that the stone has lain on the Earth or in our collections. This ferric oxyhydroxide penetrates into the finest little rifts and seams or any spaces in between but can easily be removed by acids. Apart from the meteoritic iron there are added little mineral chips, irregularly scattered and seldomly containing parallel lines, of the aggregate material that comprises the meteoritic stone. Sometimes there are water-clear, slightly cracked little remains, sometimes striated with a system of straight, parallel lines, or are traversed by jagged rips at oblique downward angles, something like it is found preserved in augite, or else by a cell network similar to certain moss lamella, a curiously elongated and transversely divided mesh structure (\emph{d}) stands out. Occasionally in a piece of the \emph{débris} a number of systems of such little parallel strips bump together. In between these larger fragments lie smaller ones entirely of the same character as the greater aggregate. In polarized light all the small parts, which in general are merely transparent, show up in variegated colors, inside which are distributed individual aggregate-like slivers, and occasionally run parallel, striped, or belt-like. Ultimately, the spheroidal inclusions already alluded to turn out to be exceptionally common components. Of the manifold forms they possess, we emphasize merely a few that are commonly found. Considerably numerous are the chondrules with an eccentric, radially-fibrous assemblage (\emph{a}), which as a rule emanates from a more granular section located near the rim and in quite a few cases is detached, in a comparable way mesh-like and cross-divided tufts of rays taper off. This structure agrees so well with those already described which we come across on other regularly defined little fragments, that we have to consider the latter as the derivatives of broken, larger chondrules. Others of the latter are composed of different systems of darker little striations traversing at acute and obtuse angles (\emph{b}), a structure that can be considered as the inception of a crystalline mode of periodic disrupted formation. In addition, other chondrules occur with a cloudy, dust-like, slightly translucent substance, often in which very numerous, densely packed, lighter little strips (\emph{c}) are noticeably dispersed groupwise following different angles. Finally, it is not uncommon for globules to occur, which seem sintered together, so to speak, from larger, lighter granules (\emph{e}) separated from each other by dark little strips in between. From all of this, it is sufficiently clear that in the stone of Eichstädt we have in front of us a chondrite of the finest kind. It can really be held as the type of this kind of structure, which is well-known as being prevalent in the meteoritic stones.

As concerns its composition, the analysis (Assistant A. Schwager) has yielded that the stone is comprised of: 
\begin{center}
    \begin{tabular}{r l}
        22.98 & meteoritic iron,\\
        3.82 & iron(II) sulfide,\\
        32.44 & decomposable in acids,\\
        40.76 & minerals not decomposable in acids.\\
    \end{tabular}
\end{center}
\paragraph{}
The composition is on the whole A, then  
\begin{center}
    \begin{tabular}{r l}
        B & silicates decomposable in HCl\\
        C & components not decomposable in HCl:\\
    \end{tabular}
\end{center}
\begin{center}
    \begin{tabular}{ |l|r|r|r| }
        \hline
         & A. & B. & C.\\\hline
        Silicon dioxide & 33.31 & 34.45 & 55.53\\\hline
        Aluminum oxide & 2.31 & 0.86 & 5.13\\\hline
        Iron(II) oxide & 15.34 & 24.52 & 16.66\\\hline
        Iron (with phosphorus) & 24.64 & - & -\\\hline
        Nickel & 0.94 & - & -\\\hline
        Calcium oxide & 0.74 & 0.68 & 1.13\\\hline
        Sulfur & 1.42 & - & -\\\hline
        Chromium oxide & 0.15 & - & 0.73\\\hline
        Magnesium oxide & 18.86 & 37.31 & 19.34\\\hline
        Potash & 0.40 & 0.68 & 0.56\\\hline
        Natron & 1.04 & 1.31 & 1.62\\\hline
         & 99.15 & 99.81 & 100.70\\
        \hline
    \end{tabular}
\end{center}
\paragraph{}
The content of the constituent parts decomposable by acids, excluding the olivine, indicates a feldspar. Though we have in it:
\begin{center}
    \begin{tabular}{l r}
        SiO\textsubscript{2} & 34.45 with 18.37 oxygen\\
        Al\textsubscript{2}O\textsubscript{3} & 0.86 with 0.40 oxygen\\
        FeO & 24.52 with 5.45 oxygen\\
        MgO & 37.31 with 14.90 oxygen\\
        CaO & 0.68 with 0.19 oxygen\\
        Ka\textsubscript{2}O & 0.68 with 0.11 oxygen\\
        Na\textsubscript{2}O & 1.31 with 0.34 oxygen\\
    \end{tabular}
\end{center}
\paragraph{}
From this, one sees that if we precipitate a unisilicate the oxygen proportion is still not fully sufficient to completely satisfy the requirements, therefore the analysis does not transmit to us any information about the nature of the silicates still present, other than some more olivine.

Finally, in the rest of that not decomposed by acids, the ratios provide the following measure:
\begin{center}
    \begin{tabular}{l r}
        Silicon dioxide & 55.53 with 29.62 O = 22.6 + 7\\
        Iron(II) oxide & 16.66 with 3.70 O = 3.58 + 0.12\\
        Magnesium oxide & 19.34 with 7.73 O\\
        Chromium oxide & 0.73 with 0.23 O\\
        Aluminum oxide & 5.13 with 2.39 O = 2.33 + 0.06\\
        Calcium oxide & 1.13 with 0.32 O\\
        Potash & 0.56 with 0.10 O\\
        Natron & 1.62 with 0.42 O\\
    \end{tabular}
\end{center}
\paragraph{}
Out of this is worked out a bisilicate, chromite (of the composition of l'Aigle), and an andesine-like feldspar in a proportion of approximately 79:1:21.

So, in total, the Eichstädt meteorite is roughly made of:
\begin{center}
    \begin{tabular}{l r}
        Meteoritic iron & 22.98\\
        Iron(II) sulfide & 3.82\\
        Chromite & 0.40\\
        Olivine & 31.00\\
        Mineral of the augite group & 31.90\\
        Andesine-like feldspar & 8.46\\
        Feldspar-like mineral & 1.54\\
    \end{tabular}
\end{center}
\paragraph{}
The frequent occurrence and relative size of the chondrules led to a special analysis of these globules. In order to be sure that the processed material was free of the smallest adhering mineral pieces, the chondrules were rubbed back and forth on a dull sanding glass plate, until their surfaces were made completely smooth and shiny. Unfortunately, the amount at my disposal was only exceedingly small (0.12 gram) and as a result the analysis was not able to be made with greater accuracy. From preliminary studies it had already been ascertained that the substance of the chondrules separates into a decomposable and an indecomposable mass in hydrochloric acid. The former additionally includes ferrous sulfide, which, as the examination in thin sections teaches, is found as tiny granules tightly grown together and, so to speak, sunk into the globules.

I found the composition as:
\begin{center}
    \begin{tabular}{l r}
        Iron(II) sulfide & 1.53\\
        1. Decomposable in hydrochloric acid & 53.05\\
        2. Indecomposable in hydrochloric acid & 45.42\\
    \end{tabular}
\end{center}
\paragraph{}
The composition of the silicates of 1 and 2 was also found  
\begin{center}
    \begin{tabular}{|l|r|r|}
        \hline
        & 1 & 2\\\hline
        Silicon dioxide & 26.26 with 14.22 O & 53.21 with 28.38 O\\\hline
        Iron(II) oxide & 30.09 with 6.67 O & 14.86 with 3.30 O\\\hline
        Magnesium oxide & 31.53 with 12.60 O & 26.42 with 10.56 O\\\hline
        Aluminum oxide & 2.70 with 1.26 O & - -\\\hline
        Calcium oxide & 1.00 with 0.29 O & 3.67 with 1.05 O\\\hline
        Alkalis & 8.00 with 1.70 O & - -\\\hline
        & 99.98 & 98.16\\
        \hline
    \end{tabular}
\end{center}
\paragraph{}
To begin with, it is noteworthy that, as has already been noted on another page, the composition of the chondrules is almost the same as that of the whole mass and can themselves be dissolved into two similar portions through treatment with acids. 

The part dissolvable in hydrochloric acid, except for some residual content of meteoritic iron and ferrous sulfide, concurs closely with olivine. Though here too, as in numerous cases of analyzed chondrites, there is a lack of silica. I would like to assume that this originates from a surplus of ferrous oxide, which, instead of decomposed olivine, stems from finely admixed meteoritic iron. Aluminum oxide, calcium oxide, and alkalis point to an admixture of small feldspar-like parts, as with the main mass of the chondrite. Yet, offering an interpretation of these components presents complications, which up till now are still not resolved.

The remaining part, undecomposed in hydrochloric acid, fits much better with the measure of a bisilicate; even if a little bit of the silica is missing here, it can be considered a consequence of losses during the analysis itself, likely due to the low amount utilized in the analysis.
\clearpage
\subsection{The Meteoritic Stone from Massing}
\begin{figure}[h]
\centering
\includegraphics[keepaspectratio, scale=2]{Fig-3.png}
\caption{Figure 3}
\end{figure}
\paragraph{}
About the nearby circumstances of the fall of these meteorites, Professor Imhof (\emph{Bavarian Electoral Palatinate Weekly Paper}, 1804, p. 3 and following)\footnote{Gilbert's \emph{Annals of Physics}, 18, p. 330.} shares:

``According to the administrative reports of the electoral provincial office, many of the country folk, who lived around the market town of Mässing (Massing) in the district of Eggenfelden, heard a bang like cannon fire, nine to ten times, on the 13\textsuperscript{th} of December 1803, in the morning between ten and eleven o'clock. A farmer at St. Nicholas, who came out of his farmhouse during this noise and looked up, glimpsed something that went by extremely high with a constant buzz in the air and eventually fell onto the rooftop of his wagon hut, shattering a number of shingles and penetrating it. He walked up to the hut and found in it a completely black stone that smelled like powder and was as hot as a stone lying in an oven. He said he heard the so-called shooting from Alten-Oetting [Altötting] (\emph{i.e.}, from the east), but the stone had come up over Heiligenstadt [Gangkofen] (\emph{i.e.}, from the west). The stone weighed over 1.5 kilograms, had a specific weight of 3.365, a dark black, slightly thicker crust than the one from Mauerkirchen, and was a lot more coarse-grained in the breaks.''

According to Imhof, as component parts it contains:
\begin{enumerate}
    \item reguline iron, which shows up as thin iron filings visibly ingrown and shiny,
    \item pyrites, which beneath the magnifying lens appear crystallized and leave a black powder when rubbed,
    \item larger and smaller flattened, angular masses, some of a deep brown, others of a darker color, which differ from those due to their shimmery quality and greater hardness,
    \item here and there one detects cubic granules and translucent flakes of a yellowish color and with a glassy luster, looking like quartz, though not possessing the hardness of quartz,
    \item also white grains of an erratic form are sprinkled, some of which are over three linie thick,
    \item under the microscope one additionally spots an off-white, blending into yellow, metal that obeys the magnet and is probably metallic nickel.
\end{enumerate}
\paragraph{}
According to the analysis of this researcher, the stone, divided into one hundred fractions, is made of:
\begin{center}
    \begin{tabular}{l r}
        Reguline iron & 1.80\\
        Reguline nickel & 1.35\\
        Brown iron(II) oxide & 32.54\\
        Magnesia & 23.25\\
        Silicas & 31.00\\
        Losses in sulfur and nickel & 10.06\\
    \end{tabular}
\end{center}
\paragraph{}
Ammler gives (Otto Büchner, \emph{ibid}., p. 17) the specific weight as 3.3636.

Professor von Schafhäutl describes (\emph{ibid}., p. 558) this stone, ``with the appearance of pumice porphyry, in which the constituent silicates occur in such large aggregates, that one is able to easily discern them with the naked eye. The stone is comprised of milky-white grains with sheet-like radial structures, of granular olivine-like pea-sized masses, and partly of dull, basalt-like fragments, which, however, from time to time show up with augite-like cleavage planes, even shiny like glass. Scattered, cracked, iridescent ferrous sulfide and granules of chromite are found sparingly. The stone does not have an effect on the magnetic needle. With the Lötrohr it quite easily melts and is covered with a glassy, shiny glaze, like the aerolite from Stannern.''

According to my observations, the stone has a brownish-black crust, shiny like glass, and its grayish-white, easily friable mass is comprised of:
\begin{enumerate}
    \item Yellowish-green to light green, somewhat cracked-in-parallel, considerably large 1 --- 1.5 millimeters wide in diameter, rounded and irregular granules (as in crystalline form) that occur only sporadically as seemingly admixed pieces, which are easily disintegrated by acids and must be held as olivine.
    \item Of a white mineral, often transparent like glass or slightly translucent like a dusty cloud, heavily cracked, seldomly with parallel stripes, furnished at times with clear cleavage surfaces that in polarized light come across as vivid single- or multi-colored patches, and that is also disintegrated by acids, in accordance with a feldspar.
    \item Of a wine-yellow to greyish-green, or faintly reddish-brown, glass-like, matt-polished mineral, 1.5 to 2 millimeters large, colored vividly in polarized light, though not dichroic, with some longitudinal fibers (but unclear, striated) and suffused with abundant small bubbles. These component parts are not decomposed by acids and belong to the augite group.
    \item Of black, intensely shining chromite, not decomposable in acids, which yields a magnificently green glass in the phosphate test.
    \item Finally, of dark, metallic granules, to some extent pulled by the magnet, which are in most cases related to ferrous sulfide, or at least meteoritic iron.
\end{enumerate}
\paragraph{}
All of these larger, prevalently roundish, irregularly cornered (not longish, spear-shaped), small pieces are situated in a fine particulate-like, granular, gray matrix, which seems to be comprised out of the same little and tiny slivers as was just mentioned. Here too, a glass-like binding mass is not detected.

The analysis of A. Schwager's yielded:
\begin{center}
    \begin{tabular}{|l|p{1.75cm}|p{3.2cm}|p{2.7cm}|}
        \hline
        Substance: & Bulk Analysis & 21.33\% decomposable in hydrochloric acid & 78.67\% not decomposable in hydrochloric acid\\\hline
        Silicon dioxide & 52.115 & 39.59 & 56.71\\\hline
        Aluminum oxide & 8.204 & 29.51 & 2.54\\\hline
        Iron(II) oxide & 19.138 & 2.83 & 23.46\\\hline
        Iron & 0.523 & 2.49 & -\\\hline
        Nickel & Traces & Traces & -\\\hline
        Chromium oxide & 0.979 & - & 1.24\\\hline
        Calcium oxide & 5.786 & 15.70 & 3.15\\\hline
        Magnesium oxide & 8,485 & 3.33 & 10.74\\\hline
        Potash & 1.188 & 4.78 & 0.85\\\hline
        Natron & 1.928 & 4.78 & 1.17\\\hline
        Sulfur & 0.374 & 1.78 & -\\\hline
        & 99.720 & 100.06 & 99.86\\
        \hline
    \end{tabular}
\end{center}
\paragraph{}
The 21.33\% fraction that can be decomposed by hydrochloric acid can be calculated, according to the observed content of sulfur, magnesium oxide, and aluminum oxide, as approximately consisting of:
\begin{center}
    \begin{tabular}{r l}
        10 & Olivine (hyalosiderite)\\
        86 & Anorthite with high alkali content\\
        4 & Iron(II) sulfide and meteoritic iron\\
    \end{tabular}
\end{center}
\paragraph{}
In rounded numbers, feldspar A and olivine B would be comprised of:
\begin{center}
    \begin{tabular}{|l|r|r|}
        \hline
        & A & B\\\hline
        Silicon dioxide & 42 & 37.25\\\hline
        Aluminum oxide & 34 & -\\\hline
        Iron(II) oxide & - & 29.75\\\hline
        Calcium oxide & 18 & -\\\hline
        Magnesium oxide & - & 33.00\\\hline
        Alkalis & 6 & -\\
        \hline
    \end{tabular}
\end{center}
\paragraph{}
As concerns the remaining 78.67\% fraction, not decomposable by acids, one must even here presume a small percentage of feldspar in addition to chromite and augite, on the order of:
\begin{center}
    \begin{tabular}{r l}
        2.5 & Chromite\\
        13.5 & Feldspar-like substance (A)\\
        84.0 & Augite mineral (B).\\
    \end{tabular}
\end{center}
\paragraph{}
Both of the latter (A and B) have come up with a composition as follows:
\begin{center}
    \begin{tabular}{|l|c|c|}
        \hline
        & A & B\\\hline
        Silicon dioxide & 66 & 86\\\hline
        Aluminum oxide & 19 & -\\\hline
        Iron(II) oxide & - & 36\\\hline
        Calcium oxide & - & 4\\\hline
        Magnesium oxide & - & 14\\\hline
        Alkalis & 15 & -\\
        \hline
    \end{tabular}
\end{center}
\paragraph{}
Furthermore, considering that the ratio decomposable and not decomposable in hydrochloric acid is 21.33 to 78.67, one is able, in accordance with the above-mentioned interpretation, to roughly imagine a composition made of: 
\begin{center}
    \begin{tabular}{l r}
        Olivine & 2.00\\
        Iron(II) sulfide & 0.75\\
        Meteoritic iron & 0.25\\
        Chromite & 2.00\\
        Anorthite & 18.00\\
        Second feldspathic substance & 11.00\\
        Augite mineral & 66.00\\
    \end{tabular}
\end{center}
\paragraph{}
Up till now the stone of Massing has been placed on the side of Luotolax and Rammelsberg (\emph{The Chemical Nature of the Meteorites}, p. 136) counts it with the Howardites (olivine-augite-anorthite meteoritic stones).

I think that it has more correspondence with the augite group of the eucrites because the olivine is very sparse in extant.

We first want to see how an understanding of the optical examination of thin sections, as shown in Figure 3, fits with such a view. Initially one notices large, irregularly cornered granules --- not like the rounded ones typical of the chondrites, and a considerably uniform, fine bulk with distinct brightly shining, metallic, steel-grey and brass-yellow accumulated veins. At first ignoring the large, irregular, abnormal additions so to speak, we come to especially large groups in the matrix of a greenish-yellow, next a faint wine-yellow, then a pale reddish-brown and at last white minerals, which we are justified to view as the main admixed components. The sparse greenish-yellow little pieces (\emph{a}) are irregularly cracked, glisten with the most vivid aggregate colors in polarized light and become decomposed by acids --- olivine. At first glance one would like to consider the amply abundant accumulated veins of the faint wine-yellow, very cracked-in-parallel mineral (\emph{b}) for olivine. But they appear undecomposed with the treatment of boiling acids and therefore are not able to belong to olivine. One also notices a kind of parallel striping that does not correspond to that of olivine but reminds one of enstatite. Additionally, there are situated numerous, often just translucent, yet also quite transparent, non-dichroic little pieces (\emph{c}), colored reddish-brown at the rim that seem to have all the behaviors of augite. I therefore think that I ought to suppose that two minerals of the augite group are represented here, namely enstatite and augite. The little glass-clear or dust-like white pieces (\emph{d}) are partly decomposable by acids, but partly they turn up as more or less unaffected in the powder treated with acids. This likewise points to the presence of two different feldspars, traces of parallel striations can be discerned in one with thin sections in polarized light. Admixed meteoritic iron, even if sparse --- contrary to Schafhäutl's information --- is also genuine (\emph{e}), since in thin sections I had detected the occurrence of two distinct granules on whose glossy steel-grey surfaces I applied copper vitriol solution, whereby one could immediately observe the excretion of metallic copper.

The nature of the large inclusions is challenging to explain, labels \emph{x} and \emph{y} point to them in the thin section. The larger, \emph{x} is parallelly streaked and cross-cracked, dark olive green to reddish-brown, a little transparent, and colorful in polarized light. It should be considered as a slightly modified augite fragment. The second fragment, \emph{y} is yellowish, exceptionally fine-grained, quite dense, weakly translucent and spread throughout with the finest dust particles. It most closely resembles the shards of a chondrite granule. Inclusions like these and others of remarkably diverse qualities of structure are still embedded in the matrix. Although a clearly chondritic structure is not present, these inclusions and the minerals of the matrix behave so similarly to the integral parts of the chondrites that the meteoritic stone from Massing must be attributed to a completely analogous formation with the latter.

The considerable content of chromite in these stones gives reason to investigate its composition in greater detail, because, as far as I know, the chromite of the meteoritic stone has not been isolated as a subject of analysis up till now. For this purpose, the chromite in the meteoritic stone of l'Aigle seemed better suited, as larger granules occur in it. It can be picked out very easily and completely clean. The analysis of this chromite yielded:
\begin{center}
    \begin{tabular}{l r}
        Chromium oxide & 52.13\\
        Iron(II) oxide & 37.68\\
        Aluminum oxide & 10.25\\
         & 100.06\\
    \end{tabular}
\end{center}
\paragraph{}
therefore, nearly the composition of the chromite of Baltimore [Emmitsburg or Nanjemoy?] (Maryland), some more evidence for the homogeneity of the formation of the cosmic and telluric minerals.
\clearpage
\subsection{The Meteoritic Stone from Schönenberg}
\begin{figure}[h]
\centering
\includegraphics[keepaspectratio, scale=2]{Fig-4.png}
\caption{Figure 4}
\end{figure}
\paragraph{}
Professor von Schafhäutl gave a very extensive account on the fall of this meteoritic stone (\emph{ibid}., p. 564). Extracting out of this, at the time of the fall on December 25, 1846, after two o'clock in the afternoon, a thunder-like noise was heard over a region of approximately sixty kilometers. In the nearby proximity of the locality where the rock fell down the noise was likened to the distant thunder of cannons, repeating more than twenty times, then fading into a drum, and after about three minutes expiring into a buzz similar to far away trumpet sounds. During this noise, a number of people in the village of Schönenberg came out of the church, in which the afternoon worship service was taking place at the time and spotted a solid fist-sized ball from north-east to south-east wheeling around as it fell down into a cabbage field near the village. Numerous inhabitants of the village hurried to the location and a black rock was found that penetrated about two feet deep into the somewhat frozen mud ground. One even thought to notice a sulfuric odor. At the same time, the heretofore overcast sky suddenly displayed a thin streak and then brightened up entirely.

Coated all over with a deep brown, roughly sintered crust, von Schafhäutl describes the form of the stone as a very irregular, four-sided pyramid with a sharpening in the overall shape, running in the direction of the longest diameter of the base and decreasing on the rear side of the pyramid. Since the crust is also found in tiny clefts, one thinks one ought to suppose that the stone reached the Earth in a softened state. Seven strips of iron-nickel wind thread-like across the stone, while an eighth, which possess a right-angle orientation to the others, crosses them. Two sides are flat and without indentations, but apart from that the surface is irregularly indented, such as a fragment of a stone that was shattered by an external force. The stone weighed eight kilograms, fifteen grams and is so malleable that it may be crumbled by the fingers. It has an effect on the magnet needle and hydrochloric acid generates hydrogen sulfide along with a gelatin formation. The mass is comprised of white, finely granulated particles, which become corroded by acid, after this of honey-yellow and greenish granular aggregates, upon which the acid has a lesser effect, furthermore of distinct tiny granules of ferrous sulfide, silvery, fimbriated flakes of iron-nickel dispersed in the mass and at the same time forming the above-mentioned lines. Nothing of augite, labradorite and the like is detected, von Schafhäutl does not seem to agree with the opinion of [Jacob] Berzelius that the admixed parts decomposed by hydrochloric acid are olivine. For the olivine-like grains are precisely the most indissoluble and the little white mineral pieces decomposable in accordance with the nature of the zeolites or equally of annealed epidote, vesuvianite, \emph{etc}. He then even adds an attempt at an explanation of the formation of the meteorite as a result of a condensation from a cloud-like mass in the vicinity of our world.

The fusion crust is, according to my perception, dully shimmering, black, and in places where the iron particles exist in proximity, quite thick (up to $\frac{1}{2}$ millimeter). The light gray, white, finely granulated, sparsely dotted black, rust-stained in patches, main mass is comprised, insofar as this provisional determination allows, out of:
\begin{enumerate}
    \item larger, greenish-yellow bits, decomposable by the use of hydrochloric acid, which give a solution containing a lot of ferrous oxide and magnesia --- also olivine-like,
    \item white, splintered little pieces, likewise dispersible by acid,
    \item greenish-grey, dully glistening, irregular granules, which are cracked and do not get decomposed by acids,
    \item various iron compounds, which are made noticeable by their metallic gloss and are frequently surrounded by a yellow, rust colored halo as a consequence of the decomposition occurring in the meteoritic iron. The content of this was ascertained through special experiments. In the leftover, the analysis gave:
\end{enumerate}
\begin{center}
    \begin{tabular}{ |l|p{1.5cm}|p{3.2cm}|p{2.9cm}| }
        \hline
        Substance: & Bulk analysis & 55.18\% decomposed by hydrochloric acid & 44.82\% not decomposed by hydrochloric acid\\\hline
        Silicon dioxide & 40.13 & 24.47 & 57.85\\\hline
        Aluminum oxide & 5.57 & 9.45 & 6.75\\\hline
        Iron & 13.77 & 30.56 & -\\\hline
        Nickel & 1.47 & 1.48 & 1.44\\\hline
        Sulfur & 1.93 & 3.52 & -\\\hline
        Phosphorus & 0.36 & 0.33 & 0.27\\\hline
        Chromium oxide & 0.60 & - & 1.35\\\hline
        Iron(II) oxide & 17.12 & 10.41 & 15.37\\\hline
        Calcium oxide & 2.31 & 3.72 & 0.56\\\hline
        Magnesium oxide & 13.81 & 11.55 & 16.63\\\hline
        Potash & 0.73 & 1.33 & Traces\\\hline
        Natron & 2.20 & 3.18 & 1.02\\\hline
        & 100.00 & 100.00 & 101.24\\
        \hline
    \end{tabular}
\end{center}
\paragraph{}
From this data it can be calculated that the fraction decomposable in hydrochloric acid is comprised out of:
\begin{center}
    \begin{tabular}{l r}
        Iron(II) sulfide & 9.64\\
        Meteoritic iron & 26.25\\
        Olivine & 34.78\\
        Feldspar mineral & 29.33\\
    \end{tabular}
\end{center}
\paragraph{}
For the olivine component part, it is established with calculation as:
\begin{center}
    \begin{tabular}{ |l|c|c| }
        \hline
        SiO\textsubscript{2} & 12.82 & 37\\\hline
        FeO & 10.41 & 30\\\hline
        MgO & 11.55 & 33\\\hline
         & 34.78 & 100\\
        \hline
    \end{tabular}
\end{center}
\paragraph{}
commensurate with the composition of the hyalosiderites.

Further, we then find for the slightly decomposed feldspar-like component part:
\begin{center}
    \begin{tabular}{ |l|c|c|c| }
        \hline
        SiO\textsubscript{2} & 11.65 & 39.71 & Oxygen 21.3\\\hline
        Al\textsubscript{2}O\textsubscript{3} & 9.45 & 32.21 & Oxygen 15.0\\\hline
        CaO & 3.72 & 12.70 & Oxygen 3.6\\\hline
        Ka\textsubscript{2}O & 1.33 & 4.54 & Oxygen 0.77\\\hline
        Na\textsubscript{2}O & 3.18 & 10.84 & Oxygen 2.8\\\hline
         & 29.33 & 100.00 & \\
        \hline
    \end{tabular}
\end{center}
\paragraph{}
The oxygen ratio of the silica, the alumina, and the alkaline bases is 3:2:1, not in agreement with that of a true feldspar, but matching that of the scapolites (meionite). The presence of minerals of this sort would better match the optical behavior than the acceptance of an anorthite or plagioclase in general, since in polarized light one cannot detect any parallel stripes in the little white or glass-clear pieces.

In the rest not decomposed by hydrochloric acid the content of nickel and phosphorus is notable. Because the assumption that this content originates from some residue of meteoritic iron, by chance undecomposed, we are forced to consider this as an indication of the admixture of schreibersite. To this end, the pertinent iron shows up naturally in the analysis among the ferrous oxide. This partly accounts for the excess in the sum being over one hundred. Although even more chromite containing alumina is certainly present, such a substantial amount of alumina, in addition to a considerable quantity of natron, turns up that in the rest a feldspathic admixed component must be implied, while its main constituent evidently constitutes an augitic mineral. If one takes an admixed bisilicate component for the latter, a balance remains, in which the oxygen ratio between the aluminum oxide and the residual lingering silicon dioxide is nearly 3:9, but then the required amount of calcium oxide and alkali is missing. As a result, the share that is not broken down by acids can only be approximately calculated as consisting out of:
\begin{center}
    \begin{tabular}{l r}
        Schreibersite & 4.5\\
        Chromite & 2.5\\
        Feldspathic mineral & 4.0\\
        Augitic mineral & 89.0\\
    \end{tabular}
\end{center}
\paragraph{}
Thus, as a whole the chondrite from Schönenberg is comprised out of:
\begin{center}
    \begin{tabular}{l r}
        Olivine & 19.0\\
        Feldspar- and scapolite-like mineral & 18.5\\
        Augitic mineral & 40.0\\
        Meteoritic iron & 14.5\\
        Iron(II) sulfide & 5.0\\
        Schreibersite & 2.0\\
        Chromite & 1.0\\
    \end{tabular}
\end{center}
\paragraph{}
Thin sections of this meteoritic stone (Figure 4 of the table) reveal to us the exceptionally fine-grained structure of the admixed constituents, which, as with all chondrites, are all irregularly splintered. Larger mineral fragments are scarce, as are the chondrules (\emph{a}) whose mass is white, cloudy, finely granulated like dust, and at the edges slightly translucent, but in polarized light they display colorful hues, less often eccentrically fibrous. Apart from these roundish granules there also occur irregularly cornered fragments of a cloudy, dust-like, and striated mass (\emph{b}) and those peculiar, utterly fine, parallelly striped and cross-divided structures, similar to the cell meshes of moss leaves (\emph{c}), which characteristically recurs in so many chondrites. The meteoritic iron often forms elongated, trail-like small heaps (\emph{d}), though also frequently wrapped around the chondrules as a thin outer layer.

Amongst the larger mineral pieces, one is able to recognize ones with yellowish, highly irregular cracks, more rounded outlines than those belonging to the olivine; they exhibit the most colorful aggregate colors in polarized light. The somewhat darker, colorful, often times slightly fading-into-red slivers of augitic minerals mark themselves by a parallel fissuring following two directions and also in polarized light are quite motley colored, while the whitish, feldspathic component parts in many cases fade into turbidity and in polarized light become dominated by blue and yellow color tones.

It follows from all of the foregoing that the Schönenberg meteorite, which was previously not looked at chemically, belongs to the major group of the chondrites and, due to its low silica content, comes very close to the Ensisheim stone, but differs from it, as does all those compiled by Rammelsberg (\emph{ibid}.), by the relatively very limited content of magnesia, and high alumina and natron content.

The string-like strips perceptible on the surface of the stone appear to correspond to fracturing of the stone, in which, like on the surface, a fusion crust seems to have formed during the fall through the atmosphere.
\clearpage

\subsection{The Meteoritic Stone from Krähenberg}
\textbf{near Zweibrücken in the Rhineland-Palatinate}
\begin{figure}[h]
\centering
\includegraphics[keepaspectratio, scale=2]{Fig-5.png}
\caption{Figure 5}
\end{figure}
\begin{figure}[h]
\centering
\includegraphics[keepaspectratio, scale=2]{Fig-6.png}
\caption{Figure 6}
\end{figure}
\paragraph{}
The stone from Krähenberg is one of the foremost falls in recent times and most thoroughly investigated meteoritic stones. On the subject of the fall itself, Dr. Georg von Neumayer (\emph{Proceedings of the Mathematical and Natural Science Class of the Academy of Sciences in Vienna}, Vol. 60, 1869, p. 229), Otto Büchner ([Johann] Poggendorf's \emph{Annals of Physics}, Vol. 137, p. 176) and [Christian Ernst] Weiss (\emph{New Yearbook}, 1869, p. 727 and Poggendorf's \emph{Annals of Physics}, Vol. 137, p. 617) gave a detailed account, on the subject of the composition [Gerhard] vom Rath (Poggendorf's \emph{Annals of Physics}, Vol. 137, p. 328), but up till now a microscopic investigation of thin sections has been absent. We learn from the above cited descriptions about the fall of this stone that, in the evening at six-thirty on the 5\textsuperscript{th} of May 1869, a most frightful, like the thunder of some cannons but vastly more powerful, bang was heard, followed by a rolling, a roaring such as coming from musket fire, and a hum similar to the noise of steam escaping a locomotive. All of a sudden, these noises, which had continued for nearly two minutes, ended with a strong thud. One observed either noises or optical phenomena in places for up to sixty to seventy kilometers distant from the Krähenberg fall spot, the latter being stated as intensely white. Two lads saw the rock plunge towards the Earth and approximately fifteen to twenty minutes after the fall dug it out of the ground, in which it had excavated a vertical, nearly 0.6 meter deep, pit and was resting upon the underlying Buntsandstein layer.\footnote{Georg von Neumayer (\emph{ibid}., p. 239) draws the conclusion from the information he has gathered that the Krähenberg stone, as it was still following the drift of its cosmic course, belongs to the meteor shower whose radiation point is located in the vicinity of $\delta$ Virginis.} The rock still felt warm, though not hot; it still weighed, after perhaps several kilograms had been chipped off, at least 15.75 kilograms and had a likeness to a loaf of bread, but with a slightly sharpened roundish form in a single direction, a larger diameter of 0.30 meter and a smaller one of 0.24 meter, the broadest off-center thickness or height is 0.18 meter; the flat, base area, considerably even, is in contrast to the curved face which is covered with numerous extremely remarkable trench shaped furrows, grooves often 0.03 meter long, up to eight millimeters deep, stretched out from the smooth apex and dispersed radially towards the sides.  In between these pits, little oblong bulges elevate themselves then narrowly undulate, so that the surface appears deeply rutted like pockmarks, so to speak. The whole surface is covered by a black, in patches foamy, slag crust from a half to one millimeter in thickness. In a spotted manner, the crust is thin and brownish colored rather than black, which, as I was convinced by the original, is due to the mix of constituent elements that are found at such locations to be more resistant to fusion, which prevented intensive melting. Weiss immediately identified the chondritic nature of the stone and also called attention to the dark gray, sharply delimited fragments lying in the whitish matrix, which, like the gray spheres, show up as a mixture of interspersed metallic particles and tiny white slivers. Vom Rath confirmed this and further added that numerous fine black lines running in all directions, sometimes interconnected in a meshed work, could be observed on the light gray fractured surface of the Krähenberg stone. They seemed to him to be rifts, which were, at least in part, formed during the entry of the meteor in the Earth's atmosphere and became filled by the melting substance of the crust. Besides these lines of glaze, curved, slender veins of another kind, comprised of iron-nickel, swarm around the stone. They are dike-like sections of considerable thickness. I was able to clearly observe such a one on a fractured surface, a metalliferous vein over three zoll long, a little curved, and $\frac{1}{3}$ --- $\frac{1}{2}$ millimeter thick. Furthermore, reflective iron occurs as well, like in the stone from Pultusk, to which the mass is very similar, but less finely granulated. As admixed components, vom Rath identified iron-nickel, pyrrhotite, chromite, olivine, and the characteristic spheres, which lay in a spherulitic matrix formed out of white and grey grains. He set the iron-nickel content (made of 84.7 iron and 15.3 nickel) at 3.5\%, so that 96.5\% came from the silicates, pyrrhotite, and chromite. Disengaged small pieces from the fusion crust have specific weight of 3.4975 at 18° C., small pieces rich in fusion crust 3.449 at 20° C., confirming the observation on the Pultusk stone, that the fusion crust is intrinsically lighter than the stony mass of the interior.

Vom Rath does not hold the ferrous sulfide for troilite, although it is not drawn by the magnet, but for pyrrhotite, because a richer amount of hydrogen sulfide arises during treatment with hydrochloric acid and a lot of sulfur is excreted. He set the content of pyrrhotite at 5.52\%.

The dark grey to black grains, up to two millimeters in size, occasionally show an utterly fine, very easily detached, whitish hull. In addition, irregularly rounded, dark grains and spherical segments occur, which, like the former, possess only an imperfect fiber composition. Still further, yellowish-white grains, up to one millimeter large appear --- presumably, olivine with rounded faces and only hints of a crystalline outline. Black, small chromite stone grains allow one to detect a seemingly octahedral form. The main mass of the stone reveals itself under the microscope as an aggregate of endless small, white, crystalline granules that are bright, vividly glisten grease-like, and display colors in polarized light; they are insoluble in acids and are essentially formed of a magnesium silicate that is richer in silica than olivine. Apart from this, a light gray substance occurs as well, which has a spherulitic form of arrangement, and like the dark spheres also at times shows a fibrous consistency.

Microscopically, unusual, admixed components are still found of extraordinarily small, crimson crystal pieces, quite a few intensely yellow granules with noticeable crystal faces, some light yellow, oblong prismatic forms and, finally, distinct, up to $\frac{1}{2}$ millimeter large, red granules with conchoidal breakage that are translucent --- likely a decomposition product of the ferrous sulfide, similar to \emph{caput mortuum} [\emph{crocus metallorum}].

The analysis of the non-magnetic part yielded, according to vom Rath:
\begin{center}
    \begin{tabular}{ |l|p{1.4cm}|p{3.5cm}| }
        \hline
        & 1. & 2.\\\hline
        & & After deduction of chromite and pyrrhotite\\\hline
        Chromite & 0.94 & -\\\hline
        Pyrrhotite sulfur & 2.25 & -\\\hline
        Pyrrhotite iron & 3.47 & -\\\hline
        Silicon dioxide & 43.29 & 46.37 oxygen 24.73\\\hline
        Aluminum oxide & 0.63 & 0.67 oxygen 0.32\\\hline
        Magnesium oxide & 25.32 & 27.13 oxygen 10.85\\\hline
        Calcium oxide & 2.01 & 2.15 oxygen 0.61\\\hline
        Iron(II) oxide & 21.06 & 22.56 oxygen 5.01\\\hline
        Manganese(II) oxide & Traces & -\\\hline
        Natron (losses) & 1.03 & 1.12 oxygen 0.29\\
        \hline
    \end{tabular}
\end{center}
\paragraph{}
According to this, the sum total of the oxygen quantities of the bases to that of the silicas is:

1:1.448,

a ratio which does not differ significantly from that of the Pultusk stone (1:1.507). As essential admixed components, the chemical analysis also gave olivine and a silica-rich mineral, whether enstatite or shepardite or both at once, vom Rath left undecided.

He holds the admixture of anorthite or labradorite as inadmissible since calcium oxide and aluminum oxide are a part of the insoluble portion and can only be stripped off in low amounts with acids.

Further, I am in debt to the information from a favorable message that the results of an analysis that the gentleman Professor Dr. Keller in Speyer performed, and which therefore is of greater importance since it was conducted with a considerable quantity, namely 5.71 grams; it was found:
\begin{center}
    \begin{tabular}{ |l|p{1.3cm}|p{1.7cm}|p{1.7cm}|p{1.7cm}|p{1.7cm}| }
        \hline
        Substances & \small{Bulk Analysis} & \small{57.69\% decomposable in hydrochloric acid individually} & \small{57.69\% decomposable in hydrochloric acid in \%} & \small{42.31\% not decomposable in hydrochloric acid\footnote{Without chromite and tin(II) oxide.} individually} & \small{42.31\% not decomposable in hydrochloric acid in \%}\\\hline
        Silicon dioxide (a) & 41.12 & 15.76 & 27.28 & 25.36 & 61.76\\\hline
        Magnesium oxide (a) & 18.62 & 14.44 & 24.99 & 4.18 & 10.18\\\hline
        Manganese(II) oxide (a) & 0.78 & 0.78 & 1.35 & - & -\\\hline
        Iron(II) oxide (a) & 17.10 & 10.69 & 18.52 & 6.41 & 15.61\\\hline
        Iron (b) & 3.93 & 3.93 & 10.85 & - & -\\\hline
        Sulfur (b) & 2.35 & 2.35 & 10.85 & - & -\\\hline
        Iron (c) & 6.44 & 6.44 & 14.31 & - & -\\\hline
        Nickel (c) & 1.36 & 1.36 & 14.31 & - & -\\\hline
        Phosphorus (c) & 0.46 & 0.46 & 14.31 & - & -\\\hline
        Chromium(II) oxide (d) & 0.89 & - & - & 0.89 & -\\\hline
        Iron(II) oxide (d) & 0.32 & - & - & 0.32 & -\\\hline
        Aluminum oxide (e) & 3.22 & 0.76 & 1.31 & 2.46 & 5.99\\\hline
        Calcium oxide (e) & 2.06 & 0.42 & 0.73 & 1.64 & 4.00\\\hline
        Potash (e) & 1.22 & 0.21 & 0.36 & 1.01 & 2.46\\\hline
        Natron (e) & 0.17 & 0.17 & 0.30 & - & -\\\hline
        Tin(II) oxide (e) & 0.18 & Traces & - & 0.18 & -\\
        \hline
    \end{tabular}
\end{center}
\paragraph{}
Out of this is calculated:
\begin{center}
    \begin{tabular}{l r}
        a) Olivine & 41.67\\
        b) Iron(II) sulfide & 6.28\\
        c) Meteoritic iron & 8.26\\
        d) Chromite & 1.21\\
        e) Other silicates & 42.58\\
    \end{tabular}
\end{center}
\paragraph{}
The specific weight was ascertained at 3.432.

We now compare the results of the latter (B) analysis with those formerly disclosed by vom Rath (A) through the simple conversion of both to silicate components so as to eliminate the impact of the admixed components of meteoritic iron, ferrous sulfide, and chromite, which clearly occurs in very unequal distributions, in this way the following numbers result:
\begin{center}
    \begin{tabular}{ |l|c|c| }
        \hline
         & A & B\\\hline
        Silicon dioxide & 46.37 & 48.78\\\hline
        Aluminum oxide & 0.67 & 3.82\\\hline
        Iron(II) oxide & 22.56 & 20.29\\\hline
        Manganese(II) oxide & Traces & 0.93\\\hline
        Magnesium oxide & 27.13 & 22.09\\\hline
        Calcium oxide & 2.15 & 2.45\\\hline
        Potash & - & 1.44\\\hline
        Natron & 1.12 & 0.20\\
        \hline
    \end{tabular}
\end{center}
\paragraph{}
Here, too, we observe extremely limited agreement in individual substances, namely in reference to alumina and magnesia, which again suggests a very uneven blend and distribution of the constituent parts. In fact, upon performing a closer examination of the stone, which is stored in the district collection at Speyer, entire sections of it, as Weiss has already stressed, conspicuously stand out as patches of darker color, greater hardness, and a compact quality when compared to the remaining light gray, friable mass. They are clean shaped inclusions, angular, irregularly defined, broken pieces on a smaller scale as it were, like the small fragments of the main mass, though also with special qualities. I was placed into the pleasant position of being able to dispose of a little bit of the Speyer stone for my further investigation. Having said this, before I make much note of these special inclusions, I still have to enter into a closer consideration of the various mineral mixtures decomposable and not decomposable in hydrochloric acid.

The silicate constituent parts decomposable in hydrochloric acid are calculated in terms of their composition:
\begin{center}
    \begin{tabular}{l r}
        (+) Silica & 36.46\\
        (+) Iron(II) oxide & 24.73\\
        (+) Magnesium oxide & 33.40\\
        (+) Manganese(II) oxide & 1.80\\
        (\^{}) Aluminum oxide & 1.76\\
        (\^{}) Calcium oxide & 0.97\\
        (\^{}) Potash & 0.48\\
        (\^{}) Natron & 0.40\\
    \end{tabular}
\end{center}
\paragraph{}
(+) almost exactly the composition of olivine (hyalosiderite). (\^{}) Residues of a difficult to decompose, feldspar-like admixed part in lesser quantities.

Accounting for the chromite, the rest not decomposed by hydrochloric acid is comprised out of, incidentally:
\begin{center}
    \begin{tabular}{ |l|c|c|c| }
        \hline
        & (1) & A & B\\\hline
        Silica & 61.7 or & 30.0 + & 31.7\\\hline
        Magnesium oxide & 10.2 & 10.2 & -\\\hline
        Iron(II) oxide & 15.6 & 15.6 & -\\\hline
        Aluminum oxide & 6.0 & - & 6.0\\\hline
        Calcium oxide & 4.0 & 2.0 + & 2.0\\\hline
        Potash & 2.5 & - & 2.5\\\hline
        & 100.00 & 57.8 & 42.2\\
        \hline
    \end{tabular}
\end{center}
\paragraph{}
We are able to break down (1) into A and B and thereby obtain as a result a mineral of the augite group and a mineral of the feldspar group, the first bronzite-like (oxygen ratio of 16:8.1), the second with an oxygen ratio of approximately 6:3:1 (more precisely 16.9:3:1) or labradorite-like, with this the alumina and alkali containing part decomposed by hydrochloric acid is estimated.

One is therefore able to assume, that on average the main mass of the meteoritic stone from Krähenberg is comprised out of:
\begin{center}
    \begin{tabular}{l r}
        Meteoritic iron & 6.27\\
        Iron(II) sulfide & 8.25\\
        Chromite & 1.21\\
        Olivine & 41.65\\
        Augite mineral (? Bronzite) & 23.48\\
        Feldspar mineral (? Labradorite) & 19.14\\
    \end{tabular}
\end{center}
\paragraph{}
Now, concerning the harder, denser, and darker sections engrained in larger chunks in the stone, which were already alluded to earlier and are possibly adherent fragments of the main masses, these are comprised, according to the analysis undertaken by assistant A. Schwager, out of:
\begin{center}
    \begin{tabular}{|l|p{1.7cm}|p{2.9cm}|p{2.9cm}|}
        \hline
        Substance: & Bulk Analysis & 64\% decomposable in hydrochloric acid & 39\% indecomposable in hydrochloric acid\\\hline
        Silica & 39.08 & 28.44 & 57.96\\\hline
        Aluminum oxide & 2.08 & 1.46 & 5.79\\\hline
        Iron(II) oxide & 28.53 & 36.20 & 13.75\\\hline
        Iron (containing nickel) & 4.43 & 6.92 & -\\\hline
        Sulfur & 1.31 & 2.04 & -\\\hline
        Manganese(II) oxide & 0.82 & 1.28 & -\\\hline
        Chromium(II) oxide & 0.39 & - & 1.08\\\hline
        Calcium oxide & 13.35 & 14.55 & 11.24\\\hline
        Magnesium oxide & 5.97 & 5.73 & 6.40\\\hline
        Potash & 1.48 & 1.73 & 1.04\\\hline
        Natron & 1.81 & 1.13 & 3.05\\\hline
        & 99.25 & 99.48 & 100.31\\
        \hline
    \end{tabular}
\end{center}
\paragraph{}
First of all, it is noteworthy that we are likewise working with a mass composed of diverse minerals, which can be separated into parts that are separable and not separable by hydrochloric acid and that as a whole have great similarity in their composition, by comparison not to be confused with the main mass. In contrast, the high content of ferrous oxide and calcium oxide and low of magnesia prove to be different if we consider the mass as a single entity, while in the extract of hydrochloric acid, besides the same proportions, even the relatively large amount of silica is visible to the eyes. Also in this remaining part is calcium oxide, which occurs in most unusual quantities. One can hardly take from this more than the assumption that, apart from hyalosiderite, an iron and calcium rich mineral of the augite group, perhaps diopside with an anorthite-like feldspar, are to be assumed as the primary admixed components.

Further investigation of the stone has brought to knowledge some interesting peculiarities of it. First of all, one's attention is directed to the numerous, traversing little black strips and small veins, which vom Rath has already accurately described. They consist, so far as I can tell, out of a substance like that of the external fusion crust, even including meteoritic iron, and appear to constitute seams and fissures in which, as on the outer surface, some melting took place. In certain ones towards the exterior, I clearly observed a blistered and foamy condition. Quite distinguished are the smooth and striated delaminated surfaces, which look exactly like the surface of a slide, though nondisplaced individual elements can be discerned against each other. They must have probably been already present, before the stone had arrived at the atmosphere of our Earth, and here obtained a fusion crust only in patches.

The thin sections, which I was able to prepare from five distinct parts of the main mass, provide us with an impression of a very composite chondrite, as depicted in the illustration in Figure 5. Lots of the round grains appear merely as shattered fragments of sphere-like pieces and are not uncommonly coated, like a crust, by a black substance whose composition also has meteoritic iron involved. In one of these, this black coating even penetrates into the grain itself. They are partly comprised out of that well-known eccentrically fibrous mass, partly made of the finest dust-like, slightly translucent granules, larger clear pieces, or out of a substance ruptured or veined in a network following different parallel directions in a great plurality of formations, in addition, angular broken pieces of entirely similar multiform formations are observed, as in the case of the spherical inclusions. Amongst these, utterly fine and dense, parallelly striated little fragments, whose tiny parallel fibers appear as if cross divided by dark small stripes (\emph{y}), stick out to the eye. They are extraordinarily characteristic of the chondrites. Individual slivers, in which are observed with strong magnification the most minute vesicles, are seldomly free from ruptures or from being traversed by frequently parallel, widely spaced dark lines. A regularity in the arrangement of these slivers, which are clearly constrained to broken pieces, does not reveal itself. All of it lies confusedly jumbled-up and connected as a tight, cohesive whole through ever emerging, smaller and more fragmented bits, down to specks of dust. In polarized light they all show up in colorful aggregate colors of various vibrancy, though free from any trace of a simple-refractive intermediate substance. Little stripes of colors, infrequently and not clearly, become visible. It still remains to be pointed out that larger spots of the mass appear stained intensely yellow. This coloration originated from ferric oxyhydroxide, as its rapid disappearance upon treatment with hydrochloric acid proved, spreading at the fine breaks, which came from the infiltration of damp air on the exceptionally susceptible meteoritic iron.

Nearly the same impression is obtained in thin sections of the dark, cleanly formed sections of the stone (Figure 6), whose analysis, which was previously discussed, was remarkable for its large calcium content and lack of magnesia. The grains and fragments situated therein merely seem larger and more densely packed together. No optical phenomenon can be detected, as one might expect, which would be able to provide information about the deviating outcome of the analysis. The limited amount of available substance hindered further tests that could perhaps account for the discovery of a lot of calcareous components. An attempt was also made to isolate and subject the yellow granules, apparently representing olivine, to a separate analysis. Treatment with hydrochloric acid immediately demonstrates that the ostensibly pure material is hardly halfway decomposed by the acid, therefore, in spite of the apparent homogeneity of the yellow fragments, they are still of a different nature, just like the stone as a whole.

If a disassembled thin section is treated for a long time with hydrochloric acid and afterwards examined under the microscope, numerous sizable, small-sized, and quite tiny voids are observed, which mark the sections of the admixed components disintegrated by the acid in the still soundly cohesive thin section. If a solution of potassium hydroxide is then additionally applied to the thin section treated like this, it immediately falls apart into separate little pieces, grains, and tiny particles, amongst which the more sizable inclusions arising from the small fragments stand out due to their firm cohesion. It is quite noteworthy, that in the chunks with a mesh-like striated structure, although they still firmly cohere, the clear strips are totally destroyed and nothing, but the dark intermediate lamellae are left undecomposed, like a frame. The little water-clear strips or lamella are therefore highly likely comprised out of olivine, the dark part of an augite mineral. This has now also fully accounted for the phenomenon that the chondrules, like the survey of the stone from Eichstädt has taught, become partially decomposed by hydrochloric acid, but partially remain unaffected.
\clearpage

\subsection{Conclusion}
\paragraph{}
If one examines the results of the investigation of this, albeit limited, group of stone meteorites, then the perception that comes to the fore is that, in spite of some differences in the nature of their conglomeration, they are nevertheless governed by completely identical structural relations. All are undoubtedly \emph{débris}, composed of small and large mineral grains, from the well-known roundish chondrules: which are usually completely preserved, but often appear as broken pieces, to the globs of metallic meteoritic substances, sulfur-iron, and chromite. All these fragments are glued together, not cemented by an intermediate substance or a binder, as there are no amorphous, glassy, or lava admixtures at all. Only the fusion crust and black constrictions, which often appear on clefts and are similar to the crust, consist of amorphous glass, which, however, originated after falling within our atmosphere. In this fusion crust, the denser meltable and larger mineral grains are usually still embedded unmelted. The mineral splinters do not bear any traces of rounding or tumbling, they are sharp-edged and pointed. As for the chondrules, their surface is not smooth, as it would have been if they were the product of tumbling, rather it is always uneven, mulberry-like, and warty, or multifaceted with a projection of crystalline surfaces. Many of them are elongated with a distinct tapering or sharpening in one direction, as is the case with hailstones. Often you encounter pieces which apparently must be regarded as parts of shattered chondrules. As an exception are twin-like connected beads, most common in those which meteoritic iron beads have grown. In numerous thin sections they are composed differently. Most often there is an eccentric, radially-fibrous structure which spreads from a point far from the center after tapering or slightly tattered lines spread like rays toward the outside. Since cuts made at various angles always reveal a columnar or needle-shaped arrangement, never leaves or lamellas in the substance forming these tufts, it seems to be columnar fibers from which such chondrules are built. With certain cuts, according to this assumption, in the cross-sections of the fibers that are perpendicular to the length direction, only irregularly angular minute fields are observed, as if the whole were composed of small polyhedral granules. Sometimes they appear as if there were several systems radiating in different directions in a sphere, as if the point of radiation were altered during its formation, so that a constant and seemingly confused elongated structure emerges. Towards the outside, against which the junction point of the radiating bundle is shifted unilaterally, the fiber structure normally becomes indistinct or replaced by a more granular aggregate formation. In none of the numerous ground-up chondrules could I observe that the tufts ran directly to the edge, as if the point of emission were outside the sphere, provided that it was completely preserved and not a mere shattered piece. The delicate transversely dividing fibers usually do not run along the entire length of the tuft, but rather they gradually sharpen, branch or end to allow others to take their place, so that in the cross-sections, a manifold, mesh-like or netted image is created. These fibrils consist, as has often been described above, of a mostly lighter core with a darker envelope that is dissolved by acids, while the latter resists. Highly curious are the bowl-shaped constructions, which seem to be meteoritic iron, which are generally only spread over a small part of the globules. The same unilateral striations, visible on the average as crescent bowed streaks, also appear inside the chondrules and provide strong evidence contrary to their being formed by a tumbling of some material, the entire arrangement of the tufted structure speaks to a resolution against their origin by tumbling.\footnote{Also, the chondrules drawn by Richard von Drasche of the meteorite from Lancé ([Gustav] Tschermak's \emph{Mineralogical Reports}, 1875, Vol. 5, Issue 1) exactly match, in reference to the inner structure and outer form, our depiction.} However, not all chondrules are the eccentrically fibrous type; many, especially the smaller ones, have a fine-grained composition, as if they are composed of a mass of aggregated dust. Here too, the one-sided formation of the spheres is sometimes noticeable by an intensely greater compression of the dust pieces.

Finally, as far as the external shape of the tiny meteoritic iron and ferrous sulfide parts admixed with the chondrites is concerned, we do not notice any regular design at all in these either, neither in little strips corresponding to the nature of ilmenite, for instance in diabase, nor in roundish spherules; isolating the meteoritic iron is easy via a light crushing of the stony mass and extraction with the magnet, with this it is revealed that the surface of the small meteoritic iron pieces is powdery, as though coated over by tiny adhesive mineral particles. In general, they are erratically shaped little pellets and nodules, which frequently proceed in fine serrations and delicate granular ramifications. The powdery mineral particles, which are chained to the surface of the tiny pellets, can be stripped off through the application of hydrofluoric acid, and then an unevenly textured, punctated surface so to speak is observed, without any trace of reflection from crystal faces. The small ferrous sulfide pieces also have a similar quality, only not as jagged as them. More mundane, though always irregularly structured, are the chromite fragments.

The most common type of stony meteorite is predominately that of the so-called chondrites, the composition and structure of which coincide so much that we do not see how a common origin and the initial cohesion of these chondrites --- if not all meteorites --- could be in doubt.

The fact is that they enter our atmosphere as highly irregular pieces --- apart from the shattering within into several fragments, which is common, but cannot be assumed in all cases, especially if, by direct observation the falling of only a single piece is confirmed; it can be further concluded that they make their orbits in the heavenly space as demolished pieces of a single larger celestial body and in their absent-mindedness occasionally fall to Earth when they enter into the region of Earth's attraction. The lack of original lava-like amorphous constituents in connection with the external irregular form is likely to exclude from the geo- or cosmological points of view the assumption that these meteorites are ejections of lunar volcanoes, as is often claimed.

The remark, which Georg von Neumayer made regarding the Krähenberg fall,\footnote{\emph{Proceedings of the Mathematical and Natural Science Class of the Academy of Sciences in Vienna}, Vol. 60, 2, 1869, p. 239.} namely, that this meteorite's cosmic course was associated with the meteor shower whose radiation point lies in the proximity of $\delta$ Virginis, can only help to make the above hypothesis more likely. Here is what the views of almost all researchers who have in recent times been concerned with the study of the meteorite just on the subject of the cause of the above destruction work out to, whether it was caused by the collision of already solid celestial bodies, or due to some operative explosion of a cosmic mass from the inside out or else by a crumbling away of loose chunks, perhaps like it occurs with desiccating clays, various notions prevail, as Tschermak so admirably describes in his outstanding treatise on the formation of the meteorite and volcanism.\footnote{\emph{Proceedings of the Mathematical and Natural Science Class of the Academy of Sciences in Vienna}, Vol. 71, 1875, April issue.} With this hypothesis it is even conceivable that a meteorite, which had already sustained a partial melting once when its orbit grazed the Earth's atmosphere, subsequently once more entered into the perigee and then actually fell down to Earth. In this way the occurrence of fusion within the individual stone meteorites might perhaps be accounted for, related to the bonds smelted in the Earth's atmosphere. Even from an astronomical point of view, the above discussed belonging of much of the meteorites to a swarm of shattered little cosmic bodies encounters no contradiction.

We have attempted to consider the chondrites as a whole to establish the plausibility of the origin of our chondrites, in so doing from the geological stand point the highly important question still remains unanswered, how could the individual chondrites have been formed as a stone mass without a lava-like cementing agent, if we envisage in detail their composition out of tiny mineral slivers, little iron pellets, and nodules (chondrules). Indeed, in recent times [Gabriel Auguste] Daubrée has been intensely occupied with the purely mineralogical parts of this question and with the most favorable experimental results.\footnote{The most important of Daubrée's publications pertinent here are: \emph{Synthetic Experiments Relating to the Meteorites}, in: \emph{Comptes Rendus}, Tact 62, 1866, \emph{Bulletin of the Geological Society of France}, 2, Series A, 26, p. 95 and \emph{Comptes Rendus}, 1877, No. 27.} It can be inferred from his classic work that the main mineral components of the chondrites can be freshly obtained in a crystallized and crystalline state (at least the two silicates) by melting the stone under certain conditions, and that through melting one may even produce with these silicates terrestrial types of rock, for instance lherzolite or olivine rock, even of serpentine. It even yields a certain structural similarity between the melted lherzolite and certain meteorites. A more essential difference is attributable to the iron components, which in the case of lherzolite are oxidized, but reguline in the meteorites. While oxygen and water took part in the formations on Earth, the impact of these molecules during the development of the meteorites has to hypothetically be disqualified. The meteorites have no affinities with the types of stone present on the surface of the Earth's crust, such as granite. To come upon analogies for them on Earth, one must go down into the deeper regions of the Earth, where the closest relations are to be found in the basic silicates of the olivine rocks. Therefore, the meteorites appear to be a kind of first process of encasing the celestial bodies, but since they contain metallic iron --- to have been produced in the absence of oxygen and water. Through direct experimentation, Daubrée has not only established the genesis of the silicates, but also has demonstrated that under the reducing action of hydrogen, iron is able to arise in a reduced state in the magnetite of the lherzolites. The little iron pieces in the meteorites are to be found not in roundish globules, but in irregular nodules, as they emerge from the molten flows amongst reducing agents. Thus, the heat of the melt during the formation of the meteorites could not have held sway over the irons, nor even the silicates. But it may also be imagined that a process counter to that of the reduction was active, if one assumes that the original compounds were not existing in an oxidized, but in a reguline state, and that at the point where the oxygen activity began to unfold, it initially combined with the most easily oxidizable compounds and if insufficient amounts were present then the compounds more resistant to oxidization --- like that of iron --- were left unoxidized.

Daubrée has even attempted with success to corroborate this hypothesis through brilliantly conducted experiments. He also ascribes the origins of the olivine rocks of the Earth, which are encountered in the lowest depths, to a similar slagging process over the course of one of the first stages of formation, but unlike the development of the meteorites containing metallic iron, oxygen was available in excess to form both the silicates as well as --- instead of the meteoritic iron --- magnetite.

Provided that in so doing the mineralogical aspect, so to speak, of the formation of the meteorites turns up confirmed, the uniquely shattered structure of the chondrites calls for further consideration.

We learn from a more recent publication of Daubrée's\footnote{\emph{Bulletin of the Geological Society of France}, 1, 26a, 1868-9, p. 98 and further on.} that he conceived of the origination of the chondrules as analogous to the deposition of olivine globules during one of his trials, in which he had melted olivine blended with coal. The comparison would be more comprehensive if the reduction process took place due to hydrogen. Only the other day did a very distinguished scholar on meteorite knowledge,\footnote{\emph{Comptes Rendus}, 1877, No. 27.} upon chance during the discussion on the subject of the peculiar breccia-like structure of the meteoritic iron from Santa Catharina, say moreover, that the fragmentation of the materials cohering the stone meteorites must be considered as an explosive effect from very compressed gases, perhaps such as it occurs from the application of dynamite. But concerning the formation of the chondrules, he refers to the trial cited above, whereby a kind of granulation gets conducted at the moment in which the substance solidified. Though most often the chondrules seem to him to be simple fragments, which are rounded down due to abrasion, such as arrived at in the investigation of these globules by Gustav Rose (paper in the \emph{Academy of Sciences in Berlin} for 1862, p. 97 and 98) and clearly set forth by [Stanislas-Étienne] Meunier regarding a number of meteorites (\emph{Comptes Rendus}, 1871, p. 346 and \emph{Research on the Composition and the Structure of the Meteorites}, 1869).

Following the procedures of [Wilhelm Carl von] Haidinger, Tschermak has also recently undertaken detailed studies on the formation of the meteorites and disclosed in further writings the findings of this highly interesting examination. These works are undoubtedly among the most important and profoundly exhaustive that we possess on this subject. Regarding the formation of the individual meteoritic pieces, Tschermak comes up with the most probable assumption that they do not owe their gestalt to a destruction of planets due to impact, but that through a force from the inside out, by an explosion analogous to volcanic activity, they were subjected to a destruction into tiny pieces that one must call atomization. Here he points out the violent, explosion-like prominences that have directly been observed in the sun and comets or reveal themselves on the lunar surface by the structure of the craters. More particularly, as far as the composition of the meteorites is concerned, Tschermak follows Haidinger's point of view, that they are assembled out of stone dust, which is likened to volcanic tuff. It is merely the occurrence \emph{en masse} of the tiny globules, which, as is well known, do not appear in the tuffs of the terrestrial volcanoes and are therefore more challenging to explain. These globules definitely do not act in accordance with his assumption, as if they had reached their form through crystallization, nor do they act like the spherulites in obsidian and perlite, or like the spheres in orbicular diorite and the round concretions of calcite, aragonite, and marcasite. They rather resemble those spheres that one frequently spots in the tuffs of volcanic formations, for example the trachyte spheres in the trachytic tuffs of Bad Gleichenberg, the spheres in the basaltic tuffs at Venusberg near Freudenthal, though especially the olivine spheres in the basaltic tuffs from Kapfenstein and Feldbach in Styria.\footnote{Only a related material was at my disposal, the trachytic tuffs with the so-called leucite nodules from the cyclopean islands. Thin sections of this rock taught me that the alleged leucite rock spherules are comprised out of the same material as the tuff itself and that they do not possess any structure akin to that of the meteorite chondrules. I additionally received samples of the rocks from Gleichenberg through Mr. Tschermak's special kindness. No analogies with the chondrules can be identified in these olivine nodules.} From the latter one may safely assume that they are the products of volcanic trituration and owe their form to the continual explosive activity of a volcanic vent, through which splintered older rocks and their tougher parts become rounded by constant collisions. At best one can envisage that the stone masses, which were subjected to the trituration, became considerably malleable and would therefore approximate the idea of Daubrée, which suggests that the stone solidified in a vortical mass of gas. Nevertheless, it should be emphasized that no meteorite has any resemblance with volcanic slag or with lava, hence the comparison of the meteorites with volcanic tuffs or breccias can only be valid up to a certain degree. The volcanic activity during the forming of the meteorites thus consisted only in the fragmentation of more rigid rocks through some explosive action as a consequence of the sudden expansion of vapor or gas, amongst which hydrogen gas may have played a major role.

So ingenious are these hypotheses of Daubrée's and Tschermak's, however, I cannot agree with their view on the formation of the globules (chondrules) on the basis of my latest research. Contrary to Tschermak's assumption, I sought to prove that the internal structure of the chondrules is not out of context with their spherical shape and that these globules cannot be regarded as pieces of a mineral crystal or solid rock. Their unsmooth, unpolished surface stands out, which, if they were formed by abrasion or tumbling, should be mirror-smooth due to the similar hardness of the material, while instead it appears rough, bumpy, often facially striated, against the theory of friction, and there is no reason at all by which to understand why the other mineral fragments are rounded like grains of sand, and why, in particular, the meteoritic iron and the very hard chromite, as I have been convinced in the meteorite of l'Aigle, are always angular, with often extremely fine, cut-leaved forms. How is it conceivable that, as if often observed, there would be a concentric accumulation of meteoritic iron within the globules? Also, the eccentrically fibrous structures of most globules in their one-sided radiating do not appear to be random in relation to the surface, but rather like the nature of the structure of hailstones. This inner structure is closely related to the act of its formation, which can only be explained as a growth of mineral forming substances with simultaneous rotation in gaseous vapors that provided the material for further support, whereby more material adhered in the direction of movement.

I have selected the facts which have come to light for all the chondrites --- and handle them here,
\begin{enumerate}
    \item that they are basically comprised out of fine or coarse little mineral fragments or out of angular, or hemispherical, shattered pieces of chondrules and of these themselves;
    \item that there is no trace of lava- or slag-like admixtures nor binding agents; all slagging that is found is only secondary phenomena resulting from the movement of the meteorite within the terrestrial atmosphere;
    \item that neither the admixed meteoritic iron nor ferrous sulfide nor chromite possess the form of the chondrules and not a trace of sustained tumbling can be detected;
    \item that the inner structure of the chondrules has a genetic connection, be it eccentrically fibrous, or granular, or merging into a powdery density, with elongated, round, reminiscent of the egg shape figure, as the nature of the bundles of rays unambiguously shows;
    \item that precipitations in the interior of the globules are occasionally found that correspond to the surface shape and
    \item finally, that the chondrules' surface is not polished, as in the case of an origination through tumbling, but rough and bumpy, as if particle after particle had outwardly settled into it,
\end{enumerate}
\paragraph{}
I have to think, in partial agreement with the cited scholars, that the material out of which the chondrites are comprised arose through a disturbed crystallization process and fragmentation as a consequence of an explosive process within a space, which was composed of a vapor providing the mineral compounds and suffused with hydrogen gas that hindered further oxidation of the meteoritic iron. The globules arose through the accumulation of mineral masses around a deposit or kernel during a continual fall or movement in vapors supplying compounds, whereby a one-sided bulge or an accretion of the materials in the direction of flight, as induced in the formation of certain hailstones or ice pellets and provides an explanation for the eccentrically fibrous structures and oblong forms. That fragmentation happened as a result of the collision of solidified masses is proven by the globules scattered in the smithereens and the abundant angular fragments, which, as with the globules, possess this fibrous structure. Perhaps the disintegration occurred as a result of rapid temperature changes. The material arising in this way fell like a shower of ash towards the surface of the emerging celestial body and compacted itself through agglutination of the \emph{débris} into a mostly loose aggregate, in a manner like that of the volcanic dry-tuffs, and, perhaps initially in this state of consolidation, was fragmented and flung apart by further explosiveness. These pieces or bits of those pieces are what ultimately arrived at Earth as meteorites. That other meteorites, namely the meteoritic iron masses and the carbonaceous ones, must have experienced another development to some extent is not disputable; they seem to have undergone a calmer process on the surface of the celestial bodies and have only this in common with the stony meteorites, that they partially involve the same material in their composition, even if in lower amounts and that they were fragmented and hurled off in a similar manner.

I encountered partially similar views, to which my study of the chondrites led, even with [Henry Clifton] Sorby, who had in the past already indicated this in the essay: ``On the Physical History of Meteorites.''\footnote{\emph{The Geological Magazine}, 2, 1865, p. 447.}

I would like to add to these remarks some observational results that I obtained in the carbonaceous meteorites from Bokkeveld and Kaba. I owe the material for this to the especial kindness of the gentleman Professor Tschermak in Vienna. I hoped through thin sections to perhaps discover some trace of organic structure in the carbonaceous constituents. In the meteorite from Bokkeveld, thin sections of which are incredibly involved and only ever restrict the method of preparation so that the carbonaceous areas become translucent only in patches, one sees a small quantity of particularly sharp-cornered, tiny water-clear mineral splinters embedded in the carbonaceous main mass. In polarized light this mineral \emph{débris} displays vivid, variegated colors and generally appears to behave like the components of the chondrites. The carbonaceous substance, wherever it is translucent, has that membranous or finely granulated microstructure, as is otherwise met with in carbonaceous substances. Small pieces which I treated with potassium chlorate and nitric acid for a few days in the cold became exceptionally soft and completely discolored. Soaking in Canada balsam allows the making of thin sections, in which the little mineral slivers now show themselves as partly blurred and non-transparent (likely decomposed olivine), but partly remaining water-clear (probably augite-like admixtures), while the carbonaceous main mass splits up into fully transparent masses and, in between these, engrained dark specks and wisps. The transparent parts allow one to perceive the same membranous-granular structure, as with the translucent sections of the untreated thin sections. Even after this procedure, indications of more organic structure could not be detected.

The carbonaceous meteorite from Kaba is a great deal harder. In thin sections one observes tiny clear mineral pieces, very numerous and with cuts through them nearly circular, thus plausibly in accordance with the chondrules, though as far as my material allows, devoid of fibrous structure. Rather, they are comprised so to speak out of an aggregate of water-clear granules, in between which usually run little non-transparent strips. Black, possibly carbonaceous, lines like this and spots also appear, mostly in concentric arrangements in and around the globules. This meteorite withstands the action of potassium chlorate and nitric acid, it decolorizes only a little, while, on the other hand with this treatment the globules have become cloudy and non-transparent as a result of the sustained corrosion, which with to some degree of probability points to their having an olivine nature. Under these circumstances, even with these carbonaceous meteorites, more organic structure is not to be seen. Perhaps one will still manage to achieve to establish the presence of organic entities on extraterrestrial celestial bodies under the application of the above cited bleaching agent with more ample material or with other carbonaceous meteorites.
\clearpage
\end{document}
