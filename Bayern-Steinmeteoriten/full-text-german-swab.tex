\documentclass[a4paper, 11pt, oneside]{article}
\usepackage[utf8]{inputenc}
\usepackage[T1]{fontenc}
\usepackage[ngerman]{babel}
\usepackage{yfonts}
%\usepackage{fbb} %Derived from Cardo, provides a Bembo-like font family in otf and pfb format plus LaTeX font support files
\usepackage{booktabs}
\setlength{\emergencystretch}{15pt}
\usepackage{fancyhdr}
\usepackage{graphicx}
\graphicspath{ {./} }
\usepackage{microtype}
\usepackage[figurename=]{caption}
\usepackage[titles]{tocloft}
\usepackage{sectsty}

\sectionfont{\Huge}
\subsectionfont{\LARGE}
\subsubsectionfont{\LARGE}
\begin{document}
\swabfamily
\renewcommand{\contentsname}{
\swabfamily{Inhaltsverzeichnis}
}
% fix toc page numbers
\let\origcftsecfont\cft
\let\origcftsecpagefont\cftsecpagefont
\let\origcftsecafterpnum\cftsecafterpnum
\renewcommand{\cftsecpagefont}{\swabfamily{\origcftsecpagefont}}
\renewcommand{\cftsecafterpnum}{\swabfamily{\origcftsecafterpnum}}
\let\origcftsubsecpagefont\cftsubsecpagefont
\let\origcftsubsecafterpnum\cftsubsecafterpnum
\renewcommand{\cftsubsecpagefont}{\swabfamily{\origcftsubsecpagefont}}
\renewcommand{\cftsubsecafterpnum}{\swabfamily{\origcftsubsecafterpnum}}
\let\origcftsubsubsecpagefont\cftsubsubsecpagefont
\let\origcftsubsubsecafterpnum\cftsubsubsecafterpnum
\renewcommand{\cftsubsubsecpagefont}{\swabfamily{\origcftsubsubsecpagefont}}
\renewcommand{\cftsubsubsecafterpnum}{\swabfamily{\origcftsubsubsecafterpnum}}
\begin{titlepage} % Suppresses headers and footers on the title page
	\centering % Centre everything on the title page
	%\scshape % Use small caps for all text on the title page

	%------------------------------------------------
	%	Title
	%------------------------------------------------
	
	\rule{\textwidth}{1.6pt}\vspace*{-\baselineskip}\vspace*{2pt} % Thick horizontal rule
	\rule{\textwidth}{0.4pt} % Thin horizontal rule
	
	\vspace{1\baselineskip} % Whitespace above the title
	
	{\Huge Sitzungsberichte der\\[1.25pt] Mathematisch-Physikalischen Klasse der\\[1.25pt] K"oniglich Bayerischen Akademie der\\[1.25pt] Wissenschaften zu M"unchen\\[1.25pt]}
	
	\vspace{1\baselineskip} % Whitespace above the title

	\rule{\textwidth}{0.4pt}\vspace*{-\baselineskip}\vspace{3.2pt} % Thin horizontal rule
	\rule{\textwidth}{1.6pt} % Thick horizontal rule
	
	\vspace{1\baselineskip} % Whitespace after the title block
	
	%------------------------------------------------
	%	Subtitle
	%------------------------------------------------
	
	{\Large Jahrgang 1878 --- Band 8} % Subtitle or further description
	
	\vspace*{1\baselineskip} % Whitespace under the subtitle
	
    {\Large In Kommission bei G. Franx} % Subtitle or further description
    
	%------------------------------------------------
	%	Editor(s)
	%------------------------------------------------
    \vspace*{\fill}

	\vspace{1\baselineskip}

	{\small\scshape M"unchen 1878}
	
	{\small\scshape{Akademische Buchdruckerei von F. Straub}}
	
	\vspace{0.5\baselineskip} % Whitespace after the title block

    \scshape Internet Archive Online Edition  % Publication year
	
	{\scshape\small Namensnennung Nicht-kommerziell Weitergabe unter gleichen Bedingungen 4.0 International} % Publisher
\end{titlepage}
\setlength{\parskip}{1mm plus1mm minus1mm}
\clearpage
\tableofcontents
\clearpage
\LARGE
\pagestyle{fancy}
\fancyhf{}
\cfoot{\swabfamily{\thepage}}
\section{\swabfamily{Sitzung vom 9. Februar 1878 --- Herr G"umbel spricht: "Uber die in Bayern gefundenen Steinmeteoriten}}
\subsection*{\swabfamily{Einleitung}}
\paragraph{}
Unter den auf bayerische Gebiete gefallenen und aufgefundenen Steinmeteoriten befinden sich mehrere, deren chemische Zusammensetzung uns nur aus "alteren Analysen bekannt ist, w"ahrend von einem derselben bis jetzt "uberhaupt noch keine chemische Untersuchung vorgenommen wurde. Da es au"serdem ihre den meisten derselben an einer ersch"opfenden Untersuchung, Zwie solche neuerdings bei Gesteinsarten mittelst D"unnschliffe und Mikroskop vorgenommen zu werden pflegt, fehlt, so schien es mir interessant genug, diese Arbeit vorzunehmen und die Ergebnisse mit dem fr"uher bekannten zusammenzustellen. Durch die besondere G"ute des Herrn Konservators der mineralogischen Staatssammlung Professor Dr. v. Kobell habe ich das hierzu erforderliche Material erhalten und ich ben"utze gerne die Gelegenheit, f"ur diese so freundliche Unterst"utzung meiner Untersuchung hier den besten Dank auszudr"ucken. Einige weitere Bemerkungen, welche an den Schl"ussen beigef"ugt sind, beziehen sich auf andere Meteorsteine, die ich gelegentlich der Vergleichung wegen in den Kreis meiner Beobachtung gezogen habe.

Es wurden im Ganzen nur 5 Steinmeteoriten von denen, welche in Bayern gefallen sind, bekannt. Darunter ist sogar noch ein Fund einbegriffen, welcher nach dem gegenw"artigen Territorialverh"altnisse nicht mehr Bayern, sondern Österreich angeh"ort, n"amlich jener von Mauerkirchen. Da jedoch zur Zeit des Falls der Ort zu Bayern geh"orte, so d"urfte es immerhin bis zu einer gewissen Grade gerechtfertigt erscheinen, diesen Stein hier unter den bayerischen aufzuf"uhren.

Diese 5 Steinmeteorite sind:
\begin{enumerate}
    \item Der Stein von Mauerkirchen im jetzt "osterreichischen Innviertel vom Falle am 20. Nov. 1768 Nach-, mittags 4 Uhr.
    \item Der Stein von Eichst"adt, welcher im sog. Wittmes 5 Kilom. von der Stadt am 19. Febr. 1785 nach 12 Uhr Mittags gefallen ist.
    \item Der Stein von Massing bei Alt"otting in S"udbayern vom Fall am 13. Dezember 1803 zwischen 10-11 Uhr Vormittags.
    \item Der Stein von Sch"onenberg bei Burgau und Schwaben, gefallen am 25. Dez. 1846 Nachmittags 2 Uhr und
    \item Der Stein von Kr"ahenberg bei Homburg in der Rheinpfalz vom Fall am 5. Mai 1869 Abends 6 1/2 Uhr.
\end{enumerate}
\paragraph{}
Vom einem 6. Meteorstein fand ich eine erste Nachricht in Gilbert's Annalen der Physic Bd. XV. S. 317, wo angef"uhrt wird, dass Casp. Schott in s. Physica curiosa l. XI Cap. XIX berichtet: "`hac in urbe nostra Herbipolensi osservatur in templo D. Jacobi trans Moenum, in monasterio Scotorum\footnote{\swabfamily{Das Schottenkloster war de 1140 gegr"undet, 1803 saecul. 1819 wurde ein Teil der Kirche zum Gottesdienst wiederhergerichtet, und zwar der Chor, das "Ubrige dient als Milit"ardepot. Ausf. Beschreibung u. Geschichte von Wieland im Archiv des hist. Vereins v. Unterfranken u. Asch. XVI. Bd.}} catenulae columna templi suspensus... durissimus est et ad ferream vergit naturam."' Daraus geht hervor, dass es wahrscheinlich ein Eisenmeteorit war. Ich habe mich, um den Spuren dieses Steines nachzuforschen an Herrn Prof. Sandberger in W"urzburg gewendet, der so freundlich war, die gr"undlichsten Nachforschungen anzustellen. Der Stein ist verschwunden. Der g"utigen Mittheilung Sandberger's verdank eiche die weitere Nachricht, welche Schnurrer in s. Seuchengeschichte Bd. II. gibt: "`Im Jahre 1103 (oder 1104) fiel in W"urzburg ein so gro"ser Meteorstein, dass vier M"anner den vierten Theil desselben kaum tragen konnten."'
\clearpage
\subsection{\swabfamily{Der Meteorstein von Mauerkirchen}}
\begin{figure}[h]
\centering
\includegraphics[keepaspectratio, scale=2]{Fig-1.png}
\caption{\swabfamily{Figur 1}}
\end{figure}
\paragraph{}
"Uber diesen Fall berichtete zuerst ein kleines Schriftchen: Nachricht und Abhandlung von einem in Bayern unfern Maurkirchen d. 20. Nov. 1768 aus der Luft gefallenen Steine (Straubingen 1769). Aus demselben teilt Chladni in seine chronologischen Verzeichnisse der mit einem Feuermeteor niedergefallenen Stein- und Eisenmassen (Gilberts Ann. d. Phys. 1803 Bd. XV. S. 316) mit, dass an dem genannten Tage Abends nach 4 Uhr bei einem gegen Occident merklich verfinsterten Himmel verschiedene ehrliche Leute zu Maurkirchen, welche dar"uber eidlich vernommen wurden, ein ungew"ohnliches Brausen und gewaltiges Krachen in der Luft gleich einem Donner und Schie"sen mit St"ucken h"orten. Unter diesem Luftget"ummel sei ein Stein aus der Luft gefallen und habe nach obrigkeitlichem Augenschein eine Grube 2 1/2 Schuh tief in die Erde gemacht. Der Stein halte nicht gar einen Schuh in die L"ange, sei 6 Zoll breit und wiege 38 bayer. Pfunde Er sei von so weicher Materie, dass er sich mit Fingern zerreiben lasse, von Farbe bl"aulich mit einem wei"sen Fluss oder Flie"serlein vermengt, au"serdem mit einer schwarzen Rinde "uberzogen n. s. w.

Professor Imhof vervollst"andigte diesen Bericht (Kurpfalzbaier. Wochenblatt. 1804. St. 4) durch folgende Angaben: "`Man fand den gefallenen Stein am Tage, nachdem man das Get"ose vernommen hatte, in dem sog. Schinperpoint in einem schr"ag einw"arts gehenden 2 1/2 Schuh tiefen Loche."' Imhof bestimmte das spec. Gewicht zu 3,452 und beschreibt die graulich schwarze 1/4 Linie dicke Rinde als am Stahl funkengebend, ferner als Gemengteile
\begin{enumerate}
    \item regulinisches Eisen, das in kleinen K"ornern und Zacken am meisten mit der "au"seren Rinde verwachsen, sehr geschmeidig und z"ahe ist und einen wei"sen stark gl"anzenden Feilenstrich gibt,
    \item Schwefelkies,
    \item kleine plattgedr"uckte, eckige K"orner, welche sich durch schwarzgraue Farbe, muschlichten Bruch, gl"anzendes Ansehen und gr"o"serer H"arte von den "andern unterscheiden,
    \item noch andere kleine K"orner von wei"ser und gelblicher Farbe, die durchscheinend und schimmernd sind. Nach seiner Analyse besteht der Meteorstein aus:
    \begin{center}
        \begin{tabular}{ |l|r| } 
        \hline
        Kiesels"aure & 25,40\\\hline
        Eisenoxyd & 40,24\\\hline
        Eisen & 2,33\\\hline
        Nickel & 1,20\\\hline
        Bittererde & 28,75\\\hline
        Schwefel und Verlust & 2,08\\
        \hline
        \end{tabular}
    \end{center}
\end{enumerate}
\paragraph{}
(Vergl. O. B"uchner die Meteoriten in Sammlungen 1863 S. 9.)

Die n"ahere Untersuchung des Steines ergab mir nun weiter, dass die mattschwarze, fleckenweis etwas Gl"anzende 0,7---0,3 mm. dicke Kruste wie bei anderen Meteorsteinen nur Schmelzrinde ist, welche ohne scharfe Grenze gegen Innen in die Hauptmasse "ubergeht, da wo Eisenteilchen an dieselbe grenzen, verst"arkt, wo gewisse gelbe K"ornchen in derselben liegen, schw"acher und an letzteren Stellen gl"anzender sich zeigt. H"aufig sind selbe Mineralteilchen eingeschmolzen und in der Rinde eingeschlossen oder ragen in dieselbe hinein. Die Hauptmasse des Steines ist lichtgrau gef"arbt, durch eingestreutes Meteoreisen schwarz punktiert und an den meisten dieser schwarzen Stellen in Folge der Oxydation des Eisens fleckig rostfarbig. Zwischen den Fingern l"asst sich der Stein ziemlich leicht zerdr"ucken und macht dem "au"seren Anschein nach dem Eindruck eines Trachyttuffs.

Aus der "au"serst feinbr"ocklichen, fast staubartigen Grundmasse heben sich ziemlich zahlreich eingestreute rundliche Mohn- bis Hirsekorn-gro"se und kleinere K"ornchen heraus, welche meist etwas dunkelschw"arzlich oder gelblich gef"arbt, au"sen matt, beim Zerschlagen glasgl"anzend ohne Spaltungsfl"achen erkennen zu lassen, den Charakter der Chondren besitzen und dem Stein daher den Stempel der Chondriten aufdr"ucken. Unter dem Mikroskop zeigen diese K"ornchen eine verschiedene Beschaffenheit. Die einen sind "au"serst fein parallel gestreift, so dass vorwaltend opake, breite Streifchen mit schmalen durchsichtigen oder durchscheinenden, wie quer gegliederten wechseln. I. p. L. erscheinen letztere mit matten feinfleckigen Farben. (y der Zeichnung der beiliegenden Tafel Fig. I. Andere K"ornchen sind wei"slich, wie aus feinstem Staub zusammengesetzt, opak, nur gegen den Rand zu etwas durchscheinend, zuweilen von feinsten, etwas durchschimmernden, einzelnen unregelm"a"sig eingestreuten N"adelchen durchzogen (x der Zeichnung). Noch andere K"ornchen besitzen eine Art radiale Faserung, die jedoch hier nicht deutlich zum Vorschein kommt. Kleinste, rundliche Teilchen sind wasserhell und erscheinen i. p. L. mit gl"anzenden bunten Farben.

Neben den Chondren lassen sich in der pulverigen Hauptmasse eingebettet noch zahlreiche meist kleine eckige l"angliche Splitterchen eines wei"sen, auf der Spaltfl"achen deutlich spiegelnden, hier und da undeutlich parallel gestreiften Minerals und mehr rundlich eckige, unregelm"a"sig rissige, selten parallelstreifende K"ornchen von gelblichem oder br"aunlichem Farbenton und von glasartigem Glanze unterscheiden. Dazu gesellen sich metallisch gl"anzende, relativ kleine traubig eckige Kl"umpchen von Meteoreisen, ferner selten solche von messinggelbem Schwefeleisen und von nicht metallisch gl"anzenden tiefschwarzen Chromeisenst"abchen. An abgeriebenen Stellen des Steins stehen die h"arteren K"ornchen hervor und lassen den Charakter den Chondriten deutlicher wahrnehmen, als auf dem Querbruche, auf dem man nur bei gr"o"serer Aufmerksamkeit die kugeligen Einlagerungen beobachtet. Die feinsten Staubteilchen, welche als das durch eine fortschreitende Zerkleinerung der gr"o"seren Splitter entstandene verbindende Material betrachtet werden m"ussen, sind teils wasserhell, teils opak, durchscheinend, und erweisen sich bis ins Kleinste i. p. L. durch wenn auch matte bunte Farben als doppelt brechende kristallinische Bruchst"ucke. Von einer glasartigen Zwischenmasse ist nicht eine Spur zu entdecken.

Nach dem Behandeln des fein zerdr"uckten (nicht zerriebenen) Materials mit Salpetersalzs"aure und Kalil"osung sind --- abgesehen von den metallischen Gemengteilen --- die gelblichen Splitterchen (Olivin) verschwunden und der R"uckstand besteht nur aus wei"sen und br"aunlichen St"ucken, die unter dem Mikroskop sich leicht unterscheiden lassen. Die br"aunlichen Fragmente sind stark rissig, selten mit Spuren von dunklen Parallelstreifchen versehen, durchsichtig und i. p. L. lebhaft buntfleckig gef"arbt. Es sind zweifelsohne Teilchen eines Minerals aus der Augitgruppe. Die wei"sen Splitterchen dagegen sind vielfach nur durchscheinend, teilweise durch die S"auren angegriffen und zeigen i. p. L. nur matte fleckige Farbent"one, welche hier und da an eine streifige Anordnung erinnern. Dass diese Splitterchen als Feldspat-artige Gemengteile gedeutet werden m"ussen, beweist auch die chemische Analyse des Restanteils nach der Einwirkung der S"auren. Kleinste schwarze Teilchen sind als Chromeisen anzusprechen. Es besteht demnach der Stein aus Olivin, einem Feldspat-artigen, augitischen Mineral, aus Meteor-, Schwefel- und Chromeisen.

Damit stimmt nun auch im Allgemeinen die chemische Analyse, welche von Hrn. Assistent Ad. Schwager unter gleichzeitig kontrollierenden eigenen Untersuchungen durchgef"uhrt wurde. Die Bestimmung des Meteoreisens und Schwefeleisens geschah durch eigene Versuche\footnote{\swabfamily{Es wurde aus dem zerdr"uckten Pulver durch den Magnet alles Ausziehbare herausgenommen, und diese Meteoreisen haltigen Bestandteile unter Anwendung von Kupfervitriol und Kupferchlorid besonders analysiert.}}. Die Analysen ergaben:
\begin{center}
\begin{tabular}{ |p{27mm}|p{35mm}|p{25mm}|p{10mm}| }
    \hline
    Stoffe: & Bauschanalyse 65,45\% durch\newline Salzs"aure zersetzbarer Anteil & 34,55\%\newline Restbestandteil & \\
    \hline\hline
    Kiesels"aure & 38,14 & 23,23 & 61,39\\\hline
    Tonerde & 2,51 & 1,20 & 5,00\\\hline
    Eisenoxydul & 25,70 & 32,72 & 17,59\\\hline
    Eisen mit Nickel & 6,30 & 9,65 & -,-\\\hline
    Schwefel & 2,09 & 3,20 & -,-\\\hline
    Phosphor & 0,14 & 0,22 & -,-\\\hline
    Chromoxyd & 0,39 & -,- & 0,84\\\hline
    Kalkerde & 2,27 & 1,51 & 4,35\\\hline
    Bittererde & 21,73 & 29,13 & 7,70\\\hline
    Kali & 0,48 & Sp. & 1,40\\\hline
    Natron & 1,00 & Sp. & 2,91\\\hline
    Summe & 100,75 & 100,86 & 101.18\\
    \hline
\end{tabular}
\end{center}
\paragraph{}
Es schlie"st sich demnach der Steinmeteorit von Mauerkirchen der Anfangsreihe der an Kiesels"aure "armsten Chondriten, wie jenen von Seres, Buchhof, Ensisheim und Chateau-Renard an. Es l"asst sich daraus der Gehalt berechnen, n"amlich an:
\begin{center}
    \begin{tabular}{ |l|r| } 
    \hline
    Meteoreisen & 2,81\%\\\hline
    Schwefeleisen & 5,72\\\hline
    Chromeisen & 0,75\\\hline
    Silikate & 90,72\\
    \hline
    \end{tabular}
\end{center}
\paragraph{}
Was die Interpretation der Silikate anbelangt, so haben wir zun"achst den durch Salzs"aure zersetzbaren Bestandteil ins Auge zu fassen. Hierin ist der relativ geringe Kiesels"auregehalt besonders auffallend. Doch wiederholt sich ein "ahnliches Verh"altnis« mehrfach wie z. B. bei den Meteorsteinen von Seres, Tjabé (Java 19. Sept. 1869), Khettre (Indien) u. A. Ziehen wir den Gehalt an Meteoreisen und Schwefeleisen ab, so erhalten wir f"ur diesen Bestandteil:
\begin{center}
    \begin{tabular}{ |l|r| } 
    \hline
    SiO\textsubscript{2} & 26,45\\\hline
    Al\textsubscript{2}O\textsubscript{3} & 1,35\\\hline
    FeO & 37,30\\\hline
    CaO & 1,70\\\hline
    MgO & 33,20\\
    \hline
    \end{tabular}
\end{center}
\paragraph{}
Worin, wenn die Tonerde und Kalkerde als wahrscheinlich zu einem zersetzten Feldspat gerechnet und ein Teil des Eisenoxyduls als noch von Meteoreisen abstammend in Abzug gebracht wird, der durch S"auren zersetzte Bestandteil nicht anders, als zu Olivin geh"orig sich auslegen l"asst. Dass ein Teil des Eisens oxydiert ist und dadurch der Gehalt an Basen etwas gesteigert erscheint, darauf weisen schon den Rostflecken hin, welche sich manchmal selbst in der Masse ziemlich verbreitet zeigen.

Was das oder die Silikate des Restbestandteils angeht, so gibt der verh"altnism"a"sig hohe Kiesels"aure- und Tonerdegehalt, neben den Alkalien wohl der Vermutung Raum, dass neben einem Augit-Mineral auch noch ein feldspattiges vorhanden sei. Gleichwohl aber bleibt auch bei dieser Annahme noch ein starker "Uberschuss an Eiesels"aure, von dem man wohl nicht voraussetzen darf, dass er in Form eines ausgeschiedenen Quarzminerals auftrete, weil bei Untersuchung des D"unnschliffs im reflektierten Lichte keine Spur einer Beimengung von durch den starken Glanz sonst erkennbarem Quarze sich bemerken l"asst. Dieses Verhalten ist vorl"aufig noch unaufgekl"art.

Derselbe Meteorstein ist bereits in neuester Zeit auch noch einer chemischen Analyse von anderer Seite unterworfen worden. Rammelsberg f"uhrt (D. chem. Nat. d. Meteoriten Abh. d. Acad. d. Wiss. in Berlin f"ur 1870 S. 148 u. ff.) als das Resultat der von Crook\footnote{\swabfamily{On the chem. constit. of meteor. stones. G"ottingen Dissert. (Mir nicht zug"anglich).}} ausgef"uhrten Untersuchung an: Zusammensetzung:
\begin{center}
    \begin{tabular}{ |l|r| } 
    \hline
    3,52\% & Meteoreisen\\\hline
    1,92 & Schwefeleisen\\\hline
    0,72 & Chromeisen\\\hline
    92,68 & Silikat\\
    \hline
    \end{tabular}
\end{center}
100,00 und zwar:
das Silikat bestehend als:
\begin{center}
\begin{tabular}{ |p{20mm}|p{24mm}|p{31mm}|p{32mm}| }
    \hline
    Stoffe: & im Ganzen Bauschanalyse & in dem 61\% durch S"auren zersetzbar. Anteil. & in dem 39\%\newline in S"auren unzersetzb. Anteil.\\
    \hline\hline
    Kiesels"aure & 44,81 & 32,68 & 3,94\\\hline
    Tonerde & 1,24 & 9,36 & 4,17\\\hline
    Eisenoxydul & 24,55 & 28,91 & 17,71\\\hline
    Bittererde & 26,10 & 37,44 & 8,20\\\hline
    Kalkerde & 2,28 & 0,61 & 4,91\\\hline
    Natron & 0,26 & -,- & 0,67\\\hline
    Kali & 0,16 & -,- & 0,40\\
    \hline
\end{tabular}
\end{center}
\paragraph{}
Diese Resultate weichen so bedeutend von den fr"uher mitgeteilten ab, dass daf"ur kein anderer Grund gefunden werden kann, als die an sich gro"se Ungleichheit in der Zusammensetzung des Meteorsteins, welche einen umso gr"o"seren Einfluss auf die Ergebnisse der Untersuchung zu "au"sern im Stande ist, mit je kleineren Quantit"aten man zu arbeiten gezwungen ist. Die mikroskopische Untersuchung der D"unnschliffe unterst"utzt direkt diese Annahme, indem sich hierbei die gr"o"ste Unregelm"a"sigkeit in der Art der Verteilung der Gemengteile erkennen l"asst. Ein gr"o"seres Korn von diesem oder jenem Gemengteil in der verwendeten Probe verr"uckt bei geringen Quantit"aten, die man ben"utzt, die Zahlen in betr"achtlicher Weise. Es lassen sich beispielsweise zackige Kn"ollchen von Meteoreisenteilchen aus der Masse herausl"osen, deren Gr"o"se in keinem Verh"altnisse steht zu dem geringen Prozentgehalte des Steins an Meteoreisen im Allgemeinen und Ganzen. Ähnlich verh"alt es sich mit den eingestreuten h"arteren Kn"ollchen und K"ornchen.

Besonders verschieden ist die Angabe bez"uglich der Zusammensetzung des in Salzs"aure zersetzbaren Gemengteils. Doch tritt auch in der Analyse Crook's die relativ geringe Menge von Kiesels"aure sehr deutlich hervor. Minder abweichend erweisen sich die Resultate der Analyse des durch S"auren unzersetzten Restes. Gerade dies beweist, dass es nicht in dem Gang der analytischen Arbeit liegt, wie es scheinen k"onnte, wenn hier der Kiesels"aurengehalt ebenso verh"altnism"a"sig hoch, wie bei dem in S"auren zersetzbaren Anteil gering gefunden wurde. Da dieser Rest, wie die mikroskopische Untersuchung desselben lehrt, aus verschiedenen Mineralsubstanzen, namentlich einem wei"sen und einem braunen Gemengteil besteht, so kann das Sauerstoff-Verh"altnis im Ganzen genommen, uns keine besonderen Aufschl"usse verschaffen.

Die wegen der leichten Zerreiblichkeit der Masse schwierig herzustellenden D"unnschliffe, welche nur durch wiederholtes Tr"anken mit sehr verd"unntem Canadabalsam in brauchbarem Zustande gewonnen werden k"onnen, geben, wie es das D"unnschliffbild auf der beiliegenden Tafel in Figur I. zeigt, bez"uglich der Zusammensetzung des Gesteins und der Verteilung der Gemengteile einige lehrreiche Aufschl"usse. Es stechen besonders die Chondren in ihrer teils staubig kr"umeligen, teils faserigen Zusammensetzung besonders hervor. Trotz der geringen Durchsichtigkeit derselben erweisen sie sich i. p. L. betrachtet stets farbig, und zwar nicht blo"s die lichteren Streifchen derselben, sondern ihre ganze Masse. Diesen Einmengungen gegen"uber sind die "ubrigen unterscheidbaren, stets unregelm"a"sig umgrenzten, gelblichen, br"aunlichen und wei"slichen Splitterchen klein. Sie sind alle von zahllosen Bissen durchzogen, die nur hier und da parallel verlaufen. Kleine St"uckchen und Staubteilchen der anscheinend gleichen Mineralien bilden die Grundmasse, in welchen die gr"o"seren Tr"ummer eingestreut liegen. I. p. L. treten bis in die feinsten Teilchen Farbenerscheinungen hervor, so dass auch in den D"unnschliffen die Abwesenheit einer glasartigen Bindemasse bestimmt beobachtet werden kann. Bemerkenswert sind zahlreiche kleinste, runde, wasserhelle K"ornchen, welche der Grundmasse beigemengt sind. Meteoreisen- und Schwefeleisen-Kn"ollchen teilen etwa die Gr"o"se der Mineralsplitterchen, machen jedoch ihren Umrissen nach nicht den Eindruck der Zertr"ummerung, wie letztere und liegen ziemlich gleichm"a"sig in der Masse zerstreut. Wir sehen also, dass der Meteorstein von Mauerkirchen seiner Struktur nach sich nicht wesentlich von anderen chondritischen Meteorsteinen unterscheidet.
\clearpage
\subsection{\swabfamily{Der Meteorstein von Eichst"adt}}
\begin{figure}[h]
\centering
\includegraphics[keepaspectratio, scale=2]{Fig-2.png}
\caption{\swabfamily{Figur 2}}
\end{figure}
\paragraph{}
"Uber den Fall dieses Steins wird berichtet, dass ein Arbeiter an einer Ziegelh"utte im sog. Wittmes, einer waldigen Gegend, etwa 5 Kil. westw"arts von Eichst"adt am 19. Feb. 1785 Nachmittags zwischen 12 und 1 Uhr nach einem donner"ahnlichen Get"ose einen gro"sen schwarzen Stein auf den mit Schnee bedeckten Erdboden, auf dem Ziegelsteine umher lagen, fallen sah. Als er zur Stelle lief, fand er den Stein, welcher einen Ziegelstein zertr"ummert hatte, eine Hand tief im Boden und so hei"s, dass er ihn erst mit Schnee abk"uhlen musste, um ihn an sich nehmen zu k"onnen. Der Stein hatte etwa ein Fu"s im Durchmesser und wog beil"aufig 3 Kilogramm. Schafh"autl (Gelehrt. Anzeige d. Ac. d. Wiss. in M"unchen 1847 S. 559.) beschreibt denselben wie folgt: "`Seine Struktur ist ziemlich grobk"ornig, die K"orner sind rundlicher, als dies bei allen "ubrigen Aerolithen der Fall ist; ja es finden sich sogar vollkommen elliptische, wie abgeschliffen aussehende K"ornchen von graulicher Farbe und dichtem ziemlich mattem ebenem Bruche darin, ohne bemerkbares kristallinisches Gef"uge. Neben diesen liegen gr"unliche olivinartige K"orner von glasig muscheligem Bruche. Schwefeleisen, Nickeleisen und Magneteisen sind zwischen diesen K"ornern eingesprengt, so dass er unter allen Meteorsteinen unserer Sammlung (M"unchner Staats-S.) am st"arksten auf die Magnetnadel wirkt."'

Das spez. Gewicht\footnote{\swabfamily{Vergl. Moll's Annal, d. Berg- u. H"uttenk. Bd. III. S. 251.}} wird angegeben:

von Schreibers zu 3,700

von Rumler zu 3,599

Klaproth hat diesen Stein analysiert und gibt (Gilberts Ann. XIII. 338) als seine Bestandteile an:
\begin{center}
    \begin{tabular}{ |l|r| } 
    \hline
    Gediegen Eisen & 19,00\\\hline
    Nickelmetall & 1,50\\\hline
    Braunes Eisenoxyd & 16,50\\\hline
    Bittersalzerde & 21,50\\\hline
    Kieselerde & 37,00\\\hline
    Verlust (mit Schwefel) & 4,50\\
    \hline
    \end{tabular}
\end{center}
\paragraph{}
Das in der M"unchener Staatssammlang verwahrte St"uck zeigt eine schwarze mattgl"anzende, runzelige Rinde und eine wei"slich graue, grobk"ornig chondritische, durch zahlreiche Rostflecken hier und da gelblich get"upfelte, leicht zerreibliche Hauptmasse, aus welcher sich die oft sehr gro"sen Chondren leicht heraus l"osen lassen. Es finden sich solche bis "uber 3 mm. im Durchmesser gro"s, sie sind sehr hart, auf der Oberfl"ache matt, erdbeerenartig h"ockerig und grubig in einer Weise, dass die angeschlossenen Mineralsplitterchen der Hauptmasse wie an die Oberfl"ache gekittet erscheinen. An vielen Stellen der Oberfl"ache bemerkt man zudem kleinen spiegelnde Streifchen, wodurch dieselben gleichsam facettiert erscheinen. Auch kommen damit fest verwachsene Meteoreisenteilchen vor, welche zuweilen selbst in die Oberfl"ache versenkt sind. Niemals zeigt sich eine Gl"attung der Oberfl"ache, wie sie Vorkommen m"usste, wenn die K"ugelchen durch Reibung und Abrollung entstanden w"aren. Vielmehr gleichen sie der "au"seren Beschaffenheit nach den in den Schlacken vorkommenden Roheisensteink"ugelchen. Zerschl"agt man sie, so zeigen sie auf der flachmuscheligen Bruchfl"ache, einen matten Glasglanz, schw"arzlichgraue Farbe und bei weit Zertr"ummerung unter dem Mikroskop erweisen sie sich nicht als eine homogene, sondern zusammengesetzte Masse. Man kann deutlich einen glashellen mit zahl- i reichen Bl"aschen erf"ullten, i. p. L. ungemein buntfarbigen Bestandteil neben einer nur durchscheinend tr"uben, wie aus kleinsten Staubteilchen zusammengesetzten, aber i. p. L. doch deutlich farbigen, zuweilen feinstreifigen Hauptmasse und einzelnen durchscheinenden intensiv gelbbraunen, i. p. L. unver"andert gef"arbten Streifchen unterscheiden. In D"unnschliffen sieht man ihre Struktur noch viel deutlicher, obwohl sie hier in einer an sich sehr dunkelgef"arbten Hauptmasse liegen und schwierig gut durchsichtig zu erhalten sind. Indem n"amlich ziemlich viel Meteoreisen als Gemengteil auftritt, dass gro"senteils bereits etwas zersetzt und mit einem H"ofchen von gelbbrauner Farbe umgeben ist, leidet auch die Klarheit derjenigen Mineralteilchen, welche sonst durch ihre Durchsichtigkeit sich auszeichnen. Die gelbe Farbe r"uhrt von Eisenoxydhydrat her, welches durch die Einwirkung der feuchten Luft unserer Atmosph"are auf das Meteoreisen erst nachtr"aglich w"ahrend der Zeit sich gebildet hat, in welcher der Stein in der Erde oder in unseren Sammlungen gelegen hat. Dieses Eisenoxydhydrat dringt in die feinsten Risschen und Spr"unge oder Zwischenr"aume ein, kann aber leicht durch S"auren entfernt werden. Neben dem Meteoreisen beteiligen sich unregelm"a"sig eingesprengte, selten von parallelen Linien eingeschlossene Mineralsplitterchen an dem Haufwerk, aus dem der Meteorstein besteht. Bald sind es wasserhelle, wenig rissige Tr"ummerchen, bald solche, welche durch ein einfaches System parallelen Linien gestreift oder von unten schiefen Winkeln sich schneidenden Rissen zerkl"uftet sind, etwa wie es bei dem Augit vorzukommen pflegt, oder aber durch eine dem Zellnetz gewisser Moosbl"attchen "ahnliche, merkw"urdig langgezogene und quergegliederte Maschenstruktur (d) sich auszeichnen. Zuweilen sto"sen in einem Tr"ummerteil mehrere Systeme solcher paralleler Streifchen zusammen. Zwischen diesen gr"o"seren Fragmenten liegen kleinere ganz von derselben Beschaffenheit, wie die gr"o"seren angeh"auft. I. p. L. erscheinen alle Teilchen, welche nur "uberhaupt durchsichtig sind, in bunten Farben, welche selbst innerhalb der einzelnen Splitter aggregatartig verteilt sind und selten streifig oder bandartig parallel verlaufen. Endlich sind als ungemein h"aufige Bestandteile die kugeligen Einschl"usse zu nennen, die schon erw"ahnt worden sind. Aus den mannichfachen Formen, welche dieselben besitzen, heben wir nur einige der am h"aufigsten vorkommenden hervor. Ziemlich zahlreich sind die Chondren mit exzentrisch strahlig faserigem Gef"uge (a), welches in der Regel von einer nahe am Rande liegenden mehr k"ornigen Partie ausgeht und in einem vielfach abgesetzten, gleichfalls maschenartigen und quergegliederten Strahlenb"uschel ausl"auft. Diese Struktur stimmt so sehr mit jener schon geschilderten "uberein, welchen wir auf "andern regelm"a"sig umgrenzten Splitterchen begegnen, dass wir letztere wohl als Abk"ommlinge zerbrochener gr"o"serer Chondren ansehen m"ussen. Andere der letzteren sind von verschiedenen Systemen sich unter spitzen und stumpfen Winkeln schneidender dunkler Streifchen beherrscht (b), eine Struktur, die sich als der Anfang einer kristallinischen periodenweis gest"orten Ausbildung betrachten l"asst. In noch anderen Chondren kommt eine staubartig tr"ube, schwach durchscheinende Substanz vor, in welcher h"aufig sehr zahlreiche dicht gedr"angte, hellere, gruppenweis nach verschiedenen Richtungen verlaufende Streifchen (c) sich bemerkbar machen. Endlich treten nicht selten K"ugelchen auf, welche aus gr"o"seren, helleren, durch dunkle Zwischenstreifchen voneinander getrennten K"ornchen (e) gleichsam zusammengebacken erscheinen. Aus alle dem geht zur Gen"uge hervor, dass wir in dem Stein von Eichst"adt einem Chondriten der ausgezeichnetsten Art vor uns haben. Derselbe kann geradezu als Typus dieser Art der Struktur, welche bei den Meteorsteinen als der vorherrschende bekannt ist, gelten.

Was seine Zusammensetzung anbelangt, so hat die Analyse (Ass. A. Schwager) ergeben, dass der Stein besteht aus:

22,98 Meteoreisen,

3,82 Schwefeleisen,

32,44 in Salzs"aure zersetzbaren,

40,76 in Salzs"aure nicht zersetzbaren Mineralien.
\paragraph{}
Die Zusammensetzung ist im Ganzen A, dann

B in den durch CI H zersetzbaren Silicaten

C in dem durch CI H nicht zersetzbaren Bestandteil:
\begin{center}
\begin{tabular}{ |p{35mm}|p{15mm}|p{15mm}|p{15mm}| }
    \hline
    & A & B & C\\
    \hline\hline
    Kieselerde & 33,31 & 34,45 & 55,53\\\hline
    Tonerde & 2,31 & 0,86 & 5,13\\\hline
    Eisenoxydul & 15,34 & 24,52 & 16,66\\\hline
    Eisen (mit Phosphor) & 24,64 & -,- & -,-\\\hline
    Nickel & 0,94 & -,- & -,-\\\hline
    Kalkerde & 0,74 & 0,68 & 1,13\\\hline
    Schwefel & 1,42 & -,- & -,-\\\hline
    Chromoxyd & 0,15 & -,- & 0,73\\\hline
    Bittererde & 18,86 & 37,31 & 19,34\\\hline
    Kali & 0,40 & 0,68 & 0,56\\\hline
    Natron & 1,04 & 1,31 & 1,62\\\hline
    & 99,15 & 99,81 & 100,70\\
    \hline
\end{tabular}
\end{center}
\paragraph{}
Der Gehalt der durch Salzs"aure zersetzbaren Gemengteile an Alkalien weist au"ser Olivin noch auf einen Feldspat hin. Wir haben aber darin:
\begin{center}
    \begin{tabular}{ |l|r| } 
    \hline
    SiO\textsubscript{2} & 34,45 mit 18,37 O\\\hline
    Al\textsubscript{2}O\textsubscript{3} & 0,86 mit 0,40\\\hline
    FeO & 24,52 mit 5,45\\\hline
    MgO & 37,31 mit 14,90\\\hline
    CaO & 0,68 mit 0,19\\\hline
    Ka\textsubscript{2}O & 0,68 mit 0,11\\\hline
    Na\textsubscript{2}O & 1,31 mit 0,34\\
    \hline
    \end{tabular}
\end{center}
\paragraph{}
Daraus er sieht man, dass, wenn wir ein Singulosilikat ausscheiden, die vorhandene Sauerstoffmenge noch nicht einmal vollst"andig ausreicht, den Bedarf ganz zu decken, dass mithin die Analyse uns keinen Aufschluss "uber die Natur des etwa noch au"ser Olivin vorhandenen Silikats weitergibt.

In dem von S"auren nicht zersetzbaren Best endlich stellen sich die Verh"altnisse folgender Maassen:
\begin{center}
\begin{tabular}{ |l|c|l| }
    \hline
    Kieselerde & 55,53 & mit 29,62 O = 22,6 + 7\\\hline
    Eisenoxydul & 16,66 & mit 3,70 O = 3,58 + 0,12\\\hline
    Bittererde & 19,34 & mit 7,73 O\\\hline
    Chromoxyd & 0,73 & mit 0,23 O\\\hline
    Tonerde & 5,13 & mit 2,39 O = 2,33 + 0.06\\\hline
    Kalkerde & 1,13 & mit 0,32 O\\\hline
    Kali & 0,56 & mit 0,10 O\\\hline
    Natron & 1,62 & mit 0,42 O\\
    \hline
\end{tabular}
\end{center}
\paragraph{}
Daraus berechnet sich ein Bisilikat, Chromeisen (von der Zusammensetzung des von L'Aigle) und ein Andesin-artiger Feldspat ungef"ahr in dem Verh"altnis wie 79:1:21.

Im Ganzen besteht also der Eichst"adter Meteorstein ungef"ahr aus:
\begin{center}
    \begin{tabular}{ |l|r| } 
    \hline
    Meteoreisen & 22,98\\\hline
    Schwefeleisen & 3,82\\\hline
    Chromeisen & 0,40\\\hline
    Olivin & 31,00\\\hline
    Mineral der Augitgruppe & 31,90\\\hline
    Andesin-artiger Feldspat & 8,46\\\hline
    Feldspatartiges Mineral & 1,54\\
    \hline
    \end{tabular}
\end{center}
\paragraph{}
Das h"aufige Vorkommen und die relative Gr"o"se der Chondren luden zu einer besonderen Analyse dieser K"ugelchen ein. Um sicher zu sein, mit einem von anhaftenden kleinsten Mineralsplitterchen freien Material zu verarbeiten, wurden die Chondren so lange auf einer mattgeschliffenen Glasplatte hin- und hergerieben, bis ihre Oberfl"ache v"ollig glatt und gl"anzend geworden war. Leider war die so mir zur Verf"ugung stehende Menge eine nur sehr geringe (0,12 Gr.) und es kann daher an die Analyse der Anspruch gr"osser Genauigkeit nicht gemacht werden. Durch Vorversuche war bereits festgestellt worden, dass auch die Substanz der Chondren sich teilt in eine von Salzs"aure zersetzbare und in eine unzersetzbare Masse. Die erstere enth"alt noch Schwefeleisen, welches, wie die Untersuchung an D"unnschliffen lehrt, in kleinen K"ornchen fest mit den K"ugelchen verwachsen und in dieselbe gleichsam eingesenkt vorkommt.

Ich fand die Zusammensetzung:

Schwefeleisen 1,53

I. In Salzs"aure zersetzbar 53,05

II. In Salzs"aure unzersetzbar 45,42

Als Zusammensetzung der Silikate I und II ergab sich ferner
\begin{center}
\begin{tabular}{ |l|r|r| }
    \hline
    & I & II\\
    \hline\hline
    Kiesels"aure & 26,26 mit 14,22 O & 53,21 mit 28,38 O\\\hline
    Eisenoxydul & 30,09 mit 6,67 O & 14,86 mit 3,30 O\\\hline
    Bittererde & 31,53 mit 12,60 O & 26,42 mit 10,56 O\\\hline
    Tonerde & 2,70 mit 1,26 O & -,-\\\hline
    Kalkerde & 1,00 mit 0,29 O & 3,67 mit 1,05 O\\\hline
    Alkalien & 8,00 mit 1,70 O & -,-\\\hline
    & 99,98 & 98,16\\
    \hline
\end{tabular}
\end{center}
\paragraph{}
Es ist zun"achst hervorzuheben, dass, wie auch schon von anderer Seite bemerkt wurde, die Zusammensetzung der Chondren nahezn die n"amliche ist, wie die der ganzen Masse und sich durch die Behandlung mit S"auren in zwei "ahnliche Teile scheiden l"asst.

Der in Salzs"aure zerlegbare Teil, abgesehen von Resten eines Gehaltes an Meteoreisen und Schwefeleisen, schlie"st sich am engsten an Olivin an. Aber es mangelt auch hier, wie in zahlreichen F"allen bei analysierten Chondriten an Kiesels"aure. Ich m"ochte vermuten, dass dies hier von einem "Uberschass an Eisenoxydul herr"uhrt, das, anstatt von zersetzem Olivin, von fein beigemengtem Meteoreisen abstammt. Tonerde, Kalkerde und Alkalien weisen auf eine Beimengung feldspatartiger Teilchen, wie bei der Hauptmasse der Chondrite hin. Doch bietet die Interpretation dieses Teils immerhin Schwierigkeiten, die bis jetzt noch nicht beseitigt sind.

Der in Salzs"aure unzersetzte Rest f"ugt sich viel besser in das Maa"s eines Bisilikates; wenn es auch hierbei um etwas weniges an Kiesels"aure fehlt, so kann dies wohl bei der geringen, zur Analyse verwendete Menge als Folge des Verlustes bei der Analyse selbst angesehen werden.
\clearpage
\subsection{\swabfamily{Der Meteorstein von Massing}}
\begin{figure}[h]
\centering
\includegraphics[keepaspectratio, scale=2]{Fig-3.png}
\caption{\swabfamily{Figur 3}}
\end{figure}
\paragraph{}
"Uber die n"aheren Umst"ande des Falls dieses Meteoriten teilt Prof. Imhof (Kurpfalzbaier. Wochenblatt 1804 St. 3 u. f.)\footnote{\swabfamily{Gilberts Ann. d. Phys. XVIII. 330.}} mit:

"`Nach den gerichtlichen Anzeigen an die kurf. Landesdirektion h"orten mehrere der Landleute, die um den Marktflecken M"assing(Massing) Ldger. Eggenfelden wohnen, am 13. Dez. 1803 Vormittag zwischen 10 und 11 Uhr neun bis zehn Mal einen Knall, wie Kanonensch"usse. Ein Bauer zu St Nicolas, der bei diesem Get"ose aus seinem Hofe trat und in die H"ohe sah, erblickte etwas, das sehr hoch unter best"andigem Sausen in der Luft daherkam und endlich auf das Dach seiner Wagenh"utte fiel, etliche Schindeln zerschlug und in dieselbe eindrang. Er ging auf die H"utte zu und fand in ihr einen Stein, der nach Pulver roch, ganz schwarz und so hei"s war, als ein Stein zu sein pflegt, der auf einem Ofen lag. Er sagte, er habe das vermeintliche Schie"sen von Alten-Oetting (d. h. von Osten) hergeh"ort, der Stein sei aber "uber Heiligenstadt (d. h. von Westen) gekommen. Der Stein wog "uber 1 1/2 Kilogramm, hat ein spec. Gew. von 3,365, eine dunkelschwarze, etwas dickere Binde, als der Mauerkirchner und ist im Bruche viel grobk"orniger. Als Gemengteile enth"alt er nach Imhof:
\begin{enumerate}
    \item regulinisches Eisen, das wie d"unne Eisenfeile sichtbar eingewachsen und gl"anzend erscheint,
    \item Schwefelkies, der unter der Loupe kristallisiert erscheint und gerieben ein schwarzes Pulver gibt,
    \item gr"o"sere und kleinere plattgedr"uckte, eckige Massen, einige von dunkelbrauner, andere von schwarzer Farbe, die sich durch ein schimmerndes Ansehen und gr"o"sere H"arte von jenen unterscheiden,
    \item hier und da bemerkt man noch kubische K"ornchen und Bl"attchen von gelblicher Farbe durchscheinend und mit Glasglanz, wie Quarz aussehend, die jedoch nicht die H"arte des Quarzes haben,
    \item auch sind wei"se K"orner von unregelm"a"siger Form eingesprengt, von denen einige "uber 3 Linien dick sind,
    \item unter dem Mikroskop sieht man auch ein wei"sgraues, ins Gelbe spielendes Metall, das die Magneten folgsam und wahrscheinlich metallisches Nickel ist.
\end{enumerate}
\paragraph{}
Nach der Analyse dieses Forschers besteht der Stein in 100 Teilen, aus:
\begin{center}
    \begin{tabular}{ |l|r| } 
    \hline
    regulinischem Eisen & 1,80\\\hline
    regulinischem Nickel & 1,35\\\hline
    braunem Eisenoxyd & 32,54\\\hline
    Magnesia & 23,25\\\hline
    Kieselerde & 31,00\\\hline
    Verlust an Schwefel u. Nickel & 10,06\\
    \hline
    \end{tabular}
\end{center}
\paragraph{}
Ammler gibt (O. B"uchner a. a. O. S. 17) das spec. Gewicht zu 3,3636 an.

Prof. v. Schafh"autl beschreibt (a. a. O. S. 558) diesen Stein "`vom Aussehen des Bimssteinporphyrs, in dem die einzelnen Silikate in so gro"sen Aggregaten auftreten, dass man sie leicht mit freiem Auge unterscheiden k"onne. Das Gestein bestehe aus milchwei"sen K"ornern von bl"attrig strahliger Struktur, aus olivinartigen k"ornigen Massen von Erbsengr"osse, und aus z. Th. matten basaltartigen Fragmenten, die jedoch "ofter auf den augitartigen Bl"atterdurchg"angen auch glasgl"anzend erscheinen. Sparsam finden sich rissiges irisirendes Schwefeleisen eingesprengt und kleine K"ornchen von Chromeisen. Der Stein wirkt nicht auf die Magnetnadel. Vor dem L"othrohr sei er ziemlich leicht schmelzbar und ebenso mit einer glasig gl"anzenden Rinde "uberzogen, wie der Aerolith von Stannern."'

Nach meinen Beobachtungen besitzt der Stein eine braunschwarze glasgl"anzende Rinde und besteht in seiner graulich wei"sen, ziemlich leicht zerreiblichen Masse aus:
\begin{enumerate}
    \item einem gelblich gr"unen bis hellgr"unen, etwas parallelrissigen, in rundlich und unregelm"a"sigen K"ornchen (wie in Krystallform) vorkommenden, ziemlich gro"sen, 1---1 1/2 mm. im Durchmesser breiten, nur sporadisch erscheinenden Gemengteil, der durch S"auren leicht zersetzt wird and als Olivin gelten muss.
    \item aus einem wei"sen, oft glasartig durchsichtigen oder staubig tr"uben, nur durchscheinenden, stark rissigen, selten parallelstreifigen, zuweilen mit deutlichen Spaltfl"achen versehenen Mineral, das i. p. L. lebhaft ein- oder fleckig vielfarbig erscheint und von S"auren gleichfalls zersetzt wird, einem Feldspat entsprechend,
    \item aus einem weingelben bis graugr"unlichen, oder blass r"otlich braunem, glasartig mattgl"anzendem Mineral, 1,5 bis 2 mm. gro"s, i. p. L. lebhaft gef"arbt, aber nicht dichroitisch, etwas l"angsfaserig (aber undeutlich, gestreift) und mit zahlreichen kleinen Bl"aschen erf"ullt. Dieser Bestandteil wird von S"auren nicht zersetzt und geh"ort der Augitgruppe an.
    \item aus schwarzem, starkgl"anzendem, in S"auren nicht zersetzbarem, in der Phosphorsalzperle ein pr"achtig gr"unes Glas lieferndem Chromeisen,
    \item endlich aus z. Th. von den Magneten gezogenen, dunklen, metallischen K"ornchen, die meist dem Schwefeleisen, im Minimum dem Meteoreisen zuzuteilen sind.
\end{enumerate}
\paragraph{}
Diese s"amtlichen gr"o"seren, vorwaltend rundlich unregelm"a"sig eckigen, (nicht l"anglich spie"sf"ormigen) Teilchen liegen in einer feinstaubartig k"ornigen, grauen Grundmasse, welche aus denselben nur kleinen und kleinsten Splitterchen, wie sie eben angef"uhrt wurden, zu bestehen scheint. Auch hier ist eine glasartige Bindemasse nicht zu erkennen.

Die Analyse A. Schwager's ergab:
\begin{center}
\begin{tabular}{ |p{20mm}|p{22mm}|p{32mm}|p{32mm}| }
    \hline
    Stoffe: & Bauschanalyse & 21,33\% in Salzs"aure zersetzbar & 78,67\% in Salzs"aure nicht zersetzbar\\
    \hline\hline
    Kiesels"aure & 52,115 & 39,59 & 56,71\\\hline
    Tonerde & 8,204 & 29,51 & 2,54\\\hline
    Eisenoxydul & 19,138 & 2,83 & 23,46\\\hline
    Eisen & 0,523 & 2,49 & -,-\\\hline
    Nickel & Spuren & Spuren & -,-\\\hline
    Chromoxyd & 0,979 & -,- & 1,24\\\hline
    Kalkerde & 5,786 & 15,70 & 3,15\\\hline
    Bittererde & 8,485 & 3,33 & 10,74\\\hline
    Kali & 1,188 & 4,78 & 0,85\\\hline
    Natron & 1,928 & 4,78 & 1,17\\\hline
    Schwefel & 0,374 & 1,78 & -,-\\\hline
    & 99,720 & 100,06 & 99,86\\
    \hline
\end{tabular}
\end{center}
\paragraph{}
Der durch Salzs"aure zersetzbare Anteil zu 21,33\% l"asst sich nach dem Gehalt an Schwefel, Bittererde und Tonerde berechnet ansehen als ungef"ahr zusammengesetzt aus:

10\% Olivin (Hyalosiderit)

86\% Anorthit mit gro"sem Alkaligehalte

4\% Schwefeleisen und Meteoreisen

In abgerundeten Zahlen best"ande der Feldspat A und der Olivin B aus:
\begin{center}
    \begin{tabular}{ |l|c|r| } 
    \hline
    & A & B\\
    \hline\hline
    Kieselerde & 42 & 37,25\\\hline
    Tonerde & 34 & -,-\\\hline
    Eisenoxydul & -,- & 29,75\\\hline
    Kalkerde & 18 & -,-\\\hline
    Bittererde & -,- & 33,00\\\hline
    Alkalien & 6 & -,-\\
    \hline
    \end{tabular}
\end{center}
\paragraph{}
Was den Rest des durch S"auren nicht zersetzbaren Anteils zu 78,67\% anbelangt, so muss man hierin noch einen kleinen Anteil Feldspat neben Chromeisen und Augit annehmen, etwa:

2,5\% Chromeisen

13,5\% feldspatartige Substanz (A)

84,0\% Augitmineral (B).

Beiden letzteren (A und B) w"urde eine Zusammsetzung zu kommen, wie folgt:
\begin{center}
    \begin{tabular}{ |l|c|r| } 
    \hline
    & A & B\\
    \hline\hline
    Kieselerde & 66 & 86\\\hline
    Tonerde & 19 & -,-\\\hline
    Eisenoxydul & -,- & 36\\\hline
    Kalkerde & -,- & 4\\\hline
    Bittererde & -,- & 14\\\hline
    Alkalien & 15 & -,-\\
    \hline
    \end{tabular}
\end{center}
\paragraph{}
Ber"ucksichtig man ferner das Verh"altnis des in Salzs"aure zersetzbaren und nicht zersetzbaren Anteils im Verh"altnis von 21,33 zu 78,67 so k"onnen wir nach der oben angef"uhrten Deutung den Meteorstein ungef"ahr zusammengesetzt uns vorstellen, aus:
\begin{center}
    \begin{tabular}{ |l|r| } 
    \hline
    Olivin & 2,00\\\hline
    Schwefeleisen & 0,75\\\hline
    Meteoreisen & 0,25\\\hline
    Chromeisen & 2,00\\\hline
    Anorthit & 18,00\\\hline
    2te feldspatige S. & 11,00\\\hline
    Augitmineral & 66,00\\
    \hline
    \end{tabular}
\end{center}
\paragraph{}
Es wurde bisher der Stein von Massing dem von Luotolaks an die Seite gestellt und Rammelsberg (d. chem. N. d Meteor. S. 136) z"ahlt ihn zu den Howarditen (Olivin- Augit- Anorthitmeteorstein).

Ich glaube, dass er mehr Analogien mit der Gruppe der Eukrite besitzt, da der Olivin sehr sp"arlich vorhanden ist.

Wir wollen nun zun"achst sehen, wie mit dieser Auffassung die optische Untersuchung der D"unnschliffe passt, wie das Bild Figur III. einen solchen darstellt. Man bemerkt zun"achst gro"se, unregelm"a"sig eckige --- nicht wie bei den typischen Chondriten abgerundete K"ornchen und eine ziemlich gleichm"a"sige, feine Hauptmasse mit einzelnen im auffallenden Lichte metallisch gl"anzenden, stahlgrauen und messinggelben Putzen. Sehen wir zun"achst ab von den gro"sen, unregelm"a"sigen, gleichsam abnormen Beimengungen, so treten uns in der Grundmasse vor Allem gr"o"sere Gruppen eines gr"unlich gelben, dann eines schwach weingelben, eines blassr"otlich braunen und wei"sen Minerals entgegen, welche wir als die Hauptgemengteile anzusehen berechtigt sind. Die wenigen gr"unlich gelben Teilchen (a) sind unregelm"a"sig rissig, gl"anzen i. p L. mit den lebhaftesten Aggregatfarben und werden durch S"auren zersetzt --- Olivin. Nach dem ersten Anschein m"ochte man auch die weit zahlreicheren Putzen des schwach weingelben, jedoch mehr parallel rissigen Minerals (b) f"ur Olivin halten. Allein in den mit kochenden S"auren anhaltend behandelten Pulvern erscheinen sie unzersetzt und k"onnen mithin nicht zum Olivin geh"oren. Auch bemerkt man in den D"unnschliffen eine Art Parallelstreifung, wie sie dem Olivin nicht zukommt, aber an Enstatit erinnert. Daneben liegen zahlreiche, oft nur durchscheinende, doch auch gut durchsichtige, an den R"andern r"otlich braun gef"arbte, nicht dichroitische Teilchen (c), die allem Verhalten nach Augit zu sein scheinen. Ich glaube demnach annehmen zu sollen, dass zwei Mineralien der Augitgruppe hier vertreten sind, n"amlich Enstatit und Augit. Die glashellen oder staubartig wei"sen Teilchen (d) sind teils durch S"auren zersetzbar, teils erscheinen sie aber auch noch in dem durch S"auren behandelten Pulver mehr oder weniger unber"uhrt. Dies deutet gleichfalls auf die Anwesenheit von zweierlei Feldspaten, von welchen der eine wohl in dem D"unnschliffe Spuren von Parallelstreifen i. p. L. erkennen l"asst. Dass --- entgegen der Angabe Schafh"autl's --- wirklich Meteoreisen, wenn auch sp"arlich beigemengt ist (e), habe ich in die D"unnschliffe, in dem zwei deutliche K"ornchen Vorkommen, dadurch festgestellt, dass ich auf die stahlgrau gl"anzenden Fl"achen Kupfervitrioll"osung brachte, wobei sich sofort die Ausscheidung metallischen Kupfers beobachten l"asst.

Schwieriger zu erkl"aren ist die Natur der gro"sen Einsprenglinge, zu denen im D"unnschliff die Parthien x und y geh"oren. Der gr"o"sere x ist parallelstreifig und querrissig, dunkelolivengr"un bis r"otlich braun, wenig durchsichtig, i. p. L. farbig. Er m"ochte als ein etwas ver"anderten Augitfragment zu betrachten sein. Das zweite Fragment y ist gelblich, sehr feink"ornig, fast dicht, schwach duschscheinend und mit feinsten schwarzen Staubteilchen durchsprengt. Es gleicht am ehesten die Bruchst"ucke eines Chondrit-k"ornchens. Dergleichen Einschl"usse m"ogen noch von sehr verschiedener Beschaffenheit in der Grundmasse eingebettet sein. Obwohl eine deutliche Chondritenstruktur nicht vorhanden ist, verhalten sich doch diese Einschl"usse und die als Grundmasse auftretenden Mineralien so "ahnlich den Bestandteilen der Chondrite, dass auch dem Meteorstein von Massing eine ganz analoge Entstehung, wie die der letzteren, zugesprochen werden muss.

Der namhafte Gehalt dieses Steins an Chromeisen gab Veranlassung, dessen Zusammensetzung n"aher zu erforschen, da, soviel ich wei"s, dass Chromeisen der Meteorsteine isoliert bis jetzt noch nicht einer Analyse unterworfen worden ist. Es schien sich hierzu das Chromeisen im Meteorstein von L'Aigle, indem es in gr"o"seren K"ornchen vorkommt, gut zu eignen. Dasselbe l"asst sich daraus sehr leicht undvollst"andig rein heraussuchen. Die Analyse dieses Chromeisens ergab:
\begin{center}
    \begin{tabular}{ |l|r| } 
    \hline
    Chromoxyd & 52,13\\\hline
    Eisenoxydul & 37,68\\\hline
    Tonerde & 10,25\\\hline
    & 100,06\\
    \hline
    \end{tabular}
\end{center}
also nahezu die Zusammensetzung des Chromeisens von Baltimore (Maryland), ein Beweis mehr f"ur die Gleichartigkeit der Bildung kosmischer und tellurischer Mineralien.
\clearpage
\subsection{\swabfamily{Der Meteorstein von Sch"onenberg}}
\begin{figure}[h]
\centering
\includegraphics[keepaspectratio, scale=2]{Fig-4.png}
\caption{\swabfamily{Figur 4}}
\end{figure}
\paragraph{}
Einen sehr ausf"uhrlichen Bericht "uber den Fall dieses Meteorsteins gibt Prof. v. Schafh"autl (a. a. O. S. 564). Daraus ist zu entnehmen, dass zur Zeit des Falls am 25. Dez. 1846 nach 2 Uhr Nachmittags auf einen Umkreis von etwa 60 Kilometer ein Donner-"ahnliches Ger"ausch geh"ort wurde. In der n"achsten N"ahe des Ortes, wo der Stein niederfiel, verglich man das Ger"auche mit fernem Kanonendonner, der nach mehr als 20maliger Wiederholung gleichsam in ein Trommeln "uberging und nach etwa 3 Minuten mit einem fernem Trompetenkl"angen "ahnlichen Sausen endete. Im Dorfe Sch"onenberg traten mehrere Leute bei diesem Ger"ausche aus der Kirche, in der gerade Nachmittagsgottesdienst stattfand, wieder heraus und sahen nun eine fest faustgro"se Kugel von N.-O.zuletzt nach S.-O. sich wendend in ein Krautfeld in der N"ahe des Dorfes niederfallen. Zahlreiche Bewohner des Dorfs eilten zur Stelle nnd es fand sich etwa 2 Fu"s tief in dem etwas gefrorenen Lehmboden eingedrungen ein schwarzer Stein. Man glaubte noch Schwefelgeruch zu sp"uren. Dabei zeigte der vordem bedeckte Himmel pl"otzlich zuerst in der Richtung des Meteorfalls einen lichten Streif und hellte sich dann g"anzlich auf.

Die Form des ringsum von einer dunkelbraunen rauhen Sinterrinde "uberzogenen Steins beschreibt v. Schafh"autl als eine sehr unregelm"a"sige in den Hauptumrissen vierseitige Pyramide mit einer Zusch"arfung, die in der Richtung des l"angsten Durchmessers der Basis l"auft und sich nach der hintern Seite der Pyramide senkt. Da die Rinde auch in kleinen Einschnitten sich vorfindet, glaubt er annehmen zu sollen, dass der Stein in einem erweichten Zustande auf die Erde kam. Merkw"urdiger Weise ziehen 7 Streifen von Nickeleisen schnurartig "uber den Stein, durchkreuzt von einem 8ten, der eine fast recktwinklige Richtung zu den anderen nimmt. Zwei Seiten sind eben und ohne Eindr"ucke, im "Ubrigen aber ist die Oberfl"ache unregelm"a"sig vertieft, wie das Bruchst"uck eines Steins, der durch eine "au"sere Gewalt zerschlagen ist. Der Stein wog 8 Kilogr. 15 Gr. und ist so weich, dass er sich mit den Fingern zerbr"ockeln l"asst. Er wirkt auf die Magnetnadel und Salzs"aure entwickelt unter Gallertbildung Schwefelwasserstoff. Die Masse besteht aus wei"sen, feink"ornigen Teilchen, welche von S"aure am meisten angegriffen wurden, dann aus honiggelben und gr"unlichen, k"ornigen Aggregaten, auf welche die S"aure weniger Wirkung aus"ubt, ferner aus einzelnen kleinen K"ornchen von Schwefeleisen, silbergl"anzenden, gefranzten Bl"attchen von Nickeleisen, in der Masse zerstreut und zugleich die oben erw"ahnten Schn"ure bildend. Von Augit, Labrador u. dgl. sei Nichts in dem Aerolithen zu entdecken, v. Schafh"autl scheint nicht der Ansicht von Berzelius zuzustimmen, dass der durch Salzs"aure zersetzte Gemengteil Olivin sei. Denn die olivinartigen K"orner seien gerade die unaufl"oslichsten und die wei"sen Mineralteilchen die zersetzbaren nach Art der Zeolithe oder gleich dem gegl"uhten Epidot, Vesuvian u. s. w. Er f"ugt dann noch einen Erkl"arungsversuch der Entstehung der Meteorite als das Resultat einer Verdichtung aus einer Wolken-artigen Masse in der N"ahe unseres Erdkreises hinzu.

Die Schmelzrinde ist nach meiner Wahrnehmung matt schimmernd, schwarz, stellenweis, wo Eisenteilchen in der N"ahe vorhanden waren, ziemlich dick (bis 1/2 mm.) Die lichtgrau wei"se, feink"ornige, sp"arlich schwarz punktierte, stellenweise rostfleckige Hauptmasse besteht, soweit sich dies vorl"aufig erkennen l"asst, aus:
\begin{enumerate}
    \item gr"o"seren, gr"unlich gelben Teilchen, welche durch Salzs"aure zersetzbar, eine viel Eisenoxydul und Bittererde haltige L"osung geben --- also olivinartig,
    \item wei"sen splittrigen Teilchen, gleichfalls durch S"aure zerlegbar,
    \item gr"unlich grauen, mattgl"anzenden, unregelm"a"sigen K"ornchen, welche rissig sind und von S"auren nicht zersetzt werden,
    \item aus verschiedenen Eisenverbindungen, die sich durch den metallischen Glanz bemerkbar machen und vielfach von einem gelben, rostfarbigen Hofe umgeben sind, als Folge der eingetretenen Zersetzung des Meteoreisens. Der Gehalt an diesem wurde durch besondere Versuche festgestellt. Im "Ubrigen ergab die Analyse:
\end{enumerate}
\begin{center}
\begin{tabular}{ |p{19mm}|p{22mm}|p{33mm}|p{33mm}| }
    \hline
    Stoffe: & Bauschanalyse & 55,18\% durch\newline Salzs"aure zersetzbar & 44,82\% durch Salzs"aurenicht zersetzbar\\
    \hline\hline
    Kiesels"aure & 40,13 & 24,47 & 57,85\\\hline
    Tonerde & 5,57 & 9,45 & 6,75\\\hline
    Eisen & 13,77 & 30,56 & -,-\\\hline
    Nickel & 1,47 & 1,48 & 1,44\\\hline
    Schwefel & 1,93 & 3,52 & -,-\\\hline
    Phosphor & 0,36 & 0,33 & 0,27\\\hline
    Chromoxyd & 0,60 & -,- & 1,35\\\hline
    Eisenoxydul & 17,12 & 10,41 & 15,37\\\hline
    Kalkerde & 2,31 & 3,72 & 0,56\\\hline
    Bittererde & 13,81 & 11,55 & 16,63\\\hline
    Kali & 0,73 & 1,33 & Spuren\\\hline
    Natron & 2,20 & 3,18 & 1,02\\\hline
    & 100,00 & 100,00 & 101,24\\
    \hline
\end{tabular}
\end{center}
\paragraph{}
Aus diesen Angaben l"asst sich berechnen, dass der in Salzs"aure zersetzbare Anteil besteht aus:
\begin{center}
    \begin{tabular}{ |l|r| } 
    \hline
    Schwefeleisen & 9,64\\\hline
    Meteoreisen & 26,25\\\hline
    Olivin & 34,78\\\hline
    Feldspat-Mineral & 29,33\\
    \hline
    \end{tabular}
\end{center}
\paragraph{}
F"ur den Olivinbestandteil ist in Rechnung zu setzen:
\begin{center}
    \begin{tabular}{ |l|c|r| } 
    \hline
    SiO\textsubscript{2} & 12,82 & 37\\\hline
    FeO & 10,41 & 30\\\hline
    MgO & 11,55 & 33\\\hline
    & 34,78 & 100\\
    \hline
    \end{tabular}
\end{center}
entsprechend der Zusammensetzung des Hyalosiderits.

Wir finden dann weiter f"ur den etwas zersetzten Feldspatartigen Bestandteil:
\begin{center}
    \begin{tabular}{ |l|c|c|l| } 
    \hline
    SiO\textsubscript{2} & 11,65 & 39,71 & Sauerstoff 21,3\\\hline
    Al\textsubscript{2}O\textsubscript{3} & 9,45 & 32,21 & Sauerstoff 15,0\\\hline
    CaO & 3,72 & 12,70 & Sauerstoff 3,6\\\hline
    Ka\textsubscript{2}O & 1,33 & 4,54 & Sauerstoff 0,77\\\hline
    Na\textsubscript{2}O & 3,18 & 10,84 & Sauerstoff 2,8\\\hline
    & 29,33 & 100,00\\
    \hline
    \end{tabular}
\end{center}
\paragraph{}
Das Sauerstoffverh"altnis der Kiesels"aure, der Tonerde und der alkalischen Basen 3:2:1 steht nicht in "Ubereinstimmung mit jenen der eigentlichen Feldspate, sondern entspricht dem der Skapolithgruppe (Mejonit). Die Anwesenheit eines derartigen Minerals w"urde aneh zu dem optischen Verhalten besser passen, als die Annahme eines Anorthits oder Plagioklases "uberhaupt, weil i. p. L. die wei"sen oder glashellen Teilchen keine parallelen Farbenstreifchen erkennen lassen.

In dem von Salzs"aure nicht zersetzten Reste ist der Gehalt an Nickel und Phosphor bemerkenswert. Wir m"ussen dies, da nicht anzunehmen ist, dass dieser Gehalt von einem Rest zuf"allig unzersetzt gebliebenen Meteoreisens herr"uhre, als ein Zeichen der Beimengung von Schreibersit ansehen. Das dazu geh"orige Eisen erscheint nat"urlich in der Analyse unter dem Eisenoxydul. Daraus mag sich auch der "Uberschuss der Summe "uber 100 z. Th. erkl"aren. Obwohl au"serdem noch sicher Tonerde-haltiges Chromeisen vorhanden ist, kommt doch eine so bedeutende Menge von Tonerde neben einem betr"achtlichen Quantum von Natron zum Vorschein, dass in dem Rest weiter auch ein feldspatiger Gemengteil vorausgesetzt werden muss, w"ahrend dessen Hauptbestandteil offenbar ein augitisches Mineral ausmacht. Bringt man f"ur letzteres die Gemengteile eines Bisilikats in Abzug, so bleibt ein Rest, in dem das Sauerstoffverh"altnis zwischen Tonerde und der "ubrig bleibenden Kiesels"aure zwar nahezu wie 3:9 verh"alt, es fehlt aber dann an der erforderlichen Menge der Kalkerde und Alkalien. Es l"asst sich daher dieser von S"auren nicht zerlegte Anteil nur ungef"ahr berechnet als bestehend aus:
\begin{center}
    \begin{tabular}{ |l|r| } 
    \hline
    Schreibersit & 4,5\\\hline
    Chromeisen & 2,5\\\hline
    feldspatiges Mineral & 4,0\\\hline
    augitisches Mineral & 89,0\\
    \hline
    \end{tabular}
\end{center}
\paragraph{}
Im Ganzen best"ande demgem"a"s der Chondrit von Schonenberg aus:
\begin{center}
    \begin{tabular}{ |l|r| } 
    \hline
    Olivin & 19,0\\\hline
    feldspatigem und Skapolithartigem Mineral & 18,5\\\hline
    augitischem Mineral & 40,0\\\hline
    Meteoreisen & 14,5\\\hline
    Schwefeleisen & 5,0\\\hline
    Schreibersit & 2,0\\\hline
    Chromeisen & 1,0\\
    \hline
    \end{tabular}
\end{center}
\paragraph{}
Der D"unnschliff dieses Meteorsteins (Figur IV. der Tafel) lehrt uns die au"sergew"ohnliche Feink"ornigkeit der Gemengteile kennen, welche alle unregelm"a"sig splittrig, wie bei allen Chondriten, sind. Gr"o"sere Mineralst"uckchen sind selten und ebenso vereinzelt die Chondren (o), deren Masse wei"s tr"ube, staubartig feink"ornig, und an den R"andern schwach durchscheinend, aber i. p. L. buntfarbig, seltener exzentrisch faserig sich zeigt. Neben diesen rundlichen K"ornchen kommen auch noch unregelm"a"sig eckige Fragmente von tr"uben, staubartigen und deutlich gestreiften Massen (b) und von jener eigent"umlichen, "au"sert fein parallelstreifigen und quergegliederten, der Zellenmaschen der Moosbl"atter "ahnlichen Struktur (c) vor, die in so vielen Chondriten als charakteristisch wiederkehrt. Das Meteoreisen bildet oft langgezogene, leistenartige H"aufchen (d), scheint aber h"aufig auch wie eine d"unne Rinde sich um die Chondren anzulegen.

Unter den gr"o"seren Mineralsplitterchen kann man die gelblichen, h"ochst unregelm"a"sig rissigen, im Umrisse mehr rundlichen als dem Olivin angeh"orig erkennen; sie zeigen i. p. L. die buntesten Aggregatfarben. Die etwas dunkler, farbigen, "ofters etwas ins R"otliche spielenden Splitter des augitischen Minerals zeichnen sich durch eine mehr parallele Zerkl"uftung nach zwei Richtungen und i. p. L. gleichfalls sehr bunte F"arbung aus, w"ahrend die wei"slichen, feldspatigen Bestandteile vielfach ins Tr"ube "ubergehen und i. p. L. von blauen und gelben Farbent"onen beherrscht werden.

Nach alledem geh"ort der fr"uher chemisch noch nicht unter sucht gewesene Meteorstein von Sch"onenberg der gro"sen Gruppe der Chondriten an und n"ahert sich unter diesen durch den niederen Kiesels"auregehalt sehr dem Stein von Ensisheim, unterscheidet sich aber von diesem, wie von allen den durch Rammeisberg (a. a. O.) zusammengestellten Arten durch den relativ sehr geringen Bittererde-, hohen Tonerde- und Natrongehalt.

Die an der Oberfl"ache des Steins bemerkbaren schnurartigen Streifen scheinen Zerkl"uftungen des Steins zu entsprechen, auf denen, wie auf der Oberfl"ache, eine Schmelzrinde beim Fall durch die Atmosph"are sich gebildet zu haben scheint.
\clearpage
\subsection{\swabfamily{Der Meteorstein von Kr"ahenberg}}
\paragraph{}
bei Zweibr"ucken in der Rheinpfalz.
\begin{figure}[h]
\centering
\includegraphics[keepaspectratio, scale=2]{Fig-5.png}
\caption{\swabfamily{Figur 5}}
\end{figure}
\begin{figure}[h]
\centering
\includegraphics[keepaspectratio, scale=2]{Fig-6.png}
\caption{\swabfamily{Figur 6}}
\end{figure}
\paragraph{}
Zu den erst in j"ungster Zeit gefallenen und am Genauesten untersuchten Meteorsteinen geh"ort der Stein von Kr"ahenberg. "Uber den Fall selbst berichten ausf"uhrlich Dr. G. Neumayer (Sitzungsb. d. Ac. d. Wiss. in Wien math. naturw. Cl. Bd. LX. 1869. S. 229), O. B"uchner (Poggendorf Ann. Bd. 137. S. 176) und Weiss (N. Jahrb. 1869. S. 727 u. Poggendorfs Ann. Bd. 137. S. 617), "uber die Zusammensetzung [Gerhard] vom Rath (Poggendorfs Ann. Bd. 137. S. 328), an einer mikroskopischen Untersuchung der D"unnschliffe fehlte es jedoch bis jetzt. Wir entnehmen den oben angef"uhrten Angaben "uber den Fall des Steins, dass am 5. Mai 1869 Abends 6 1/2 Uhr ein furchtbarer, einem Kanonendonner "ahnlicher, aber weit st"arkerer Knall geh"ort wurde, dem ein Rollen, ein Geknatter, wie von Musketenfeuer herr"uhrend und ein Brausen, "ahnlich dem Ger"ausche, des aus einer Lokomotive ausstr"omenden Dampfes folgte. Mit einem starken Schlag endigte pl"otzlich diese Ger"ausche, welches gegen 2 Minuten angedauert hatte. Man beobachtete an Orten bis auf 60 bis 70 Kilometer Entfernung vom Fallpunkte Kr"ahenberg entweder Ger"ausch oder Lichterscheinungen, welch letztere als intensiv wei"s angegeben werden. Zwei Knaben sahen den Stein zur Erde fallen und etwa 15-20 Minuten nach dem Fall grub man denselben aus der Erde, in die er ein senkrechtes, gegen 0,6 M. tiefes Loch sich gegraben hatte und auf einer Platte des unterliegenden Buntsandsteins liegen geblieben war.* Der Stein f"uhlte sich noch warm, aber nicht heisa an; er wog, nachdem wohl einige Kilogramm abgeschlagen worden waren, immerhin noch 15,75 Kilogramm und besa"s einen Brodlaib "ahnliche, aber etwas einseitig erh"ohte rundliche Form, mit einem gr"o"seren Durchmesser von 0,30 m. und einem kleineren von 0,24 m., die au"ser der Mitte liegende gr"o"ste Dicke oder H"ohe ist 0,18 m.; die Grundfl"ache flach, ziemlich eben, die gew"olbte Fl"ache dagegen h"ochst merkw"urdig mit zahlreichen, vom glatten Scheitel aus, gegen den Rand strahlig verteilten, grubenf"ormigen, oft zu 0,03 m. langen Rinnen ausgestreckten, bis 8 mm. tiefen Furchen bedeckt. Zwischen diesen Gruben erheben sich dann schmale wellige W"ulstchen, so dass die Oberfl"ache gleichsam tief blatternarbig durchfurcht erscheint. Die ganze Oberfl"ache ist mit einer schwarzen, stellenweis schaumigen Schlackenrinde vom 1/2---1 mm. Dicke bedeckt. Fleckenweis ist die Rinde d"unn und br"aunlich statt schwarz gef"arbt, was, wie ich mich am Original "uberzeugte, daher r"uhrt, dass an solchen Stellen schwerer schmelzbare Gemengteile sich vorfinden, die ein intensiveres Schmelzen verhinderten. Wei"s hatte sogleich die Chondritennatur des Steins erkannt und macht auch auf die in der wei"sen Grundmasse liegenden dunkelgrauen, scharf abgegrenzten Fragmente aufmerksam, welche sich durch eingesprengte metallische Teilchen und wei"sliche Splitterchen ebenfalls als Gemenge, wie die grauen Kugeln erweisen. Vom Rath best"atigt dies und f"uhrt weiter an, dass der Kr"ahenberger Stein auf der lichtgrauen Bruchfl"ache zahlreiche, in allen Richtungen ziehende, zuweilen zu einer Masche werke verbundene, feine schwarze Linien bemerken l"asst. Es scheinen ihm Spalten zu sein, welche wenigstens z. Th. beim Eintritt des Meteors in die Erdatmosph"are sich bildeten und mit der schmelzenden Substanz der Rinde erf"ullt wurden. Au"ser diesen Schmelzlinien schw"armen in den Steinen gekr"ummte schmale G"ange anderer Art umher, die aus Nickeleisen bestehen. Es sind gang"ahnliche Parthieen von ansehnlicher Dicke. Ich konnte eine solche "uber 3 Zoll lange, wenig Gekr"ummte 1/3-1/2 mm. dicke Erzader auf einer Bruchfl"ache deutlich beobachten. Au"serdem kommen auch Eisenspiegel, wie im Stein von Pultusk vor, dem auch die Masse sehr "ahnlich, doch weniger feink"ornig ist. Als Gemengteile erkannte vom Rath Nickeleisen, Magnetkies, Chromeisen, Olivin und die charakteristischen Kugeln, welche Gemengteile in einer aus wei"sen und grauen K"ornern gebildeten sph"arolithischen Grundmasse liegen. Den Gehalt an Nickeleisen (aus 84,7 Eisen und 15,3 Nickel) bestimmte er zu 3,5\%, so dass 96,5\% auf die Silikate, Magnetkies and Chromeisen kommen. Von Schmelzrinde freie St"uckchen besitzen das spec. Gew. 3,4975 bei 18$^{\circ}$ C., an Schmelzrinde reiche St"uckchen 3,449 bei 20$^{\circ}$ C., wonach sich die Beobachtung am Pultusker Stein best"atigt, dass die Schmelzrinde spezifisch leichter ist als die steinige Masse des Innern.

Das Schwefeleisen h"alt vom Rath, obwohl es nicht vom Magnet gezogen wird, nicht f"ur Troilit, sondern f"ur Magnetkies, weil sich bei der Behandlung mit Salzs"aure in reichlicher Menge Schwefelwasserstoff entwickelt und eine Menge Schwefel ausgeschieden wird. Er bestimmte den Gehalt an Magnetkies zu 5,52\%.

Die dunkelgrauen bis schwarzen K"orner, bis 2 mm. gro"s, zeigen bisweilen eine "au"serst feine, sich sehr leicht abl"osende, wei"se H"ulle. Dazu kommen unregelm"a"sig gerundete, dunkle K"orner und Kugelsegmente, welche wie erstere, wenn gleich nur unvollkommene Faserzusammensetzung besitzen. Weiter zeigen sich bis 1 mm. gro"se, gelblich wei"se K"orner --- wahrscheinlich Olivin mit gerundeten Oberfl"achen und nur Andeutungen von kristallinischer Umgr"anzung. Schwarze, kleine Chromeisensteink"orner scheinen eine oktaedrische Form erkennen zu lassen. Die Hauptmasse des Steins stellt sich unter dem Mikroskop als ein Haufwerk unendlich kleiner, wei"ser, kristallinischer K"ornchen dar Sie sind hell, lebhaft fettartig gl"anzend, zeigen Farben i. p. L.; sind in S"auren unl"oslich und bestellen wesentlich aus einem Magnesiasilikate, das reicher an Kiesels"aure, als Olivin ist. Daneben kommt auch noch eine lichtgraue Substanz, welche Anlage zu sph"arolithischer Bildung besitzt, und wie die dunklen Kugeln auch zuweilen faserige Zusammensetzung zeigt, vor.

Mikroskopisch fanden sich noch als seltene Gemengteile vor: au"serordentlich kleine, purpurrote Kristallteilchen, mehrere intensiv gelbe K"ornchen mit deutlichen Krystallfl"achen, einige lichtgelbe, langprismatische Formen und endlich einzelne, bis 1/2 mm. gro"se, rote K"ornchen, von muscheligem Bruche und durchscheinend --- wahrscheinlich Zersetzungsprodukt des Schwefeleisens, dem Caput mortuum "ahnlich.

Die Analyse des nicht magnetischen Anteils ergab nach vom Rath:
\begin{center}
    \begin{tabular}{ |l|r|p{50mm}| }
    \hline
    & I & II\\\hline
    & & Nach Abzug von Chromeisen und Magnetkies\\
    \hline\hline
    Chromeisen & 0,94 & -,-\\\hline
    Magnetkies Schwefel & 2,25 & -,-\\\hline
    Magnetkies Eisen & 3,47 & -,-\\\hline
    Kiesels"aure & 43,29 & 46,37 Sauerstoff 24,73\\\hline
    Tonerde & 0,63 & 0,67 Sauerstoff 0,32\\\hline
    Magnesia & 25,32 & 27,13 Sauerstoff 10,85\\\hline
    Kalkerde & 2,01 & 2,15 Sauerstoff 0,61\\\hline
    Eisenoxydul & 21,06 & 22,56 Sauerstoff 5,01\\\hline
    Manganoxydul & Spur. & -,-\\\hline
    Natron (Verlust) & 1,03 & 1,12 Sauerstoff 0,29\\
    \hline
    \end{tabular}
\end{center}
\paragraph{}
Demnach verh"alt sich die Summe der Sauerstoffmengen der Basen gegen die der Kiesels"aure wie:

1:1,448,

welches Verh"altnis gegen das des Pultusker Steins (1:1,507) auf keine wesentliche Verschiedenheit schlie"sen l"asst. Als wesentliche Gemengteile ergeben sich auch nach der chemischen Analyse: Olivin und ein kiesels"aurereiches Mineral, ob Enstatit oder Shepardit oder beide gleichzeitig, l"asst vom Rath unentschieden.

Die Beimengung von Anorthit oder Labrador h"alt er f"ur unzul"assig, weil Kalk- und Tonerde dem unl"oslichen Anteil angeh"oren und nur in geringer Menge mit S"auren sich ausziehen lassen.

Einer gef"alligen Mitteilung verdanke ichferner die Kenntnisnahme der Resultate einer Analyse, welche Herr Professor Dr. Keller in Speyer vorgenommen hat und welche deshalb von gr"osser Wichtigkeit ist, weil sie mit einer bedeutenden Quantit"at durchgef"uhrt wurde, n"amlich mit 5,71 Gramm; gefunden wurden:
\begin{center}
    \begin{tabular}{ |p{30mm}|p{14mm}|p{14mm}|p{14mm}|p{14mm}|p{14mm}| }
    \hline
    Stoffe & Bausch-Analyse & 57,69\% in Salzs"aure zersetzbar einzeln & 57,69\% in Salzs"aure zersetzbar in \% & 42,31\% in Salzs"aure nicht zersetzbar* einzeln & 42,31\% in Salzs"aure nicht zersetzbar\footnote{\swabfamily{Ohne Chromeisen und Zinnoxyd.}} in \%\\
    \hline\hline
    Kieselerde (a) & 41,12 & 15,76 & 27,28 & 25,36 & 61,76\\\hline
    Bittererde (a) & 18,62 & 14,44 & 24,99 & 4,18 & 10,18\\\hline
    Manganoxydul (a) & 0,78 & 0,78 & 1,35 & -,- & -,-\\\hline
    Eisenoxydul (a) & 17,10 & 10,69 & 18,52 & 6,41 & 15,61\\\hline
    Eisen (b) & 3,93 & 3,93 & 10,85 & -,- & -,-\\\hline
    Schwefel (b) & 2,35 & 2,35 & 10,85 & -,- & -,-\\\hline
    Eisen (c) & 6,44 & 6,44 & 14,31 & -,- & -,-\\\hline
    Nickel (c) & 1,36 & 1,36 & 14,31 & -,- & -,-\\\hline
    Phosphor (c) & 0,46 & 0,46 & 14,31 & -,- & -,-\\\hline
    Chromoxyd (d) & 0,89 & -,- & -,- & 0,89 & -,-\\\hline
    Eisenoxydul (d) & 0,32 & -,- & -,- & 0,32 & -,-\\\hline
    Tonerde (e) & 3,22 & 0,76 & 1,31 & 2,46 & 5,99\\\hline
    Kalk (e) & 2,06 & 0,42 & 0,73 & 1,64 & 4,00\\\hline
    Kali (e) & 1,22 & 0,21 & 0,36 & 1,01 & 2,46\\\hline
    Natron (e) & 0,17 & 0,17 & 0,30 & -,- & -,-\\\hline
    Zinnoxyd (e) & 0,18 & Spuren & -,- & 0,18 & -,-\\
    \hline
    \end{tabular}
\end{center}
\paragraph{}
Daraus wird berechnet:

a) Olivin 41,67

b) Schwefeleisen 6,28

c) Meteoreisen 8,26

d) Chromeisen 1,21

e) Weitere Silikate 42,58

Das spec. Gewicht wurde zu 3,432 ermittelt.

Vergleichen wir nun die Resultate der letzteren (B) Analyse mit jener fr"uher mittgeteilten vom Rath's (A), indem wir beide blo"s auf die Silikatbestandteile umrechnen, um den Einfluss der offenbar in sehr ungleicher Verteilung vorkommenden Gemengteilen des Meteor-, Schwefel- und Chromeisen zu eliminieren, so ergeben sich folgende Zahlen:
\begin{center}
    \begin{tabular}{ |l|r|r| }
    \hline
    & A & B\\
    \hline\hline
    Kieselerde & 46,37 & 48,78\\\hline
    Tonerde & 0,67 & 3,82\\\hline
    Eisenoxydul & 22,56 & 20,29\\\hline
    Manganoxydul & Spur. & 0,93\\\hline
    Magnesia & 27,13 & 22,09\\\hline
    Kalkerde & 2,15 & 2,45\\\hline
    Kali & -,- & 1,44\\\hline
    Natron & 1,12 & 0,20\\
    \hline
    \end{tabular}
\end{center}
\paragraph{}
Auch hier bemerken wir in einzelnen Stoffen eine sehr geringe "Ubereinstimmung, so namentlich in Bezug auf Tonerde und Bittererde, was wieder auf eine sehr ungleiche Mengung und Verteilung der Bestandteile hinweist. In der Tat ergab sich nun bei n"aherer Untersuchung des Steins, welcher in der Kreissammlung zu Speyer verwahrt ist, dass, wie schon Weiss hervorgehoben hat, ganze Parthieen desselben flecken weise durch dunklere Farbe, gr"o"sere H"arte und kompaktere Beschaffenheit vor den "ubrigen hellgrauen, zerreiblichen Massen auffallend sich hervorheben. Es sind diese putzenf"ormigen Einschl"usse, eckig, unregelm"a"sig umgrenzt, gleichsam Bruchst"ucke im Gro"sen, wie die Splitter der Hauptmasse im Kleinen, jedoch auch von besonderer Beschaffenheit. Ich wurde in die angenehme Lage versetzt, "uber St"uckchen des Speyerer Steins f"ur meine weitere Untersuchung verf"ugen zu k"onnen. Ehe ich jedoch "uber diese besonderen Einschl"usse weitere Mitteilung mache, habe ich noch in die n"ahere Er"orterung bez"uglich der in Salzs"aure zersetzbaren und nicht zersetzbaren, verschiedenen Mineralgemenge einzutreten.

Die in Salzs"aure zersetzbaren Silikatbestandteile berechnen sich in ihrer Zusammensetzung:

(+) Kieselerde 36,46

(+) Eisenoxydul 24,73

(+) Bittererde 33,40

(+) Manganoxydul 1,80

(\^{}) Tonerde 1,76

(\^{}) Kalkerde 0,97

(\^{}) Kali 0,48

(\^{}) Natron 0,40\\

(+) nahezu genau die Zusammensetzung des Olivins (Hyalosiderit).\\

(\^{}) Reste eines schwer zersetzbaren, feldspatartigen Gemengteils in geringer Menge.

Der von Salzs"aure nicht zersetzte Rest besteht, das Chromeisen abgerechnet, aus beil"aufig:
\begin{center}
    \begin{tabular}{ |l|r|c|r| }
    \hline
    & I & A & B\\
    \hline\hline
    Kieselerde & 61,7 oder & 30,0 + & 31,7\\\hline
    Bittererde & 10,2 & 10,2 & -,-\\\hline
    Eisenoxydul & 15,6 & 15,6 & -,-\\\hline
    Tonerde & 6,0 & -,- & 6,0\\\hline
    Kalkerde & 4,0 & 2,0+ & 2,0\\\hline
    Kali & 2,5 & -,- & 2,5\\\hline
    & 100,00 & 57,8 & 42,2\\
    \hline
    \end{tabular}
\end{center}
\paragraph{}
Wir k"onnen dieses I. zerlegen in A und B und erhalten dadurch ein Mineral der Augitgruppe und ein Mineral der Feldspatgruppe, das erste bronzitartig (Sauerstoffverh"altnis wie 16:8,1), das zweite mit einem Sauerstoffverh"altnis nahezu wie 6:3:1 (genauer 16,9:3:1) oder labradorartig, zu dem der Tonerde- und Alkali-haltige Anteil des durch Salzs"aure zerlegten Teiles zu rechnen w"are.

Man kann mithin annehmen, dass im Durchschnitt der Meteorstein von Er"ahenberg in seiner Hauptmasse besteht aus:
\begin{center}
    \begin{tabular}{ |l|r| } 
    \hline
    Meteoreisen & 6,27\\\hline
    Schwefeleisen & 8,25\\\hline
    Chromeisen & 1,21\\\hline
    Olivin & 41,65\\\hline
    Augitmineral (? Bronizt) & 23,48\\\hline
    Feldspatmineral (? Labrador) & 19,14\\
    \hline
    \end{tabular}
\end{center}
\paragraph{}
Was nun die in gr"o"seren Brocken im Gestein eingebetteten h"arteren, dichteren und dunkleren Teile anbelangt, welche bereits fr"uher erw"ahnt wurden, so bestehen diese, m"oglichst von den anhaftenden Splittern der Hauptmassen befreit, nach der von Ass. A. Schwager vorgenommenen Analyse aus:
\begin{center}
    \begin{tabular}{ |p{32mm}|p{22mm}|p{20mm}|p{20mm}| }
    \hline
    Stoffe: & Bauschanalyse & 64\% in Salzs"aure zersetzbar & 39\% in Salzs"aure unzersetzbar\\
    \hline\hline
    Kieselerde & 39,08 & 28,44 & 57,96\\\hline
    Tonerde & 2,08 & 1,46 & 5,79\\\hline
    Eisenoxydul & 28,53 & 36,20 & 13,75\\\hline
    Eisen (Nickelhaltig) & 4,43 & 6,92 & -,-\\\hline
    Schwefel & 1,31 & 2,04 & -,-\\\hline
    Manganoxydul & 0,82 & 1,28 & -,-\\\hline
    Chromoxyd & 0,39 & -,- & 1,08\\\hline
    Kalkerde & 13,35 & 14,55 & 11,24\\\hline
    Bittererde & 5,97 & 5,73 & 6,40\\\hline
    Kali & 1,48 & 1,73 & 1,04\\\hline
    Natron & 1,81 & 1,13 & 3,05\\\hline
    & 99,25 & 99,48 & 100,31\\
    \hline
    \end{tabular}
\end{center}
\paragraph{}
Zun"achst ist bemerkenswert, dass wir es gleichfalls mit einer aus verschiedenen Mineralien zusammengesetzten Masse zu tun haben, welche sich in einen durch Salzs"aure zerlegbaren und nicht zerlegbaren Anteil trennen l"asst und dass im Ganzen eine gro"se Ähnlichkeit in ihrer Zusammensetzung im Vergleiche mit jener der Hauptmasse nicht zu verkennen ist. Abweichend erweist sich dagegen besonders der hohe Gehalt an Eisenoxydul und Kalkerde und der geringe an Bittererde, wenn wir die Masse als Ganzes betrachten, w"ahrend in dem Salzs"aureauszug neben denselben Verh"altnissen noch die relativ gro"se Menge an Kiesels"aure in die Augen f"allt. Auch in dem Restanteil ist es die Kalkerde, welche in ungew"ohnlicher Menge auftritt. Es l"asst sich daraus kaum mehr, als die Vermutung sch"opfen, dass neben Hyalosiderit ein eisen- und kalkreiches Mineral der Augitgruppe vielleicht Diopsid mit Anorthit-artigem Feldspat als Hauptgemengteile anzunehmen sind.

Die weitere Untersuchung des Steins hat einige interessante Eigent"umlichkeiten desselben kennen gelehrt. Zun"achst lenken (die zahlreichen, denselben durchziehenden schwarzen Streifchen und Äderchen, welche schon vom Rath genau beschrieben, hat, die Aufmerksamkeit auf sich. Sie bestehen, soweit ich sehen konnte, aus einer der "au"seren Schmelzrinde gleichen, auch Meteoreisen enthaltenden Substanz und scheinen mir Spr"unge und Zerkl"uftungen darzustellen, auf welchen, wie an der Au"senfl"ache, eine Schmelzung stattfand. An einzelnen derselben bemerkte ich gegen Au"sen deutlich eine blasige und schaumige Beschaffenheit. Ganz ausgezeichnet sind glatte und gestreifte Abl"osungsfl"achen, die genau Rutschfl"achen gleichsehen, ohne dass sich jedoch eine Verschiebung einzelner Teile gegen einander erkennen l"asst. Sie m"ussen wohl schon vorhanden gewesen sein, ehe der Stein in die Atmosph"are unserer Erde gelangt war und hier nur stellenweis eine Schmelzrinde erhalten haben.

Die D"unnschliffe, deren ich aus verschiedenen Teilen der Hauptmasse 5 habe herstellen lassen, geben uns "uber das Gef"uge das Bild eines sehr zusammengesetzten Chondriten, wie es die Zeichnung in Figur V darstellt. Viele der runden K"orner erscheinen nur als zersprungene Fragmente kugelartiger Teile und sind nicht selten von einer schwarzen Substanz, an deren Zusammensetzung auch Meteoreisen beteiligt ist, wie von einer Rinde, "uberzogen. An einem derselben dringt dieser schwarze "Uberzug auch in das Korn selbst ein. Sie bestehen teils aus der bekannten exzentrisch faserigen Masse, teils aus feinsten, staub"ahnlichen, wenig durchscheinenden K"ornchen, gr"o"seren hellen Teilchen oder aus einer nach verschiedenen Richtungen parallel zerrissenen oder netzaderigen Substanz in gr"osser Mannigfaltigkeit der Ausbildung Au"serdem bemerkt man eckige Bruchst"ucke von ganz gleicher vielgestaltiger Ausbildung wie bei den kugeligen Einschl"ussen. Unter denselben stechen besonders die "au"serst fein und dicht parallel gestreiften Splitterchen in die Augen, deren Parallelfaserchen durch dunkle Streifchen wie quer gegliedert erscheinen (y). Sie sind f"ur die Chondrite au"serordentlich charakteristisch. Selten sind einzelne St"uckchen frei von Rissen oder von regelm"a"sig parallelen, weit auseinander stehenden, dunklen Linien durchzogen, an denen man bei starker Vergr"o"serung kleinste Bl"aschen bemerkt. Eine Regelm"a"sigkeit in der Anordnung dieser deutlich nur als Splitter eingemengten Bruchst"ucke gibt sich nicht zu erkennen. Alles liegt wirr durcheinander und wird durch immer kleiner werdende und bis zu St"aubchen zerst"uckelte Teilchen zu einem dicht geschlossenen Ganzen verbunden. I. p. L. zeigt sich Alles in bunten Aggregat-Farben von verschiedener Lebhaftigkeit, aber ohne von einer Spur einfach brechender Zwischensubstanz unterbrochen zu werden. Farbenstreifchen kommen selten und nicht deutlich zum Vorschein. Noch bleibt hervorzuheben, dass gr"o"sere Flecke der Masse intensiv gelb gef"arbt erscheinen. Diese F"arbung r"uhrt, wie das rasche Verschwinden derselben beim Behandeln mit Salzs"aure beweist, von infiltriertem, auf den feinen Rissen sich ausbreitendem Eisenoxydhydrat her, das von dem sich in feuchter Luft ungemein leicht zersetzenden Meteoreisen abstammt.

Fast dasselbe Bild gewinnt man auch in dem D"unnschliff der dunklen putzenformigen Parthieen des Steins, von welchen vorher die durch den gro"sen Kalkgehalt und den Mangel an Bittererde auffallende Analyse mitgeteilt wurde (Figur VI.). Es scheinen darin nur die K"orner und Fragmente gr"osser und dichter gedr"angt bei einander zu liegen. Es l"asst sich keine optische Erscheinung auffinden, welche "uber das so abweichende Ergebnis der Analyse Aufschluss zu liefern im Stande w"are, wie man erwarten d"urfte. Die geringe Menge der zur Verf"ugung stehenden Substanz verhindert weitere Untersuchungen anzustellen, die vielleicht das Auffinden eines sehr kalkhaltigen Bestandteils ergeben w"urde. Es wurde auch der Versuch gemacht, die gelben, anscheinend Olivin darstellenden K"ornchen zu isolieren und getrennt einer Analyse zu unterwerfen. Die Behandlung mit Salzs"aure zeigte aber sofort, dass das anscheinend rein herausgelesene Material kaum zur H"alfte von der S"aure zersetzt wird, mithin immer noch trotz der anscheinenden Gleichartigkeit der gelben Splitter verschiedener Natur ist, fast wie der Stein im Ganzen.

Behandelt man einen losgel"osten D"unnschliff l"angere Zeit mit Salzs"aure und untersucht ihn nachher unter dem Mikroskop, so bemerkt man in dem noch gut zusammenhaltenden D"unnschliffe zahlreiche gr"o"sere, kleinere und kleinste L"ucken, welche die Stelle der durch die S"aure zersetzten Gemengteile bezeichnen. Bringt man nun noch weitere Kalil"osung auf den so behandelten D"unnschliff, so zerfallt derselbe sofort in einzelne St"uckchen, K"ornchen und Staubteilchen, unter welchen die von den gr"o"seren Einschl"ussen, abstammenden Splitterchen sich durch ihren festeren Zusammenhalt auszeichnen. Sehr bemerkenswert ist es, dass in den St"ucken von maschenartig streifiger Struktur, obwohl sie noch fest Zusammenhalten, die hellen Streifchen vollst"andig zerst"ort sind und nur die dunklen Zwischenlamellen, wie ein Gerippe unzersetzt geblieben sind. Es l"asst sich dies i. p. L. unzweifelhaft feststellen. Es bestehen demnach die wasserhellen Streifchen oder Lamellen sehr wahrscheinlich aus Olivin, die dunklen Teile aus einem Augitmineral. Daraus erkl"art sich nunmehr auch vollst"andig die Erscheinung, dass die Chondren, wie die Untersuchung an jenen des Steins von Eichst"adt gelehrt hat, teilweise von Salzs"aure zersetzt werden, teilweise aber unangegriffen bleiben.

"Uberblickt man die Resultate der Untersuchung dieser wenn auch beschr"ankten Gruppe von Steinmeteoriten, so dr"angt sich die Wahrnehmung in den Vordergrund, dass sie, trotz einiger Verschiedenheit in der Natur ihrer Gemengteile, doch von vollst"andig gleichen Strukturverh"altnissen beherrscht sind. Alle sind unzweifelhafte Tr"ummergesteine, zusammengesetzt ans kleinen und gr"o"seren Mineralsplitterchen, aus den bekannten rundlichen Chondren, welche meist vollst"andig erhalten, aber oft auch in St"ucke zersprungen Vorkommen und aus Gr"aupchen von metallischen Substanzen Meteoreisen, Schwefeleisen, Chromeisen. Alle diese Fragmente sind aneinandergeklebt, nicht durch eine Zwischensubstanz oder durch ein Bindemittel verkittet, wie sich "uberhaupt keine amorphen, glas- oder lavaartigen Beimengungen vorfinden. Nur die Schmelzrinde und die oft auf Kl"uften auftretenden, der Schmelzrinde "ahnlich entstandenen schwarzen "Uberrundungen bestehen aus amorpher Glasmasse, die aber erst beim Nieder fallen innerhalb unserer Atmosph"are nachtr"aglich entstanden ist. In dieser Schmelzrinde sind die schwerer schmelzbaren und gr"o"seren Mineralk"ornchen meist noch ungeschmolzen eingebettet. Die Mineralsplitterchen tragen durchaus keine Spuren einer Abrundung oder Abrollung an sich, sie sind scharfkantig und spitzeckig. Was die Chondren anbelangt, so ist ihre Oberfl"ache nie gegl"attet, wie sie sein m"usste, wenn die K"ugelchen das Produkt einer Abrollung w"aren, sie ist vielmehr stets h"ockerig uneben, maulbeerartig rauh und warzig oder facettenartig mit einem Ansatz von Krystallfl"achen versehen. Viele derselben sind l"anglich, mit einer deutlichen Verj"ungung oder Zuspitzung nach einer Richtung, wie es bei Hagelk"ornern vorkommt. Oft begegnet man St"uckchen, welche offenbar als Teile zertr"ummerter oder zersprungener Chondren gelten m"ussen. Als Ausnahme kommen zwillingsartig verbundene K"ugelchen vor, h"aufiger solche, in welchen Meteoreisenst"uckchen ein- oder angewachsen sind. Nach zahlreichen D"unnschliffen sind sie verschiedenartig zusammengesetzt. Am h"aufigsten findet sich eine exzentrisch strahlig faserige Struktur in der Art, dass von einer weit aus der Mitte nach dem sich verj"ungenden oder etwas zugespitzten Teil hin verr"uckten Punkte aus ein Strahlenb"uschel gegen au"sen sich verbreitet. Da die in den verschiedensten Richtungen gef"uhrten Schnitte immer s"aulen- oder nadelf"ormige, nie bl"atter- oder lamellenartige Anordnung in der diesen B"uschel bildenden Substanz erkennen lassen, so scheinen es in der Tat s"aulenf"ormige Fasern zu sein, aus welchen sich solche Chondren aufbauen. Bei gewissen Schnitten gewahrt man, dieser Annahme entsprechend, in den senkrecht zur L"angenrichtung gehenden Querschnitten der Fasern nur unregelm"a"sig eckige, kleinste Feldchen, als ob das Ganze aus lauter kleinen polyedrischen K"ornchen zusammengesetzt sei. Zuweilen sieht es aus, als ob in einem K"ugelchen gleichsam mehrere nach verschiedener Richtung hin strahlende Systeme vorhanden w"aren oder als ob gleichsam der Ausstrahlungspunkt sich w"ahrend ihrer Bildung ge"andert habe, wodurch bei Durchschnitten nach gewissen Richtungen eine scheinbar wirre, st"angliche Struktur zum Vorschein kommt. Gegen die Au"senseite hin, gegen welche der Vereinigungspunkt des Strahlenb"uschels einseitig verschoben ist, zeigt sich die Faserstruktur meist undeutlich oder durch eine mehr k"ornige Aggregatbildung ersetzt. Bei keinen der zahlreichen angeschliffenen Chondren konnte ich beobachten, dass die B"uschel so unmittelbar bis zum Rande verlaufen, als ob der Ausstrahlungspunkt gleichsam au"serhalb des K"ugelchens l"age, soferne nur dasselbe vollst"andig erhalten und nicht etwa ein blo"ses zersprungenes St"uck vorhanden war. Die zierlich quergegliederten F"aserchen verlaufen meist nicht nach der ganzen L"ange des B"uschels in gleicher Weise, sondern sie spitzen sich allm"ahlich zu, ver"asteln sich oder endigen, um andere an ihre Stelle treten zu lassen, so dass in dem Querschnitte eine mannichfache, maschenartige oder netzf"ormige Zeichnung entsteht. Diese F"aserchen bestehen, wie dies schon vielfach im Vorausgehenden geschildert wurde, aus einem meist helleren Kern und einer dunkleren Umh"ullung, jener durch S"auren mehr oder weniger zerlegbar, letztere dagegen dieser Einwirkung widerstehend. H"ochst merkw"urdig sind die schalenf"ormigen "Uberrundungen, welche aus Meteoreisen zu bestehen scheinen und in der Regel nur "uber einen kleineren Teil der K"ugelchen sich ausbreiten. Die gleichen einseitigen, im Durchschnitt mithin als bogenf"ormig gekr"ummte Streifchen sichtbaren "Uberrundungen, kommen auch im Innern der Chondren vor und liefern einen starken Gegenbeweis gegen die Annahme, dass die Chondren durch Abrollung irgend eines Materials entstanden seien, wie denn "uberhaupt die ganze Anordnung der btischeligen Struktur mit Entschiedenheit gegen ihre Entstehung durch Abrollung spricht.\footnote{\swabfamily{Auch die von G. v. Drasche aus dem Meteorit von Lancé gezeichneten fasrigen Chondren (Tschermak's Miner. Mittb. 1875. Bd. V. 1. H.) entsprechen in Bezug auf innere Struktur und "au"sere Form genau unserer Schilderung.}} Doch nicht alle Chondren sind exzentrisch faserig; viele, namentlich die kleineren besitzen eine feink"ornige Zusammensetzung, als best"anden sie aus einer zusammengeballten Staubmasse. Auch hierbei macht sich zuweilen die einseitige Ausbildung der K"ugelchen durch eine exzentrisch gr"o"sere Verdichtung der Staubteile bemerkbar.

Was endlich die "au"sere Form der den Chondriten beigemengten Meteor- und Schwefeleisenteilchen anbelangt, so bemerken wir auch bei diesen durchaus keine regelm"a"sige Gestaltung, weder in Leistchen nach Art des Titaneisens etwa im Dolerit, noch in rundlichen K"ugelchen, isoliert man das Meteoreisen einfach durch leichtes Zerdr"uckender Steinmasse und Herausziehen mit dem Magnet, so zeigen sich die Meteoreisenteilchen an der Oberfl"ache staubig, von anhaftenden Mineralteilchen wie "uberkleidet. Im Allgemeinen sind es unregelm"a"sig gestaltete Gr"aupchen und Kn"ollchen, welche vielfach in feine Z"ackchen und zarte gek"ornlte Ver"astelungen verlaufen. Durch Anwenden von Flusss"aure kann man die staubigen Mineralteilchen, welche auf der Oberfl"ache der Gr"aupchen wie angekittet sind, entfernen und man bemerkt nun eine uneben grubige, gleichsam punktierte Oberfl"ache, ohne Spur einer Spiegelung von Krystallfl"achen. Ähnliche Beschaffenheit besitzen auch die Schwefeleisenteilchen, nur sind sie nicht so zackig. Noch einfacher, aber auch stets unregelm"a"sig gestaltet sind die Chromeisenfragmente.

Der gew"ohnliche Typus der Meteorite von steiniger Beschaffenheit ist so weit "uberwiegend derjenige der sog Chondrite und die Zusammensetzung sowie die Struktur aller dieser Steine so sehr "ubereinstimmend, dass wir den gemeinsamen Ursprung und die uranf"angliche Zusammengeh"origkeit aller dieser Art Meteorite --- wenn nicht aller --- wohl nicht weiter in Zweifel ziehen k"onnen.

Der Umstand, dass sie s"amtlich in h"ochst unregelm"a"sig geformten St"uckchen in unsere Atmosph"are gelangen --- abgesehen von dem Zerspringen innerhalb der letzteren in mehrere Fragmente, was zwar h"aufig vor kommt, aber doch nicht in allen F"allen angenommen werden kann, namentlich nicht, wenn durch direkte Beobachtung das Fallen nur eines St"uckes konstatiert ist, --- l"asst weiter schlie"sen, dass sie bereits in regellos zertr"ummerten St"ucken als Abk"ommlinge von einem einzigen gr"o"seren Himmelsk"orper ihre Bahnen im Himmelsraume ziehen und in ihrer Zerstreutheit einzeln zuweilen in das Attraktionsbereich der Erde geratend zur Erde niederfallen. Der Mangel urspr"unglicher, lavaartiger, amorpher Bestandteile in Verbindung mit der "au"sern unregelm"a"sigen Form d"urfte von geo- oder kosmologischen Standpunkten aus der Annahme ausschlie"sen, dass diese Meteorite Ausw"urflinge aus Mondvulkanen, wie vielfach behauptet wird, sein k"onnen.

Die Bemerkung, welche G. Neumayer bez"uglich des Falls von Kr"ahenberg macht,\footnote{\swabfamily{Sitzb. d. Acad. in Wien math.-naturw. Cl. Bd. 60, 2. 1869. S. 239.}} dass n"amlich dieser Meteorit auf seinem kosmischen Laufe dem Meteorschauer angeh"ort habe, dessen Radiationspunkt in der N"ahe von $\delta$ Virginis liegt, kann nur dazu dienen, obige Annahme wahrscheinlicher zu machen. Darauf laufen auch die Ansichten fast aller Forscher hinaus, welche sich in neuerer Zeit mit dem Studium der Meteorite befasst haben, nur "uber die Ursache der Zertr"ummerung ob sie durch den Zusammensto"s bereits fester Himmelsk"orper, oder durch eine von innen nach au"sen wirkende Explosion einer kosmischen Masse oder aber durch ein Zerbr"ockeln von freien St"ucken, etwa wie es bei austrocknendem Tone eintritt, erfolgt sei, herrscht verschiedene Meinung, wie es Tschermak in seiner ausgezeichneten Arbeit "uber die Bildung der Meteorite und des Vulkanismus\footnote{\swabfamily{Sitz. d. Ac. d. Wiss. in Wien math.-nat. Cl. Bd. 71. 1875. Aprilheft.}} so vortrefflich schildert. Es ist bei dieser Annahme sogar denkbar, dass ein Meteorit, der schon einmal die Erdatmosph"are auf seiner Bahn gestreift und dabei eine partielle Schmelzung erlitten hat, sp"ater wieder in die Erdn"ahe ger"ath und nun wirklich zur Erde niederfallt. So lie"se sich vielleicht das Vorkommen von Schmelzmasse, "ahnlich wie die in der Erdatmosph"are geschmolzenen Binde, im Innern einzelnen Steinmeteorite erkl"aren. Auch von astronomischer Seite scheint die oben besprochene Zugeh"origkeit vieler Meteorite zu einem aus zertr"ummerten kosmischen K"orperchen bestehenden Schwarme auf keinen Widerspruch zu sto"sen.

Haben wir die Wahrscheinlichkeit des Ursprungs unsere Chondrite als Ganzes betrachtet nachzuweisen versucht, so bleibt uns vom geologischen Standpunkte die weit wichtigere Frage noch zu beantworten "ubrig, wie der einzelne Chondrit als Gestein seiner Masse nach sich gebildet haben mag, wenn wir seine Zusammensetzung aus kleinen Mineralsplitterchen, Eisengr"aupchen und rundlichen Kn"ollchen (Chondren) ohne laya"ahnliches Kittmittel n"aher ins Auge fassen. Mit den rein mineralogischen Teilen dieser Frage hat sich wohl in neuerer Zeit am intensivsten und mit den gl"ucklichsten Erfolgen experimentellen Nachweises Daubrée befasst.\footnote{\swabfamily{Die wichtigsten der hierher geh"origen Publikationen Daubrée's sind: Expériences synthétiques relatives aux météorites in: Comptes rendus t. 62. 1866, Bulletin de la soc. géologique d. France II. Ser. A. 26. p. 95 und Comptes rendus 1877. N. 27.}} Aus seinen. klassischen Arbeiten l"asst sich entnehmen, dass sich die Hauptmineralbestandteile der Chondrite, Olivin, Enstatit und metallisches Eisen durch Schmelzen der Steine unter gewissen Bedingungen in kristallisiertem und kristallinischem Zustande (wenigstens die zwei Silikate) wieder gewinnen lassen und dass man diese Silikate auch aus irdischen Felsarten z. B. Lherzolith oder Olivinfels, sogar aus Serpentin durch Schmelzen herstellen kann. Es ergibt sich selbst eine gewisse Struktur"ahnlichkeit zwischen geschmolzenem Lherzolith und gewissen Meteoriten. Ein wesentlicher Unterschied wird durch den Eisenbestandteil bedingt, der bei dem Lherzolith ein oxydiertes Eisen, bei den Meteoriten aber ein regulinisches ist. W"ahrend bei den Bildungen auf Erden Sauerstoff und Wasser mitwirkten, muss der Einfluss dieser Stoffe bei der Entstehung der Meteorite ausgeschlossen angenommen werden. Die Meteorite haben keine Ähnlichkeit mit unseren auf der Oberfl"ache der Erdrinde vorfindlichen Gesteinsarten, wie Granit. Um Analogien f"ur sie auf Erden zu finden, muss man in die tiefere Region der Erde hinabgehen, wo in den basischen Silikaten der Olivingesteine die n"achsten Verwandten sich finden. Es scheinen daher die Meteoriten aus einer Art erstem Verschlackungsprozess der Himmelsk"orper --- aber, da sie metallisches Eisen enthalten --- bei Mangel von Sauerstoff und Wasser hervorgegangen zu sein. Daubrée hat durch direkte Experimente nicht blo"s die Entstehung der Silikate nachgewiesen, sondern auch gezeigt, dass unter dem reduzierenden Einfluss von Wasserstoff aus dem Magneteisen des Lherzoliths Eisen in reduziertem Zustande sich bilden kann. Die Eisenteilchen in den Meteoriten finden sich aber nicht in rundlichen K"ugelchen, wie sie aus den Schmelzfl"ussen bei Reduktionsmittel hervorgehen, sondern in unregelm"a"sigen Kn"ollchen. Es kann daher bei der Bildung der Meteoriten nicht die Schmelzhitze des Eisens, selbst nicht die der Silikate geherrscht haben. Es l"asst sich aber auch denken, dass ein der Reduktion entgegengesetzter Prozess wirksam war, wenn man annimmt, dass die Stoffe urspr"unglich nicht in oxydiertem, sondern in regulinischem Zustande vorhanden waren, und dass im Momente, wo der Sauerstoff anfing seine Wirksamkeit zu entfalten, derselbe zuerst sich mit den am leichtesten oxydierbaren Stoffen verband und wenn er in nicht zureichender Menge vorhanden war, welche die schwieriger oxydierbaren Stoffe unoxydiert --- so das Eisen --- "ubrig lie"s.

Auch diese Hypothese hat Daubrée durch gl"anzend durchgef"uhrte Experimente mit Erfolg zu erh"arten versucht. Einem "ahnlichen Verschlackungsprozess w"ahrend einer der ersten Bildungsstadien schreibt er auch die Entstehung der Olivingesteine der Erde zu, welche in gr"o"ster Tiefe sich vorfinden, wobei jedoch abweichend von der Entstehung der metallisches Eisen enthaltenden Meteoriten, Sauerstoff im "Uberschuss vorhanden war, um sowohl die Silikate als auch --- anstatt des Meteoreisens --- Magneteisen zu bilden.

Wenn auf diese Weise gleichsam die mineralogische Seite der Bildung der Meteorite erkl"art erscheint, so erfordert die eigent"umliche tr"ummerige Struktur der Chondrite noch eine weitere Er"orterung.

Wir entnehmen einer neueren Publikation Daubrée's,\footnote{\swabfamily{Bull. d. 1. société géol. d. France 26a. l868-1869 S. 98 u. ffd.}} dass er die Entstehung der Chondren sich analog denkt, wie die Abscheidung von Olivink"ugelchen bei einem Versuche, bei welchem er Olivin mit Kohlen gemengt, geschmolzen hat. Vollst"andiger ist der Vergleich, wenn der Reduktionsprozess durch Wasserstoff erfolgt. Erst neulich spricht sich der um die Kenntnis der Meteorite so sehr verdiente Gelehrte\footnote{\swabfamily{Comptes rendus 1877. No. 27.}} "uber diesen Gegenstand bei Gelegenheit der Er"orterung einer merkw"urdigen Breccien-"ahnlichen Struktur an dem Meteoreisen von St. Catharina weiter aus, dass die Zertr"ummerung des die Steinmeteoriten zusammenhaltenden Materials wohl als Sprengwirkung sehr zusammengedr"uckter Grase angesehen werden m"usse, etwa wie sie bei Anwendung von Dynamit stattfindet. Was aber die Bildung der Chondren anbelangt, so beruft er sich auf den oben angef"uhrten Versuch, wobei eine Art K"ornelung in dem Moment sich vollzieht, in dem die Substanz sich verfestigt. Aber am "oftesten scheinen ihm die Chondren einfache Fragmente zu sein, welche sich durch Reibung abrundeten, wie dies aus der Untersuchung dieser K"ugelchen durch G. Rose (Abh. der Ac. d. Wiss. in Berlin f"ur 1862 S. 97 u. 98) hervorgehe und St. Meunier (Comptes rendus 1871. 346 u. Recherches sur la composition et la structure de Météorites 1869) f"ur mehrere Meteorite klargelegt habe.

Nach dem Vorg"ange Haidinger's hat sich neuerdings auch Tschermak mit dem Studium der Bildung der Meteorite eingehend befasst und die Ergebnisse seiner h"ochst interessanten Untersuchungen in mehreren Schriften mitgeteilt. Diese Arbeiten geh"oren unstreitig zu den wichtigsten und tief gr"undlichsten, die wir "uber diesen Gegenstand besitzen. Tschermak kommt bez"uglich der Entstehung der einzelnen Meteorst"ucke zu der am wahrscheinlichsten sich ergebenden Annahme, dass sie ihre Gestalt nicht einer Zertr"ummerung von Planeten durch Sto"s verdanken, sondern dass durch eine Wirkung von Innen nach Au"sen, durch eine Explosion analog der vulkanischen T"atigkeit jene Zertr"ummerung bis zu winzigen St"ucken, die man ein Zerst"auben nennen muss, bewirkt werde. Er weist hierbei auf die gewaltsamen explosionsartigen Erhebungen hin, welche bei der Sonne und bei Kometen direkt beobachtet worden sind, oder auf der Mondoberfl"ache durch den Aufbau der Krater sich verraten. Was nun die Zusammensetzung der Meteorite insbesondere anbelangt, so schlie"st sich auch in dieser Richtung Tschermak der Ansicht Haidinger's an, dass sie aus Gesteinsstaub zusammengef"ugt sind, welcher dem vulkanischen Tuff zu vergleichen ist. Nur das massenhafte Erscheinen der kleinen K"ugelchen, der Chondrite, ist es, welche, so viel bekannt, in den Tuffen der irdischen Vulkane nicht auftreten und deshalb schwieriger zu erkl"aren sind. Diese K"ugelchen verhalten sich nach seiner Annahme durchaus nicht, als ob sie durch Kristallisation zu ihrer Form gekommen w"aren, sie verhalten sich auch nicht wie die Sph"arolithe im Obsidian und Perlstein, oder wie die Kugeln im Kugeldiorit, und die runden Konkretionen vom Calcit, Aragonit, Markasit. Sie gleichen vielmehr den Kugeln, welche man "ofters in Tuffen der vulkanischen Bildungen sieht, z. B. die Trachytkugeln in dem Gleichenberger Trachyttuff, die Kugeln in dem Basalttuff am Venusberg bei Freudenthal, besonders aber den Olivinkugeln in dem Basalttuff von Kapfenstein und Feldbach in Steiermark.\footnote{\swabfamily{Es stand mir nur ein "ahnliches Material, der Trachyttuff mit sog. Leucitkn"ollchen von den zyklopischen Inseln, zur Verf"ugung. D"unnschliffe dieses Gesteins lehrten mich, dass die vermeintlichen Leucite Gesteinsk"ugelchen sind, welche aus demselben Material bestehen, wie die Tuffmasse selbst und keine den Meteoriten-Chondren "ahnliche Struktur besitzen. Nachtr"aglich erhielt ich durch Hrn. Tschermak's besondere 1G"ute auch Proben des Gesteins von Gleichenberg. Diese Olivinknollen a"sen keine Analogien mit den Chondren erkennen.}} Von letzteren darf man sicher annehmen, dass sie Produkte der vulkanischen Zerreibung sind und ihre Form einer kontinuierlichen explosiven T"atigkeit eines vulkanischen Schlotes verdanken, durch welche "altere Gesteine zersplittert und deren z"ahere Teile durch best"andiges Zusammensto"sen abgerundet wurden. Man k"onne allenfalls sich vorstellen, dass die Steinmassen, welche der Zerreibung ausgesetzt waren, ziemlich weich gewesen seien, und w"urde sich dadurch der Vorstellung Daubrée's n"ahern, welcher auf ein Gestein hinweise, das in einer Gasmasse wirbelnd erstarrte. Doch sei hervorzuheben, dass kein Meteorit irgendeine Ähnlichkeit mit vulkanischer Schlacke oder mit Lava besitze, daher k"onne der Vergleich der Meteoriten mit vulkanischen Tuffen oder Breccien nur bis zu einer gewissen Grade gelten. Die vulkanische T"atigkeit bei der Bildung der Meteoriten bestand daher nur in der Zertr"ummerung starrer Gesteine durch eine explosive T"atigkeit in Folge pl"otzlicher Ausdehnung von D"ampfen oder Gasen, unter welchen das Wasserstoffgas eine bedeutende Rolle gespielt haben d"urfte.

So geistreich diese Hypothesen Daubrée's und Tschermak's sind, so kann ich mich doch in Bezug auf die Entstehung der K"ugelchen (Chondren) ihrer Ansicht auf Grund meiner neuesten Untersuchungen nicht anschlie"sen. Ich habe in den Gegens"atzen zu Tschermak's Annahme nachzuweisen gesucht, dass das innere Gef"uge der Chondren nicht au"ser Zusammenhang mit ihrer kugeligen Gestalt stehe, und dass man diese K"ugelchen weder als St"ucke eines Mineralkrystalls noch eines festen Gesteins ansehen k"onne. Spricht schon ihre nicht gegl"attete, nicht polierte Oberfl"ache, welche wenn durch Abreibung oder Abrollung gebildet, bei solcher H"arte des Materials spiegelglatt sein m"usste, w"ahrend sie rauh, h"ockerig, oft strichweise kristallinisch facettiert erscheint, gegen die Abreibungstheorie, so ist auch gar kein Grund einzusehen, weshalb nicht alle anderen Mineralsplitterchen wie Sandk"orner abgerundet seien und weshalb namentlich das Meteoreisen, das Schwefeleisen und das sehr harte Chromeisen, wie ich in dem Meteorit von L'Aigle mich "uberzeugt habe, stets nichtgerundete, oft "au"serst fein zerschlitzte Formen besitzen. Wie w"are es zudem denkbar, dass, wie h"aufig beobachtet wird, innerhalb der K"ugelchen konzentrische Anh"aufung von Meteoreisenteilchen vorkommen? Auch erscheint die exzentrisch faserige Struktur der meisten K"ugelchen in ihrem einseitig gelegenen Ausstrahlungspunkte in Bezug auf die Oberfl"ache nicht als zuf"allig, sondern der Art der Struktur der Hagelk"orner nachgebildet. Dieses innere Gefiige steht im engsten Zusammenh"ange mit dem Akt ihrer Entstehung, welche nur als eine Verdichtung Mineral bildender Stoffe unter gleichzeitiger drehender Bewegung in D"ampfen, welche das Material zur Fortbildung lieferten, sich erkl"aren l"asst, wobei in der Richtung der Bewegung einseitig mehr Material sich ansetzte.

Indem ich auf die Tatsachen mich berufe, welche bei allen Chondriten --- und um diese handelt es sich hier --- zum Vorschein kommen,
\begin{enumerate}
    \item dass sie nur aus feinen oder gr"oberen Mineralsplitterchen oder aus eckigen oder halbkugeligen, zersprengten St"ucken von Chondren und ans diesen selbst bestehen;
    \item dass jede Spur von Lava- oder Schlacken-"ahnlichen Beimengungen oder Bindemittel fehlt und alle Verschlackungen, welche sich vorfinden, nur sekund"are Erscheinungen in Folge der Bewegung der Meteorite innerhalb der irdischen Atmosph"are sind;
    \item dass weder das beigemengte Meteoreisen noch Schwefeleisen noch Chromeisen die Form der Chondren besitzen und keine Spur erlittener Abrollung erkennen lassen;
    \item dass die innere Struktur der Chondren, sei sie exzentrisch faserig, oder k"ornig oder staubig in's Dichte "ubergehend, mit der l"anglich runden, an die Eiform erinnernden Gestalt in genetischem Zusammenhange steht, wie die Beschaffenheit der Strahlenb"uschel unzweideutig lehrt;
    \item dass zuweilen der Oberfl"achenform entsprechende Ausscheidungen im Innern der K"ugelchen sich vorfinden und
    \item endlich, dass die Oberfl"ache der Chondren nicht, wie bei Entstehung durch Abrollung, poliert, sondern rauh und h"ockerig ist, wie wenn Teilchen um Teilchen nach Au"sen sich gesetzt h"atten,
\end{enumerate}
glaube ich z. Th. in "Ubereinstimmung mit den genannten Gelehrten annehmen zu m"ussen, dass das Material, aus welchem die Chondrite bestehen, durch eine gest"orte Kristallisation und Zertr"ummerung in Folge von explosiven Vorg"angen innerhalb eines Raumes sich bildete, welcher von die Mineral bildenden Stoffe liefernden D"ampfen und von die weitere Oxydation des Meteoreisens verhinderndem Wasserstoffgas erf"ullt war. Die K"ugelchen bildeten sich durch Anh"aufung von Mineralmasse um einen Ansatz oder Kern bei fortdauerndem Fall oder Bewegung in den Stoff liefernden D"ampfen, wodurch eine einseitige Zunahme oder ein Ansatz des Materials in der Richtung des Flugs, wie bei der Entstehung gewisser Hagelk"orner oder Eisgraupen bedingt ist und die exzentrisch faserige Struktur und l"anglichrunde Form ihre Erkl"arung findet. Dass hierbei Zertr"ummerungen in Folge des Zusammensto"ses der verfestigten Massen stattfanden, beweisen die in St"ucke zersprengten K"ugelchen und die zahlreichen eckigen Fragmente, welche dieselbe faserige Struktur, wie die K"ugelchen selbst, besitzen. Vielleicht, dass ein Zerfallen auch in Folge raschen Temperaturwechsels eingetreten ist. Das so entstandene Material fiel, wie ein Aschenregen, zur Oberfl"ache des sich bildenden Himmelk"orpers und verfestigte sich nach Art der vulkanischen Trockentuffe durch Agglutinieren der Tr"ummerchen zu einem meist lockeren Aggregat und wurde vielleicht erst in diesem Zustande der Verfestigung durch weitere Explosionst"atigkeit zerst"uckelt und abgeschleudert. Diese St"ucke oder Teile dieser St"ucke sind es, welche als Meteorite endlich zur Erde gelangen. Dass andere Meteoriten namentlich die Meteoreisenmassen und die kohligen eine teilweise andere Entstehung gehabt haben m"ussen, ist nicht zweifelhaft; sie m"ogen einen ruhigeren Prozess an der Oberfl"ache des Himmelsk"orpers durchgemacht und nur das mit den steinigen Meteoriten gemein haben, dass sich z. Th. dasselbe Material an ihrer Zusammensetzung beteiligte, wenn auch in geringerer Menge und dass sie auf gleiche Weise zerst"uckelt und abgeschleudert wurden.

Ich begegne z. Th. "ahnlichen Ansichten, zu welchen mich das Studium der Chondrite gef"uhrt hat, auch bei Sorby, welcher dieselben schon fr"uher in die Aufs"atze: "`On the Physical History of Meteorites"'\footnote{\swabfamily{The geological Magazin, II. 1865 S. 447.}} angedeutet hat.

Ich f"uge diesen Bemerkungen noch einige Beobachtungsresultate hinzu, welche ich an den kohligen Meteoriten von Bokkeveld und Kaba erhalten habe. Das Material hierf"ur verdanke ich der besonderen G"ute des Hm. Prof. Tschermak in Wien. Ich hoffte durch D"unnschliffe vielleicht eine Spur organischer Struktur in dem kohligen Bestandteile zu entdecken. In dem Meteorit von Bokkeveld, von den D"unnschliffen sehr schwierig und immer nur in der beschr"ankten Weise herzustellen sind, dass die kohligen Parthieen nur hier und da durchscheinend werden, sieht man eine Menge kleiner, besonders scharfeckiger, wasserheller Mineralsplitterchen in der kohligen Hauptmasse eingebettet. I. p. L. zeigen diese Mineraltr"ummerchen lebhafte bunte Farben und scheinen sich "uberhaupt wie die Bestandteile der Chondrite zu verhalten. Die kohlige Substanz, wo sie durchscheinend ist, besitzt jenes h"autige oder feink"ornige Gef"uge, wie man es sonst auch bei kohligen Substanzen trifft. St"uckchen, welche ich w"ahrend einiger Tage mit chlorsaurem Kali und Salpeters"aure in der K"alte behandelte, entf"arbten sich vollst"andig und wurden sehr weich. Mit Kanadabalsam getr"ankt gestatteten sie die Herstellung von· D"unnschliffen, in welchen nunmehr die Mineralsplitterchen z. Th. tr"ube und undurchsichtig sich zeigen (wahrscheinlich zersetzter Olivin), z. Th. aber wasserhell geblieben sind (wahrscheinlich Augit-artige Beimengungen), w"ahrend die kohlige Hauptmasse sich teilte in eine vollst"andigIdurchsichtige Masse und in zwischen diese eingebetteten dunkleren Flecke und W"olkchen. Die durchsichtigen Teile lassen dieselbe membran"osk"ornige Struktur erkennen, wie bei dem durchscheinenden Parthien der nicht behandelten D"unnschliffe. Von Andeutungen organischer Struktur konnte auch nach dieser Behandlung nichts entdeckt werden.

Der kohlige Meteorit von Kaba ist ungleich h"arter. In den D"unnschliffen beobachtet man sehr zahlreiche hell-Mineralteilchen, fast alle von kreisrunden Durchschnitten, also wahrscheinlich Chondren entsprechend, jedoch, soweit mein Material erkennen lie"s, ohne Faserstruktur. Sie bestehen vielmehr gleichsam aus einem Aggregat von wasserhellen K"ornchen, zwischen welchen gew"ohnlich undurchsichtige Streifchen verlaufen. Dergleichen schwarze, vielleicht kohlige Linien und Flecken erscheinen meist auch in konzentrischer Anordnung in den K"ugelchen und um diese herum. Die helle Mineralsubstanz zeigt i. p. L. bunte Farben. Der Einwirkung von chlorsaurem Kali und Salpeters"aure leistet dieser Meteorit Widerstand, er entf"arbt sich nur wenig, dagegen werden bei dieser Behandlung die K"ugelchen in Folge erlittener Zersetzung tr"ub und undurchsichtig, was mit einiger Wahrscheinlichkeit auf ihre Olivinnatur zu deuten sein wird. Von organischer Struktur ist unter diesen Umst"anden auch bei diesen kohligen Meteoriten nichts zu sehen. Vielleicht gelingt es dennoch unter Anwendung des oben angef"uhrten Entf"arbungsmittels bei reichlicherem Material oder an anderen kohligen Meteoriten die Anwesenheit organischer Wesen auf au"serirdischen Himmelsk"orpern nachzuweisen.
\clearpage
\end{document}
