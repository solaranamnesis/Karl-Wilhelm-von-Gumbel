\documentclass[a4paper, 11pt, oneside]{article}
\usepackage[utf8]{inputenc}
\usepackage[T1]{fontenc}
\usepackage[ngerman]{babel}
\usepackage{fbb} %Derived from Cardo, provides a Bembo-like font family in otf and pfb format plus LaTeX font support files
\usepackage{booktabs}
\setlength{\emergencystretch}{15pt}
\usepackage{fancyhdr}
\usepackage{graphicx}
\graphicspath{ {./} }
\begin{document}
\begin{titlepage} % Suppresses headers and footers on the title page
	\centering % Centre everything on the title page
	%\scshape % Use small caps for all text on the title page

	%------------------------------------------------
	%	Title
	%------------------------------------------------
	
	\rule{\textwidth}{1.6pt}\vspace*{-\baselineskip}\vspace*{2pt} % Thick horizontal rule
	\rule{\textwidth}{0.4pt} % Thin horizontal rule
	
	\vspace{1\baselineskip} % Whitespace above the title
	
	{\scshape\LARGE Sitzungsberichte der\\[1.25pt] Mathematisch-Physikalischen Klasse der\\[1.25pt] Königlich Bayerischen Akademie der\\[1.25pt] Wissenschaften zu München\\[1.25pt]}
	
	\vspace{1\baselineskip} % Whitespace above the title

	\rule{\textwidth}{0.4pt}\vspace*{-\baselineskip}\vspace{3.2pt} % Thin horizontal rule
	\rule{\textwidth}{1.6pt} % Thick horizontal rule
	
	\vspace{1\baselineskip} % Whitespace after the title block
	
	%------------------------------------------------
	%	Subtitle
	%------------------------------------------------
	
	{\scshape Band 8 --- Jahrgang 1878} % Subtitle or further description
	
	\vspace*{1\baselineskip} % Whitespace under the subtitle
	
    {\scshape\small In Kommission bei G. Franx} % Subtitle or further description
    
	%------------------------------------------------
	%	Editor(s)
	%------------------------------------------------
    \vspace*{\fill}

	\vspace{1\baselineskip}

	{\small\scshape München 1878}
	
	{\small\scshape{Akademische Buchdruckerei von F. Straub}}
	
	\vspace{0.5\baselineskip} % Whitespace after the title block

    \scshape Internet Archive Online Edition  % Publication year
	
	{\scshape\small Namensnennung Nicht-kommerziell Weitergabe unter gleichen Bedingungen 4.0 International} % Publisher
\end{titlepage}
\setlength{\parskip}{1mm plus1mm minus1mm}
\clearpage
\tableofcontents
\clearpage
\section{Sitzung vom 9. Februar 1878 --- Über die in Bayern gefundenen Steinmeteoriten}
\subsection*{Einleitung}
\paragraph{}
Unter den auf bayerische Gebiete gefallenen und aufgefundenen Steinmeteoriten befinden sich mehrere, deren chemische Zusammensetzung uns nur aus älteren Analysen bekannt ist, während von einem derselben bis jetzt überhaupt noch keine chemische Untersuchung vorgenommen wurde. Da es außerdem ihre den meisten derselben an einer erschöpfenden Untersuchung, Zwie solche neuerdings bei Gesteinsarten mittelst Dünnschliffe und Mikroskop vorgenommen zu werden pflegt, fehlt, so schien es mir interessant genug, diese Arbeit vorzunehmen und die Ergebnisse mit dem früher bekannten zusammenzustellen. Durch die besondere Güte des Herrn Konservators der mineralogischen Staatssammlung Professor Dr. v. Kobell habe ich das hierzu erforderliche Material erhalten und ich benütze gerne die Gelegenheit, für diese so freundliche Unterstützung meiner Untersuchung hier den besten Dank auszudrücken. Einige weitere Bemerkungen, welche an den Schlüssen beigefügt sind, beziehen sich auf andere Meteorsteine, die ich gelegentlich der Vergleichung wegen in den Kreis meiner Beobachtung gezogen habe.

Es wurden im Ganzen nur 5 Steinmeteoriten von denen, welche in Bayern gefallen sind, bekannt. Darunter ist sogar noch ein Fund einbegriffen, welcher nach dem gegenwärtigen Territorialverhältnisse nicht mehr Bayern, sondern Österreich angehört, nämlich jener von Mauerkirchen. Da jedoch zur Zeit des Falls der Ort zu Bayern gehörte, so dürfte es immerhin bis zu einer gewissen Grade gerechtfertigt erscheinen, diesen Stein hier unter den bayerischen aufzuführen.

Diese 5 Steinmeteorite sind:
\begin{enumerate}
    \item Der Stein von Mauerkirchen im jetzt österreichischen Innviertel vom Falle am 20. Nov. 1768 Nach-, mittags 4 Uhr.
    \item Der Stein von Eichstädt, welcher im sog. Wittmes 5 Kilom. von der Stadt am 19. Febr. 1785 nach 12 Uhr Mittags gefallen ist.
    \item Der Stein von Massing bei Altötting in Südbayern vom Fall am 13. Dezember 1803 zwischen 10-11 Uhr Vormittags.
    \item Der Stein von Schönenberg bei Burgau und Schwaben, gefallen am 25. Dez. 1846 Nachmittags 2 Uhr und
    \item Der Stein von Krähenberg bei Homburg in der Rheinpfalz vom Fall am 5. Mai 1869 Abends 6 1/2 Uhr.
\end{enumerate}
\paragraph{}
Vom einem 6. Meteorstein fand ich eine erste Nachricht in Gilbert's Annalen der Physic Bd. XV. S. 317, wo angeführt wird, dass Casp. Schott in s. Physica curiosa l. XI Cap. XIX berichtet: „hac in urbe nostra Herbipolensi osservatur in templo D. Jacobi trans Moenum, in monasterio Scotorum\footnote{Das Schottenkloster war de 1140 gegründet, 1803 saecul. 1819 wurde ein Teil der Kirche zum Gottesdienst wiederhergerichtet, und zwar der Chor, das Übrige dient als Militärdepot. Ausf. Beschreibung u. Geschichte von Wieland im Archiv des hist. Vereins v. Unterfranken u. Asch. XVI. Bd.} catenulae columna templi suspensus... durissimus est et ad ferream vergit naturam.“ Daraus geht hervor, dass es wahrscheinlich ein Eisenmeteorit war. Ich habe mich, um den Spuren dieses Steines nachzuforschen an Herrn Prof. Sandberger in Würzburg gewendet, der so freundlich war, die gründlichsten Nachforschungen anzustellen. Der Stein ist verschwunden. Der gütigen Mittheilung Sandberger's verdank eiche die weitere Nachricht, welche Schnurrer in s. Seuchengeschichte Bd. II. gibt: „Im Jahre 1103 (oder 1104) fiel in Würzburg ein so großer Meteorstein, dass vier Männer den vierten Theil desselben kaum tragen konnten.“
\clearpage
\subsection{Der Meteorstein von Mauerkirchen}
(Beiliegende Tafel Figur I.)

Über diesen Fall berichtete zuerst ein kleines Schriftchen: Nachricht und Abhandlung von einem in Bayern unfern Maurkirchen d. 20. Nov. 1768 aus der Luft gefallenen Steine (Straubingen 1769). Aus demselben teilt Chladni in seine chronologischen Verzeichnisse der mit einem Feuermeteor niedergefallenen Stein- und Eisenmassen (Gilberts Ann. d. Phys. 1803 Bd. XV. S. 316) mit, dass an dem genannten Tage Abends nach 4 Uhr bei einem gegen Occident merklich verfinsterten Himmel verschiedene ehrliche Leute zu Maurkirchen, welche darüber eidlich vernommen wurden, ein ungewöhnliches Brausen und gewaltiges Krachen in der Luft gleich einem Donner und Schießen mit Stücken hörten. Unter diesem Luftgetümmel sei ein Stein aus der Luft gefallen und habe nach obrigkeitlichem Augenschein eine Grube 2 1/2 Schuh tief in die Erde gemacht. Der Stein halte nicht gar einen Schuh in die Länge, sei 6 Zoll breit und wiege 38 bayer. Pfunde Er sei von so weicher Materie, dass er sich mit Fingern zerreiben lasse, von Farbe bläulich mit einem weißen Fluss oder Fließerlein vermengt, außerdem mit einer schwarzen Rinde überzogen n. s. w.

Professor Imhof vervollständigte diesen Bericht (Kurpfalzbaier. Wochenblatt. 1804. St. 4) durch folgende Angaben: „Man fand den gefallenen Stein am Tage, nachdem man das Getöse vernommen hatte, in dem sog. Schinperpoint in einem schräg einwärts gehenden 2 1/2 Schuh tiefen Loche.“ Imhof bestimmte das spec. Gewicht zu 3,452 und beschreibt die graulich schwarze 1/4 Linie dicke Rinde als am Stahl funkengebend, ferner als Gemengteile
\begin{enumerate}
    \item regulinisches Eisen, das in kleinen Körnern und Zacken am meisten mit der äußeren Rinde verwachsen, sehr geschmeidig und zähe ist und einen weißen stark glänzenden Feilenstrich gibt,
    \item Schwefelkies,
    \item kleine plattgedrückte, eckige Körner, welche sich durch schwarzgraue Farbe, muschlichten Bruch, glänzendes Ansehen und größerer Härte von den ändern unterscheiden,
    \item noch andere kleine Körner von weißer und gelblicher Farbe, die durchscheinend und schimmernd sind. Nach seiner Analyse besteht der Meteorstein aus:
    \begin{center}
        \begin{tabular}{ |l|r| } 
        \hline
        Kieselsäure & 25,40\\\hline
        Eisenoxyd & 40,24\\\hline
        Eisen & 2,33\\\hline
        Nickel & 1,20\\\hline
        Bittererde & 28,75\\\hline
        Schwefel und Verlust & 2,08\\
        \hline
        \end{tabular}
    \end{center}
\end{enumerate}
\paragraph{}
(Vergl. O. Büchner die Meteoriten in Sammlungen 1863 S. 9.)

Die nähere Untersuchung des Steines ergab mir nun weiter, dass die mattschwarze, fleckenweis etwas Glänzende 0,7 — 0,3 mm. dicke Kruste wie bei anderen Meteorsteinen nur Schmelzrinde ist, welche ohne scharfe Grenze gegen Innen in die Hauptmasse übergeht, da wo Eisenteilchen an dieselbe grenzen, verstärkt, wo gewisse gelbe Körnchen in derselben liegen, schwächer und an letzteren Stellen glänzender sich zeigt. Häufig sind selbe Mineralteilchen eingeschmolzen und in der Rinde eingeschlossen oder ragen in dieselbe hinein. Die Hauptmasse des Steines ist lichtgrau gefärbt, durch eingestreutes Meteoreisen schwarz punktiert und an den meisten dieser schwarzen Stellen in Folge der Oxydation des Eisens fleckig rostfarbig. Zwischen den Fingern lässt sich der Stein ziemlich leicht zerdrücken und macht dem äußeren Anschein nach dem Eindruck eines Trachyttuffs.

Aus der äußerst feinbröcklichen, fast staubartigen Grundmasse heben sich ziemlich zahlreich eingestreute rundliche Mohn- bis Hirsekorn-große und kleinere Körnchen heraus, welche meist etwas dunkelschwärzlich oder gelblich gefärbt, außen matt, beim Zerschlagen glasglänzend ohne Spaltungsflächen erkennen zu lassen, den Charakter der Chondren besitzen und dem Stein daher den Stempel der Chondriten aufdrücken. Unter dem Mikroskop zeigen diese Körnchen eine verschiedene Beschaffenheit. Die einen sind äußerst fein parallel gestreift, so dass vorwaltend opake, breite Streifchen mit schmalen durchsichtigen oder durchscheinenden, wie quer gegliederten wechseln. I. p. L. erscheinen letztere mit matten feinfleckigen Farben. (y der Zeichnung der beiliegenden Tafel Fig. I. Andere Körnchen sind weißlich, wie aus feinstem Staub zusammengesetzt, opak, nur gegen den Rand zu etwas durchscheinend, zuweilen von feinsten, etwas durchschimmernden, einzelnen unregelmäßig eingestreuten Nädelchen durchzogen (x der Zeichnung). Noch andere Körnchen besitzen eine Art radiale Faserung, die jedoch hier nicht deutlich zum Vorschein kommt. Kleinste, rundliche Teilchen sind wasserhell und erscheinen i. p. L. mit glänzenden bunten Farben.

Neben den Chondren lassen sich in der pulverigen Hauptmasse eingebettet noch zahlreiche meist kleine eckige längliche Splitterchen eines weißen, auf der Spaltflächen deutlich spiegelnden, hier und da undeutlich parallel gestreiften Minerals und mehr rundlich eckige, unregelmäßig rissige, selten parallelstreifende Körnchen von gelblichem oder bräunlichem Farbenton und von glasartigem Glanze unterscheiden. Dazu gesellen sich metallisch glänzende, relativ kleine traubig eckige Klümpchen von Meteoreisen, ferner selten solche von messinggelbem Schwefeleisen und von nicht metallisch glänzenden tiefschwarzen Chromeisenstäbchen. An abgeriebenen Stellen des Steins stehen die härteren Körnchen hervor und lassen den Charakter den Chondriten deutlicher wahrnehmen, als auf dem Querbruche, auf dem man nur bei größerer Aufmerksamkeit die kugeligen Einlagerungen beobachtet. Die feinsten Staubteilchen, welche als das durch eine fortschreitende Zerkleinerung der größeren Splitter entstandene verbindende Material betrachtet werden müssen, sind teils wasserhell, teils opak, durchscheinend, und erweisen sich bis ins Kleinste i. p. L. durch wenn auch matte bunte Farben als doppelt brechende kristallinische Bruchstücke. Von einer glasartigen Zwischenmasse ist nicht eine Spur zu entdecken.

Nach dem Behandeln des fein zerdrückten (nicht zerriebenen) Materials mit Salpetersalzsäure und Kalilösung sind — abgesehen von den metallischen Gemengteilen — die gelblichen Splitterchen (Olivin) verschwunden und der Rückstand besteht nur aus weißen und bräunlichen Stücken, die unter dem Mikroskop sich leicht unterscheiden lassen. Die bräunlichen Fragmente sind stark rissig, selten mit Spuren von dunklen Parallelstreifchen versehen, durchsichtig und i. p. L. lebhaft buntfleckig gefärbt. Es sind zweifelsohne Teilchen eines Minerals aus der Augitgruppe. Die weißen Splitterchen dagegen sind vielfach nur durchscheinend, teilweise durch die Säuren angegriffen und zeigen i. p. L. nur matte fleckige Farbentöne, welche hier und da an eine streifige Anordnung erinnern. Dass diese Splitterchen als Feldspat-artige Gemengteile gedeutet werden müssen, beweist auch die chemische Analyse des Restanteils nach der Einwirkung der Säuren. Kleinste schwarze Teilchen sind als Chromeisen anzusprechen. Es besteht demnach der Stein aus Olivin, einem Feldspat-artigen, augitischen Mineral, aus Meteor-, Schwefel- und Chromeisen.

Damit stimmt nun auch im Allgemeinen die chemische Analyse, welche von Hrn. Assistent Ad. Schwager unter gleichzeitig kontrollierenden eigenen Untersuchungen durchgeführt wurde. Die Bestimmung des Meteoreisens und Schwefeleisens geschah durch eigene Versuche\footnote{Es wurde aus dem zerdrückten Pulver durch den Magnet alles Ausziehbare herausgenommen, und diese Meteoreisen haltigen Bestandteile unter Anwendung von Kupfervitriol und Kupferchlorid besonders analysiert.}. Die Analysen ergaben:
\begin{center}
\begin{tabular}{ |p{30mm}|p{35mm}|p{25mm}|p{15mm}| }
    \hline
    Stoffe: & Bauschanalyse 65,45\% durch\newline Salzsäure zersetzbarer Anteil & 34,55\%\newline Restbestandteil & \\
    \hline\hline
    Kieselsäure & 38,14 & 23,23 & 61,39\\\hline
    Tonerde & 2,51 & 1,20 & 5,00\\\hline
    Eisenoxydul & 25,70 & 32,72 & 17,59\\\hline
    Eisen mit Nickel & 6,30 & 9,65 & -,-\\\hline
    Schwefel & 2,09 & 3,20 & -,-\\\hline
    Phosphor & 0,14 & 0,22 & -,-\\\hline
    Chromoxyd & 0,39 & -,- & 0,84\\\hline
    Kalkerde & 2,27 & 1,51 & 4,35\\\hline
    Bittererde & 21,73 & 29,13 & 7,70\\\hline
    Kali & 0,48 & Sp. & 1,40\\\hline
    Natron & 1,00 & Sp. & 2,91\\\hline
    Summe & 100,75 & 100,86 & 101.18\\
    \hline
\end{tabular}
\end{center}
\paragraph{}
Es schließt sich demnach der Steinmeteorit von Mauerkirchen der Anfangsreihe der an Kieselsäure ärmsten Chondriten, wie jenen von Seres, Buchhof, Ensisheim und Chateau-Renard an. Es lässt sich daraus der Gehalt berechnen, nämlich an:
\begin{center}
    \begin{tabular}{ |l|r| } 
    \hline
    Meteoreisen & 2,81\%\\\hline
    Schwefeleisen & 5,72\\\hline
    Chromeisen & 0,75\\\hline
    Silikate & 90,72\\
    \hline
    \end{tabular}
\end{center}
\paragraph{}
Was die Interpretation der Silikate anbelangt, so haben wir zunächst den durch Salzsäure zersetzbaren Bestandteil ins Auge zu fassen. Hierin ist der relativ geringe Kieselsäuregehalt besonders auffallend. Doch wiederholt sich ein ähnliches Verhältnis« mehrfach wie z. B. bei den Meteorsteinen von Seres, Tjabé (Java 19. Sept. 1869), Khettre (Indien) u. A. Ziehen wir den Gehalt an Meteoreisen und Schwefeleisen ab, so erhalten wir für diesen Bestandteil:
\begin{center}
    \begin{tabular}{ |l|r| } 
    \hline
    SiO\textsubscript{2} & 26,45\\\hline
    Al\textsubscript{2}O\textsubscript{3} & 1,35\\\hline
    FeO & 37,30\\\hline
    CaO & 1,70\\\hline
    MgO & 33,20\\
    \hline
    \end{tabular}
\end{center}
\paragraph{}
Worin, wenn die Tonerde und Kalkerde als wahrscheinlich zu einem zersetzten Feldspat gerechnet und ein Teil des Eisenoxyduls als noch von Meteoreisen abstammend in Abzug gebracht wird, der durch Säuren zersetzte Bestandteil nicht anders, als zu Olivin gehörig sich auslegen lässt. Dass ein Teil des Eisens oxydiert ist und dadurch der Gehalt an Basen etwas gesteigert erscheint, darauf weisen schon den Rostflecken hin, welche sich manchmal selbst in der Masse ziemlich verbreitet zeigen.

Was das oder die Silikate des Restbestandteils angeht, so gibt der verhältnismäßig hohe Kieselsäure- und Tonerdegehalt, neben den Alkalien wohl der Vermutung Raum, dass neben einem Augit-Mineral auch noch ein feldspattiges vorhanden sei. Gleichwohl aber bleibt auch bei dieser Annahme noch ein starker Überschuss an Eieselsäure, von dem man wohl nicht voraussetzen darf, dass er in Form eines ausgeschiedenen Quarzminerals auftrete, weil bei Untersuchung des Dünnschliffs im reflektierten Lichte keine Spur einer Beimengung von durch den starken Glanz sonst erkennbarem Quarze sich bemerken lässt. Dieses Verhalten ist vorläufig noch unaufgeklärt.

Derselbe Meteorstein ist bereits in neuester Zeit auch noch einer chemischen Analyse von anderer Seite unterworfen worden. Rammelsberg führt (D. chem. Nat. d. Meteoriten Abh. d. Acad. d. Wiss. in Berlin für 1870 S. 148 u. ff.) als das Resultat der von Crook\footnote{On the chem. constit. of meteor. stones. Göttingen Dissert. (Mir nicht zugänglich).} ausgeführten Untersuchung an: Zusammensetzung:
\begin{center}
    \begin{tabular}{ |l|r| } 
    \hline
    3,52\% & Meteoreisen\\\hline
    1,92\% & Schwefeleisen\\\hline
    0,72\% & Chromeisen\\\hline
    92,68\% & Silikat\\
    \hline
    \end{tabular}
\end{center}
100,00 und zwar:
das Silikat bestehend als:
\begin{center}
\begin{tabular}{ |p{20mm}|p{24mm}|p{31mm}|p{32mm}| }
    \hline
    Stoffe: & im Ganzen Bauschanalyse & in dem 61\% durch Säuren zersetzbar. Anteil. & in dem 39\%\newline in Säuren unzersetzb. Anteil.\\
    \hline\hline
    Kieselsäure & 44,81 & 32,68 & 3,94\\\hline
    Tonerde & 1,24 & 9,36 & 4,17\\\hline
    Eisenoxydul & 24,55 & 28,91 & 17,71\\\hline
    Bittererde & 26,10 & 37,44 & 8,20\\\hline
    Kalkerde & 2,28 & 0,61 & 4,91\\\hline
    Natron & 0,26 & - & 0,67\\\hline
    Kali & 0,16 & - & 0,40\\
    \hline
\end{tabular}
\end{center}
\paragraph{}
Diese Resultate weichen so bedeutend von den früher mitgeteilten ab, dass dafür kein anderer Grund gefunden werden kann, als die an sich große Ungleichheit in der Zusammensetzung des Meteorsteins, welche einen umso größeren Einfluss auf die Ergebnisse der Untersuchung zu äußern im Stande ist, mit je kleineren Quantitäten man zu arbeiten gezwungen ist. Die mikroskopische Untersuchung der Dünnschliffe unterstützt direkt diese Annahme, indem sich hierbei die größte Unregelmäßigkeit in der Art der Verteilung der Gemengteile erkennen lässt. Ein größeres Korn von diesem oder jenem Gemengteil in der verwendeten Probe verrückt bei geringen Quantitäten, die man benützt, die Zahlen in beträchtlicher Weise. Es lassen sich beispielsweise zackige Knöllchen von Meteoreisenteilchen aus der Masse herauslösen, deren Größe in keinem Verhältnisse steht zu dem geringen Prozentgehalte des Steins an Meteoreisen im Allgemeinen und Ganzen. Ähnlich verhält es sich mit den eingestreuten härteren Knöllchen und Körnchen.

Besonders verschieden ist die Angabe bezüglich der Zusammensetzung des in Salzsäure zersetzbaren Gemengteils. Doch tritt auch in der Analyse Crook's die relativ geringe Menge von Kieselsäure sehr deutlich hervor. Minder abweichend erweisen sich die Resultate der Analyse des durch Säuren unzersetzten Restes. Gerade dies beweist, dass es nicht in dem Gang der analytischen Arbeit liegt, wie es scheinen könnte, wenn hier der Kieselsäurengehalt ebenso verhältnismäßig hoch, wie bei dem in Säuren zersetzbaren Anteil gering gefunden wurde. Da dieser Rest, wie die mikroskopische Untersuchung desselben lehrt, aus verschiedenen Mineralsubstanzen, namentlich einem weißen und einem braunen Gemengteil besteht, so kann das Sauerstoff-Verhältnis im Ganzen genommen, uns keine besonderen Aufschlüsse verschaffen.

Die wegen der leichten Zerreiblichkeit der Masse schwierig herzustellenden Dünnschliffe, welche nur durch wiederholtes Tränken mit sehr verdünntem Canadabalsam in brauchbarem Zustande gewonnen werden können, geben, wie es das Dünnschliffbild auf der beiliegenden Tafel in Figur I. zeigt, bezüglich der Zusammensetzung des Gesteins und der Verteilung der Gemengteile einige lehrreiche Aufschlüsse. Es stechen besonders die Chondren in ihrer teils staubig krümeligen, teils faserigen Zusammensetzung besonders hervor. Trotz der geringen Durchsichtigkeit derselben erweisen sie sich i. p. L. betrachtet stets farbig, und zwar nicht bloß die lichteren Streifchen derselben, sondern ihre ganze Masse. Diesen Einmengungen gegenüber sind die übrigen unterscheidbaren, stets unregelmäßig umgrenzten, gelblichen, bräunlichen und weißlichen Splitterchen klein. Sie sind alle von zahllosen Bissen durchzogen, die nur hier und da parallel verlaufen. Kleine Stückchen und Staubteilchen der anscheinend gleichen Mineralien bilden die Grundmasse, in welchen die größeren Trümmer eingestreut liegen. I. p. L. treten bis in die feinsten Teilchen Farbenerscheinungen hervor, so dass auch in den Dünnschliffen die Abwesenheit einer glasartigen Bindemasse bestimmt beobachtet werden kann. Bemerkenswert sind zahlreiche kleinste, runde, wasserhelle Körnchen, welche der Grundmasse beigemengt sind. Meteoreisen- und Schwefeleisen-Knöllchen teilen etwa die Größe der Mineralsplitterchen, machen jedoch ihren Umrissen nach nicht den Eindruck der Zertrümmerung, wie letztere und liegen ziemlich gleichmäßig in der Masse zerstreut. Wir sehen also, dass der Meteorstein von Mauerkirchen seiner Struktur nach sich nicht wesentlich von anderen chondritischen Meteorsteinen unterscheidet.
\clearpage
\subsection{Der Meteorstein von Eichstädt}
\paragraph{}
(Figur II.)

Über den Fall dieses Steins wird berichtet, dass ein Arbeiter an einer Ziegelhütte im sog. Wittmes, einer waldigen Gegend, etwa 5 Kil. westwärts von Eichstädt am 19. Feb. 1785 Nachmittags zwischen 12 und 1 Uhr nach einem donnerähnlichen Getöse einen großen schwarzen Stein auf den mit Schnee bedeckten Erdboden, auf dem Ziegelsteine umher lagen, fallen sah. Als er zur Stelle lief, fand er den Stein, welcher einen Ziegelstein zertrümmert hatte, eine Hand tief im Boden und so heiß, dass er ihn erst mit Schnee abkühlen musste, um ihn an sich nehmen zu können. Der Stein hatte etwa ein Fuß im Durchmesser und wog beiläufig 3 Kilogramm. Schafhäutl (Gelehrt. Anzeige d. Ac. d. Wiss. in München 1847 S. 559.) beschreibt denselben wie folgt: „Seine Struktur ist ziemlich grobkörnig, die Körner sind rundlicher, als dies bei allen übrigen Aerolithen der Fall ist; ja es finden sich sogar vollkommen elliptische, wie abgeschliffen aussehende Körnchen von graulicher Farbe und dichtem ziemlich mattem ebenem Bruche darin, ohne bemerkbares kristallinisches Gefüge. Neben diesen liegen grünliche olivinartige Körner von glasig muscheligem Bruche. Schwefeleisen, Nickeleisen und Magneteisen sind zwischen diesen Körnern eingesprengt, so dass er unter allen Meteorsteinen unserer Sammlung (Münchner Staats-S.) am stärksten auf die Magnetnadel wirkt.“

Das spez. Gewicht\footnote{Vergl. Moll's Annal, d. Berg- u. Hüttenk. Bd. III. S. 251.} wird angegeben:

von Schreibers zu 3,700

von Rumler zu 3,599

Klaproth hat diesen Stein analysiert und gibt (Gilberts Ann. XIII. 338) als seine Bestandteile an:
\begin{center}
    \begin{tabular}{ |l|r| } 
    \hline
    Gediegen Eisen & 19,00\\\hline
    Nickelmetall & 1,50\\\hline
    Braunes Eisenoxyd & 16,50\\\hline
    Bittersalzerde & 21,50\\\hline
    Kieselerde & 37,00\\\hline
    Verlust (mit Schwefel) & 4,50\\
    \hline
    \end{tabular}
\end{center}
\paragraph{}
Das in der Münchener Staatssammlang verwahrte Stück zeigt eine schwarze mattglänzende, runzelige Rinde und eine weißlich graue, grobkörnig chondritische, durch zahlreiche Rostflecken hier und da gelblich getüpfelte, leicht zerreibliche Hauptmasse, aus welcher sich die oft sehr großen Chondren leicht heraus lösen lassen. Es finden sich solche bis über 3 mm. im Durchmesser groß, sie sind sehr hart, auf der Oberfläche matt, erdbeerenartig höckerig und grubig in einer Weise, dass die angeschlossenen Mineralsplitterchen der Hauptmasse wie an die Oberfläche gekittet erscheinen. An vielen Stellen der Oberfläche bemerkt man zudem kleinen spiegelnde Streifchen, wodurch dieselben gleichsam facettiert erscheinen. Auch kommen damit fest verwachsene Meteoreisenteilchen vor, welche zuweilen selbst in die Oberfläche versenkt sind. Niemals zeigt sich eine Glättung der Oberfläche, wie sie Vorkommen müsste, wenn die Kügelchen durch Reibung und Abrollung entstanden wären. Vielmehr gleichen sie der äußeren Beschaffenheit nach den in den Schlacken vorkommenden Roheisensteinkügelchen. Zerschlägt man sie, so zeigen sie auf der flachmuscheligen Bruchfläche, einen matten Glasglanz, schwärzlichgraue Farbe und bei weit Zertrümmerung unter dem Mikroskop erweisen sie sich nicht als eine homogene, sondern zusammengesetzte Masse. Man kann deutlich einen glashellen mit zahl- i reichen Bläschen erfüllten, i. p. L. ungemein buntfarbigen Bestandteil neben einer nur durchscheinend trüben, wie aus kleinsten Staubteilchen zusammengesetzten, aber i. p. L. doch deutlich farbigen, zuweilen feinstreifigen Hauptmasse und einzelnen durchscheinenden intensiv gelbbraunen, i. p. L. unverändert gefärbten Streifchen unterscheiden. In Dünnschliffen sieht man ihre Struktur noch viel deutlicher, obwohl sie hier in einer an sich sehr dunkelgefärbten Hauptmasse liegen und schwierig gut durchsichtig zu erhalten sind. Indem nämlich ziemlich viel Meteoreisen als Gemengteil auftritt, dass großenteils bereits etwas zersetzt und mit einem Höfchen von gelbbrauner Farbe umgeben ist, leidet auch die Klarheit derjenigen Mineralteilchen, welche sonst durch ihre Durchsichtigkeit sich auszeichnen. Die gelbe Farbe rührt von Eisenoxydhydrat her, welches durch die Einwirkung der feuchten Luft unserer Atmosphäre auf das Meteoreisen erst nachträglich während der Zeit sich gebildet hat, in welcher der Stein in der Erde oder in unseren Sammlungen gelegen hat. Dieses Eisenoxydhydrat dringt in die feinsten Risschen und Sprünge oder Zwischenräume ein, kann aber leicht durch Säuren entfernt werden. Neben dem Meteoreisen beteiligen sich unregelmäßig eingesprengte, selten von parallelen Linien eingeschlossene Mineralsplitterchen an dem Haufwerk, aus dem der Meteorstein besteht. Bald sind es wasserhelle, wenig rissige Trümmerchen, bald solche, welche durch ein einfaches System parallelen Linien gestreift oder von unten schiefen Winkeln sich schneidenden Rissen zerklüftet sind, etwa wie es bei dem Augit vorzukommen pflegt, oder aber durch eine dem Zellnetz gewisser Moosblättchen ähnliche, merkwürdig langgezogene und quergegliederte Maschenstruktur (d) sich auszeichnen. Zuweilen stoßen in einem Trümmerteil mehrere Systeme solcher paralleler Streifchen zusammen. Zwischen diesen größeren Fragmenten liegen kleinere ganz von derselben Beschaffenheit, wie die größeren angehäuft. I. p. L. erscheinen alle Teilchen, welche nur überhaupt durchsichtig sind, in bunten Farben, welche selbst innerhalb der einzelnen Splitter aggregatartig verteilt sind und selten streifig oder bandartig parallel verlaufen. Endlich sind als ungemein häufige Bestandteile die kugeligen Einschlüsse zu nennen, die schon erwähnt worden sind. Aus den mannichfachen Formen, welche dieselben besitzen, heben wir nur einige der am häufigsten vorkommenden hervor. Ziemlich zahlreich sind die Chondren mit exzentrisch strahlig faserigem Gefüge (a), welches in der Regel von einer nahe am Rande liegenden mehr körnigen Partie ausgeht und in einem vielfach abgesetzten, gleichfalls maschenartigen und quergegliederten Strahlenbüschel ausläuft. Diese Struktur stimmt so sehr mit jener schon geschilderten überein, welchen wir auf ändern regelmäßig umgrenzten Splitterchen begegnen, dass wir letztere wohl als Abkömmlinge zerbrochener größerer Chondren ansehen müssen. Andere der letzteren sind von verschiedenen Systemen sich unter spitzen und stumpfen Winkeln schneidender dunkler Streifchen beherrscht (b), eine Struktur, die sich als der Anfang einer kristallinischen periodenweis gestörten Ausbildung betrachten lässt. In noch anderen Chondren kommt eine staubartig trübe, schwach durchscheinende Substanz vor, in welcher häufig sehr zahlreiche dicht gedrängte, hellere, gruppenweis nach verschiedenen Richtungen verlaufende Streifchen (c) sich bemerkbar machen. Endlich treten nicht selten Kügelchen auf, welche aus größeren, helleren, durch dunkle Zwischenstreifchen voneinander getrennten Körnchen (e) gleichsam zusammengebacken erscheinen. Aus alle dem geht zur Genüge hervor, dass wir in dem Stein von Eichstädt einem Chondriten der ausgezeichnetsten Art vor uns haben. Derselbe kann geradezu als Typus dieser Art der Struktur, welche bei den Meteorsteinen als der vorherrschende bekannt ist, gelten.

Was seine Zusammensetzung anbelangt, so hat die Analyse (Ass. A. Schwager) ergeben, dass der Stein besteht aus:

22,98 Meteoreisen,

3,82 Schwefeleisen,

32,44 in Salzsäure zersetzbaren,

40,76 in Salzsäure nicht zersetzbaren Mineralien.
\paragraph{}
Die Zusammensetzung ist im Ganzen A, dann

B in den durch CI H zersetzbaren Silicaten

C in dem durch CI H nicht zersetzbaren Bestandteil:
\begin{center}
\begin{tabular}{ |p{35mm}|p{20mm}|p{20mm}|p{20mm}| }
    \hline
    & A. & B. & C.\\
    \hline\hline
    Kieselerde & 33,31 & 34,45 & 55,53\\\hline
    Tonerde & 2,31 & 0,86 & 5,13\\\hline
    Eisenoxydul & 15,34 & 24,52 & 16,66\\\hline
    Eisen (mit Phosphor) & 24,64 & - & -\\\hline
    Nickel & 0,94 & - & -\\\hline
    Kalkerde & 0,74 & 0,68 & 1,13\\\hline
    Schwefel & 1,42 & - & -\\\hline
    Chromoxyd & 0,15 & - & 0,73\\\hline
    Bittererde & 18,86 & 37,31 & 19,34\\\hline
    Kali & 0,40 & 0,68 & 0,56\\\hline
    Natron & 1,04 & 1,31 & 1,62\\\hline
    & 99,15 & 99,81 & 100,70\\
    \hline
\end{tabular}
\end{center}
\paragraph{}
Der Gehalt der durch Salzsäure zersetzbaren Gemengteile an Alkalien weist außer Olivin noch auf einen Feldspat hin. Wir haben aber darin:
\begin{center}
    \begin{tabular}{ |l|r| } 
    \hline
    SiO\textsubscript{2} & 34,45 mit 18,37 O\\\hline
    Al\textsubscript{2}O\textsubscript{3} & 0,86 mit 0,40\\\hline
    FeO & 24,52 mit 5,45\\\hline
    MgO & 37,31 mit 14,90\\\hline
    CaO & 0,68 mit 0,19\\\hline
    Ka\textsubscript{2}O & 0,68 mit 0,11\\\hline
    Na\textsubscript{2}O & 1,31 mit 0,34\\
    \hline
    \end{tabular}
\end{center}
\paragraph{}
Daraus er sieht man, dass, wenn wir ein Singulosilikat ausscheiden, die vorhandene Sauerstoffmenge noch nicht einmal vollständig ausreicht, den Bedarf ganz zu decken, dass mithin die Analyse uns keinen Aufschluss über die Natur des etwa noch außer Olivin vorhandenen Silikats weitergibt.

In dem von Säuren nicht zersetzbaren Best endlich stellen sich die Verhältnisse folgender Maassen:
\begin{center}
\begin{tabular}{ |l|c|r| }
    \hline
    Kieselerde & 55,53 & mit 29,62 O = 22,6 + 7\\\hline
    Eisenoxydul & 16,66 & mit 3,70 O = 3,58 + 0,12\\\hline
    Bittererde & 19,34 & mit 7,73 O\\\hline
    Chromoxyd & 0,73 & mit 0,23 O\\\hline
    Tonerde & 5,13 & mit 2,39 O = 2,33 + 0.06\\\hline
    Kalkerde & 1,13 & mit 0,32 O\\\hline
    Kali & 0,56 & mit 0,10 O\\\hline
    Natron & 1,62 & mit 0,42 O\\
    \hline
\end{tabular}
\end{center}
\paragraph{}
Daraus berechnet sich ein Bisilikat, Chromeisen (von der Zusammensetzung des von L'Aigle) und ein Andesin-artiger Feldspat ungefähr in dem Verhältnis wie 79:1:21.

Im Ganzen besteht also der Eichstädter Meteorstein ungefähr aus:
\clearpage
\section{Abbildungen}
\clearpage
\pagestyle{fancy}
\fancyhf{}
\rhead{Figur 1}
\cfoot{\thepage}
\begin{figure}[t]
%\includegraphics[width=\textwidth,height=\textheight,keepaspectratio]{fig1.jpeg}
\centering
\end{figure}
\clearpage
\rhead{Figur 2}
\begin{figure}[t]
\centering
%\includegraphics[width=\textwidth,height=\textheight,keepaspectratio]{fig2.jpeg}
\end{figure}
\clearpage
\end{document}
