\documentclass[a4paper, 11pt, oneside]{article}
\usepackage[utf8]{inputenc}
\usepackage[T1]{fontenc}
\usepackage[ngerman]{babel}
\usepackage{fbb} %Derived from Cardo, provides a Bembo-like font family in otf and pfb format plus LaTeX font support files
\usepackage{booktabs}
\setlength{\emergencystretch}{15pt}
\usepackage{fancyhdr}
\usepackage{graphicx}
\graphicspath{ {./} }
\begin{document}
\begin{titlepage} % Suppresses headers and footers on the title page
	\centering % Centre everything on the title page
	%\scshape % Use small caps for all text on the title page

	%------------------------------------------------
	%	Title
	%------------------------------------------------
	
	\rule{\textwidth}{1.6pt}\vspace*{-\baselineskip}\vspace*{2pt} % Thick horizontal rule
	\rule{\textwidth}{0.4pt} % Thin horizontal rule
	
	\vspace{1\baselineskip} % Whitespace above the title
	
	{\scshape\LARGE Sitzungsberichte der\\[1.25pt] Mathematisch-Physikalischen Klasse der\\[1.25pt] Königlich Bayerischen Akademie der\\[1.25pt] Wissenschaften zu München\\[1.25pt]}
	
	\vspace{1\baselineskip} % Whitespace above the title

	\rule{\textwidth}{0.4pt}\vspace*{-\baselineskip}\vspace{3.2pt} % Thin horizontal rule
	\rule{\textwidth}{1.6pt} % Thick horizontal rule
	
	\vspace{1\baselineskip} % Whitespace after the title block
	
	%------------------------------------------------
	%	Subtitle
	%------------------------------------------------
	
	{\scshape Band 8 --- Jahrgang 1878} % Subtitle or further description
	
	\vspace*{1\baselineskip} % Whitespace under the subtitle
	
    {\scshape\small In Kommission bei G. Franx} % Subtitle or further description
    
	%------------------------------------------------
	%	Editor(s)
	%------------------------------------------------
    \vspace*{\fill}

	\vspace{1\baselineskip}

	{\small\scshape München 1878}
	
	{\small\scshape{Akademische Buchdruckerei von F. Straub}}
	
	\vspace{0.5\baselineskip} % Whitespace after the title block

    \scshape Internet Archive Online Edition  % Publication year
	
	{\scshape\small Namensnennung Nicht-kommerziell Weitergabe unter gleichen Bedingungen 4.0 International} % Publisher
\end{titlepage}
\setlength{\parskip}{1mm plus1mm minus1mm}
\clearpage
\tableofcontents
\clearpage
\section{Sitzung vom 9. Februar 1878 --- Herr Gümbel spricht: Über die in Bayern gefundenen Steinmeteoriten}
\subsection*{Einleitung}
\paragraph{}
Unter den auf bayerische Gebiete gefallenen und aufgefundenen Steinmeteoriten befinden sich mehrere, deren chemische Zusammensetzung uns nur aus älteren Analysen bekannt ist, während von einem derselben bis jetzt überhaupt noch keine chemische Untersuchung vorgenommen wurde. Da es außerdem ihre den meisten derselben an einer erschöpfenden Untersuchung, Zwie solche neuerdings bei Gesteinsarten mittelst Dünnschliffe und Mikroskop vorgenommen zu werden pflegt, fehlt, so schien es mir interessant genug, diese Arbeit vorzunehmen und die Ergebnisse mit dem früher bekannten zusammenzustellen. Durch die besondere Güte des Herrn Konservators der mineralogischen Staatssammlung Professor Dr. v. Kobell habe ich das hierzu erforderliche Material erhalten und ich benütze gerne die Gelegenheit, für diese so freundliche Unterstützung meiner Untersuchung hier den besten Dank auszudrücken. Einige weitere Bemerkungen, welche an den Schlüssen beigefügt sind, beziehen sich auf andere Meteorsteine, die ich gelegentlich der Vergleichung wegen in den Kreis meiner Beobachtung gezogen habe.

Es wurden im Ganzen nur 5 Steinmeteoriten von denen, welche in Bayern gefallen sind, bekannt. Darunter ist sogar noch ein Fund einbegriffen, welcher nach dem gegenwärtigen Territorialverhältnisse nicht mehr Bayern, sondern Österreich angehört, nämlich jener von Mauerkirchen. Da jedoch zur Zeit des Falls der Ort zu Bayern gehörte, so dürfte es immerhin bis zu einer gewissen Grade gerechtfertigt erscheinen, diesen Stein hier unter den bayerischen aufzuführen.

Diese 5 Steinmeteorite sind:
\begin{enumerate}
    \item Der Stein von Mauerkirchen im jetzt österreichischen Innviertel vom Falle am 20. Nov. 1768 Nach-, mittags 4 Uhr.
    \item Der Stein von Eichstädt, welcher im sog. Wittmes 5 Kilom. von der Stadt am 19. Febr. 1785 nach 12 Uhr Mittags gefallen ist.
    \item Der Stein von Massing bei Altötting in Südbayern vom Fall am 13. Dezember 1803 zwischen 10-11 Uhr Vormittags.
    \item Der Stein von Schönenberg bei Burgau und Schwaben, gefallen am 25. Dez. 1846 Nachmittags 2 Uhr und
    \item Der Stein von Krähenberg bei Homburg in der Rheinpfalz vom Fall am 5. Mai 1869 Abends 6 1/2 Uhr.
\end{enumerate}
\paragraph{}
Vom einem 6. Meteorstein fand ich eine erste Nachricht in Gilbert's Annalen der Physic Bd. XV. S. 317, wo angeführt wird, dass Casp. Schott in s. Physica curiosa l. XI Cap. XIX berichtet: „hac in urbe nostra Herbipolensi osservatur in templo D. Jacobi trans Moenum, in monasterio Scotorum\footnote{Das Schottenkloster war de 1140 gegründet, 1803 saecul. 1819 wurde ein Teil der Kirche zum Gottesdienst wiederhergerichtet, und zwar der Chor, das Übrige dient als Militärdepot. Ausf. Beschreibung u. Geschichte von Wieland im Archiv des hist. Vereins v. Unterfranken u. Asch. XVI. Bd.} catenulae columna templi suspensus... durissimus est et ad ferream vergit naturam.“ Daraus geht hervor, dass es wahrscheinlich ein Eisenmeteorit war. Ich habe mich, um den Spuren dieses Steines nachzuforschen an Herrn Prof. Sandberger in Würzburg gewendet, der so freundlich war, die gründlichsten Nachforschungen anzustellen. Der Stein ist verschwunden. Der gütigen Mittheilung Sandberger's verdank eiche die weitere Nachricht, welche Schnurrer in s. Seuchengeschichte Bd. II. gibt: „Im Jahre 1103 (oder 1104) fiel in Würzburg ein so großer Meteorstein, dass vier Männer den vierten Theil desselben kaum tragen konnten.“
\clearpage
\subsection{Der Meteorstein von Mauerkirchen}
(Beiliegende Tafel Figur I.)

Über diesen Fall berichtete zuerst ein kleines Schriftchen: Nachricht und Abhandlung von einem in Bayern unfern Maurkirchen d. 20. Nov. 1768 aus der Luft gefallenen Steine (Straubingen 1769). Aus demselben teilt Chladni in seine chronologischen Verzeichnisse der mit einem Feuermeteor niedergefallenen Stein- und Eisenmassen (Gilberts Ann. d. Phys. 1803 Bd. XV. S. 316) mit, dass an dem genannten Tage Abends nach 4 Uhr bei einem gegen Occident merklich verfinsterten Himmel verschiedene ehrliche Leute zu Maurkirchen, welche darüber eidlich vernommen wurden, ein ungewöhnliches Brausen und gewaltiges Krachen in der Luft gleich einem Donner und Schießen mit Stücken hörten. Unter diesem Luftgetümmel sei ein Stein aus der Luft gefallen und habe nach obrigkeitlichem Augenschein eine Grube 2 1/2 Schuh tief in die Erde gemacht. Der Stein halte nicht gar einen Schuh in die Länge, sei 6 Zoll breit und wiege 38 bayer. Pfunde Er sei von so weicher Materie, dass er sich mit Fingern zerreiben lasse, von Farbe bläulich mit einem weißen Fluss oder Fließerlein vermengt, außerdem mit einer schwarzen Rinde überzogen n. s. w.

Professor Imhof vervollständigte diesen Bericht (Kurpfalzbaier. Wochenblatt. 1804. St. 4) durch folgende Angaben: „Man fand den gefallenen Stein am Tage, nachdem man das Getöse vernommen hatte, in dem sog. Schinperpoint in einem schräg einwärts gehenden 2 1/2 Schuh tiefen Loche.“ Imhof bestimmte das spec. Gewicht zu 3,452 und beschreibt die graulich schwarze 1/4 Linie dicke Rinde als am Stahl funkengebend, ferner als Gemengteile
\begin{enumerate}
    \item regulinisches Eisen, das in kleinen Körnern und Zacken am meisten mit der äußeren Rinde verwachsen, sehr geschmeidig und zähe ist und einen weißen stark glänzenden Feilenstrich gibt,
    \item Schwefelkies,
    \item kleine plattgedrückte, eckige Körner, welche sich durch schwarzgraue Farbe, muschlichten Bruch, glänzendes Ansehen und größerer Härte von den ändern unterscheiden,
    \item noch andere kleine Körner von weißer und gelblicher Farbe, die durchscheinend und schimmernd sind. Nach seiner Analyse besteht der Meteorstein aus:
    \begin{center}
        \begin{tabular}{ |l|r| } 
        \hline
        Kieselsäure & 25,40\\\hline
        Eisenoxyd & 40,24\\\hline
        Eisen & 2,33\\\hline
        Nickel & 1,20\\\hline
        Bittererde & 28,75\\\hline
        Schwefel und Verlust & 2,08\\
        \hline
        \end{tabular}
    \end{center}
\end{enumerate}
\paragraph{}
(Vergl. O. Büchner die Meteoriten in Sammlungen 1863 S. 9.)

Die nähere Untersuchung des Steines ergab mir nun weiter, dass die mattschwarze, fleckenweis etwas Glänzende 0,7 — 0,3 mm. dicke Kruste wie bei anderen Meteorsteinen nur Schmelzrinde ist, welche ohne scharfe Grenze gegen Innen in die Hauptmasse übergeht, da wo Eisenteilchen an dieselbe grenzen, verstärkt, wo gewisse gelbe Körnchen in derselben liegen, schwächer und an letzteren Stellen glänzender sich zeigt. Häufig sind selbe Mineralteilchen eingeschmolzen und in der Rinde eingeschlossen oder ragen in dieselbe hinein. Die Hauptmasse des Steines ist lichtgrau gefärbt, durch eingestreutes Meteoreisen schwarz punktiert und an den meisten dieser schwarzen Stellen in Folge der Oxydation des Eisens fleckig rostfarbig. Zwischen den Fingern lässt sich der Stein ziemlich leicht zerdrücken und macht dem äußeren Anschein nach dem Eindruck eines Trachyttuffs.

Aus der äußerst feinbröcklichen, fast staubartigen Grundmasse heben sich ziemlich zahlreich eingestreute rundliche Mohn- bis Hirsekorn-große und kleinere Körnchen heraus, welche meist etwas dunkelschwärzlich oder gelblich gefärbt, außen matt, beim Zerschlagen glasglänzend ohne Spaltungsflächen erkennen zu lassen, den Charakter der Chondren besitzen und dem Stein daher den Stempel der Chondriten aufdrücken. Unter dem Mikroskop zeigen diese Körnchen eine verschiedene Beschaffenheit. Die einen sind äußerst fein parallel gestreift, so dass vorwaltend opake, breite Streifchen mit schmalen durchsichtigen oder durchscheinenden, wie quer gegliederten wechseln. I. p. L. erscheinen letztere mit matten feinfleckigen Farben. (y der Zeichnung der beiliegenden Tafel Fig. I. Andere Körnchen sind weißlich, wie aus feinstem Staub zusammengesetzt, opak, nur gegen den Rand zu etwas durchscheinend, zuweilen von feinsten, etwas durchschimmernden, einzelnen unregelmäßig eingestreuten Nädelchen durchzogen (x der Zeichnung). Noch andere Körnchen besitzen eine Art radiale Faserung, die jedoch hier nicht deutlich zum Vorschein kommt. Kleinste, rundliche Teilchen sind wasserhell und erscheinen i. p. L. mit glänzenden bunten Farben.

Neben den Chondren lassen sich in der pulverigen Hauptmasse eingebettet noch zahlreiche meist kleine eckige längliche Splitterchen eines weißen, auf der Spaltflächen deutlich spiegelnden, hier und da undeutlich parallel gestreiften Minerals und mehr rundlich eckige, unregelmäßig rissige, selten parallelstreifende Körnchen von gelblichem oder bräunlichem Farbenton und von glasartigem Glanze unterscheiden. Dazu gesellen sich metallisch glänzende, relativ kleine traubig eckige Klümpchen von Meteoreisen, ferner selten solche von messinggelbem Schwefeleisen und von nicht metallisch glänzenden tiefschwarzen Chromeisenstäbchen. An abgeriebenen Stellen des Steins stehen die härteren Körnchen hervor und lassen den Charakter den Chondriten deutlicher wahrnehmen, als auf dem Querbruche, auf dem man nur bei größerer Aufmerksamkeit die kugeligen Einlagerungen beobachtet. Die feinsten Staubteilchen, welche als das durch eine fortschreitende Zerkleinerung der größeren Splitter entstandene verbindende Material betrachtet werden müssen, sind teils wasserhell, teils opak, durchscheinend, und erweisen sich bis ins Kleinste i. p. L. durch wenn auch matte bunte Farben als doppelt brechende kristallinische Bruchstücke. Von einer glasartigen Zwischenmasse ist nicht eine Spur zu entdecken.

Nach dem Behandeln des fein zerdrückten (nicht zerriebenen) Materials mit Salpetersalzsäure und Kalilösung sind — abgesehen von den metallischen Gemengteilen — die gelblichen Splitterchen (Olivin) verschwunden und der Rückstand besteht nur aus weißen und bräunlichen Stücken, die unter dem Mikroskop sich leicht unterscheiden lassen. Die bräunlichen Fragmente sind stark rissig, selten mit Spuren von dunklen Parallelstreifchen versehen, durchsichtig und i. p. L. lebhaft buntfleckig gefärbt. Es sind zweifelsohne Teilchen eines Minerals aus der Augitgruppe. Die weißen Splitterchen dagegen sind vielfach nur durchscheinend, teilweise durch die Säuren angegriffen und zeigen i. p. L. nur matte fleckige Farbentöne, welche hier und da an eine streifige Anordnung erinnern. Dass diese Splitterchen als Feldspat-artige Gemengteile gedeutet werden müssen, beweist auch die chemische Analyse des Restanteils nach der Einwirkung der Säuren. Kleinste schwarze Teilchen sind als Chromeisen anzusprechen. Es besteht demnach der Stein aus Olivin, einem Feldspat-artigen, augitischen Mineral, aus Meteor-, Schwefel- und Chromeisen.

Damit stimmt nun auch im Allgemeinen die chemische Analyse, welche von Hrn. Assistent Ad. Schwager unter gleichzeitig kontrollierenden eigenen Untersuchungen durchgeführt wurde. Die Bestimmung des Meteoreisens und Schwefeleisens geschah durch eigene Versuche\footnote{Es wurde aus dem zerdrückten Pulver durch den Magnet alles Ausziehbare herausgenommen, und diese Meteoreisen haltigen Bestandteile unter Anwendung von Kupfervitriol und Kupferchlorid besonders analysiert.}. Die Analysen ergaben:
\begin{center}
\begin{tabular}{ |p{30mm}|p{35mm}|p{25mm}|p{15mm}| }
    \hline
    Stoffe: & Bauschanalyse 65,45\% durch\newline Salzsäure zersetzbarer Anteil & 34,55\%\newline Restbestandteil & \\
    \hline\hline
    Kieselsäure & 38,14 & 23,23 & 61,39\\\hline
    Tonerde & 2,51 & 1,20 & 5,00\\\hline
    Eisenoxydul & 25,70 & 32,72 & 17,59\\\hline
    Eisen mit Nickel & 6,30 & 9,65 & -,-\\\hline
    Schwefel & 2,09 & 3,20 & -,-\\\hline
    Phosphor & 0,14 & 0,22 & -,-\\\hline
    Chromoxyd & 0,39 & -,- & 0,84\\\hline
    Kalkerde & 2,27 & 1,51 & 4,35\\\hline
    Bittererde & 21,73 & 29,13 & 7,70\\\hline
    Kali & 0,48 & Sp. & 1,40\\\hline
    Natron & 1,00 & Sp. & 2,91\\\hline
    Summe & 100,75 & 100,86 & 101.18\\
    \hline
\end{tabular}
\end{center}
\paragraph{}
Es schließt sich demnach der Steinmeteorit von Mauerkirchen der Anfangsreihe der an Kieselsäure ärmsten Chondriten, wie jenen von Seres, Buchhof, Ensisheim und Chateau-Renard an. Es lässt sich daraus der Gehalt berechnen, nämlich an:
\begin{center}
    \begin{tabular}{ |l|r| } 
    \hline
    Meteoreisen & 2,81\%\\\hline
    Schwefeleisen & 5,72\\\hline
    Chromeisen & 0,75\\\hline
    Silikate & 90,72\\
    \hline
    \end{tabular}
\end{center}
\paragraph{}
Was die Interpretation der Silikate anbelangt, so haben wir zunächst den durch Salzsäure zersetzbaren Bestandteil ins Auge zu fassen. Hierin ist der relativ geringe Kieselsäuregehalt besonders auffallend. Doch wiederholt sich ein ähnliches Verhältnis« mehrfach wie z. B. bei den Meteorsteinen von Seres, Tjabé (Java 19. Sept. 1869), Khettre (Indien) u. A. Ziehen wir den Gehalt an Meteoreisen und Schwefeleisen ab, so erhalten wir für diesen Bestandteil:
\begin{center}
    \begin{tabular}{ |l|r| } 
    \hline
    SiO\textsubscript{2} & 26,45\\\hline
    Al\textsubscript{2}O\textsubscript{3} & 1,35\\\hline
    FeO & 37,30\\\hline
    CaO & 1,70\\\hline
    MgO & 33,20\\
    \hline
    \end{tabular}
\end{center}
\paragraph{}
Worin, wenn die Tonerde und Kalkerde als wahrscheinlich zu einem zersetzten Feldspat gerechnet und ein Teil des Eisenoxyduls als noch von Meteoreisen abstammend in Abzug gebracht wird, der durch Säuren zersetzte Bestandteil nicht anders, als zu Olivin gehörig sich auslegen lässt. Dass ein Teil des Eisens oxydiert ist und dadurch der Gehalt an Basen etwas gesteigert erscheint, darauf weisen schon den Rostflecken hin, welche sich manchmal selbst in der Masse ziemlich verbreitet zeigen.

Was das oder die Silikate des Restbestandteils angeht, so gibt der verhältnismäßig hohe Kieselsäure- und Tonerdegehalt, neben den Alkalien wohl der Vermutung Raum, dass neben einem Augit-Mineral auch noch ein feldspattiges vorhanden sei. Gleichwohl aber bleibt auch bei dieser Annahme noch ein starker Überschuss an Eieselsäure, von dem man wohl nicht voraussetzen darf, dass er in Form eines ausgeschiedenen Quarzminerals auftrete, weil bei Untersuchung des Dünnschliffs im reflektierten Lichte keine Spur einer Beimengung von durch den starken Glanz sonst erkennbarem Quarze sich bemerken lässt. Dieses Verhalten ist vorläufig noch unaufgeklärt.

Derselbe Meteorstein ist bereits in neuester Zeit auch noch einer chemischen Analyse von anderer Seite unterworfen worden. Rammelsberg führt (D. chem. Nat. d. Meteoriten Abh. d. Acad. d. Wiss. in Berlin für 1870 S. 148 u. ff.) als das Resultat der von Crook\footnote{On the chem. constit. of meteor. stones. Göttingen Dissert. (Mir nicht zugänglich).} ausgeführten Untersuchung an: Zusammensetzung:
\begin{center}
    \begin{tabular}{ |l|r| } 
    \hline
    3,52\% & Meteoreisen\\\hline
    1,92\% & Schwefeleisen\\\hline
    0,72\% & Chromeisen\\\hline
    92,68\% & Silikat\\
    \hline
    \end{tabular}
\end{center}
100,00 und zwar:
das Silikat bestehend als:
\begin{center}
\begin{tabular}{ |p{20mm}|p{24mm}|p{31mm}|p{32mm}| }
    \hline
    Stoffe: & im Ganzen Bauschanalyse & in dem 61\% durch Säuren zersetzbar. Anteil. & in dem 39\%\newline in Säuren unzersetzb. Anteil.\\
    \hline\hline
    Kieselsäure & 44,81 & 32,68 & 3,94\\\hline
    Tonerde & 1,24 & 9,36 & 4,17\\\hline
    Eisenoxydul & 24,55 & 28,91 & 17,71\\\hline
    Bittererde & 26,10 & 37,44 & 8,20\\\hline
    Kalkerde & 2,28 & 0,61 & 4,91\\\hline
    Natron & 0,26 & - & 0,67\\\hline
    Kali & 0,16 & - & 0,40\\
    \hline
\end{tabular}
\end{center}
\paragraph{}
Diese Resultate weichen so bedeutend von den früher mitgeteilten ab, dass dafür kein anderer Grund gefunden werden kann, als die an sich große Ungleichheit in der Zusammensetzung des Meteorsteins, welche einen umso größeren Einfluss auf die Ergebnisse der Untersuchung zu äußern im Stande ist, mit je kleineren Quantitäten man zu arbeiten gezwungen ist. Die mikroskopische Untersuchung der Dünnschliffe unterstützt direkt diese Annahme, indem sich hierbei die größte Unregelmäßigkeit in der Art der Verteilung der Gemengteile erkennen lässt. Ein größeres Korn von diesem oder jenem Gemengteil in der verwendeten Probe verrückt bei geringen Quantitäten, die man benützt, die Zahlen in beträchtlicher Weise. Es lassen sich beispielsweise zackige Knöllchen von Meteoreisenteilchen aus der Masse herauslösen, deren Größe in keinem Verhältnisse steht zu dem geringen Prozentgehalte des Steins an Meteoreisen im Allgemeinen und Ganzen. Ähnlich verhält es sich mit den eingestreuten härteren Knöllchen und Körnchen.

Besonders verschieden ist die Angabe bezüglich der Zusammensetzung des in Salzsäure zersetzbaren Gemengteils. Doch tritt auch in der Analyse Crook's die relativ geringe Menge von Kieselsäure sehr deutlich hervor. Minder abweichend erweisen sich die Resultate der Analyse des durch Säuren unzersetzten Restes. Gerade dies beweist, dass es nicht in dem Gang der analytischen Arbeit liegt, wie es scheinen könnte, wenn hier der Kieselsäurengehalt ebenso verhältnismäßig hoch, wie bei dem in Säuren zersetzbaren Anteil gering gefunden wurde. Da dieser Rest, wie die mikroskopische Untersuchung desselben lehrt, aus verschiedenen Mineralsubstanzen, namentlich einem weißen und einem braunen Gemengteil besteht, so kann das Sauerstoff-Verhältnis im Ganzen genommen, uns keine besonderen Aufschlüsse verschaffen.

Die wegen der leichten Zerreiblichkeit der Masse schwierig herzustellenden Dünnschliffe, welche nur durch wiederholtes Tränken mit sehr verdünntem Canadabalsam in brauchbarem Zustande gewonnen werden können, geben, wie es das Dünnschliffbild auf der beiliegenden Tafel in Figur I. zeigt, bezüglich der Zusammensetzung des Gesteins und der Verteilung der Gemengteile einige lehrreiche Aufschlüsse. Es stechen besonders die Chondren in ihrer teils staubig krümeligen, teils faserigen Zusammensetzung besonders hervor. Trotz der geringen Durchsichtigkeit derselben erweisen sie sich i. p. L. betrachtet stets farbig, und zwar nicht bloß die lichteren Streifchen derselben, sondern ihre ganze Masse. Diesen Einmengungen gegenüber sind die übrigen unterscheidbaren, stets unregelmäßig umgrenzten, gelblichen, bräunlichen und weißlichen Splitterchen klein. Sie sind alle von zahllosen Bissen durchzogen, die nur hier und da parallel verlaufen. Kleine Stückchen und Staubteilchen der anscheinend gleichen Mineralien bilden die Grundmasse, in welchen die größeren Trümmer eingestreut liegen. I. p. L. treten bis in die feinsten Teilchen Farbenerscheinungen hervor, so dass auch in den Dünnschliffen die Abwesenheit einer glasartigen Bindemasse bestimmt beobachtet werden kann. Bemerkenswert sind zahlreiche kleinste, runde, wasserhelle Körnchen, welche der Grundmasse beigemengt sind. Meteoreisen- und Schwefeleisen-Knöllchen teilen etwa die Größe der Mineralsplitterchen, machen jedoch ihren Umrissen nach nicht den Eindruck der Zertrümmerung, wie letztere und liegen ziemlich gleichmäßig in der Masse zerstreut. Wir sehen also, dass der Meteorstein von Mauerkirchen seiner Struktur nach sich nicht wesentlich von anderen chondritischen Meteorsteinen unterscheidet.
\clearpage
\subsection{Der Meteorstein von Eichstädt}
\paragraph{}
(Figur II.)

Über den Fall dieses Steins wird berichtet, dass ein Arbeiter an einer Ziegelhütte im sog. Wittmes, einer waldigen Gegend, etwa 5 Kil. westwärts von Eichstädt am 19. Feb. 1785 Nachmittags zwischen 12 und 1 Uhr nach einem donnerähnlichen Getöse einen großen schwarzen Stein auf den mit Schnee bedeckten Erdboden, auf dem Ziegelsteine umher lagen, fallen sah. Als er zur Stelle lief, fand er den Stein, welcher einen Ziegelstein zertrümmert hatte, eine Hand tief im Boden und so heiß, dass er ihn erst mit Schnee abkühlen musste, um ihn an sich nehmen zu können. Der Stein hatte etwa ein Fuß im Durchmesser und wog beiläufig 3 Kilogramm. Schafhäutl (Gelehrt. Anzeige d. Ac. d. Wiss. in München 1847 S. 559.) beschreibt denselben wie folgt: „Seine Struktur ist ziemlich grobkörnig, die Körner sind rundlicher, als dies bei allen übrigen Aerolithen der Fall ist; ja es finden sich sogar vollkommen elliptische, wie abgeschliffen aussehende Körnchen von graulicher Farbe und dichtem ziemlich mattem ebenem Bruche darin, ohne bemerkbares kristallinisches Gefüge. Neben diesen liegen grünliche olivinartige Körner von glasig muscheligem Bruche. Schwefeleisen, Nickeleisen und Magneteisen sind zwischen diesen Körnern eingesprengt, so dass er unter allen Meteorsteinen unserer Sammlung (Münchner Staats-S.) am stärksten auf die Magnetnadel wirkt.“

Das spez. Gewicht\footnote{Vergl. Moll's Annal, d. Berg- u. Hüttenk. Bd. III. S. 251.} wird angegeben:

von Schreibers zu 3,700

von Rumler zu 3,599

Klaproth hat diesen Stein analysiert und gibt (Gilberts Ann. XIII. 338) als seine Bestandteile an:
\begin{center}
    \begin{tabular}{ |l|r| } 
    \hline
    Gediegen Eisen & 19,00\\\hline
    Nickelmetall & 1,50\\\hline
    Braunes Eisenoxyd & 16,50\\\hline
    Bittersalzerde & 21,50\\\hline
    Kieselerde & 37,00\\\hline
    Verlust (mit Schwefel) & 4,50\\
    \hline
    \end{tabular}
\end{center}
\paragraph{}
Das in der Münchener Staatssammlang verwahrte Stück zeigt eine schwarze mattglänzende, runzelige Rinde und eine weißlich graue, grobkörnig chondritische, durch zahlreiche Rostflecken hier und da gelblich getüpfelte, leicht zerreibliche Hauptmasse, aus welcher sich die oft sehr großen Chondren leicht heraus lösen lassen. Es finden sich solche bis über 3 mm. im Durchmesser groß, sie sind sehr hart, auf der Oberfläche matt, erdbeerenartig höckerig und grubig in einer Weise, dass die angeschlossenen Mineralsplitterchen der Hauptmasse wie an die Oberfläche gekittet erscheinen. An vielen Stellen der Oberfläche bemerkt man zudem kleinen spiegelnde Streifchen, wodurch dieselben gleichsam facettiert erscheinen. Auch kommen damit fest verwachsene Meteoreisenteilchen vor, welche zuweilen selbst in die Oberfläche versenkt sind. Niemals zeigt sich eine Glättung der Oberfläche, wie sie Vorkommen müsste, wenn die Kügelchen durch Reibung und Abrollung entstanden wären. Vielmehr gleichen sie der äußeren Beschaffenheit nach den in den Schlacken vorkommenden Roheisensteinkügelchen. Zerschlägt man sie, so zeigen sie auf der flachmuscheligen Bruchfläche, einen matten Glasglanz, schwärzlichgraue Farbe und bei weit Zertrümmerung unter dem Mikroskop erweisen sie sich nicht als eine homogene, sondern zusammengesetzte Masse. Man kann deutlich einen glashellen mit zahl- i reichen Bläschen erfüllten, i. p. L. ungemein buntfarbigen Bestandteil neben einer nur durchscheinend trüben, wie aus kleinsten Staubteilchen zusammengesetzten, aber i. p. L. doch deutlich farbigen, zuweilen feinstreifigen Hauptmasse und einzelnen durchscheinenden intensiv gelbbraunen, i. p. L. unverändert gefärbten Streifchen unterscheiden. In Dünnschliffen sieht man ihre Struktur noch viel deutlicher, obwohl sie hier in einer an sich sehr dunkelgefärbten Hauptmasse liegen und schwierig gut durchsichtig zu erhalten sind. Indem nämlich ziemlich viel Meteoreisen als Gemengteil auftritt, dass großenteils bereits etwas zersetzt und mit einem Höfchen von gelbbrauner Farbe umgeben ist, leidet auch die Klarheit derjenigen Mineralteilchen, welche sonst durch ihre Durchsichtigkeit sich auszeichnen. Die gelbe Farbe rührt von Eisenoxydhydrat her, welches durch die Einwirkung der feuchten Luft unserer Atmosphäre auf das Meteoreisen erst nachträglich während der Zeit sich gebildet hat, in welcher der Stein in der Erde oder in unseren Sammlungen gelegen hat. Dieses Eisenoxydhydrat dringt in die feinsten Risschen und Sprünge oder Zwischenräume ein, kann aber leicht durch Säuren entfernt werden. Neben dem Meteoreisen beteiligen sich unregelmäßig eingesprengte, selten von parallelen Linien eingeschlossene Mineralsplitterchen an dem Haufwerk, aus dem der Meteorstein besteht. Bald sind es wasserhelle, wenig rissige Trümmerchen, bald solche, welche durch ein einfaches System parallelen Linien gestreift oder von unten schiefen Winkeln sich schneidenden Rissen zerklüftet sind, etwa wie es bei dem Augit vorzukommen pflegt, oder aber durch eine dem Zellnetz gewisser Moosblättchen ähnliche, merkwürdig langgezogene und quergegliederte Maschenstruktur (d) sich auszeichnen. Zuweilen stoßen in einem Trümmerteil mehrere Systeme solcher paralleler Streifchen zusammen. Zwischen diesen größeren Fragmenten liegen kleinere ganz von derselben Beschaffenheit, wie die größeren angehäuft. I. p. L. erscheinen alle Teilchen, welche nur überhaupt durchsichtig sind, in bunten Farben, welche selbst innerhalb der einzelnen Splitter aggregatartig verteilt sind und selten streifig oder bandartig parallel verlaufen. Endlich sind als ungemein häufige Bestandteile die kugeligen Einschlüsse zu nennen, die schon erwähnt worden sind. Aus den mannichfachen Formen, welche dieselben besitzen, heben wir nur einige der am häufigsten vorkommenden hervor. Ziemlich zahlreich sind die Chondren mit exzentrisch strahlig faserigem Gefüge (a), welches in der Regel von einer nahe am Rande liegenden mehr körnigen Partie ausgeht und in einem vielfach abgesetzten, gleichfalls maschenartigen und quergegliederten Strahlenbüschel ausläuft. Diese Struktur stimmt so sehr mit jener schon geschilderten überein, welchen wir auf ändern regelmäßig umgrenzten Splitterchen begegnen, dass wir letztere wohl als Abkömmlinge zerbrochener größerer Chondren ansehen müssen. Andere der letzteren sind von verschiedenen Systemen sich unter spitzen und stumpfen Winkeln schneidender dunkler Streifchen beherrscht (b), eine Struktur, die sich als der Anfang einer kristallinischen periodenweis gestörten Ausbildung betrachten lässt. In noch anderen Chondren kommt eine staubartig trübe, schwach durchscheinende Substanz vor, in welcher häufig sehr zahlreiche dicht gedrängte, hellere, gruppenweis nach verschiedenen Richtungen verlaufende Streifchen (c) sich bemerkbar machen. Endlich treten nicht selten Kügelchen auf, welche aus größeren, helleren, durch dunkle Zwischenstreifchen voneinander getrennten Körnchen (e) gleichsam zusammengebacken erscheinen. Aus alle dem geht zur Genüge hervor, dass wir in dem Stein von Eichstädt einem Chondriten der ausgezeichnetsten Art vor uns haben. Derselbe kann geradezu als Typus dieser Art der Struktur, welche bei den Meteorsteinen als der vorherrschende bekannt ist, gelten.

Was seine Zusammensetzung anbelangt, so hat die Analyse (Ass. A. Schwager) ergeben, dass der Stein besteht aus:

22,98 Meteoreisen,

3,82 Schwefeleisen,

32,44 in Salzsäure zersetzbaren,

40,76 in Salzsäure nicht zersetzbaren Mineralien.
\paragraph{}
Die Zusammensetzung ist im Ganzen A, dann

B in den durch CI H zersetzbaren Silicaten

C in dem durch CI H nicht zersetzbaren Bestandteil:
\begin{center}
\begin{tabular}{ |p{35mm}|p{20mm}|p{20mm}|p{20mm}| }
    \hline
    & A. & B. & C.\\
    \hline\hline
    Kieselerde & 33,31 & 34,45 & 55,53\\\hline
    Tonerde & 2,31 & 0,86 & 5,13\\\hline
    Eisenoxydul & 15,34 & 24,52 & 16,66\\\hline
    Eisen (mit Phosphor) & 24,64 & - & -\\\hline
    Nickel & 0,94 & - & -\\\hline
    Kalkerde & 0,74 & 0,68 & 1,13\\\hline
    Schwefel & 1,42 & - & -\\\hline
    Chromoxyd & 0,15 & - & 0,73\\\hline
    Bittererde & 18,86 & 37,31 & 19,34\\\hline
    Kali & 0,40 & 0,68 & 0,56\\\hline
    Natron & 1,04 & 1,31 & 1,62\\\hline
    & 99,15 & 99,81 & 100,70\\
    \hline
\end{tabular}
\end{center}
\paragraph{}
Der Gehalt der durch Salzsäure zersetzbaren Gemengteile an Alkalien weist außer Olivin noch auf einen Feldspat hin. Wir haben aber darin:
\begin{center}
    \begin{tabular}{ |l|r| } 
    \hline
    SiO\textsubscript{2} & 34,45 mit 18,37 O\\\hline
    Al\textsubscript{2}O\textsubscript{3} & 0,86 mit 0,40\\\hline
    FeO & 24,52 mit 5,45\\\hline
    MgO & 37,31 mit 14,90\\\hline
    CaO & 0,68 mit 0,19\\\hline
    Ka\textsubscript{2}O & 0,68 mit 0,11\\\hline
    Na\textsubscript{2}O & 1,31 mit 0,34\\
    \hline
    \end{tabular}
\end{center}
\paragraph{}
Daraus er sieht man, dass, wenn wir ein Singulosilikat ausscheiden, die vorhandene Sauerstoffmenge noch nicht einmal vollständig ausreicht, den Bedarf ganz zu decken, dass mithin die Analyse uns keinen Aufschluss über die Natur des etwa noch außer Olivin vorhandenen Silikats weitergibt.

In dem von Säuren nicht zersetzbaren Best endlich stellen sich die Verhältnisse folgender Maassen:
\begin{center}
\begin{tabular}{ |l|c|l| }
    \hline
    Kieselerde & 55,53 & mit 29,62 O = 22,6 + 7\\\hline
    Eisenoxydul & 16,66 & mit 3,70 O = 3,58 + 0,12\\\hline
    Bittererde & 19,34 & mit 7,73 O\\\hline
    Chromoxyd & 0,73 & mit 0,23 O\\\hline
    Tonerde & 5,13 & mit 2,39 O = 2,33 + 0.06\\\hline
    Kalkerde & 1,13 & mit 0,32 O\\\hline
    Kali & 0,56 & mit 0,10 O\\\hline
    Natron & 1,62 & mit 0,42 O\\
    \hline
\end{tabular}
\end{center}
\paragraph{}
Daraus berechnet sich ein Bisilikat, Chromeisen (von der Zusammensetzung des von L'Aigle) und ein Andesin-artiger Feldspat ungefähr in dem Verhältnis wie 79:1:21.

Im Ganzen besteht also der Eichstädter Meteorstein ungefähr aus:
\begin{center}
    \begin{tabular}{ |l|r| } 
    \hline
    Meteoreisen & 22,98\\\hline
    Schwefeleisen & 3,82\\\hline
    Chromeisen & 0,40\\\hline
    Olivin & 31,00\\\hline
    Mineral der Augitgruppe & 31,90\\\hline
    Andesin-artiger Feldspat & 8,46\\\hline
    Feldspatartiges Mineral & 1,54\\
    \hline
    \end{tabular}
\end{center}
\paragraph{}
Das häufige Vorkommen und die relative Größe der Chondren luden zu einer besonderen Analyse dieser Kügelchen ein. Um sicher zu sein, mit einem von anhaftenden kleinsten Mineralsplitterchen freien Material zu verarbeiten, wurden die Chondren so lange auf einer mattgeschliffenen Glasplatte hin- und hergerieben, bis ihre Oberfläche völlig glatt und glänzend geworden war. Leider war die so mir zur Verfügung stehende Menge eine nur sehr geringe (0,12 Gr.) und es kann daher an die Analyse der Anspruch grösser Genauigkeit nicht gemacht werden. Durch Vorversuche war bereits festgestellt worden, dass auch die Substanz der Chondren sich teilt in eine von Salzsäure zersetzbare und in eine unzersetzbare Masse. Die erstere enthält noch Schwefeleisen, welches, wie die Untersuchung an Dünnschliffen lehrt, in kleinen Körnchen fest mit den Kügelchen verwachsen und in dieselbe gleichsam eingesenkt vorkommt.

Ich fand die Zusammensetzung:

Schwefeleisen 1,53

I. In Salzsäure zersetzbar 53,05

II. In Salzsäure unzersetzbar 45,42

Als Zusammensetzung der Silikate I und II ergab sich ferner
\begin{center}
\begin{tabular}{ |l|r|r| }
    \hline
    & I & II\\
    \hline\hline
    Kieselsäure & 26,26 mit 14,22 O & 53,21 mit 28,38 O\\\hline
    Eisenoxydul & 30,09 mit 6,67 O & 14,86 mit 3,30 O\\\hline
    Bittererde & 31,53 mit 12,60 O & 26,42 mit 10,56 O\\\hline
    Tonerde & 2,70 mit 1,26 O & - -\\\hline
    Kalkerde & 1,00 mit 0,29 O & 3,67 mit 1,05 O\\\hline
    Alkalien & 8,00 mit 1,70 O & - -\\\hline
    & 99,98 & 98,16\\
    \hline
\end{tabular}
\end{center}
\paragraph{}
Es ist zunächst hervorzuheben, dass, wie auch schon von anderer Seite bemerkt wurde, die Zusammensetzung der Chondren nahezn die nämliche ist, wie die der ganzen Masse und sich durch die Behandlung mit Säuren in zwei ähnliche Teile scheiden lässt.

Der in Salzsäure zerlegbare Teil, abgesehen von Resten eines Gehaltes an Meteoreisen und Schwefeleisen, schließt sich am engsten an Olivin an. Aber es mangelt auch hier, wie in zahlreichen Fällen bei analysierten Chondriten an Kieselsäure. Ich möchte vermuten, dass dies hier von einem Überschass an Eisenoxydul herrührt, das, anstatt von zersetzem Olivin, von fein beigemengtem Meteoreisen abstammt. Tonerde, Kalkerde und Alkalien weisen auf eine Beimengung feldspatartiger Teilchen, wie bei der Hauptmasse der Chondrite hin. Doch bietet die Interpretation dieses Teils immerhin Schwierigkeiten, die bis jetzt noch nicht beseitigt sind.

Der in Salzsäure unzersetzte Rest fügt sich viel besser in das Maaß eines Bisilikates; wenn es auch hierbei um etwas weniges an Kieselsäure fehlt, so kann dies wohl bei der geringen, zur Analyse verwendete Menge als Folge des Verlustes bei der Analyse selbst angesehen werden.
\clearpage
\subsection{Der Meteorstein von Massing}
\paragraph{}
(Figur III.)

Über die näheren Umstände des Falls dieses Meteoriten teilt Prof. Imhof (Kurpfalzbaier. Wochenblatt 1804 St. 3 u. f.)\footnote{Gilberts Ann. d. Phys. XVIII. 330.} mit:

„Nach den gerichtlichen Anzeigen an die kurf. Landesdirektion hörten mehrere der Landleute, die um den Marktflecken Mässing(Massing) Ldger. Eggenfelden wohnen, am 13. Dez. 1803 Vormittag zwischen 10 und 11 Uhr neun bis zehn Mal einen Knall, wie Kanonenschüsse. Ein Bauer zu St Nicolas, der bei diesem Getöse aus seinem Hofe trat und in die Höhe sah, erblickte etwas, das sehr hoch unter beständigem Sausen in der Luft daherkam und endlich auf das Dach seiner Wagenhütte fiel, etliche Schindeln zerschlug und in dieselbe eindrang. Er ging auf die Hütte zu und fand in ihr einen Stein, der nach Pulver roch, ganz schwarz und so heiß war, als ein Stein zu sein pflegt, der auf einem Ofen lag. Er sagte, er habe das vermeintliche Schießen von Alten-Oetting (d. h. von Osten) hergehört, der Stein sei aber über Heiligenstadt (d. h. von Westen) gekommen. Der Stein wog über 1 1/2 Kilogramm, hat ein spec. Gew. von 3,365, eine dunkelschwarze, etwas dickere Binde, als der Mauerkirchner und ist im Bruche viel grobkörniger. Als Gemengteile enthält er nach Imhof:
\begin{enumerate}
    \item regulinisches Eisen, das wie dünne Eisenfeile sichtbar eingewachsen und glänzend erscheint,
    \item Schwefelkies, der unter der Loupe kristallisiert erscheint und gerieben ein schwarzes Pulver gibt,
    \item größere und kleinere plattgedrückte, eckige Massen, einige von dunkelbrauner, andere von schwarzer Farbe, die sich durch ein schimmerndes Ansehen und größere Härte von jenen unterscheiden,
    \item hier und da bemerkt man noch kubische Körnchen und Blättchen von gelblicher Farbe durchscheinend und mit Glasglanz, wie Quarz aussehend, die jedoch nicht die Härte des Quarzes haben,
    \item auch sind weiße Körner von unregelmäßiger Form eingesprengt, von denen einige über 3 Linien dick sind,
    \item unter dem Mikroskop sieht man auch ein weißgraues, ins Gelbe spielendes Metall, das die Magneten folgsam und wahrscheinlich metallisches Nickel ist.
\end{enumerate}
\paragraph{}
Nach der Analyse dieses Forschers besteht der Stein in 100 Teilen, aus:
\begin{center}
    \begin{tabular}{ |l|r| } 
    \hline
    regulinischem Eisen & 1,80\\\hline
    regulinischem Nickel & 1,35\\\hline
    braunem Eisenoxyd & 32,54\\\hline
    Magnesia & 23,25\\\hline
    Kieselerde & 31,00\\\hline
    Verlust an Schwefel u. Nickel & 10,06\\
    \hline
    \end{tabular}
\end{center}
\paragraph{}
Ammler gibt (O. Büchner a. a. O. S. 17) das spec. Gewicht zu 3,3636 an.

Prof. v. Schafhäutl beschreibt (a. a. O. S. 558) diesen Stein „vom Aussehen des Bimssteinporphyrs, in dem die einzelnen Silikate in so großen Aggregaten auftreten, dass man sie leicht mit freiem Auge unterscheiden könne. Das Gestein bestehe aus milchweißen Körnern von blättrig strahliger Struktur, aus olivinartigen körnigen Massen von Erbsengrösse, und aus z. Th. matten basaltartigen Fragmenten, die jedoch öfter auf den augitartigen Blätterdurchgängen auch glasglänzend erscheinen. Sparsam finden sich rissiges irisirendes Schwefeleisen eingesprengt und kleine Körnchen von Chromeisen. Der Stein wirkt nicht auf die Magnetnadel. Vor dem Löthrohr sei er ziemlich leicht schmelzbar und ebenso mit einer glasig glänzenden Rinde überzogen, wie der Aerolith von Stannern.“

Nach meinen Beobachtungen besitzt der Stein eine braunschwarze glasglänzende Rinde und besteht in seiner graulich weißen, ziemlich leicht zerreiblichen Masse aus:
\begin{enumerate}
    \item einem gelblich grünen bis hellgrünen, etwas parallelrissigen, in rundlich und unregelmäßigen Körnchen (wie in Krystallform) vorkommenden, ziemlich großen, 1 — 1 1/2 mm. im Durchmesser breiten, nur sporadisch erscheinenden Gemengteil, der durch Säuren leicht zersetzt wird and als Olivin gelten muss.
    \item aus einem weißen, oft glasartig durchsichtigen oder staubig trüben, nur durchscheinenden, stark rissigen, selten parallelstreifigen, zuweilen mit deutlichen Spaltflächen versehenen Mineral, das i. p. L. lebhaft ein- oder fleckig vielfarbig erscheint und von Säuren gleichfalls zersetzt wird, einem Feldspat entsprechend,
    \item aus einem weingelben bis graugrünlichen, oder blass rötlich braunem, glasartig mattglänzendem Mineral, 1,5 bis 2 mm. groß, i. p. L. lebhaft gefärbt, aber nicht dichroitisch, etwas längsfaserig (aber undeutlich, gestreift) und mit zahlreichen kleinen Bläschen erfüllt. Dieser Bestandteil wird von Säuren nicht zersetzt und gehört der Augitgruppe an.
    \item aus schwarzem, starkglänzendem, in Säuren nicht zersetzbarem, in der Phosphorsalzperle ein prächtig grünes Glas lieferndem Chromeisen,
    \item endlich aus z. Th. von den Magneten gezogenen, dunklen, metallischen Körnchen, die meist dem Schwefeleisen, im Minimum dem Meteoreisen zuzuteilen sind.
\end{enumerate}
\paragraph{}
Diese sämtlichen größeren, vorwaltend rundlich unregelmäßig eckigen, (nicht länglich spießförmigen) Teilchen liegen in einer feinstaubartig körnigen, grauen Grundmasse, welche aus denselben nur kleinen und kleinsten Splitterchen, wie sie eben angeführt wurden, zu bestehen scheint. Auch hier ist eine glasartige Bindemasse nicht zu erkennen.

Die Analyse A. Schwager's ergab:
\begin{center}
\begin{tabular}{ |p{20mm}|p{22mm}|p{32mm}|p{32mm}| }
    \hline
    Stoffe: & Bauschanalyse & 21,33\% in Salzsäure zersetzbar & 78,67\% in Salzsäure nicht zersetzbar\\
    \hline\hline
    Kieselsäure & 52,115 & 39,59 & 56,71\\\hline
    Tonerde & 8,204 & 29,51 & 2,54\\\hline
    Eisenoxydul & 19,138 & 2,83 & 23,46\\\hline
    Eisen & 0,523 & 2,49 & -\\\hline
    Nickel & Spuren & Spuren & -\\\hline
    Chromoxyd & 0,979 & - & 1,24\\\hline
    Kalkerde & 5,786 & 15,70 & 3,15\\\hline
    Bittererde & 8,485 & 3,33 & 10,74\\\hline
    Kali & 1,188 & 4,78 & 0,85\\\hline
    Natron & 1,928 & 4,78 & 1,17\\\hline
    Schwefel & 0,374 & 1,78 & -\\\hline
    & 99,720 & 100,06 & 99,86\\
    \hline
\end{tabular}
\end{center}
\paragraph{}
Der durch Salzsäure zersetzbare Anteil zu 21,33\% lässt sich nach dem Gehalt an Schwefel, Bittererde und Tonerde berechnet ansehen als ungefähr zusammengesetzt aus:

10\% Olivin (Hyalosiderit)

86\% Anorthit mit großem Alkaligehalte

4\% Schwefeleisen und Meteoreisen

In abgerundeten Zahlen bestände der Feldspat A und der Olivin B aus:
\begin{center}
    \begin{tabular}{ |l|c|r| } 
    \hline
    & A & B\\
    \hline\hline
    Kieselerde & 42 & 37,25\\\hline
    Tonerde & 34 & -\\\hline
    Eisenoxydul & - & 29,75\\\hline
    Kalkerde & 18 & -\\\hline
    Bittererde & - & 33,00\\\hline
    Alkalien & 6 & -\\
    \hline
    \end{tabular}
\end{center}
\paragraph{}
Was den Rest des durch Säuren nicht zersetzbaren Anteils zu 78,67\% anbelangt, so muss man hierin noch einen kleinen Anteil Feldspat neben Chromeisen und Augit annehmen, etwa:

2,5\% Chromeisen

13,5\% feldspatartige Substanz (A)

84,0\% Augitmineral (B).

Beiden letzteren (A und B) würde eine Zusammsetzung zu kommen, wie folgt:
\begin{center}
    \begin{tabular}{ |l|c|r| } 
    \hline
    & A & B\\
    \hline\hline
    Kieselerde & 66 & 86\\\hline
    Tonerde & 19 & -\\\hline
    Eisenoxydul & - & 36\\\hline
    Kalkerde & - & 4\\\hline
    Bittererde & - & 14\\\hline
    Alkalien & 15 & -\\
    \hline
    \end{tabular}
\end{center}
\paragraph{}
Berücksichtig man ferner das Verhältnis des in Salzsäure zersetzbaren und nicht zersetzbaren Anteils im Verhältnis von 21,33 zu 78,67 so können wir nach der oben angeführten Deutung den Meteorstein ungefähr zusammengesetzt uns vorstellen, aus:
\begin{center}
    \begin{tabular}{ |l|r| } 
    \hline
    Olivin & 2,00\\\hline
    Schwefeleisen & 0,75\\\hline
    Meteoreisen & 0,25\\\hline
    Chromeisen & 2,00\\\hline
    Anorthit & 18,00\\\hline
    2te feldspatige S. & 11,00\\\hline
    Augitmineral & 66,00\\
    \hline
    \end{tabular}
\end{center}
\paragraph{}
Es wurde bisher der Stein von Massing dem von Luotolaks an die Seite gestellt und Rammelsberg (d. chem. N. d Meteor. S. 136) zählt ihn zu den Howarditen (Olivin- Augit- Anorthitmeteorstein).

Ich glaube, dass er mehr Analogien mit der Gruppe der Eukrite besitzt, da der Olivin sehr spärlich vorhanden ist.

Wir wollen nun zunächst sehen, wie mit dieser Auffassung die optische Untersuchung der Dünnschliffe passt, wie das Bild Figur III. einen solchen darstellt. Man bemerkt zunächst große, unregelmäßig eckige — nicht wie bei den typischen Chondriten abgerundete Körnchen und eine ziemlich gleichmäßige, feine Hauptmasse mit einzelnen im auffallenden Lichte metallisch glänzenden, stahlgrauen und messinggelben Putzen. Sehen wir zunächst ab von den großen, unregelmäßigen, gleichsam abnormen Beimengungen, so treten uns in der Grundmasse vor Allem größere Gruppen eines grünlich gelben, dann eines schwach weingelben, eines blassrötlich braunen und weißen Minerals entgegen, welche wir als die Hauptgemengteile anzusehen berechtigt sind. Die wenigen grünlich gelben Teilchen (a) sind unregelmäßig rissig, glänzen i. p L. mit den lebhaftesten Aggregatfarben und werden durch Säuren zersetzt — Olivin. Nach dem ersten Anschein möchte man auch die weit zahlreicheren Putzen des schwach weingelben, jedoch mehr parallel rissigen Minerals (b) für Olivin halten. Allein in den mit kochenden Säuren anhaltend behandelten Pulvern erscheinen sie unzersetzt und können mithin nicht zum Olivin gehören. Auch bemerkt man in den Dünnschliffen eine Art Parallelstreifung, wie sie dem Olivin nicht zukommt, aber an Enstatit erinnert. Daneben liegen zahlreiche, oft nur durchscheinende, doch auch gut durchsichtige, an den Rändern rötlich braun gefärbte, nicht dichroitische Teilchen (c), die allem Verhalten nach Augit zu sein scheinen. Ich glaube demnach annehmen zu sollen, dass zwei Mineralien der Augitgruppe hier vertreten sind, nämlich Enstatit und Augit. Die glashellen oder staubartig weißen Teilchen (d) sind teils durch Säuren zersetzbar, teils erscheinen sie aber auch noch in dem durch Säuren behandelten Pulver mehr oder weniger unberührt. Dies deutet gleichfalls auf die Anwesenheit von zweierlei Feldspaten, von welchen der eine wohl in dem Dünnschliffe Spuren von Parallelstreifen i. p. L. erkennen lässt. Dass — entgegen der Angabe Schafhäutl's — wirklich Meteoreisen, wenn auch spärlich beigemengt ist (e), habe ich in die Dünnschliffe, in dem zwei deutliche Körnchen Vorkommen, dadurch festgestellt, dass ich auf die stahlgrau glänzenden Flächen Kupfervitriollösung brachte, wobei sich sofort die Ausscheidung metallischen Kupfers beobachten lässt.

Schwieriger zu erklären ist die Natur der großen Einsprenglinge, zu denen im Dünnschliff die Parthien x und y gehören. Der größere x ist parallelstreifig und querrissig, dunkelolivengrün bis rötlich braun, wenig durchsichtig, i. p. L. farbig. Er möchte als ein etwas veränderten Augitfragment zu betrachten sein. Das zweite Fragment y ist gelblich, sehr feinkörnig, fast dicht, schwach duschscheinend und mit feinsten schwarzen Staubteilchen durchsprengt. Es gleicht am ehesten die Bruchstücke eines Chondrit-körnchens. Dergleichen Einschlüsse mögen noch von sehr verschiedener Beschaffenheit in der Grundmasse eingebettet sein. Obwohl eine deutliche Chondritenstruktur nicht vorhanden ist, verhalten sich doch diese Einschlüsse und die als Grundmasse auftretenden Mineralien so ähnlich den Bestandteilen der Chondrite, dass auch dem Meteorstein von Massing eine ganz analoge Entstehung, wie die der letzteren, zugesprochen werden muss.

Der namhafte Gehalt dieses Steins an Chromeisen gab Veranlassung, dessen Zusammensetzung näher zu erforschen, da, soviel ich weiß, dass Chromeisen der Meteorsteine isoliert bis jetzt noch nicht einer Analyse unterworfen worden ist. Es schien sich hierzu das Chromeisen im Meteorstein von L'Aigle, indem es in größeren Körnchen vorkommt, gut zu eignen. Dasselbe lässt sich daraus sehr leicht undvollständig rein heraussuchen. Die Analyse dieses Chromeisens ergab:
\begin{center}
    \begin{tabular}{ |l|r| } 
    \hline
    Chromoxyd & 52,13\\\hline
    Eisenoxydul & 37,68\\\hline
    Tonerde & 10,25\\\hline
    & 100,06\\
    \hline
    \end{tabular}
\end{center}
also nahezu die Zusammensetzung des Chromeisens von Baltimore (Maryland), ein Beweis mehr für die Gleichartigkeit der Bildung kosmischer und tellurischer Mineralien.
\clearpage
\subsection{Der Meteorstein von Schönenberg}
\paragraph{}
(Figur IV.)

Einen sehr ausführlichen Bericht über den Fall dieses Meteorsteins gibt Prof. v. Schafhäutl (a. a. O. S. 564). Daraus ist zu entnehmen, dass zur Zeit des Falls am 25. Dez. 1846 nach 2 Uhr Nachmittags auf einen Umkreis von etwa 60 Kilometer ein Donner-ähnliches Geräusch gehört wurde. In der nächsten Nähe des Ortes, wo der Stein niederfiel, verglich man das Geräuche mit fernem Kanonendonner, der nach mehr als 20maliger Wiederholung gleichsam in ein Trommeln überging und nach etwa 3 Minuten mit einem fernem Trompetenklängen ähnlichen Sausen endete. Im Dorfe Schönenberg traten mehrere Leute bei diesem Geräusche aus der Kirche, in der gerade Nachmittagsgottesdienst stattfand, wieder heraus und sahen nun eine fest faustgroße Kugel von N.-O.zuletzt nach S.-O. sich wendend in ein Krautfeld in der Nähe des Dorfes niederfallen. Zahlreiche Bewohner des Dorfs eilten zur Stelle nnd es fand sich etwa 2 Fuß tief in dem etwas gefrorenen Lehmboden eingedrungen ein schwarzer Stein. Man glaubte noch Schwefelgeruch zu spüren. Dabei zeigte der vordem bedeckte Himmel plötzlich zuerst in der Richtung des Meteorfalls einen lichten Streif und hellte sich dann gänzlich auf.

Die Form des ringsum von einer dunkelbraunen rauhen Sinterrinde überzogenen Steins beschreibt v. Schafhäutl als eine sehr unregelmäßige in den Hauptumrissen vierseitige Pyramide mit einer Zuschärfung, die in der Richtung des längsten Durchmessers der Basis läuft und sich nach der hintern Seite der Pyramide senkt. Da die Rinde auch in kleinen Einschnitten sich vorfindet, glaubt er annehmen zu sollen, dass der Stein in einem erweichten Zustande auf die Erde kam. Merkwürdiger Weise ziehen 7 Streifen von Nickeleisen schnurartig über den Stein, durchkreuzt von einem 8ten, der eine fast recktwinklige Richtung zu den anderen nimmt. Zwei Seiten sind eben und ohne Eindrücke, im Übrigen aber ist die Oberfläche unregelmäßig vertieft, wie das Bruchstück eines Steins, der durch eine äußere Gewalt zerschlagen ist. Der Stein wog 8 Kilogr. 15 Gr. und ist so weich, dass er sich mit den Fingern zerbröckeln lässt. Er wirkt auf die Magnetnadel und Salzsäure entwickelt unter Gallertbildung Schwefelwasserstoff. Die Masse besteht aus weißen, feinkörnigen Teilchen, welche von Säure am meisten angegriffen wurden, dann aus honiggelben und grünlichen, körnigen Aggregaten, auf welche die Säure weniger Wirkung ausübt, ferner aus einzelnen kleinen Körnchen von Schwefeleisen, silberglänzenden, gefranzten Blättchen von Nickeleisen, in der Masse zerstreut und zugleich die oben erwähnten Schnüre bildend. Von Augit, Labrador u. dgl. sei Nichts in dem Aerolithen zu entdecken, v. Schafhäutl scheint nicht der Ansicht von Berzelius zuzustimmen, dass der durch Salzsäure zersetzte Gemengteil Olivin sei. Denn die olivinartigen Körner seien gerade die unauflöslichsten und die weißen Mineralteilchen die zersetzbaren nach Art der Zeolithe oder gleich dem geglühten Epidot, Vesuvian u. s. w. Er fügt dann noch einen Erklärungsversuch der Entstehung der Meteorite als das Resultat einer Verdichtung aus einer Wolken-artigen Masse in der Nähe unseres Erdkreises hinzu.

Die Schmelzrinde ist nach meiner Wahrnehmung matt schimmernd, schwarz, stellenweis, wo Eisenteilchen in der Nähe vorhanden waren, ziemlich dick (bis 1/2 mm.) Die lichtgrau weiße, feinkörnige, spärlich schwarz punktierte, stellenweise rostfleckige Hauptmasse besteht, soweit sich dies vorläufig erkennen lässt, aus:
\begin{enumerate}
    \item größeren, grünlich gelben Teilchen, welche durch Salzsäure zersetzbar, eine viel Eisenoxydul und Bittererde haltige Lösung geben — also olivinartig,
    \item weißen splittrigen Teilchen, gleichfalls durch Säure zerlegbar,
    \item grünlich grauen, mattglänzenden, unregelmäßigen Körnchen, welche rissig sind und von Säuren nicht zersetzt werden,
    \item aus verschiedenen Eisenverbindungen, die sich durch den metallischen Glanz bemerkbar machen und vielfach von einem gelben, rostfarbigen Hofe umgeben sind, als Folge der eingetretenen Zersetzung des Meteoreisens. Der Gehalt an diesem wurde durch besondere Versuche festgestellt. Im Übrigen ergab die Analyse:
\end{enumerate}
\begin{center}
\begin{tabular}{ |p{19mm}|p{22mm}|p{33mm}|p{33mm}| }
    \hline
    Stoffe: & Bauschanalyse & 55,18\% durch\newline Salzsäure zersetzbar & 44,82\% durch Salzsäurenicht zersetzbar\\
    \hline\hline
    Kieselsäure & 40,13 & 24,47 & 57,85\\\hline
    Tonerde & 5,57 & 9,45 & 6,75\\\hline
    Eisen & 13,77 & 30,56 & -\\\hline
    Nickel & 1,47 & 1,48 & 1,44\\\hline
    Schwefel & 1,93 & 3,52 & -\\\hline
    Phosphor & 0,36 & 0,33 & 0,27\\\hline
    Chromoxyd & 0,60 & - & 1,35\\\hline
    Eisenoxydul & 17,12 & 10,41 & 15,37\\\hline
    Kalkerde & 2,31 & 3,72 & 0,56\\\hline
    Bittererde & 13,81 & 11,55 & 16,63\\\hline
    Kali & 0,73 & 1,33 & Spuren\\\hline
    Natron & 2,20 & 3,18 & 1,02\\\hline
    & 100,00 & 100,00 & 101,24\\
    \hline
\end{tabular}
\end{center}
\paragraph{}
Aus diesen Angaben lässt sich berechnen, dass der in Salzsäure zersetzbare Anteil besteht aus:
\begin{center}
    \begin{tabular}{ |l|r| } 
    \hline
    Schwefeleisen & 9,64\\\hline
    Meteoreisen & 26,25\\\hline
    Olivin & 34,78\\\hline
    Feldspat-Mineral & 29,33\\
    \hline
    \end{tabular}
\end{center}
\paragraph{}
Für den Olivinbestandteil ist in Rechnung zu setzen:
\begin{center}
    \begin{tabular}{ |l|c|r| } 
    \hline
    SiO\textsubscript{2} & 12,82 & 37\\\hline
    FeO & 10,41 & 30\\\hline
    MgO & 11,55 & 33\\\hline
    & 34,78 & 100\\
    \hline
    \end{tabular}
\end{center}
entsprechend der Zusammensetzung des Hyalosiderits.

Wir finden dann weiter für den etwas zersetzten Feldspatartigen Bestandteil:
\begin{center}
    \begin{tabular}{ |l|c|c|l| } 
    \hline
    SiO\textsubscript{2} & 11,65 & 39,71 & Sauerstoff 21,3\\\hline
    Al\textsubscript{2}O\textsubscript{3} & 9,45 & 32,21 & Sauerstoff 15,0\\\hline
    CaO & 3,72 & 12,70 & Sauerstoff 3,6\\\hline
    Ka\textsubscript{2}O & 1,33 & 4,54 & Sauerstoff 0,77\\\hline
    Na\textsubscript{2}O & 3,18 & 10,84 & Sauerstoff 2,8\\\hline
    & 29,33 & 100,00\\
    \hline
    \end{tabular}
\end{center}
\paragraph{}
Das Sauerstoffverhältnis der Kieselsäure, der Tonerde und der alkalischen Basen 3:2:1 steht nicht in Übereinstimmung mit jenen der eigentlichen Feldspate, sondern entspricht dem der Skapolithgruppe (Mejonit). Die Anwesenheit eines derartigen Minerals würde aneh zu dem optischen Verhalten besser passen, als die Annahme eines Anorthits oder Plagioklases überhaupt, weil i. p. L. die weißen oder glashellen Teilchen keine parallelen Farbenstreifchen erkennen lassen.

In dem von Salzsäure nicht zersetzten Reste ist der Gehalt an Nickel und Phosphor bemerkenswert. Wir müssen dies, da nicht anzunehmen ist, dass dieser Gehalt von einem Rest zufällig unzersetzt gebliebenen Meteoreisens herrühre, als ein Zeichen der Beimengung von Schreibersit ansehen. Das dazu gehörige Eisen erscheint natürlich in der Analyse unter dem Eisenoxydul. Daraus mag sich auch der Überschuss der Summe über 100 z. Th. erklären. Obwohl außerdem noch sicher Tonerde-haltiges Chromeisen vorhanden ist, kommt doch eine so bedeutende Menge von Tonerde neben einem beträchtlichen Quantum von Natron zum Vorschein, dass in dem Rest weiter auch ein feldspatiger Gemengteil vorausgesetzt werden muss, während dessen Hauptbestandteil offenbar ein augitisches Mineral ausmacht. Bringt man für letzteres die Gemengteile eines Bisilikats in Abzug, so bleibt ein Rest, in dem das Sauerstoffverhältnis zwischen Tonerde und der übrig bleibenden Kieselsäure zwar nahezu wie 3:9 verhält, es fehlt aber dann an der erforderlichen Menge der Kalkerde und Alkalien. Es lässt sich daher dieser von Säuren nicht zerlegte Anteil nur ungefähr berechnet als bestehend aus:
\begin{center}
    \begin{tabular}{ |l|r| } 
    \hline
    Schreibersit & 4,5\\\hline
    Chromeisen & 2,5\\\hline
    feldspatiges Mineral & 4,0\\\hline
    augitisches Mineral & 89,0\\
    \hline
    \end{tabular}
\end{center}
\paragraph{}
Im Ganzen bestände demgemäß der Chondrit von Schonenberg aus:
\begin{center}
    \begin{tabular}{ |l|r| } 
    \hline
    Olivin & 19,0\\\hline
    feldspatigem und Skapolithartigem Mineral & 18,5\\\hline
    augitischem Mineral & 40,0\\\hline
    Meteoreisen & 14,5\\\hline
    Schwefeleisen & 5,0\\\hline
    Schreibersit & 2,0\\\hline
    Chromeisen & 1,0\\
    \hline
    \end{tabular}
\end{center}
\paragraph{}
Der Dünnschliff dieses Meteorsteins (Figur IV. der Tafel) lehrt uns die außergewöhnliche Feinkörnigkeit der Gemengteile kennen, welche alle unregelmäßig splittrig, wie bei allen Chondriten, sind. Größere Mineralstückchen sind selten und ebenso vereinzelt die Chondren (o), deren Masse weiß trübe, staubartig feinkörnig, und an den Rändern schwach durchscheinend, aber i. p. L. buntfarbig, seltener exzentrisch faserig sich zeigt. Neben diesen rundlichen Körnchen kommen auch noch unregelmäßig eckige Fragmente von trüben, staubartigen und deutlich gestreiften Massen (b) und von jener eigentümlichen, äußert fein parallelstreifigen und quergegliederten, der Zellenmaschen der Moosblätter ähnlichen Struktur (c) vor, die in so vielen Chondriten als charakteristisch wiederkehrt. Das Meteoreisen bildet oft langgezogene, leistenartige Häufchen (d), scheint aber häufig auch wie eine dünne Rinde sich um die Chondren anzulegen.

Unter den größeren Mineralsplitterchen kann man die gelblichen, höchst unregelmäßig rissigen, im Umrisse mehr rundlichen als dem Olivin angehörig erkennen; sie zeigen i. p. L. die buntesten Aggregatfarben. Die etwas dunkler, farbigen, öfters etwas ins Rötliche spielenden Splitter des augitischen Minerals zeichnen sich durch eine mehr parallele Zerklüftung nach zwei Richtungen und i. p. L. gleichfalls sehr bunte Färbung aus, während die weißlichen, feldspatigen Bestandteile vielfach ins Trübe übergehen und i. p. L. von blauen und gelben Farbentönen beherrscht werden.

Nach alledem gehört der früher chemisch noch nicht unter sucht gewesene Meteorstein von Schönenberg der großen Gruppe der Chondriten an und nähert sich unter diesen durch den niederen Kieselsäuregehalt sehr dem Stein von Ensisheim, unterscheidet sich aber von diesem, wie von allen den durch Rammeisberg (a. a. O.) zusammengestellten Arten durch den relativ sehr geringen Bittererde-, hohen Tonerde- und Natrongehalt.

Die an der Oberfläche des Steins bemerkbaren schnurartigen Streifen scheinen Zerklüftungen des Steins zu entsprechen, auf denen, wie auf der Oberfläche, eine Schmelzrinde beim Fall durch die Atmosphäre sich gebildet zu haben scheint.
\clearpage
\subsection{Der Meteorstein von Krähenberg}
\paragraph{}
bei Zweibrücken in der Rheinpfalz.

(Figur V. und VI.)

Zu den erst in jüngster Zeit gefallenen und am Genauesten untersuchten Meteorsteinen gehört der Stein von Krähenberg. Über den Fall selbst berichten ausführlich Dr. G. Neumayer (Sitzungsb. d. Ac. d. Wiss. in Wien math. naturw. Cl. Bd. LX. 1869. S. 229), O. Büchner (Poggendorf Ann. Bd. 137. S. 176) und Weiss (N. Jahrb. 1869. S. 727 u. Poggendorfs Ann. Bd. 137. S. 617), über die Zusammensetzung [Gerhard] vom Rath (Poggendorfs Ann. Bd. 137. S. 328), an einer mikroskopischen Untersuchung der Dünnschliffe fehlte es jedoch bis jetzt. Wir entnehmen den oben angeführten Angaben über den Fall des Steins, dass am 5. Mai 1869 Abends 6 1/2 Uhr ein furchtbarer, einem Kanonendonner ähnlicher, aber weit stärkerer Knall gehört wurde, dem ein Rollen, ein Geknatter, wie von Musketenfeuer herrührend und ein Brausen, ähnlich dem Geräusche, des aus einer Lokomotive ausströmenden Dampfes folgte. Mit einem starken Schlag endigte plötzlich diese Geräusche, welches gegen 2 Minuten angedauert hatte. Man beobachtete an Orten bis auf 60 bis 70 Kilometer Entfernung vom Fallpunkte Krähenberg entweder Geräusch oder Lichterscheinungen, welch letztere als intensiv weiß angegeben werden. Zwei Knaben sahen den Stein zur Erde fallen und etwa 15-20 Minuten nach dem Fall grub man denselben aus der Erde, in die er ein senkrechtes, gegen 0,6 M. tiefes Loch sich gegraben hatte und auf einer Platte des unterliegenden Buntsandsteins liegen geblieben war.* Der Stein fühlte sich noch warm, aber nicht heisa an; er wog, nachdem wohl einige Kilogramm abgeschlagen worden waren, immerhin noch 15,75 Kilogramm und besaß einen Brodlaib ähnliche, aber etwas einseitig erhöhte rundliche Form, mit einem größeren Durchmesser von 0,30 m. und einem kleineren von 0,24 m., die außer der Mitte liegende größte Dicke oder Höhe ist 0,18 m.; die Grundfläche flach, ziemlich eben, die gewölbte Fläche dagegen höchst merkwürdig mit zahlreichen, vom glatten Scheitel aus, gegen den Rand strahlig verteilten, grubenförmigen, oft zu 0,03 m. langen Rinnen ausgestreckten, bis 8 mm. tiefen Furchen bedeckt. Zwischen diesen Gruben erheben sich dann schmale wellige Wülstchen, so dass die Oberfläche gleichsam tief blatternarbig durchfurcht erscheint. Die ganze Oberfläche ist mit einer schwarzen, stellenweis schaumigen Schlackenrinde vom 1/2 — 1 mm. Dicke bedeckt. Fleckenweis ist die Rinde dünn und bräunlich statt schwarz gefärbt, was, wie ich mich am Original überzeugte, daher rührt, dass an solchen Stellen schwerer schmelzbare Gemengteile sich vorfinden, die ein intensiveres Schmelzen verhinderten. Weiß hatte sogleich die Chondritennatur des Steins erkannt und macht auch auf die in der weißen Grundmasse liegenden dunkelgrauen, scharf abgegrenzten Fragmente aufmerksam, welche sich durch eingesprengte metallische Teilchen und weißliche Splitterchen ebenfalls als Gemenge, wie die grauen Kugeln erweisen. Vom Rath bestätigt dies und führt weiter an, dass der Krähenberger Stein auf der lichtgrauen Bruchfläche zahlreiche, in allen Richtungen ziehende, zuweilen zu einer Masche werke verbundene, feine schwarze Linien bemerken lässt. Es scheinen ihm Spalten zu sein, welche wenigstens z. Th. beim Eintritt des Meteors in die Erdatmosphäre sich bildeten und mit der schmelzenden Substanz der Rinde erfüllt wurden. Außer diesen Schmelzlinien schwärmen in den Steinen gekrümmte schmale Gänge anderer Art umher, die aus Nickeleisen bestehen. Es sind gangähnliche Parthieen von ansehnlicher Dicke. Ich konnte eine solche über 3 Zoll lange, wenig Gekrümmte 1/3-1/2 mm. dicke Erzader auf einer Bruchfläche deutlich beobachten. Außerdem kommen auch Eisenspiegel, wie im Stein von Pultusk vor, dem auch die Masse sehr ähnlich, doch weniger feinkörnig ist. Als Gemengteile erkannte vom Rath Nickeleisen, Magnetkies, Chromeisen, Olivin und die charakteristischen Kugeln, welche Gemengteile in einer aus weißen und grauen Körnern gebildeten sphärolithischen Grundmasse liegen. Den Gehalt an Nickeleisen (aus 84,7 Eisen und 15,3 Nickel) bestimmte er zu 3,5\%, so dass 96,5\% auf die Silikate, Magnetkies and Chromeisen kommen. Von Schmelzrinde freie Stückchen besitzen das spec. Gew. 3,4975 bei 18$^{\circ}$ C., an Schmelzrinde reiche Stückchen 3,449 bei 20$^{\circ}$ C., wonach sich die Beobachtung am Pultusker Stein bestätigt, dass die Schmelzrinde spezifisch leichter ist als die steinige Masse des Innern.

Das Schwefeleisen hält vom Rath, obwohl es nicht vom Magnet gezogen wird, nicht für Troilit, sondern für Magnetkies, weil sich bei der Behandlung mit Salzsäure in reichlicher Menge Schwefelwasserstoff entwickelt und eine Menge Schwefel ausgeschieden wird. Er bestimmte den Gehalt an Magnetkies zu 5,52\%.

Die dunkelgrauen bis schwarzen Körner, bis 2 mm. groß, zeigen bisweilen eine äußerst feine, sich sehr leicht ablösende, weiße Hülle. Dazu kommen unregelmäßig gerundete, dunkle Körner und Kugelsegmente, welche wie erstere, wenn gleich nur unvollkommene Faserzusammensetzung besitzen. Weiter zeigen sich bis 1 mm. große, gelblich weiße Körner — wahrscheinlich Olivin mit gerundeten Oberflächen und nur Andeutungen von kristallinischer Umgränzung. Schwarze, kleine Chromeisensteinkörner scheinen eine oktaedrische Form erkennen zu lassen. Die Hauptmasse des Steins stellt sich unter dem Mikroskop als ein Haufwerk unendlich kleiner, weißer, kristallinischer Körnchen dar Sie sind hell, lebhaft fettartig glänzend, zeigen Farben i. p. L.; sind in Säuren unlöslich und bestellen wesentlich aus einem Magnesiasilikate, das reicher an Kieselsäure, als Olivin ist. Daneben kommt auch noch eine lichtgraue Substanz, welche Anlage zu sphärolithischer Bildung besitzt, und wie die dunklen Kugeln auch zuweilen faserige Zusammensetzung zeigt, vor.

Mikroskopisch fanden sich noch als seltene Gemengteile vor: außerordentlich kleine, purpurrote Kristallteilchen, mehrere intensiv gelbe Körnchen mit deutlichen Krystallflächen, einige lichtgelbe, langprismatische Formen und endlich einzelne, bis 1/2 mm. große, rote Körnchen, von muscheligem Bruche und durchscheinend — wahrscheinlich Zersetzungsprodukt des Schwefeleisens, dem Caput mortuum ähnlich.

Die Analyse des nicht magnetischen Anteils ergab nach vom Rath:
\begin{center}
    \begin{tabular}{ |l|r|r| }
    \hline
    & I & II\\\hline
    & & Nach Abzug von Chromeisen und Magnetkies\\
    \hline\hline
    Chromeisen & 0,94 & -\\\hline
    Magnetkies Schwefel & 2,25 & -\\\hline
    Magnetkies Eisen & 3,47 & -\\\hline
    Kieselsäure & 43,29 & 46,37 Sauerstoff 24,73\\\hline
    Tonerde & 0,63 & 0,67 Sauerstoff 0,32\\\hline
    Magnesia & 25,32 & 27,13 Sauerstoff 10,85\\\hline
    Kalkerde & 2,01 & 2,15 Sauerstoff 0,61\\\hline
    Eisenoxydul & 21,06 & 22,56 Sauerstoff 5,01\\\hline
    Manganoxydul & Spur. & -\\\hline
    Natron (Verlust) & 1,03 & 1,12 Sauerstoff 0,29\\
    \hline
    \end{tabular}
\end{center}
\paragraph{}
Demnach verhält sich die Summe der Sauerstoffmengen der Basen gegen die der Kieselsäure wie:

1:1,448,

welches Verhältnis gegen das des Pultusker Steins (1:1,507) auf keine wesentliche Verschiedenheit schließen lässt. Als wesentliche Gemengteile ergeben sich auch nach der chemischen Analyse: Olivin und ein kieselsäurereiches Mineral, ob Enstatit oder Shepardit oder beide gleichzeitig, lässt vom Rath unentschieden.

Die Beimengung von Anorthit oder Labrador hält er für unzulässig, weil Kalk- und Tonerde dem unlöslichen Anteil angehören und nur in geringer Menge mit Säuren sich ausziehen lassen.

Einer gefälligen Mitteilung verdanke ichferner die Kenntnisnahme der Resultate einer Analyse, welche Herr Professor Dr. Keller in Speyer vorgenommen hat und welche deshalb von grösser Wichtigkeit ist, weil sie mit einer bedeutenden Quantität durchgeführt wurde, nämlich mit 5,71 Gramm; gefunden wurden:
\clearpage
\section{Abbildungen}
\clearpage
\pagestyle{fancy}
\fancyhf{}
\rhead{Figur 1}
\cfoot{\thepage}
\begin{figure}[t]
%\includegraphics[width=\textwidth,height=\textheight,keepaspectratio]{fig1.jpeg}
\centering
\end{figure}
\clearpage
\rhead{Figur 2}
\begin{figure}[t]
\centering
%\includegraphics[width=\textwidth,height=\textheight,keepaspectratio]{fig2.jpeg}
\end{figure}
\clearpage
\end{document}
