\documentclass[a4paper, 12pt, oneside]{article}
\usepackage[utf8]{inputenc}
\usepackage{fouriernc}
\usepackage{booktabs}
\setlength{\emergencystretch}{15pt}
\usepackage{fancyhdr}
\usepackage{graphicx}
\graphicspath{ {./} }
\usepackage{booktabs}
\begin{document}
\begin{titlepage} % Suppresses headers and footers on the title page
	\centering % Centre everything on the title page
	\scshape % Use small caps for all text on the title page

	%------------------------------------------------
	%	Title
	%------------------------------------------------
	
	\rule{\textwidth}{1.6pt}\vspace*{-\baselineskip}\vspace*{2pt} % Thick horizontal rule
	\rule{\textwidth}{0.4pt} % Thin horizontal rule
	
	\vspace{1\baselineskip} % Whitespace above the title
	
	{\LARGE Meeting Reports of the\\[0.05in] Mathematical-Physical Class of the\\[0.05in] Royal Bavarian Academy of Sciences\\[0.15in] in Munich.}
	
	\vspace{1\baselineskip} % Whitespace above the title

	\rule{\textwidth}{0.4pt}\vspace*{-\baselineskip}\vspace{3.2pt} % Thin horizontal rule
	\rule{\textwidth}{1.6pt} % Thick horizontal rule
	
	\vspace{1\baselineskip} % Whitespace after the title block
	
	%------------------------------------------------
	%	Subtitle
	%------------------------------------------------
	
	{Volume 5 --- Year 1875} % Subtitle or further description
	
	\vspace*{1\baselineskip} % Whitespace under the subtitle
	
    {\small In Commission with G. Franz} % Subtitle or further description
    
	%------------------------------------------------
	%	Editor(s)
	%------------------------------------------------
	
	\vspace{1\baselineskip}

    \vspace*{\fill}

	Munich 1875
	
	Academic Publishing Office of F. Straub

    \vspace{0.5\baselineskip} % Whitespace after the title block

    Internet Archive Online Edition  % Publication year
	
	{\small Attribution NonCommercial ShareAlike 4.0 International } % Publisher
\end{titlepage}
\setlength{\parskip}{1mm plus1mm minus1mm}
\clearpage
\tableofcontents
\clearpage
\section{Mr. Carl Wilhelm von Gümbel Gives a Lecture: On the Nature of the Stone Meteorites from the Fall of February 12, 1875 in Iowa County North America}
\subsection*{Introduction}
\paragraph{}
There took place, according to information from John Lawrence Smith,\footnote{\emph{Meeting Reports of the Academy of Sciences in Paris}. Vol. 80. No. 23. 1875. p. 1451.} on February 12\textsuperscript{th} of this year, in Iowa County of North America, in the evening around ten-thirty under a slightly cloudy sky, a powerful bang\footnote{\emph{The American Journal of Science and Arts}. Dana and Silliman. May 1875. Vol. 9. No. 53. p. 407.} from a meteor fall visible for miles around, which delivered a large number of stones. Smith reported that by then approximately 150 kilograms of stones had been collected, of which 25 kilograms belonged to Prof. [Gustavus Detlef] Hinrichs. The academy is in debt to his charitableness, for he donated a splendid part weighing approximately 1,500 grams, which gave rise to the following description on the nature of this highly peculiar meteorite.

The Iowa [Homestead] meteorite belongs to that most commonly occurring class of stones, which one refers to by the name of chondrites, or according to [Gabriel Auguste] Daubrée, in the domain of the sporadosiderite and in the group of the oligosiderite, as Prof. Hinrichs had correctly noted\footnote{\emph{Meeting Reports of the Academy of Sciences}. 1875. p. 1175.} in his accompanying letter to the Paris Academy, which included a portion of all these meteorites, and corroborated by Daubrée himself.

The considerably sharp-edged, acute-angled, unevenly tetrahedral stone is coated all over with a black fusion crust, and inside light grayish white, furnished with abundant small black nodules and granules of meteoritic and sulphuric iron and seemingly scattered small rust stains. The stone is rather hard and cannot be crushed with the hand. Its overall character is not much different from the stone meteorite of Pultusk, in that, apart from the meteoritic and sulphuric iron, its main mass is whitish and yellowish, in which the individual shiny glass olivine granules contrast with the partly dark, partly light, sometimes opaque globules (chondrule spheres). Daubrée\footnote{Ibid.} compares it with the stone meteorites of Vouillé (May 13, 1831) and of Aumale in Algiers (August 25, 1865). Through this fall the tally of this type of most prevalent meteoritic stone, the chondrite, already above all others, is again increased by one and the impression of a unitary source of all these fragments from a once connected whole, which Meunier\footnote{Course on Comparative Geology. Compare with: Tschermak, \emph{The Formation of the Meteorites, Meeting Reports of the Academy of Sciences in Vienna}. Vol. 71, Sec. 2. 1875.} also recently so strongly stressed, is significantly reinforced.

The exterior, fairly sharp-edged and angular form of the stones in these falls, which is only slightly obscured by the thin, superficial fusion crust, undoubtedly indicates debris of a larger, fragmented stone mass, which was formed from the destruction of an already completely finished solid substance. That this dismemberment partly took place during the fall through the Earth's atmosphere is implied by Smith's\footnote{A. a. O. p. 1453.} observation that stated that a number of the stones looked as if they were freshly broken and that melting had only started to appear on these fractured surfaces. Incidentally, however, one detects neither rounding, nor filamentous expanding or cord-like twisting, striped formations, such as a soft, malleable body would obtain while moving along a cosmic orbit, or in flight during a volcano-like eruption, as one is obliged to suppose, like the lapilli and volcanic bombs. Even the inner, grainy debris-like nature without traces of glass- or lava-like particles, which cannot be brought into direct agreement with a molten liquid fire mass, decidedly rejects the notion of an eruption product in the style of our volcanoes. The external form and internal nature of this kind of meteorite does not speak, from a petrographic standpoint, in favor of the conjecture that these meteoritic stones were ejected from the Moon as creations of huge volcano-like eruptions. Also, equally implausible is their origin out of the host of shooting stars, because the times of the meteorite falls, insofar as the observations suffice, do not coincide with the times at which the shooting stars appear to fall at their maximum. What's more, this conjecture barely explains the very striking homogeneity in the composition of the stone meteorites. Hence the point of view is gaining more feasibility, that they are fragments from a celestial body, which through a destruction, engendered as a consequence of collision or due to a kind of pulverization from interior sources, whereby the centrifugal force of the excess of weight exceeds the original ability of attraction and the debris managed to come into the vicinity of Earth's pull, forced them to fall. Whether they are members of asteroid bodies or, as Meunier desires, a second satellite of the Earth reserves to be decided on by astronomical considerations, and is far from the point.
\clearpage
\subsection{Crust}
\paragraph{}
The available stone meteorite from Iowa is externally, apart from a minor man-made break, coated all over with a black, matte lustered, slightly rugose crust on average 0.05 m thick. This glass-like coating is coarsely cracked, fissured, and quite easily detaches from the main mass, whereby pieces of the latter remain adhered to it. In the interior of the stones one does not detect the presence of any veins or smooth surfaces similar to the crust, which for example so often pervade the stones of Pultusk.

This crust, based on closer examination, is comprised of a highly transparent, glass-like mass, which easily refracts the light and in numerous places encloses vesicles and porosities, but not in so singular a manner, as I have observed in the crust substance from the stones of Pultusk. The crust is not completely spread out in a uniform way; at distinct locations one discerns, with a gentle rub, protruding meteoritic iron particles with a metallic gleam, on shifting it is very thin and tinted a little brighter, or else even thicker and at the same time usually shining even stronger. As thin sections indicate, finely crusted places match up with olivine grains intruding into the crust region, while a thicker fusion crust is formed where sulphuric iron occurs.

It is very challenging, due to the deep coloration, to obtain transparent crust in thin sections. It works out more easily to crush smaller chippings between two glass plates. They reveal thereupon a deep bottle-green up to a brownish-red color and behave in polarized light like an amorphous glass mass. These qualities validate the assumption that the crust was formed by the surface becoming molten as it flew through the Earth's atmosphere, in other words it represents a genuine fusion crust. For comparison, melting small fragments from the interior of the stone can be simply accomplished with very thin pieces at the fine points. The melted mass displays the full nature of the fusion crust, the same color and the same vesicles. The stone behaves peculiarly when one exposes it, without melting, to an intense red heat for a long time. In the process it takes on a dark, brownish-black color and shows distinct patches with a molten appearance when pierced. These are around the edges of the furbished pyrites, which have endured through the action of melting. If one produces thin sections of such annealed pieces, then one can see in them that the majority of the mass, of which the stone consists, has taken on a brown color due to the annealing, which as I have already emphasized earlier\footnote{\emph{The Paleolithic}. \emph{Eruptive Stones of the Fichtel Mountains}. 1874. p. 39.}, makes for a very good indicator of olivine admixtures. The black edges around the pyrite particles are nearly opaque, colored deep brown, and refract light in a simple way, like the fusion crust. This darkened color, which the stone acquires with heating, is not found naturally in the stone beneath the fusion crust, demonstrating that the heat of melting restricted its action to an exceptionally thin layer of the surface, without transferring degrees of heat towards deeper parts of the stone. Compared with this appearance, the well-done veining of some meteoritic stones from other places of recovery with very thin black little strips is highly remarkable. In the stone of Pultusk, of which I had material at my disposal, I detected that these small veins likewise consisted of an amorphous glass substance. They also seem to be related to the black, nearly opaque marks which are found scattered in some meteoritic stones and presumably represent minor melt flows that generated mixtures, for instance pyrite.

Having said this, I do not think that the fine small veins mentioned above were a molten mass that infiltrated into the interior of the stones from the crust, but that the stone was broken or fissured in such places, and that these breaks were accessible to the atmosphere which performed the same melting process through friction, as on the surface itself.
\clearpage
\subsection{Stone Mass}
\paragraph{}
The main mass of the stone, which is rather hard and not friable with the fingers, is made of an aggregate of debris particles, which are agglutinated together without any intermediate substance, as neither a glass-like nor even a distinctive binding agent between the distinct granules can be observed. In great number in the main mass are found tiny little slivers of minerals with totally irregular contours, such as those resulting from the destruction of crystals or crystalline masses. Only very seldom does one see --- in thin section --- such little pieces, which are delimited by regular straight lines and could be held as small crystals or small regular cleavage objects (\emph{k} of the lithograph table). To this is added irregular, angular granules that can be quite safely identified as olivine by their glassy luster and their color (\emph{o}), whitish plaster of an opaque substance, small granules of lead-grey, meteoritic iron with metallic luster (\emph{f}), tombac yellow little heaps of sulphuric iron in many cases perforated (\emph{s}), the fine granules of which rarely account for the inferred mass and finally those small, rounded, almost dark-, almost light-colored globules (spheroidal chondrules \emph{c}), which impress upon the stone the character of [Gustav] Rose's chondrite. Sparsely positioned or concentrated into tiny clusters, there are utterly fine, black dust particles without a metallic luster (\emph{ch}), which either are associated with chrome iron or a carbonaceous substance, since they resist all action of acids.
\clearpage
\begin{figure}[t]
\centering
\includegraphics[keepaspectratio]{Fig1.png}
\caption{Table 1}
\end{figure}
The image included in the lithograph table shows the sort of distribution of these constituent minerals in a thin section at 25 times magnification.

Explanation of the Annotations of the Lithograph\\
\begin{minipage}[t]{0.54\textwidth}
\emph{o} --- Olivine,

\emph{a} --- Augite piece,

\emph{f} --- Meteoritic iron,

\emph{s} --- Sulphuric iron,

\emph{ch} --- Chrome iron,

\emph{k} --- Piece with well-behaved crystal contours,

\emph{io} --- Olivine granule in meteoritic iron,

\emph{g} --- Reddish garnet-like inclusion,
\end{minipage}
\begin{minipage}[t]{0.46\textwidth}
\emph{c} --- Spheroidal chondrules, namely:

\emph{cc} --- with concentric structure,

\emph{sc} --- with fibrous structure,

\emph{fc} --- with radial structure,

\emph{kc} --- with granular structure,

\emph{oc} --- consisting of olivine,

\emph{dc} --- opaque finely granulated globules.
\end{minipage}
\paragraph{}
A peculiar occurrence with practically all the constituent minerals, excluding the metallic ones, is demonstrated by the existence of an astonishing quantity of thin and very fine cracks that permeate individual pieces. With some constituent minerals, a certain regularity is seen in the direction of these unending fissures due to a parallel progression of the cracks, which probably are related to the cleavage direction of the relevant minerals. But at the same time, alongside these more regular cracks emerge others that cross them at right angles or obliquely and create a veritable network of breaks, so that even otherwise clear mineral pieces show up clouded. They must be viewed as a sign of destruction incurred by impact, pressure, or rapid changes of temperature.

Due to this cracked condition of most of the constituent minerals, the comprehensive inner nature is often obscured, so that it is but rarely in individual larger particles that what seem to be common vesicles can be discerned --- only so far as my observations suffice --- devoid of fluid inclusions. Utterly fine, dust-like mixtures are also frequently present in the otherwise clear mineral particles, while actual microliths seem to be missing.

As far as the mineralogical nature of the distinct constituent minerals is concerned, a great number of them cannot be associated with simple minerals, but rather represent stone fragments composed of a more or less regular intergrowth of different minerals.

Olivine undoubtedly takes first place among the simple mineral parts. Not only in the exterior appearance, the color, the peculiar sheen pointed out on lots of the larger granules, and the tiny crystal fragments of olivine, but also this stipulation finds confirmation in the decomposition of these particles by hydrochloric acid, in the turning-brown through annealing, and in the motley play of colors with the application of polarized light in thin sections. Much of the finely granulated, fissured fragments in the figure belong to olivine (\emph{o}), as well as many of the crystal-like regularly defined slivers and even a number of the spherical depositions turn out to be reliably identified as olivine. Even more, olivine pieces are also noticeable in the fine powder-like intermediate mass, which appears to join the constituent larger fragments, as can be detected during the turning-brown of annealing. Most curiously, the olivine substance in some panel-like striated globules (\emph{sc} in the figure) with a white, feather-like straight-grained substance, such as occurs in the radiating fibrous globules, are intergrown in a lamellar entangling like a kind of graphic granite. The narrow, intersecting depositional little olivine lamellae come out very clearly after the annealing due to their dark brown coloration. That they are associated with an olivine substance is revealed through treatment with hydrochloric acid, whereby they are corroded, though many intermediate lamellae remain unaltered.

I was not able to detect feldspathic component parts with certainty, even though individual water-clear small needles in polarized light exhibit the peculiar pale yellow and blue colors, so characteristic of feldspar, and even though I, with all certainty, observed them in great quantity in the meteorite of L'Aigle (fell on April 26, 1803), which incorporated numerous little feldspar needles in the stone debris. The chemical analysis also confirms that at any rate feldspathic components are only contributing to the composition in a most minor way.

If one treats quite a lot of fine powder with hydrochloric acid in heat for a long time, a large part of the stone mass --- of the olivine portion --- separates into a slimier silicic acid without actually forming a gel. In the remains released by boiling silicic acid, one can now spot very numerous, often water-clear, little pieces with parallel striations, alongside cloudy, powdery-grained residues, most of which originate from shattered globules. The fine, black granules, which are deposited here and there in groups, have also been left undissolved, while, along with olivine, the meteoritic and sulphuric iron have gone into solution. The more or less water-clear small pieces, the ones that remained undissolved, turn out to be birefringent and exhibit the most beautiful aggregate colors in polarized light, and if the rest is treated still further with hydrofluoric acid, it completely breaks down into fine black granules, which are associated with chrome iron or a carbonaceous substance. Since the dissolution of the stone mass by means of dehydrated barite produces a substance with chrome, it is highly likely that the black granules are chrome iron. To be sure, I noted that several times during the annealing of the pulverized stone a sporadic smoldering occurred, such as from carbonaceous bits, and I was unable to ascertain whether or not this was caused by dust particles that did not initially belong to the stone, but only adhered mechanically.

If one modifies the experiment in such a way that one boils sheets, not too thinly ground but decently transparent, of the stone in hydrochloric acid, they will be preserved through their cohesion. Included in a glass slide and then treated carefully with caustic potash, in order to get rid of the released silicic acid, it produces a preparation full of holes from which the olivine, meteoritic iron, and sulphuric iron have disappeared, while the white mineral and a lot of the globules have remained unaltered. If one tries to preserve the preparation obtained this way, by Canada balsam under a coverslip, the slight pressure applied by placing the coverslip breaks apart the mass into separate little heaps of the white minerals, into isolated flakes and into round little balls which often protrude loosely and reveal an uneven, rough surface. Furthermore, very sparse, tiny, light garnet red, rather regularly 5-6 sided objects become noticeable, which I also observed in the thin sections (\emph{g}). They remind one of garnets, but show double refraction. The color is even reminiscent of noseau [noselite]. Yet, even so, the optical properties are not right.

Nothing but a chemical analysis has the ability to provide information about the nature of the clear, small mineral pieces undecomposed in hydrochloric acid, which likely belong to the augite group. Though, sure enough, even here uncertainty sets in, because there is also the presence of numerous globules, intact in the hydrochloric acid (apart from the olivine grains), that are neither composed identically to the clear mineral nor correspond to any simple mineral. Many of these globules approximate the white mineral in their physical characteristics, but still exhibit a strange type of fissure. Others are noticeably comprised of distinct lamellae of intergrown minerals and still others a little transparent, white, powdery, granular, and in many cases showing a concentric structure with dark and light zones, often even with a dark rind-like shell or a partly dark, partly light center. Black, dust-like specks that are found in them are likewise usually organized concentrically or radically. Nonetheless, these specks are not amorphous, since the shine of polarized light appears considerably tinged. Finally, these are concluded by the strangest kind of these globules, which seem to be very finely radially-striped and finely-granulated, slightly transparent, and whitish in color. The beaming little strips are eccentric and maintain no relation with the external form of the globules. In some globules, there is often a number of systems of little strips next to each other in a panel-like manner. In polarized light, despite the low transparency, noticeably tuft-shaped stains show up, which are reminiscent of the well-known phenomenon of many variolite nodules, though without them being quite the same. The lamellar intergrowth of small olivine-like strips with a likewise fibrous white substance has already been mentioned.

Concerning the the formation of these curious constituent components of the meteoritic stones, Daubrée\footnote{\emph{Journal des Savants}. 1870. p. 38.} assumed that they had formed by a solidification during a vortical flight through gases, Tschermak\footnote{\emph{Meeting Reports of the Royal Academy of Sciences in Vienna}. Vol. 71, Sec. 2. 1875. April Issue. pp. 9-10.} was in favor of a development as a result of a tumbling of already solid debris through a prolonged flow, such as is produced during a volcanic explosion, referring to similar such round globules in the trachytic tuffs of [Bad] Gleichenberg, etc. The latter hypothesis explains the peculiarities perceived in many of the globules, that their inner chamfered structure is devoid of any relation to the external spherical shape. Even for the globules with a clearly concentric structure, this mode of formation may be held, if one assumes that, as is quite likely, the concentric strips and shelled dissociations are merely secondary phenomena, as a result of mechanical and chemical variations, that are to be understood as incurred only after the tumbling of the rounded grain.

Sulphuric iron makes up a significant portion of the composition of the stone from Iowa. It shows up spread into tiny irregularly defined spots, sandwiched, so to speak, between the constituent pieces. When the stone powder is treated with hydrochloric acid, hydrogen sulfide emerges, without sulphur precipitating. Hence, it is justified to denote this sulphuric iron as troilite. Appearing even more frequently are granules of the stone mass consisting of admixed little clumps of meteoritic iron, which are usually jagged, angularly bent, and often tapered into fine points, and, wherever they are found, cling tightly to the non-metallic portions such as if this iron had only been deposited lastly, perhaps due to reduction at the location. This meteoritic iron contains nickel, is a little bit phosphoric, very malleable, as it can be easily broken into thin little sheets with a hammer, and active, as revealed when a polished piece is immersed in vitriol of copper, whereby the iron surface is rapidly coated with a copper precipitate. Whether Widmanstätten lines appeared with a slight etching, I was not able to clearly discern due to the smallness of the iron granules. Nevertheless, lighter and darker marks were present.

That the stone incorporates water requires no further evidence, as the not so rare rust stains --- hydrated iron oxide --- reveal.

Various types of gas have already been accounted for by Wriht\footnote{\emph{The American Journal of Science and Arts}. James Dana and Silliman. May 1875. Vol. 9. No. 54. p. 459; also \emph{Annals of Chemistry and of Physics}. Ergänz. Vol. 7, Part 2.} in this meteorite from Iowa. The provisional experiments of Wriht yielded a gas content of which almost half was made of carbon dioxide and carbon monoxide (CO\textsubscript{2} = 35; CO = 14), with the remaining being comprised primarily of hydrogen.

The specific weight of the stone in its interior mass amounts to 3.75; that of a piece of crust is 3.55 at 20° C.
\clearpage
\subsection{Chemical Analysis}
\paragraph{}
I had slightly more than 1.5 grams of material available to carry out a chemical analysis. To begin with, the meteoritic iron was extracted from the finely pulverized powder with all due care and by repeating this process as much as possible to liberate all the adhering stone pieces, thereupon analyzed in particular. One portion served for the measurement of sulfur, while the leftover was first treated with boiling hydrochloric acid, and in this way a decomposed and an undecomposed fraction, still further dissolved by means of hydrated barite, was analyzed.

The findings were as follows here:

The stone is comprised of
\begin{center}
    \begin{tabular}{ |l|r| } 
    \hline
    Meteoritic iron & 12.32\\\hline
    Troilite & 5.25\\\hline
    the portion decomposable in hydrochloric acid & 48.11\\\hline
    the portion undecomposable in hydrochloric acid & 34.32\\
    \hline
    \end{tabular}
\end{center}
\paragraph{}
Excluding traces of copper and sulphur, the latter presumably stemming from bits adhering to the troilite, the nickel iron is comprised of

Iron 83.38

Nickel (containing a little cobalt with sulphur and phosphorus) 16.62

hence, likely Fe\textsubscript{5}Ni

The part\footnote{These analyses were performed by assistant Mr. Ad. Schwager. (1875. 3. Math.-Phys. Class)} decomposable in hydrochloric acid (calculated without meteoritic and sulphuric iron) is made of
\begin{center}
    \begin{tabular}{ |l|r|r| } 
    \hline
    Silicon dioxide & 38.38 & Oxygen: 19.76\\\hline
    Iron(II) oxide & 28.58 & 6.33\\\hline
    Manganese(II) oxide & 0.53 & 0.12\\\hline
    Magnesium oxide & 31.49 & 12.59\\\hline
    Aluminum oxide & 1.01 & 0.47\\\hline
    calcium oxide, alkalis, water & Traces &\\
    \hline
    \end{tabular}
\end{center}
\paragraph{}
The rest, undecomposed in hydrochloric acid, consists of\footnote{Ibid.}
\begin{center}
    \begin{tabular}{ |l|r|r| } 
    \hline
    Silicon dioxide & 53.96 & Oxygen: 28.74\\\hline
    Aluminum oxide & 2.01 & 0.94\\\hline
    Iron(II) oxide & 25.18 & 5.57\\\hline
    Magnesium oxide & 8.91 & 3.56\\\hline
    Calcium oxide & 4.04 & 1.16\\\hline
    Manganese(II) oxide & Traces &\\\hline
    Chromium(II) oxide & 1.42 & 1.16\\\hline
    Natron & 2.39 & 0.59\\\hline
    Potash & 1.67 & 0.29\\
    \hline
    \end{tabular}
\end{center}
\paragraph{}
As concerns the meteoritic iron and the ordinary sulphuric iron, there is not much need for discussion over this. In the portion decomposable by hydrochloric acid, the oxygen ratio of the bases and acids is nearly 1:1 and indeed, here as well, it hardly requires any further explanation that this portion is largely derived from an olivine with a preponderance of rich iron(II) oxide. Far more difficult is the interpretation of the best of those undecomposable in hydrochloric acid, whose constituent parts and their oxygen ratios do not match any defined mineral. This also completely agrees with the optical analysis in which, following the removal of the parts soluble in hydrochloric acid, a light, cracked mineral and tiny black grains were detected, besides the spheroidal chondrules with their highly diverse nature. That the former are comprised of chrome iron is now hardly in doubt, according to the results of the analysis. The light, cracked mineral is likely sure to belong to the augite group. Totally unusual is the high iron(II) oxide content, even if one makes an allowance for an appropriate portion being associated with chromium(II) oxide on the chromium iron, whereas the lack of magnesium oxide and calcium oxide on the other hand is striking. The high content of alkali still seems to have more connection to the composition of the globules and indicates their feldspathic compounds. Presumably, the aluminum oxide is part of these constituent components in correspondence with the amount of silicon dioxide, as was finally figured out --- though always just incidentally corresponding to an iron-rich augite composition, such as found in the eucrites, for instance as highlighted in those of Juvinas. Still the intimate nature of these augitic constituent components remains difficult to determine. Even though the analysis of the Iowa meteorites that J. L. Smith\footnote{\emph{Meeting Reports of the Academy of Sciences in Paris}. Vol. 80. No. 23. 1875. p. 1452.} communicated does not exactly hold true with the above, it nevertheless indicates an unusually high iron(II) oxide content in the portion insoluble in acid, namely 27.41\%. In order to compare, Smith's statements are included here:

The entire stone is comprised out of:
\begin{center}
    \begin{tabular}{ |l|r| } 
    \hline
    Stony mass & 81.64\\\hline
    Troilite & 5.82\\\hline
    Nickel iron & 12.54\\
    \hline
    \end{tabular}
\end{center}
\paragraph{}
The stony part contains:

A) 54.15 decomposable in acids,

B) 45.85 substances undecomposed in acids.

This is further comprised of
\begin{center}
    \begin{tabular}{ |l|r|r| } 
    \hline
    Silicon dioxide & 35.61 & 55.02\\\hline
    Iron(II) oxide & 27.20 & 27.41\\\hline
    Magnesium oxide & 33.45 & 13.12\\\hline
    Aluminum oxide & 0.71 & 0.84\\\hline
    Alkalis, iron, etc. & 1.45 & 2.01\\
    \hline
    \end{tabular}
\end{center}
\paragraph{}
Smith then calculates the composition of the meteorite as:
\begin{center}
    \begin{tabular}{ |l|r| } 
    \hline
    Olivine & 44.09\\\hline
    Pyroxene & 37.55\\\hline
    Troilite & 5.82\\\hline
    Nickel iron & 12.54\\
    \hline
    \end{tabular}
\end{center}
\paragraph{}
The round globules did not get further consideration in the account, which certainly does not seem natural, because these globules cannot be considered as consisting of augite.

Among the chondrites analyzed up till now, it is only that of Tadjera with a similar composition,\footnote{Rammelsberg. \emph{The Chemical Nature of the Meteorites}. p. 157.} though poorer in silicon dioxide and richer in calcium oxide.

Bringing together the findings of this survey of the stone meteorite of Iowa, they justify the following conclusions:
\begin{enumerate}
    \item The stone mass is comprised of irregular little mineral fragments of olivine and a substance related to augite, and appears to have been taken from a shattered rock. These same distinct small pieces are assembled from different admixed minerals. Also, a feldspathic substance seems to be present in low quantities. Finely pulverized pieces of these minerals seem to surrender the filling agent.
    \item Aside from the alluded to small mineral pieces, a significant part of the substance of the stone is made of the roundish globules. They partly belong to olivine and partly represent lamellar intergrowth of minerals or exist as a radial, fibrous mass. A portion of these appear to be of a feldspathic substance. They owe their form to a mechanical rounding.
    \item The meteoritic iron granules are nestled between the little mineral slivers and globules, as if they were formed retroactively due to reduction.
    \item There is nothing to be found in the rock of glass or lava-like admixtures (with the exception of the fusion crust). It is not a crystalline rock that solidified from a melt flow, but rather a clastic rock, the aggregate particles of which do not have the properties of volcanic ash.
\end{enumerate}
\clearpage
\end{document}
