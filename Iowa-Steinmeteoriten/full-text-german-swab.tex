\documentclass[a4paper, 11pt, oneside]{article}
\usepackage[utf8]{inputenc}
\usepackage[T1]{fontenc}
\usepackage[ngerman]{babel}
\usepackage{yfonts}
%\usepackage{fbb} %Derived from Cardo, provides a Bembo-like font family in otf and pfb format plus LaTeX font support files
\usepackage{booktabs}
\setlength{\emergencystretch}{15pt}
\usepackage{fancyhdr}
\usepackage{graphicx}
\graphicspath{ {./} }
\usepackage{microtype}
\usepackage[figurename=]{caption}
\begin{document}
\swabfamily
\begin{titlepage} % Suppresses headers and footers on the title page
	\centering % Centre everything on the title page
	%\scshape % Use small caps for all text on the title page

	%------------------------------------------------
	%	Title
	%------------------------------------------------
	
	\rule{\textwidth}{1.6pt}\vspace*{-\baselineskip}\vspace*{2pt} % Thick horizontal rule
	\rule{\textwidth}{0.4pt} % Thin horizontal rule
	
	\vspace{1\baselineskip} % Whitespace above the title
	
	{\scshape\LARGE Sitzungsberichte der\\[1.25pt] Mathematisch-Physikalischen Klasse der\\[1.25pt] K"oniglich Bayerischen Akademie der\\[1.25pt] Wissenschaften zu M"unchen\\[1.25pt]}
	
	\vspace{1\baselineskip} % Whitespace above the title

	\rule{\textwidth}{0.4pt}\vspace*{-\baselineskip}\vspace{3.2pt} % Thin horizontal rule
	\rule{\textwidth}{1.6pt} % Thick horizontal rule
	
	\vspace{1\baselineskip} % Whitespace after the title block
	
	%------------------------------------------------
	%	Subtitle
	%------------------------------------------------
	
	{\scshape Jahrgang 1875 --- Band 5} % Subtitle or further description
	
	\vspace*{1\baselineskip} % Whitespace under the subtitle
	
    {\scshape\small In Kommission bei G. Franz} % Subtitle or further description
    
	%------------------------------------------------
	%	Editor(s)
	%------------------------------------------------
    \vspace*{\fill}

	\vspace{1\baselineskip}

	{\small\scshape M"unchen 1875}
	
	{\small\scshape{Akademische Buchdruckerei von F. Straub}}
	
	\vspace{0.5\baselineskip} % Whitespace after the title block

    \scshape Internet Archive Online Edition  % Publication year
	
	{\scshape\small Namensnennung Nicht-kommerziell Weitergabe unter gleichen Bedingungen 4.0 International} % Publisher
\end{titlepage}
\setlength{\parskip}{1mm plus1mm minus1mm}
\clearpage
\tableofcontents
\clearpage
\section{\swabfamily{Herr Karl Wilhelm von G"umbel h"alt einen Vortrag: "Uber die Beschaffenheit des Steinmeteoriten vom Fall am 12. Februar 1875 in der Grafschaft Iowa Nordamerika}}
\subsection*{\swabfamily{Einleitung}}
\paragraph{}
Am 12. Februar dieses Jahres ereignete sich in der Grafschaft Iowa in Nordamerika nach den Angaben von John Lawrence Smith\footnote{\swabfamily{Comptes rendus d. seance de l'Academie d. sc. à Paris. T. LXXX. Nr. 23. 1875. p. 1451.}} Abends um 10 1/2 Uhr bei leicht bew"olktem Himmel unter starkem Knall\footnote{\swabfamily{The Americ. Journ. ob. sc. a arts. f. Dana a. Silliman. Mai 1875. Vol. IX. Nr. 53. p. 407.}} der Fall eines weithin sichtbaren Meteors, welcher eine gro"se Anzahl von Steine lieferte. Smith berichtet, dass bis dahin ungef"ahr 150 Kilogramm Steine gesammelt wurden, von denen 25 Kilogramm Prof. Hinrichs zukamen. Seiner G"ute verdankt die Akademie ein ungef"ahr 1500 Gramm schweres pr"achtiges St"uck, welches die Veranlassung, zu der nach folgender n"aherer Beschreibung der Beschaffenheit dieses h"ochst merkw"urdigen Meteorsteins gab.

Der Meteorit von Iowa [Homestead] geh"ort zu der Klasse jener am h"aufigsten vorkommenden Steine, welche man als Chondrite zu bezeichnen pflegt, oder nach [Gabriel Auguste] Daubrée in die Abteilung der Sporadosiderite und in die Gruppe der Oligosiderite, einreiht wie bereits Prof. Hinrichs\footnote{\swabfamily{Comptes rendus d. s. de l'Acad. d. sc. 1875. p. 1175.}} in dem Begleitbriefe zu einem der Pariser Akademie "uber-schickten St"uck dieses Meteorsteins ganz richtig bemerkt hatte und Daubrée selbst best"atigte.

Der ziemlich scharfkantige, spitzwinkelige, unregelm"a"sig tetraedrische Stein ist mit einer schwarzen Schmelzrinde rings bedeckt, und im Innern licht gr"aulich wei"s, mit zahlreichen kleinen schwarzen Kn"ollchen und K"ornchen von Meteor- und Schwefeleisen, und zerstreut vorkommenden kleinen Rostflecken versehen. Der Stein ist ziemlich hart und l"asst sich nicht mit der Hand zerreiben. Er gleicht dem allgemeinen Charakter nach sehr dem Steinmeteoriten von Pultusk, indem er wie dieser, abgesehen von Meteor- und Schwefeleisen, aus einer wei"slichen und gelblichen Hauptmasse besteht, in welcher einzelne glasgl"anzende Olivink"ornchen und teils dunklere, teils hellere, zuweilen opake K"ugelchen (Sphaerochondren) sich abheben. Daubrée\footnote{\swabfamily{Comptes rendus d. s. de l'Acad. d. sc. 1875. 1175.}} vergleicht ihn mit den Steinmeteoriten von Vouillé (13. Mai 1831) und von Aumale in Algier (25. August 1865). Es wird durch diesen Fall die bereits "uber alle andern Arten von Meteorsteinen weit "uberwiegende Zahl der Chondrite wiederum um eine vergr"o"sert und der Eindruck des einheitlichen Ursprungs aller dieser Fragmente von einem einst zusammengeh"origen Ganzen, den auch neulich Meunier\footnote{\swabfamily{Cours de géologie comparée. Vergleiche auch: Tschermak, Bildung der Meteoriten, Sitzungsberichte der Akademie der Wissenschaften in Wien. Bd. LXXI, 2. Abth. 1875.}} so stark betont, wesentlich verst"arkt.

Die "au"sere, ziemlich scharfkantige und eckige Form der Steine dieses Falls, welche durch die d"unne, oberfl"achliche Schmelzrinde nur wenig verwischt wird, deutet unzweifelhaft auf Bruchst"ucke einer zersplitterten gr"o"seren Steinmasse hin, welche durch Zertr"ummerung einer bereits vollst"andig fertigen festen Substanz entstanden sind. Dass diese Zerst"uckelung zum Teil w"ahrend des Falls durch die Atmosph"are der Erde erfolgte, wird durch die Beobachtung Smith's angedeutet, welcher angibt, dass mehrere der gefallenen Steine wie frisch gebrochen aussahen und dass sich auf diesen Bruchfl"achen eine erst beginnende Schmelzung zeigt. Im "Ubrigen aber bemerkt man keine Abrundung, keine fadenf"ormige Ausbreitung oder strickartig gewundene, streifige Ausbildung, wie sie ein weicher, formbarer K"orper bei einer Bewegung auf kosmischen Bahnen erhalten, oder aber bei vulkanenartigen Eruptionen im Fluge annehmen m"usste, "ahnlich den Rapilli und vulkanischen Bomben. Auch weist die innere tr"ummerig k"ornige Beschaffenheit ohne Spur von glas- oder lavaartigen Teilchen, welche mit einer feuerfl"ussigen Schmelzung der Masse direkt nicht in "Ubereinstimmung gebracht werden kann, jeden Gedanken an ein Eruptionsprodukt im Styl unserer Vulkane entschieden zur"uck. "Au"sere Form und innere Beschaffenheit dieser Art Meteorite sprechen demnach von petrographischen Standpunkten nicht zu Gunsten der Annahme, dass diese Meteorsteine als Erzeugnisse von gewaltigen vulkanenartigen Eruptionen etwa von den Monden ausgeworfen seien. Ebenso unwahrscheinlich ist ihre Abstammung aus dem Schwarm der Sternschnuppen schon deshalb, weil die Zeit der Meteoriten-f"alle, soweit die Beobachtungen reichen, nicht mit der Zeit zusammentrifft, in welche das Maximum des Erscheinens der Sternschnuppen fallt. Auch w"are bei dieser Annahme die so auffallende Gleichartigkeit in der Zusammensetzung der Steinmeteoriten kaum zu erkl"aren. Es gewinnt daher die Ansicht sehr an Wahrscheinlichkeit, dass wir es mit Bruchst"ucken von Himmelsk"orpern zu tun haben, welche durch eine Zertr"ummerung, sei es in Folge von Zusammensto"s oder durch eine Art Zerst"aubung aus inneren Ursachen entstanden sind, wobei die Schleuderkraft das "Ubergewicht "uber die urspr"ungliche Attraktionskraft erlangte und die Tr"ummer in die Anziehungsn"ahe der Erde gelangt, auf diese fallen mussten. Ob sie Teile eines Asteroidenk"orpers, oder, wie Meunier will, eines zweiten Erdtrabanten sind, bleibt astronomischen Er"orterungen, die hier ferne liegen, zu entscheiden vorbehalten.
\clearpage
\subsection{\swabfamily{Rinde}}
\paragraph{}
Der vorliegende Steinmeteorit von Iowa ist "au"serlich, abgesehen von einer kleinen k"unstlichen Bruchstelle, allseitig mit einer schwarzen, mattschimmernden, schwach-runzeligen Rinde von durchschnittlich 0,05 m Dicke "uberzogen. Dieser glasartige "Uberzug ist einfach rissig, zerkl"uftet und lasst sich ziemlich leicht von der Hauptmasse abl"osen, wobei jedoch Teile des letzteren daran haften bleiben. Im Innern des Steines bemerkt man an dem vorliegenden Stein keine der Rinde "ahnliche Adern oder glatte Flachen, welche z. B. die Steine von Pultusk so h"aufig durchziehen.

Diese Rinde besteht nach n"aherer Untersuchung ans einer schwer durchsichtigen, glasartigen Masse, welche das Licht einfach bricht und stellenweise zahlreiche Bl"aschen und Poren umschlie"st, doch nicht in so ausgezeichneter Weise, wie ich dies an der Rindensubstanz des Steins von Pultusk beobachtet habe. Die Rinde ist "uber die Oberfl"ache des Steins nicht ganz in gleicher Weise ausgebreitet; an einzelnen Stellen erkennt man die bei gelinde Reiben mit metallischem Glanz hervortretenden Meteoreisenteilchen, an "andern ist sie "au"serst d"unn und etwas heller gef"arbt, oder aber auch dicker und zugleich meist auch st"arker gl"anzend. Wie D"unnschliffe zeigen, entsprechen die d"unnrindigen Stellen dem Hineinragen von Olivink"ornchen in die Rindenregion, w"ahrend da, wo Schwefeleisen hier vorkommt, eine dickere Schmelzrinde entstanden ist.

Es ist wegen der Tiefe der F"arbung sehr schwierig, die Rinde in D"unnschliffen durchsichtig zu erhalten. Leichter gelingt dies durch Zerdr"ucken kleiner Splitterchen zwischen zwei Glasbl"attchen. Sie zeigt alsdann eine tief bouteillengr"une bis braunrote Farbe und verh"alt sich im polarisierten Lichte wie eine amorphe Glasmasse. Diese Beschaffenheit best"atigt die Annahme, dass die Rinde durch eine oberfl"achliche Schmelzung beim Fliegen durch die Atmosph"are der Erde gebildet worden sei, also eine echte Schmelzrinde darstellt. Zu den Vergleichen wurden kleine Splitterchen aus dem Innern des Steins v. d. L. geschmolzen, was nur in ganz d"unnen St"uckchen an den feinen Spitzen gelingt. Die geschmolzene Masse zeigt ganz die Beschaffenheit der Schmelzrinde, dieselbe Farbe und dieselben Bl"aschen. Eigent"umlich verh"alt sich der Stein, wenn man ihn, ohne zu schmelzen, l"angere Zeit einer starken Rotglut aussetzt. Er nimmt dabei eine dunkle, braunschwarze Farbe an und zeigt beim Durchschlagen einzelne Flecken, die wie geschmolzen aussehen. Es sind dies die R"ander um die Schwefelkiesputzen, welche in der Tat eine Schmelzung erlitten haben. Verfertigt man von solchen gegl"uhtem St"uckchen D"unnschliffe, so sieht man in denselben, dass die gr"o"sere Masse, woraus der Stein besteht, durch das Gl"uhen eine tief braune Farbe angenommen hat, welche, wie ich fr"uher\footnote{\swabfamily{Die pal"aol. Eruptivgesteine des Fichtelgebirges 1874. S. 39.}} schon hervorgehoben habe, ein sehr gutes Kennzeichen f"ur die Olivinbeimengung abgibt. Die schwarzen R"ander um die Schwefelkiesteilchen sind fast undurchsichtig, tiefbraun-gef"arbt und brechen das Licht gleichfalls einfach, genau wie die Schmelzrinde. Diese dunkle Farbe, welche der Stein beim Erhitzen annimmt, die sich aber am nat"urlichen Stein selbst dicht unter der Schmelzrinde nicht vorfindet, beweist, dass die Schmelzhitze ihre Wirkung auf eine au"serordentlich d"unne Lage der Oberfl"ache beschr"ankte, ohne tiefere Teile des Gesteins in h"ohere Hitzegrade zu versetzen. Dieser Erscheinung gegen"uber ist die Durchaderung mancher Meteorsteine anderer Fundstellen von ganz d"unnen schwarzen Streifchen h"ochst bemerkenswert. Bei dem Stein von Pultusk, von dem mir Material zur Verf"ugung stand, fand ich, dass diese "Aderchen gleichfalls aus amorpher Glassubstanz bestehen. "Ahnlich scheinen sich auch die schwarzen, fast undurchsichtigen Flecke zu verhalten, welche in manchen Meteorsteinen durch die ganze Masse zerstreut vorkommen und wahrscheinlich die R"ander um leichteren Schmelzfluss erzeugende Einmengungen z. B. Schwefelkies darstellen.

Ich glaube jedoch nicht, dass die oben erw"ahnten feinen "Aderchen eine geschmolzene Masse ist, die von der Rinde aus ins Innere des Gesteins eingedrungen ist, sondern dass an solchen Stellen der Stein zersprungen oder rissig war, und dass sich auf diesen der Atmosph"are zug"anglichen Rissen derselbe Schmelzprozess durch Reibung vollzog, wie auf der Oberfl"ache selbst.
\clearpage
\subsection{\swabfamily{Gesteinsmasse}}
\paragraph{}
Die ziemlich harte, zwischen den Fingern nicht zerreibliche Hauptmasse des Steins besteht aus einer Zusammenh"aufung von Tr"ummerteilchen, die ohne jede Zwischensubstanz aneinander agglutiniert sind, da sich weder ein glasartiges noch "uberhaupt ein ausgesprochenes Bindemittel zwischen den einzelnen K"ornchen beobachten l"asst. In gr"o"ster Anzahl finden sich in der Hauptmasse kleine Splitterchen von Mineralien mit v"ollig unregelm"a"sigen Umrissen, wie sie durch Zertr"ummerung von Kristallen oder kristallinischen Massen entstehen. Nur h"ochst selten sieht man --- im D"unnschliffe --- solche St"uckchen, welche von regelm"a"sigen geraden Linien begrenzt, als Krist"allchen oder regelm"a"sige Spaltungsk"orperchen gelten k"onnen (k der lithogr. Tafel). Dazu gesellen sich unregelm"a"sige eckige K"ornchen, die an ihrem Glasglanz und an ihrer Farbe ziemlich sicher als Olivin zu bestimmen sind (o), wei"sliche Putzen einer opaken Substanz, kleine K"ornchen von bleigrauem metallgl"anzendem Meteoreisen (f), tombackgelbe vielfach durchbrochene H"aufchen von Schwefeleisen (s), deren feine K"ornchen selten geschlossene Massen ausmachen und endlich jene kleinen abgerundeten bald dunkel-, bald hellfarbigen K"ugelchen (Sph"arochondren c), welche dem Stein den Charakter der Chondrite Rose's aufdr"ucken. Zerstreut oder zu kleinen Gruppen vereinigt stellen sich weiter noch "au"serst feine schwarze, nicht metallischgl"anzende Staubteilchen (ch) ein, die entweder Chromeisen oder einer kohligen Substanz angeh"oren, da sie jeder Einwirkung von S"auren Widerstand leisten.
\clearpage
\begin{figure}[t]
\centering
\includegraphics[keepaspectratio]{Fig1.png}
\caption{Tafel 1}
\end{figure}
Das Bild auf der beigegebenen lithographirten Tafel zeigt die Art der Verteilung dieser Gemengteile in einem D"unnschliff in 25-maliger Vergr"o"serung.

Erkl"arung der Randbezeichnungen der Lithographie\\
\begin{minipage}[t]{0.54\textwidth}
o --- Olivin,

a --- Augitische Teile,

f --- Meteoreisen,

s --- Schwefeleisen,

ch --- Chromeisen,

k --- Teile mit kristallartigem Umrisse,

io --- Olivink"ornchen im Meteoreisen,

g --- r"otliche granat"ahnliche Einschl"usse,
\end{minipage}
\begin{minipage}[t]{0.46\textwidth}
c --- Sph"arochondren und zwar:

cc --- mit konzentrischer Struktur,

sc --- mit fasriger Struktur,

fc --- mit strahliger Struktur,

kc --- mit k"orniger Struktur,

oc --- aus Olivin bestehend,

dc --- opake feink"ornige K"ugelchen.
\end{minipage}
\paragraph{}
Eine der merkw"urdigsten Erscheinungen bei fast s"amtlichen Gemengteilen, die metallischen abgerechnet, zeigt sich darin, dass die einzelnen St"uckchen von einer erstaunlichen Menge feiner und feinster Risse durchzogen sind. Bei manchen Gemengteilen zeigt sich in der Richtung dieser endlosen Zerkl"uftung eine gewisse Regelm"a"sigkeit durch einen parallelen Verlauf der Risse, welche vermutlich im Zusammenhang mit der Spaltungsrichtung der betreffenden Mineralien steht. Aber gleichzeitig treten neben diesen mehr regelm"a"sig verlaufenden andere Risse hervor, die jene rechtwinklig oder schief durchkreuzen und ein wahres Netzwerk von Rissen erzeugen, so dass selbst sonst helle Mineralteilchen dadurch getr"ubt erscheinen. Sie m"ussen als ein Zeichen erlittener Zertr"ummerung durch Sto"s, Druck oder raschen Temperaturwechsel angesehen werden.

Durch diese rissige Beschaffenheit der meisten Gemengteile wird die weitere innere Natur vielfach verdeckt, so dass man nur selten in einzelnen gr"o"seren Teilchen die, wie es scheint, h"aufig vorkommenden Bl"aschen --- aber soweit meine Beobachtungen reichen --- ohne Fl"ussigkeitseinschl"usse, erkennen kann. Auch "au"serst feine, staubartige Einmengungen zeigen sich h"aufig in den sonst hellen Mineralteilchen, obwohl eigentliche Mikrolithe zu fehlen scheinen.

Was die mineralogische Natur der einzelnen Gemengteile anbelangt, so d"urfte eine gro"se Anzahl derselben nicht einfachen Mineralien angeh"oren, sondern Gesteinssplitter, die aus mehreren Mineralien zusammengesetzt sind, oder eine mehr oder weniger regelm"a"sige Verwachsung verschiedener Mineralien darstellen.

Olivin nimmt unter den einfachen Mineralteilchen zweifelsohne die erste Stelle ein. Nicht blo"s das "au"sere Ansehen, die Farbe, der eigent"umliche Glanz weisen viele der gr"o"seren K"ornchen und Kristallsplitterchen dem Olivin zu, sondern diese Bestimmung findet ihre Best"atigung auch in der Zersetzbarkeit dieser Teilchen durch Salzs"aure, in dem Braunwerden beim Gl"uhen, und in dem bunten Farbenspiel bei Anwendung des polarisierten Lichtes in D"unnschliffen. Die meisten feink"ornig zerkl"ufteten Teile in der Abbildung geh"oren dem Olivin (o) an, ebenso viele der kristallartig regelm"a"sig umgrenzten St"uckchen und selbst von den kugeligen Ausscheidungen wurden mehrere sicher als Olivin erkannt. Aber auch in der staubartig feinen Zwischenmasse, welche die einzelnen gr"o"seren Fragmente zu verbinden scheint, machen sich Olivinteilchen bemerkbar, wie das Braunwerden derselben beim Gl"uhen erkennen l"asst. Am eigent"umlichsten ist die Olivinsubstanz in manchen felderartig gestreiften K"ugelchen (sc der Abbildung) mit einer wei"sen, federartig streifigen Substanz, wie solche f"ur sich in den strahlig fasrigen K"ugelchen vorkommt, in lamellarer Verwachsung nach Art des Schriftgranits verbunden. Die schmalen, abgesetzt verlaufenden Olivinlamellchen treten sehr deutlich nach dem Gl"uhen durch ihre dunkelbraune F"arbung hervor. Dass sie einer Olivinsubstanz angeh"oren, ergibt sich bei Behandeln mit Salzs"aure, wobei sie zersetzt werden, w"ahrend die meisten Zwischenlamellen unver"andert stehen bleiben.

Feldspatige Bestandteile vermochte ich mit Sicherheit nicht nachzuweisen, obwohl einzelne wasserhelle N"adelchen i. p. L. die eigent"umlichen fahlgelben und blauen Farben zeigen, wie solche f"ur den Feldspat so charakteristisch sind und wie ich sie mit aller Bestimmtheit in gr"o"ser Menge in dem Meteorstein von L'Aigle (Fall am 26. April 1803) beobachtete, der zahlreiche, von Feldspatn"adelchen vollgespickte Gesteinstr"ummer enth"alt. Auch die chemische Analyse best"atigt, dass jedenfalls feldspatige Teile nur in h"ochst untergeordneter Weise an der Zusammensetzung beteiligt sind.

Behandelt man massig feines Pulver l"angere Zeit mit Salzs"aure in der W"arme, so zersetzt sich ein gro"ser Teil der Gesteinsmasse --- der Olivinanteil --- unter Abscheidung schleimiger Kiesels"aure ohne eigentliche Gallerte zu bilden. In dem durch Kochen mit Alkalien von Kiesels"aure befreiten R"uckstande erkennt man nun sehr zahlreiche, oft wasserhelle, parallelstreifige Teilchen, neben tr"uben pulvrigk"ornigen Resten, die meistenteils von zertr"ummerten K"ugelchen herr"uhren. Auch die feinen, schwarzen K"ornchen, welche hie und da gruppenweise Vorkommen, sind ungel"ost geblieben, w"ahrend neben Olivin das Meteor- und Schwefeleisen in L"osung "ubergegangen sind. Die mehr oder weniger wasserhellen St"uckchen, die ungel"ost geblieben sind, erweisen sich als doppeltbrechend und zeigen die sch"onsten Aggregatfarben i. p. L. Behandelt man diesen Rest noch weiter mit Flusss"aure, so zersetzt er sich vollst"andig bis auf die feinen schwarzen K"ornchen, welche Chromeisen oder einer kohligen Substanz angeh"oren. Da beim Aufschluss der Gesteinsmasse mittelst Baryterdehydrat sich ein Gehalt an Chrom ergibt, so ist es h"ochst wahrscheinlich, dass die schwarzen K"ornchen Chromeisen sind. Ich beobachtete zwar mehrfach beim Gl"uhen des Gesteinpulvers ein sporadisches Verglimmen wie von kohligen Teilchen, ich konnte mich jedoch nicht bestimmt "uberzeigen, ob dies nicht von Staubteilchen herr"uhrt, die dem Stein nicht urspr"unglich angeh"oren, sondern nur mechanisch anhaften.

"Andert man den Versuch in der Weise ab, dass man ein nicht zu d"unngeschliffenes, jedoch gut durchsichtiges Bl"attchen des Gesteins erst in Salzs"aure kocht, so beh"alt dasselbe noch seinen Zusammenhalt. Auf eine Glasplatte aufgenommen und nun mit kaustischer Kalilauge sorgf"altig behandelt, um die freigewordene Kiesels"aure zu beseitigen, stellt dasselbe ein l"ocheriges Pr"aparat dar, aus dem der Olivin, Meteoreisen und Schwefeleisen verschwunden sind, w"ahrend das wei"se Mineral und die meisten K"ugelchen unver"andert sich erhalten haben. Versucht man das auf diese Weise erhaltene Pr"aparat mittelst Kanadabalsam unter einem Deckgl"aschen zu konservieren, so zerteilt sich bei dem geringen Druck, den das Auflegen des Deckgl"aschens verursacht, die Masse in einzelne H"aufchen des wei"sen Minerals, in einzelne Flocken und in die runden K"ugelchen, welche oft ganz frei hervortreten und eine unebene rauhe Oberfl"ache erkennen lassen. Au"serdem machen sich "au"serst sp"arlich kleine, lichtgranatrote, ziemlich regelm"a"sig 5-6 seitige K"orperchen bemerkbar, die ich auch in den D"unnschliffen (g) beobachtete. Sie erinnern an Granaten, erweisen sich aber als doppelt brechend. Die Farbe l"asst auch an Noseau denken. Doch stimmt auch damit die optische Eigenschaft derselben nicht.

"Uber die Natur der in Salzs"aure unzersetzten, hellen Mineralteilchen, die wahrscheinlich der Gruppe des Augits angeh"oren, vermag nur die chemische Analyse Aufschluss zu geben. Aber auch hierbei stellt sich eine gewisse Unsicherheit wegen der Anwesenheit der zahlreichen, gleichfalls in Salzs"aure unzersetzten K"ugelchen (abgesehen von den Olivink"ornchen) ein, die weder mit dem hellen Mineral identisch zusammengesetzt sind, noch "uberhaupt einem einfachen Mineral entsprechen. Manche dieser K"ugelchen n"ahern sich in ihrem physikalischen Verhalten dem wei"sen Mineral, zeigen aber doch eine eigent"umliche Art der Zerkl"uftung. Andere bestehen deutlich aus Lamellen verschiedener verwachsener Mineralien und noch andere sind wenig durchsichtig wei"s, pulvrig k"ornig und zeigen h"aufig eine konzentrische Struktur mit dunkleren und helleren Zonen, oft auch mit einer rindenartigen dunklen Umh"ullung oder einem teils dunklen, teils hellen Zentrum. Schwarze staubartige K"ornchen, die in denselben vorkommen, sind meist gleichfalls konzentrisch oder radikal geordnet. Doch sind diese K"ugelchen nicht amorph, da der Lichtschimmer in p. L. deutlich gef"arbt erscheint. Ihnen schlie"st sich endlich die merkw"urdigste Art dieser K"ugelchen an, welche "au"serst fein radialgestreift und feingek"ornelt, schwach durchscheinend, wei"slich gef"arbt erscheinen. Die strahligen Streifchen sind exzentrisch und stehen in keinen Zusammenhang mit der "au"seren Form der K"ugelchen. Oft treten in einem K"ugelchen mehrere Systeme von Streifchen felderweise neben einander auf. Im p. L. erscheinen trotz der geringen Durchsichtigkeit deutlich b"uschelf"ormige Farben, welche entfernt an die bekannte Erscheinung bei vielen Variolitkn"ollchen erinnert, ohne ihr jedoch ganz gleich zu kommen. Von der lamellaren Verwachsung olivinartiger Streifchen mit einer "ahnlich fasrigen wei"sen Substanz ist schon fr"uher berichtet worden.

Was nun die Entstehung dieser merkw"urdigsten unter den Gemengteilen der Meteorsteine anbelangt, so nimmt Daubrée\footnote{\swabfamily{Journ. d. savants. 1870. p. 38.}} an, dass sie durch ein Festwerden w"ahrend eines wirbelnden Flugs durch Gase sich gebildet h"atten, w"ahrend Tschermak\footnote{\swabfamily{Sitz. d. k. Ac. d. Wiss. Wien. Bd. LXXI. 2. Abth. 1875. Aprilheft S. 9 u. 10.}} sich f"ur eine Entstehung in Folge einer Abrollung bereits fester Tr"ummer durch eine anhaltende Bewegung, wie sie durch vulkanische Explosion erzeugt wird, unter Hinweis auf "ahnliche runde K"ugelchen im trachytischen Tuff von Gleichenberg etc. ausspricht. Durch die letztere Annahme erkl"art sich die bei vielen K"ugelchen wahrgenommene Eigent"umlichkeit, dass ihre innere Fasenstruktur ohne alle Beziehung steht mit der "au"seren Kugelform. Selbst bei den K"ugelchen mit deutlich konzentrischer Struktur d"urfte an dieser Art der Entstehung festzuhalten sein, wenn man annimmt, dass, wie sehr wahrscheinlich ist, die konzentrischen Streifchen und schaligen Absonderungen nur als sekund"are Erscheinungen, als Folgen mechanischer und chemischer Ver"anderungen, welche das abgerundete Korn erst nach der Abrollung erlitten habe, aufzufassen sind.

Einen wesentlichen Anteil an der Zusammensetzung des Steins von Iowa nimmt das Schwefeleisen. Es erscheint auf kleine, unregelm"a"sig umgrenzte Flecke verteilt zwischen die "ubrigen Gemengteile gleichsam eingezw"angt. Bei dem Behandeln des Gesteinspulvers mit Salzs"aure entwickelt sich Schwefelwasserstoff, ohne dass sich Schwefel ausscheidet. Es d"urfte daher dieses Schwefeleisen als Troilit zu bezeichnen sein. Noch h"aufiger erscheinen die aus Meteoreisen bestehenden K"ornchen der Gesteinsmasse in meist zackigen, winkelig gebogenen, oft in feine Spitzen auslaufenden Kl"umpchen beigemengt, welche so innig an die nicht metallischen Teile sich anschmiegen, als ob das Eisen erst zuletzt etwa durch Reduktion an der Stelle ausgeschieden worden w"are, wo es sich vorfindet. Dieses Meteoreisen ist nickel- und etwas phosphorhaltig, sehr dehnbar, indem es sich mit dem Hammer leicht in d"unne Bl"attchen ausschlagen l"asst und aktiv, wie sich zeigt, wenn man ein geschliffenes St"uckchen in Kupfervitriol taucht, wobei die Eisenfl"ache sich rasch mit einem Kupferniederschlag bedeckt. Ob durch An"atzen die Widmanst"atten’schen Linien zum Vorschein kommen, konnte ich bei der Kleinheit der Eisenk"ornchen nicht deutlich erkennen. Doch zeigten sich hellere und dunklere Flecke.

Dass das Gestein Wasser enth"alt, bedarf nicht erst eines Nachweises, da dies die nicht seltenen Rostflecke --- Eisenoxydhydrat --- zum Voraus verraten.

Auch verschiedene Gasarten sind bereits durch Wriht\footnote{\swabfamily{The Americ. Journ. o. sc. a. arts. J. Dana a. Silliman May 1875. Vol. IX. Nr. 54. p. 459; auch Ann. d. Phys. u. Chem. Erg"anz. Bd. VII St"uck 2.}} in diesem Meteorite von Iowa nachgewiesen. Die vorl"aufigen Versuche Wriht's ergaben einen Gehalt an Gas, welches fast zur H"alfte aus Kohlens"aure und Kohlenoxyd (CO2 = 35; CO = 14), im "Ubrigen haupts"achlich aus Wasserstoff besteht.

Das spezifische Gewicht des Steins in der inneren Masse betr"agt = 3,75; das eines Rindenst"uckes = 3,55 bei 20$^{\circ}$ C.
\clearpage
\subsection{\swabfamily{Chemische Analyse}}
\paragraph{}
Zur Vornahme einer chemischen Analyse stand mir etwas "uber 1,5 Gramm Substanz zur Verf"ugung. Aus dem fein zerriebenen Pulver wurde zuerst mit aller Sorgfalt das Meteoreisen mit dem Magnet ausgezogen und dies durch wiederholtes Verfahren m"oglichst von allen anhaftenden Gesteinsteilen befreit, alsdann besonders analysiert. Ein Teil diente zur Schwefelbestimmung, das "ubrige wurde zuerst mit kochender Salzs"aure behandelt, der auf diese Weise zersetzte Anteil und ebenso der unzersetzte mittelst Barythydrat aufgeschlossene weiter analysiert.

Er ergab sich hierbei folgendes Resultat:

Das Gestein besteht aus
\begin{center}
    \begin{tabular}{ |l|r| } 
    \hline
    Meteoreisen & 12,32\\\hline
    Troilit & 5,25\\\hline
    in Salzs"aure zersetzbarem Teil & 48,11\\\hline
    in Salzs"aure unzersetzbarem Teil & 34,32\\
    \hline
    \end{tabular}
\end{center}
\paragraph{}
Das Nickeleisen besteht au"ser Spuren von Kupfer und Schwefel, letzterer wahrscheinlich von etwas anh"angenden Troïlit abstammend, aus

Eisen 83,38

Nickel (etwas cobalthaltig mit Schwefel und Phosphor) 16,62

also ann"ahernd Fe\textsubscript{5}Ni

Der in Salzs"aure zersetzbare Teil\footnote{\swabfamily{Diese Analysen wurden von Hrn. Assistent Ad. Schwager ausgef"uhrt. (1875. 3. math.-phys. Cl.)}} (ohne Meteor-und Schwefeleisen berechnet) aus
\begin{center}
    \begin{tabular}{ |l|r|r| } 
    \hline
    Kiesels"aure & 38,38 & Sauerstoff: 19,76\\\hline
    Eisenoxydul & 28,58 & 6,33\\\hline
    Manganoxydul & 0,53 & 0,12\\\hline
    Bittererde & 31,49 & 12,59\\\hline
    Tonerde & 1,01 & 0,47\\\hline
    Kalkerde, Alkalien, Wasser & Spuren &\\
    \hline
    \end{tabular}
\end{center}
Der in Salzs"aure unzersetzte Rest besteht aus
\begin{center}
    \begin{tabular}{ |l|r|r| } 
    \hline
    Kieselerde & 53,96 & Sauerstoff: 28,74\\\hline
    Tonerde & 2,01 & 0,94\\\hline
    Eisenoxydul & 25,18 & 5,57\\\hline
    Bittererde & 8,91 & 3,56\\\hline
    Kalkerde & 4,04 & 1,16\\\hline
    Manganoxydul & Spuren &\\\hline
    Chromoxyd & 1,42 & 1,16\\\hline
    Natron & 2,39 & 0,59\\\hline
    Kali & 1,67 & 0,29\\
    \hline
    \end{tabular}
\end{center}
\paragraph{}
Was das Meteoreisen und das einfache Schwefeleisen anbelangt, so bedarf es hier"uber keiner weiteren Er"orterungen. In dem durch Salzs"aure zersetzbaren Anteil stellt sich ein Sauerstoffverh"altniss der Basen und S"aure nahezu wie 1:1 heraus und es bedarf auch hier wohl kaum einer weiteren Ausf"uhrung, dass dieser Anteil von einem eisenoxydulreichem Olivin weit vorherrschend herstammt. Weit schwieriger ist die Deutung des in Salzs"aure nicht zersetzbares Bestes, dessen Bestandteile und ihre Sauerstoffverh"altniss zu keinem bestimmten Minerale passen. Dies stimmt auch vollst"andig mit der optischen Analyse "uberein, bei welcher nach Entfernung der in Salzs"aure l"oslichen Anteile neben den Sph"arochondren in ihrer sehr verschiedenartigen Beschaffenheit noch ein helles rissiges Mineral und kleine schwarze K"ornchen nachgewiesen wurden. Dass die letzteren aus Chromeisen bestehen, ist nach den Resultaten der Analyse jetzt kaum mehr zu bezweifeln. Das helle, rissige Mineral geh"ort wohl sicher der Augitgruppe an. Ganz au"sergew"ohnlich ist der hohe Eisenoxydulgehalt, auch wenn man einen entsprechenden Teil als am Chromoxyd zu Chromeisen verbunden in Abzug bringt, wogegen die Armut an Bittererde und Kalkerde auf der anderen Seite auffallend ist. Der hohe Gehalt an Alkali scheint weiter mehr Bezug auf die Zusammensetzung der K"ugelchen zu haben und auf deren feldspatige Zusammensetzung hinzudeuten. Geh"ort die Tonerde, wie wahrscheinlich, diesem Gemengteile mit der entsprechenden Menge Kiesels"aure an, so k"onnte schlie"slich sich eine, --- aber immer nur beil"aufig entsprechende Zusammensetzung eines eisenreichen Augits, wie solcher in den Eukriten, z. B. dem von Juvinas gefunden wird, herausstellen. Immerhin scheint die n"ahere Natur dieses augitischen Gemengteiles schwierig ermittelt werden zu k"onnen. Obwohl die Analyse, welche J. L. Smith* von dem Iowa-Meteoriten mitteilt, nicht genau mit der obigen stimmt, so wird doch auch in dieser ein ungew"ohnlich hoher Eisenoxydulgehalt in dem in S"auren unl"oslichen Anteil n"amlich 27,41\% angegeben. Des Vergleichs wegen folgen hier die Smith'schen Angaben:

Der ganze Stein besteht aus:
\begin{center}
    \begin{tabular}{ |l|r| } 
    \hline
    Steinige Masse & 81,64\\\hline
    Troilit & 5,82\\\hline
    Nickeleisen & 12,54\\
    \hline
    \end{tabular}
\end{center}
Der steinige Anteil enth"alt:

A) 54,15 in S"auren zersetzbare,

B) 45,85 in S"auren unzersetzbare Substanzen.

Diese bestehen nun weiter
\begin{center}
    \begin{tabular}{ |l|r|r| } 
    \hline
    Kiesels"aure & 35,61 & 55,02\\\hline
    Eisenoxydul & 27,20 & 27,41\\\hline
    Magnesia & 33,45 & 13,12\\\hline
    Tonerde & 0,71 & 0,84\\\hline
    Alkalien, Eisen etc. & 1,45 & 2,01\\
    \hline
    \end{tabular}
\end{center}
Darnach berechnet Smith die Zusammensetzung des Meteoriten aus:
\begin{center}
    \begin{tabular}{ |l|r| } 
    \hline
    Olivin & 44,09\\\hline
    Pyroxen & 37,55\\\hline
    Troilit & 5,82\\\hline
    Nickeleisen & 12,54\\
    \hline
    \end{tabular}
\end{center}
\paragraph{}
Dabei haben die runden K"ugelchen keine weitere Ber"ucksichtigung gefunden, was gewiss nicht naturgem"a"s erscheint, da sich diese K"ugelchen nicht ohne Weiters als aus Augit bestehend ansehen lassen.

Unter den bisher analysierten Chondriten ist nur jener von Tadjera von einer "ahnlichen Zusammensetzung\footnote{\swabfamily{Rammelsberg, D. chem. Natur d. Meteoriten, S. 157.}}, jedoch kiesels"aure"armer und kalkreichen.

Fasst man die Ergebnisse der Untersuchung des Steinmeteoriten von Iowa zusammen, so berechtigen sie zu folgenden Schl"ussen:
\begin{enumerate}
    \item Die Steinmasse besteht aus unregelm"a"sigen Mineralsplitterchen von Olivin und einer augit"ahnlichen Substanz, welche von einem zertr"ummerten Gestein hergenommen scheinen. Denselben sind einzelne aus verschiedenen Mineralien zusammengesetzte St"uckchen beigemengt. Auch scheint eine feldspatige Substanz in geringer Menge vorhanden zu sein. Fein zerriebene Teilchen dieser Mineralien schienen das Kittmittel abzugeben.
    \item Die rundlichen K"ugelchen machen neben den erw"ahnten Mineralteilchen einen wesentlichen Teil der Substanz des Steins aus. Sie geh"oren teils dem Olivin an, teils stellen sie lamellare Verwachsungen von Mineralien dar oder bestehen aus strahlig fasriger Masse. Ein Teil derselben scheint aus feldspatiger Substanz zu bestehen. Ihre Form verdanken sie einer mechanischen Abrundung.
    \item Die Meteoreisenk"ornchen liegen so zwischen den Mineralsplitterchen und K"ugelchen angeschmiegt, als seien sie erst nachtr"aglich durch Reduktion entstanden.
    \item Von Glas- oder Lava-"ahnlichen Beimengungen (die Schmelzrinde ausgenommen) ist in dem Gestein nichts zu finden. Es ist kein aus dem Schmelzfluss hervorgegangenes, kristallinisches, sondern ein klastisches Gestein, dessen Gemengteilchen nicht die Eigenschaften einer vulkanischen Asche an sich tragen.
\end{enumerate}
\clearpage
\end{document}
